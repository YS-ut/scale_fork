\section{Quick Start}
%##########################################################################################

\subsection{Preparation}

Please see the Chapter 2.1 and 2.2 of ``SCALE USERS GUIDE'' 
as to the details of required system environment, library, and settings of environmental parameters.

SCALE-GM requires numerical computation library, Lapack.
Please install Lapack to your system in addition to other required libraries; HDF5, netCDF, and MPI.


\subsection{Compile}
%------------------------------------------------------------------------------
Please refer the Chapter 2.3.1 of ``SCALE USERS GUIDE'' for how to get the source code
and the Chapter 2.3.2 for the system environmental parameters.

Change directly to test case directly, for example,
\begin{verbatim}
  > cd ${TOP}/src
\end{verbatim}

\noindent Compile the program using make command.
\begin{verbatim}
  > make -j 4
\end{verbatim}
When the compile is finished correctly, a "\verb|nhm_driver|" is created
in the current directly, which is an executable binary of SCALE-GM.

\begin{itemize}
  \item[*] The number of -j option is a number of parallel compile processes.
   To reduce elapsed time of compile, you can specify the number
   as more than two. We recommend 2 ~ 8 for the -j option.
\end{itemize}

\subsection{Run experiments}
%------------------------------------------------------------------------------
\subsubsection{Test cases}

\noindent In the \${TOP}/test/case, a lot of test cases sets are prepared.
In those, the cases for DCMIP2016 are shown in Table 1.
%Procedures for three ideal experiments are explained assuming the Yellowstone environment.

 \begin{table}[b]
 \begin{center}
 \caption{Corresponding test cases}
 \begin{tabularx}{150mm}{|l|X|} \hline
 \rowcolor[gray]{0.9} teat case name in NICAM & test case type \\ \hline
  DCMIP2016-11 & moist baroclinic wave test (161)       \\ \hline
  DCMIP2016-12 & idealized tropical cyclone test (162)  \\ \hline
  DCMIP2016-13 & supercell test (163)                   \\ \hline
 \end{tabularx}
 \end{center}
 \end{table}


\subsubsection{How to run}

To run the model, type "make run", and then the job script is
displayed in standard output. Hit "q" to quit.

 \begin{verbatim}
   > make run
   #! /bin/bash -x
   ##################################################################
   #
   # for NCAR Yellowstone (IBM iDataPlex Sandybridge)
   #
   ##################################################################
   #BSUB -a poe                  # set parallel operating environment
   #BSUB -P SCIS0006             # project code
   #BSUB -J nicamdc              # job name
   #BSUB -W 00:10                # wall-clock time (hrs:mins)
   #BSUB -n 10                   # number of tasks in job
   #BSUB -R "span[ptile=10]"     # run four MPI tasks per node
   #BSUB -q regular              # queue
   #BSUB -e errors.%J.nicamdc    # error file name
   #BSUB -o output.%J.nicamdc    # output file name
 \end{verbatim}

 \noindent If the job script is OK, submit a job to the machine.
 \begin{verbatim}
  > bsub < run.sh
 \end{verbatim}
 \noindent \textcolor{blue}{{\sf Caution}} : Do not miss the symbol "\verb|<|". \\


\subsection{Post process}
%------------------------------------------------------------------------------
 After finish of test run, create the lat-lon grid data from
 the original icosahedral grid data.
 Before submit a job of post process, edit \verb|ico2ll_netcdf.sh|
 following your experimental settings.
 \begin{verbatim}
   > vi ico2ll_netcdf.sh

   [at Line 22]
   # User Settings
   # ---------------------------------------------------------------------

   glev=5          # g-level of original grid
   case=161        # test case number
   out_intev='day' # output interval (format: "1hr", "6hr", "day", "100s")
 \end{verbatim}

 \noindent If the job script is OK, submit a job to the machine.
 \begin{verbatim}
   > bsub < ico2ll_netcdf.sh
 \end{verbatim}

 \noindent The netcdf format data such as "\verb|nicam.161.200.L30.interp_latlon.nc|"
 is created by an "ico2ll" post-process program.


