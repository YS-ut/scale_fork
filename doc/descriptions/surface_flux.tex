%\section{Surface flux}
{\bf \Large 
\begin{tabular}{ccc}
\hline
  Corresponding author & : & Seiya Nishizawa\\
\hline
\end{tabular}
}


\subsection{Monin-Obukhov similarity}
The first of the all,
we assume that in the boundary layer
1. fluxes are constant,
and 2. variables are horizontally uniform.


Relations between flux and vertical gradient are
\begin{align}
  \frac{kz}{u_*} \frac{\partial u}{\partial z} &= \phi_m\left(\frac{z}{L}\right), \label{eq: flux-gradient u} \\
  \frac{kz}{\theta_*} \frac{\partial \theta}{\partial z} &= \phi_h\left(\frac{z}{L}\right), \label{eq: flux-gradient t} \\
  \frac{kz}{q_*} \frac{\partial q}{\partial z} &= \phi_q\left(\frac{z}{L}\right),
\end{align}
where $k$ is the Von Karman constant.
$L$ is the Monin-Obukhov scale height, which is
\begin{equation}
  L = \frac{\theta u_*^2}{kg\theta_*},
\end{equation}
where $g$ is the gravity.
The scaling velocity, $u_*$, temperature, $\theta_*$,
and water vapor, $q_*$, are defined
from the vertical eddy fluxes of momentum, sensible heat and water vapor:
\begin{align}
  \overline{u'w'} &= -u_*u_*, \\
  \overline{w'\theta'} &= - u_*\theta_*, \\
  \overline{w'q'} &= - u_*q_*.
\end{align}

The integration between the roughness length $z_0$ to the height $z$ of the lowest model level, eqs. (\ref{eq: flux-gradient u}) and (\ref{eq: flux-gradient t}) become
\begin{align}
  u(z) &= \frac{u_*}{k} \left\{\ln(z/z_0)-\Phi_m(z/L)+\Phi_m(z_0/L)\right\}, \\
  \Delta\theta &= R\frac{\theta_*}{k} \left\{\ln(z/z_0)-\Phi_h(z/L)+\Phi_h(z_0/L)\right\},
\end{align}
where $\Delta\theta = \theta-\theta_0$,
and
\begin{align}
  \Phi_m(z) = \int^z \frac{1-\phi_m(z')}{z'} dz', \\
  \Phi_h(z) = \int^z \frac{R-\phi_h(z')}{R z'} dz'.
\end{align}


\subsection{Louis (1979) Model}
Louis (1979) introduced a parametric model of vertical eddy fluxes.

The $L$ becomes
\begin{equation}
  L = \frac{\theta u^2}{g\Delta\theta}
    \frac{\ln(z/z_0)-\Phi_h(z/L)+\Phi_h(z_0/L)}{\left\{\ln(z/z_0)-\Phi_m(z/L)+\Phi_m(z/L)\right\}^2}.
\end{equation}
The bulk Richardson number for the layer $Ri_B$ is
\begin{equation}
  Ri_B = \frac{gz\Delta\theta}{\theta u^2},
\end{equation}
and its form implies relationship with the Monin-Obukhov scale height $L$.
Then the fluxes could be written as
\begin{align}
  u_*^2 &= a^2 u^2 F_m\left(\frac{z}{z_0},Ri_B\right), \label{eq: u_*^2} \\
  u_*\theta_* &= \frac{a^2}{R} u \Delta \theta F_h\left(\frac{z}{z_0},Ri_B\right), \label{eq: u_*t_*}
\end{align}
where
$R$ is ratio of the drag coefficients for momentum and heat in the neutral limit, and
\begin{equation}
  a^2 = \frac{k^2}{\left\{\ln\left(z/z_0\right)\right\}^2}
\end{equation}
is the drag coefficient in neutral conditions.

For the unstable condition ($Ri_B<0$),
$F_i$s ($i=m,h$) could be
\begin{equation}
  F_i = 1 - \frac{b Ri_B}{1 + c_i \sqrt{|Ri_B|}},
  \label{eq: F_i unstable}
\end{equation}
under the consideration that
$F_i$ must behave as $1/u$ (i.e. $\sqrt{|Ri_B|}$) in the free convection limit ($u \to 0$),
and becomes $1$ in neutral conditions ($Ri_B \to 0$).
In the stable conditions ($Ri_b$), on the other hand,
Louis (1979) adopted the following form for $F_i$:
\begin{equation}
  F_i = \frac{1}{(1 + b' Ri_B)^2}.
  \label{eq: F_i stable}
\end{equation}

The constants are estimated as
$R=0.74$ by Businger et al. (1971),
and $b=2b'=9.4$ by Louis (1979).
By the dimensional analysis,
\begin{equation}
  c_i = C^*_i a^2 b \sqrt{\frac{z}{z_0}},
\end{equation}
and $C^*_m = 7.4, C^*_h = 5.3$, which result best fit curves.


\subsection{Uno et al. (1995) Model}
Uno et al. (1995) extended the Louis Model,
which considers difference of the roughness length
between for momentum, $z_0$, and temperature, $z_t$.

The potential temperature difference between $z=z$ and $z=z_t$,
$\Delta\theta_t$, is
\begin{align}
  \Delta\theta_t
  &= R\frac{\theta_*}{k}\left\{\ln(z_0/z_t) - \Phi_h(z_0/L) + \Phi_h(z_t/L)\right\} + \Delta\theta_0, \nonumber \\
  &= R\frac{\theta_*}{k}{\ln(z_0/z_t)} + \Delta\theta_0, \nonumber \\
  &= \Delta\theta_0 \left\{\frac{R\ln(z_0/z_t)}{\Psi_h} + 1\right\},
\end{align}
where $\Delta\theta_0 = \theta_z - \theta_{z_0} (=\Delta\theta)$,
\begin{equation}
  \Psi_h = \int_{z_0}^z\frac{\phi_h}{z'}dz', \label{eq: Psi_h}
\end{equation}
and $\phi_h$ is assumed to be $R$ in the range $z_t < z < z_0$.
Thus
\begin{equation}
  \Delta\theta_0 = \Delta\theta_t \left\{\frac{R\ln(z_0/z_t)}{\Psi_h}+1\right\}^{-1},
  \label{eq: Delta t_0}
\end{equation}
or equivalently,
\begin{equation}
  Ri_{B0} = Ri_{Bt} \left\{\frac{R\ln(z_0/z_t)}{\Psi_h}+1\right\}^{-1}.
  \label{eq: Ri_B0}
\end{equation}

From the eqs. (\ref{eq: u_*^2}) and (\ref{eq: u_*t_*}),
\begin{equation}
  \Delta\theta_0 = \frac{R\theta_*}{k}\ln\left(\frac{z}{z_0}\right)\frac{\sqrt{F_m}}{F_h},
\end{equation}
while
\begin{equation}
  \Delta\theta_0 = \frac{\theta_*}{k}\Psi_h,
\end{equation}
from eqs. (\ref{eq: flux-gradient t}) and (\ref{eq: Psi_h}).
Therefore
\begin{equation}
  \Psi_h = R\ln\left(\frac{z}{z_0}\right)\frac{\sqrt{F_m}}{F_h}.
  \label{eq: Psi}
\end{equation}

Because $\Psi_h$ depends on $Ri_{B0}$,
$Ri_{B0}$ cannot be calculated from $Ri_{Bt}$ with eq. (\ref{eq: Ri_B0})
directly,
so numerical iteration is required to obtain $Ri_{B0}$
\footnote{In the stable case, it can be solved analytically
with eq. (\ref{eq: F_i stable}),
but the solution is too complicated.}.
Starting from $Ri_{Bt}$ as the first estimation of $Ri_{B0}$,
the second estimate by the Newton-Raphson iteration becomes
\begin{equation}
  \hat{Ri}_{B0} = Ri_{Bt} - \frac{Ri_{Bt}R\ln(z_0/z_t)}{\ln(z_0/z_t) + \hat{\Psi}_h},
  \label{eq: Ri_B0 estimation}
\end{equation}
where $\hat{\Psi}_h$ is the estimate of $\Psi_h$ using $Ri_{Bt}$ instead of $Ri_{B0}$.
Approximate values for $F_m, F_h$, and $\Psi_h$ are re-calculated
based on the $\hat{Ri}_{B0}$,
and then
$\Delta\theta_0$, and the surface fluxes $u_*^2$ and $u_*\theta_*$
are calculated from eqs. (\ref{eq: Delta t_0}), (\ref{eq: u_*^2}),
and (\ref{eq: u_*t_*}), respectively.


R\subsection{Roughness length}
Miller et al. (1992) provides the roughness length over the tropical ocean,
based on the numerical calculations
by combining the smooth surface values with the Charnock relation
for the aerodynamic roughness length
and the constant values for heat and moisture in accordance with Smith (1988,1989) suggestions:
\begin{align}
  z_0 &= 0.11u/\nu_* + 0.018u_*^2/g, \label{eq: z_0} \\
  z_t &= 0.40u/\nu_* + 1.4 \times 10^{-5}, \label{eq: z_t} \\
  z_q &= 0.62u/\nu_* + 1.3 \times 10^{-4}, \label{eq: z_q}
\end{align}
where $\nu_*$ is the kinematic viscosity of air ($\sim 1.5 \times 10^{-5}$),
and $z_0, z_t$, and $z_q$ are the roughness length
for the momentum, heat, and vapor, respectively.



\subsection{Discretization}

\def\half{\frac{1}{2}}

All the fluxes are calculated based on the velocity at the first full-level (k=1)
($z=\Delta z/2$).
The absolute velocities $U$ are
\begin{align}
  U_{i+\half,j,1}^2 &=
    \left\{\frac{2(\rho u)_{i+\half,j,1}}{\rho_{i,j,1}+\rho_{i+1,j,1}}\right\}^2
  + \left\{\frac{(\rho v)_{i,j-\half,1} + (\rho v)_{i,j+\half,1} + (\rho v)_{i+1,j-\half,1} + (\rho v)_{i+1,j+\half,1}}{2(\rho_{i,j,1}+\rho_{i+1,j,1})}\right\}^2 \nonumber \\
 &+ \left\{\frac{(\rho w)_{i,j,1+\half} + (\rho w)_{i+1,j,1+\half}}{2(\rho_{i,j,1}+\rho_{i+1,j,1})}\right\}^2, \\
  U_{i,j+\half,1}^2 &=
    \left\{\frac{(\rho u)_{i-\half,j,1} + (\rho u)_{i+\half,j,1} + (\rho u)_{i-\half,j-1,1} + (\rho u)_{i+\half,j+1,1}}{2(\rho_{i,j,1}+\rho_{i,j+1,1})}\right\}^2 \nonumber\\
 &+ \left\{\frac{2(\rho v)_{i,j+\half,1}}{\rho_{i,j,1}+\rho_{i,j+1,1}}\right\}^2
  + \left\{\frac{(\rho w)_{i,j,1+\half} + (\rho w)_{i,j+1,1+\half}}{2(\rho_{i,j,1}+\rho_{i,j+1,1})}\right\}^2, \\
  U_{i,j,1}^2 &=
    \left\{\frac{(\rho u)_{i-\half,j,1} + (\rho u)_{i+\half,j,1}}{\rho_{i,j,1}}\right\}^2
  + \left\{\frac{(\rho v)_{i,j-\half,1} + (\rho v)_{i,j+\half,1}}{\rho_{i,j,1}}\right\}^2
  + \left\{\frac{(\rho w)_{i,j,1+\half}}{2\rho_{i,j,1}}\right\}^2,
\end{align}
here it is note that $(\rho w)_{i,j,\half}=0$.
The potential temperatures $\theta$ are
\begin{align}
  \theta_{i,j,1} &= \frac{(\rho \theta)_{i,j,1}}{\rho_{i,j,1}}, \\
  \bar{\theta}_{i+\half,j,1} &= \frac{\theta_{i,j,1}+\theta_{i+1,j,1}}{2}, \\
  \bar{\theta}_{i,j+\half,1} &= \frac{\theta_{i,j,1}+\theta_{i,j+1,1}}{2}.
\end{align}

The roughness lengthes, $z_0, z_t$, and $z_q$ are caluclated from
the eqs. (\ref{eq: z_0}), (\ref{eq: z_t}), and (\ref{eq: z_q}),
in which the friction verociy $u_*$ is estimated as
\begin{equation}
  u_* = \frac{k}{\ln(z_1/z_0)}U.
\end{equation}

From eq. (\ref{eq: Ri_B0})
The $Ri_{Bt})$, which is the first guess of the $Ri_{B0}$, is
\begin{equation}
  Ri_{Bt} = \frac{gz_1(\theta_1-\theta_sfc)}{\Theta_{sfc}U^2},
\end{equation}
with the assumption that $\theta_{z_t} = \theta_{sfc}$.
The estimation of the $\hat{\Psi}_h$ is calculated with $Ri_{Bt}$ from
the eqs. (\ref{eq: Psi}), (\ref{eq: F_i unstable}), and (\ref{eq: F_i stable}).
The final estimation of $Ri_{B0}$ is obtained from the eq. (\ref{eq: Ri_B0 estimation}),
and the final estimation of $\Psi_h$ is obtaind with the $Ri_{B0}$.

Now we can calculate the bulk coefficients, $C_m, C_h$, and $C_e$ for the moments, heat, and vapor:
\begin{align}
  C_m &= \frac{k^2}{\ln(z_1/z_0)}F_m(Ri_{B0}), \\
  C_h &= \frac{k^2}{R\ln(z_1/z_0)}F_h(Ri_{B0})\left\{\frac{R\ln(z_0/z_t)}{\Psi_h}-1\right\}^{-1}, \\
  C_e &= \frac{k^2}{R\ln(z_1/z_0)}F_h(Ri_{B0})\left\{\frac{R\ln(z_0/z_e)}{\Psi_h}-1\right\}^{-1}.
\end{align}
The fluxes are
\begin{align}
  \overline{\rho u'w'} &= - C_m U \rho u, \\
  \overline{\rho v'w'} &= - C_m U \rho v, \\
  \overline{\rho w'w'} &= - C_m U \rho w, \\
  \overline{\rho \theta'w'} &= - C_h U \{\rho \theta - \rho \theta_{sfc}\}, \\
  \overline{\rho q'w'} &= -C_e U \rho ( q - q_{evap} ),
\end{align}
where $q_{evap}$ is the saturation value at surface.

