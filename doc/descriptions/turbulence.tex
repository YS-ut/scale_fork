%\documentclass{article}
%\usepackage{amsmath}
%\begin{document}

\section{Turbulence}
{\bf \Large 
\begin{tabular}{ccc}
\hline
  Correnspoinding author & : & Seiya Nishizawa\\
\hline
\end{tabular}
}
\subsection{Spatial filter}

The governing euqations are the followings:
\begin{align}
  \frac{\partial\rho}{\partial t} + \frac{\partial u_i \rho}{\partial x_i} 
  &= 0 \\
  \frac{\partial\rho u_i}{\partial t}
  + \frac{\partial u_j \rho u_i}{\partial x_j}
  &= -\frac{\partial p}{\partial x_i} + g \rho \delta_{i3} \\
  \frac{\partial\rho \theta}{\partial t}
  + \frac{\partial u_i \rho \theta}{\partial x_i}
  &= Q
\end{align}

Spatial filtering the continuity equation yields
\begin{equation}
  \frac{\partial \overline{\rho}}{\partial t} + \frac{\partial \overline{u_i \rho}}{\partial x_i} = 0, \label{eq: spatial filtered rho}
\end{equation}
where $\overline{\phi}$ means the spatial filtered quantity of an arbitrary variable $\phi$.
The Favre filtering (Favre 1983) defined by
\begin{equation}
  \widetilde{\phi} = \frac{\overline{\rho \phi}}{\overline{\rho}}
\end{equation}
makes the equation (\ref{eq: spatial filtered rho})
\begin{equation}
  \frac{\partial \overline{\rho}}{\partial t} + \frac{\partial \widetilde{u_i}\overline{\rho}}{\partial x_i} = 0.
\end{equation}


The momentam equations become
\begin{align}
  \frac{\partial \overline{\rho u_i}}{\partial t} + \frac{\partial \overline{u_j\rho u_i}}{\partial x_j} &= -\frac{\partial \overline{p}}{\partial x_i} + \overline{\rho} g\delta_{i3} \\
  \frac{\partial \overline{\rho}\widetilde{u_i}}{\partial t} + \frac{\partial \widetilde{u_j}\:\overline{\rho}\widetilde{u_i}}{\partial x_j} &= -\frac{\partial \overline{p}}{\partial x_i} + g\overline{\rho} \delta_{i3}
    -\frac{\partial}{\partial x_j}\left(\overline{u_i \rho u_j} - \widetilde{u_j}\overline{\rho}\widetilde{u_i}\right) \\
  \frac{\partial \overline{\rho}\widetilde{u_i}}{\partial t} + \frac{\partial \widetilde{u_j}\:\overline{\rho}\widetilde{u_i}}{\partial x_j} &= -\frac{\partial \overline{p}}{\partial x_i} + g\overline{\rho} \delta_{i3}
    -\frac{\partial}{\partial x_j}\overline{\rho}\left(\widetilde{u_i u_j} - \widetilde{u_j}\widetilde{u_i}\right).
\end{align}


As the same matter, the thermal equation becomes
\begin{equation}
  \frac{\partial \overline{\rho}\widetilde{\theta}}{\partial t}
  + \frac{\partial \widetilde{u_i}\overline{\rho}\widetilde{\theta}}{\partial x_i}
  = Q -\frac{\partial}{\partial x_i}\overline{\rho}\left(\widetilde{u_i\theta}-\widetilde{u_i}\widetilde{\theta}\right).
\end{equation}

Then, the govering equations for the prognositic variables
($\overline{\rho}, \overline{\rho}\widetilde{u_i}, $ and $\overline{\rho}\widetilde{\theta}$) are
\begin{align}
  \frac{\partial \overline{\rho}}{\partial t}
  + \frac{\partial \widetilde{u_i}\overline{\rho}}{\partial x_i} &= 0, \\
  \frac{\partial \overline{\rho}\widetilde{u_i}}{\partial t}
  + \frac{\partial \widetilde{u_j}\overline{\rho}\widetilde{u_i}}{\partial x_j}
  &= -\frac{\partial \overline{p}}{\partial x_i} + g\overline{\rho}\delta_{i3}
  -\frac{\partial \overline{\rho}\tau_{ij}}{\partial x_j}, \\
  \frac{\partial \overline{\rho}\widetilde{\theta}}{\partial t}
  + \frac{\partial \widetilde{u_i}\overline{\rho}\widetilde{\theta}}{\partial x_i}
  &= Q -\frac{\partial \overline{\rho}\tau^*_{i}}{\partial x_i},
\end{align}
where
\begin{align}
  \tau_{ij} &= \widetilde{u_iu_j}-\widetilde{u_i}\widetilde{u_j}, \\
  \tau^*_{i} &= \widetilde{u_i\theta}-\widetilde{u_i}\widetilde{\theta}.
\end{align}

\subsection{SGS model}
\subsubsection{Smagorinsky-Lilly model}
\begin{equation}
  \tau_{ij} - \frac{1}{3}\tau_{kk}\delta_{ij} = -2\nu_{SGS}\left(S_{ij}-\frac{1}{3}S_{kk}\delta_{ij}\right),
\end{equation}
where $S_{ij}$ is the strain tensor,
\begin{equation}
  S_{ij} = \frac{1}{2}\left(\frac{\partial \widetilde{u_i}}{\partial x_j} + \frac{\partial \widetilde{u_j}}{\partial x_i}\right),
\end{equation}
and
\begin{equation}
  \nu_{SGS} = \left(C_s\Delta\right)^2 \left|S\right|.
\end{equation}
$C_s$ is the Smagorinsky constant,
$\Delta$ is the grid spacing,
and $\left|S\right|$ is scale of the tensor $S$,
\begin{equation}
  \left|S\right| = \sqrt{2S_{ij}S_{ij}}.
\end{equation}
\begin{equation}
  \tau_{ij} = -2\nu_{SGS}\left(S_{ij}-\frac{1}{3}S_{kk}\delta_{ij}\right)
             + \frac{2}{3} TKE\delta_{ij},
\end{equation}
where
\begin{equation}
  TKE = \frac{1}{2}\left(\widetilde{u_i^2} - \widetilde{u_i}^2\right)
   = \frac{1}{2}\tau_{ii}
   = \left(C_s\Delta\right)^2\left|S\right|^2.
\end{equation}


\begin{equation}
  \tau^*_i = -\nu^*_{SGS} \frac{\partial \theta}{\partial x_i},
\end{equation}
where
\begin{equation}
  \nu^*_{SGS} = \frac{1}{Pr}\nu_{SGS}.
\end{equation}
$Pr$ is the turbulent Prandtl number.
For the other scalar constants such as water vaper,
$\nu^*_{SGS}$ is also used as their diffusivity.

To include buoyancy effects,
the extension of the basic Smagorinsky developed by Brown et al. (1994)
is used.
\begin{equation}
  \nu_{SGS} = \lambda_r^2 |S| \sqrt{1-Rf},
\end{equation}
where $Rf$ is the flux Richardson number ($Rf = Ri/Pr$),
and $\lambda_r$ is a characteristic subgrid length scale.
$Ri$ is the Richardson number ($Ri = N^2/|S|^2$),
and $N^2$ is the Brunt-Visala frequency ($N^2 = g/\theta\:\partial\theta/\partial z$).
The Prandtl number is an unknow parameter,
and it depends on the Richardson number,
while it is offten assumed a constant value.
For the unstable conditions ($Ri < 0$),
\begin{align}
  \nu_{SGS} &= \left(C_s\Delta\right)^2 |S| \sqrt{1 - c Ri}, \\
  \nu^*_{SGS} &= \frac{1}{Pr_N} \left(C_s\Delta\right)^2 |S| \sqrt{1 - b Ri},
\end{align}
where $Pr_N$ is the Prandtl number in neutral condtions.
The values of $c, b, Pr_N$ are set 16, 40, and 0.7, respectively.
Then the Prandtl number is
\begin{equation}
  Pr = Pr_N \sqrt{\frac{1-c Ri}{1-b Ri}}.
\end{equation}
For the stable condtions,
when the Richardson number is smaller than the critical Richardson number, $Ri_c (=0.25)$,
\begin{align}
  \nu_{SGS} &= \left(C_s\Delta\right)^2 \left(1-\frac{Ri}{Ri_c}\right)^4, \\
  \nu^*_{SGS} &= \frac{1}{Pr_N}\left(C_s\Delta\right)^2 \left(1-\frac{Ri}{Ri_c}\right)^4\left(1-g Ri\right).
\end{align}
The constant $g$ is determined as the Prandtl number becomes 1
in the limit of $Ri \to Ri_C$ and then is $(1-Pr_N)/Ri_c$.
The Prandtl number is
\begin{equation}
  Pr = Pr_N \left(1-\frac{1-Pr_N}{Ri_c}Ri\right)^{-1}.
\end{equation}
For the strongly stable condistions ($Ri > Ri_c$),
the eddy viscosity and the diffusivity for scalars are 0;
\begin{align}
  \nu_{SGS} &= 0, \\
  \nu^*_{SGS} &= 0.
\end{align}

%\end{document}
