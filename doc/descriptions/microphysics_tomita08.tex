\subsection{Six-class Single-moment Bulk Scheme (Tomita 2008)}
Six-class single-moment bulk scheme in the SCALE was developed by \citet{tomita_2008}.
This scheme predicts mass exchange among six categories of water substances (water vapor, cloud water, rain, cloud ice, snow, and graupel), which is largely based on the method of \citet{lin_etal_1983}.
However, there are some modifications from the original method of \citet{lin_etal_1983}: Both cloud water and cloud ice are generated only by a saturation adjustment process, and wet growth process of graupel is omitted.
According to \citet{tomita_2008}, these modifications result in 20\% reduction in computational cost compared to \citet{lin_etal_1983} without significant changes on physical performance.
In this subsection, the formulations of the microphysical scheme of \citet{tomita_2008} are described.
Mass concentrations of water vapor, cloud water, rain, cloud ice, snow, and graupel are indicated by $q_{v}$, $Q_{W}$, $Q_{R}$, $Q_{I}$, $Q_{S}$, and $Q_{G}$ respectively.

In \citet{tomita_2008}, cloud microphysics processes except for the saturation adjustment process consist of auto-conversion, accretion, evaporation, sublimation, deposition, melting, freezing, and Bergeron process. When $T < T_{0}$ (= 273.15 K), the tendency of each water substance by these processes can be written as follows:
\begin{align}
  \frac{\partial Q_{W}}{\partial t} &= -P_{RAUT}-P_{RACW}-P_{SACW}-P_{GACW}-P_{SFW}, \\
  \frac{\partial Q_{I}}{\partial t} &= -P_{SAUT}-P_{RACI}-P_{SACI}-P_{GACI}-P_{SFI}, \\
  \frac{\partial Q_{R}}{\partial t} &= P_{RAUT}+P_{RACW}-P_{IACR}-P_{SACR}-P_{GACR}-P_{GFRZ}-P_{REVP}, \\
  \frac{\partial Q_{S}}{\partial t} &= P_{SAUT}-P_{GAUT} \nonumber \\
  &+P_{SACW}+P_{SACI}+(1-\delta_{1})P_{RACI}+(1-\delta_{1})P_{IACR}+\delta_{2}P_{SACR}-(1-\delta_{2})P_{RACS}-P_{GACS} \nonumber \\
  &-(1-\delta_{3})P_{SSUB}+\delta_{3}P_{SDEP}+P_{SFW}+P_{SFI}, \\
  \frac{\partial Q_{G}}{\partial t} &= P_{GAUT} \nonumber \\
  &+P_{GACW}+P_{GACI}+P_{GACR}+P_{GACS}+\delta_{1}P_{RACI}+\delta_{1}P_{IACR}+(1-\delta_{2})P_{SACR}+(1-\delta_{2})P_{RACS} \nonumber \\
  &-(1-\delta_{3})P_{GSUB}+\delta_{3}P_{GDEP}+P_{GFRZ}, \\
  \frac{\partial q_{v}}{\partial t} &= P_{REVP}+(1-\delta_{3})P_{SSUB}+(1-\delta_{3})P_{GSUB}-\delta_{3}P_{SDEP}-\delta_{3}P_{GDEP},
\end{align}
where $P_{*}$ in the right hand sides are conversion terms listed in Table \ref{table-tomita08-1}, and $\delta_{1}$, $\delta_{2}$, and $\delta_{3}$ are defined as
\begin{align}
\delta_{1} &=
\begin{cases}
  1, &{\rm for\ } Q_{R} \geq 10^{-4} {\rm \ kg/kg}\\
  0, &{\rm otherwise}\\
\end{cases},
\\
\delta_{2} &=
\begin{cases}
  1, &{\rm for\ } Q_{R} \leq 10^{-4} {\rm \ kg/kg\ and\ } Q_{S} \leq 10^{-4} {\rm \ kg/kg}\\
  0, &{\rm otherwise}\\
\end{cases},
\\
\delta_{3} &=
\begin{cases}
  1, &{\rm for\ } S_{ice} \geq 1\\
  0, &{\rm otherwise}\\
\end{cases},
\end{align}
where $S_{ice}$ is saturation ratio over ice. When $T \geq T_{0}$, the tendencies of each water substance can be written as follows:
\begin{align}
  \frac{\partial Q_{W}}{\partial t} &= -P_{RAUT}-P_{RACW}-P_{SACW}-P_{GACW}, \\
  \frac{\partial Q_{I}}{\partial t} &= 0, \\
  \frac{\partial Q_{R}}{\partial t} &= P_{RAUT}+P_{RACW}+P_{SACW}+P_{GACW}+P_{SMLT}+P_{GMLT}-P_{REVP}, \\
  \frac{\partial Q_{S}}{\partial t} &= -P_{GACS}-P_{SMLT}, \\
  \frac{\partial Q_{G}}{\partial t} &= P_{GACS}-P_{GMLT}, \\
  \frac{\partial q_{v}}{\partial t} &= P_{REVP}.
\end{align}
Formulation of each term is described later.
\begin{table}[tbh]
\begin{center}
\caption{List of conversion terms used in six-class single-moment bulk scheme of \citet{tomita_2008}}
\label{table-tomita08-1}
\scalebox{0.7}{
\begin{tabular}{llll}
\hline
Notation&Description&Direction&Conditions \\ \hline \hline
$P_{RAUT}$&Auto-conversion rate of cloud water to form rain&$Q_{W} \longrightarrow Q_{R}$& \\ \hline
$P_{SAUT}$&Auto-conversion rate of cloud ice to form snow&$Q_{I} \longrightarrow Q_{S}$&$T < T_{0}$ \\ \hline
$P_{GAUT}$&Auto-conversion rate of snow to form graupel&$Q_{S} \longrightarrow Q_{G}$&$T < T_{0}$ \\ \hline
$P_{RACW}$&Accretion rate of cloud water by rain&$Q_{W} \longrightarrow Q_{R}$& \\ \hline
$P_{SACW}$&Accretion rate of cloud water by snow&$Q_{W} \longrightarrow Q_{S}$&$T < T_{0}$ \\
&&$Q_{W} \longrightarrow Q_{R}$&$T \geq T_{0}$ \\ \hline
$P_{GACW}$&Accretion rate of cloud water by graupel&$Q_{W} \longrightarrow Q_{G}$&$T < T_{0}$ \\
&&$Q_{W} \longrightarrow Q_{R}$&$T \geq T_{0}$ \\ \hline
$P_{RACI}$&Accretion rate of cloud ice by rain&$Q_{I} \longrightarrow Q_{S}$&$T < T_{0}$ and $Q_{R} < 10^{-4}$ kg/kg \\
&&$Q_{I} \longrightarrow Q_{G}$&$T < T_{0}$ and $Q_{R} \geq 10^{-4}$ kg/kg \\ \hline
$P_{SACI}$&Accretion rate of cloud ice by snow&$Q_{I} \longrightarrow Q_{S}$&$T < T_{0}$ \\ \hline
$P_{GACI}$&Accretion rate of cloud ice by graupel&$Q_{I} \longrightarrow Q_{G}$&$T < T_{0}$ \\ \hline
$P_{IACR}$&Accretion rate of rain by cloud ice&$Q_{R} \longrightarrow Q_{S}$&$T < T_{0}$ and $Q_{R} < 10^{-4}$ kg/kg \\
&&$Q_{R} \longrightarrow Q_{G}$&$T < T_{0}$ and $Q_{R} \geq 10^{-4}$ kg/kg \\ \hline
$P_{SACR}$&Accretion rate of rain by snow&$Q_{R} \longrightarrow Q_{S}$&$T < T_{0}$, $Q_{R} \leq 10^{-4}$ kg/kg, and $Q_{S} \leq 10^{-4}$ kg/kg \\
&&$Q_{R} \longrightarrow Q_{G}$&$T < T_{0}$ and ($Q_{R}$ or $Q_{S}$) $> 10^{-4}$ kg/kg \\ \hline
$P_{GACR}$&Accretion rate of rain by graupel&$Q_{R} \longrightarrow Q_{G}$&$T < T_{0}$ \\ \hline
$P_{RACS}$&Accretion rate of snow by rain&$Q_{S} \longrightarrow Q_{G}$&$T < T_{0}$ and ($Q_{R}$ or $Q_{S}$) $> 10^{-4}$ kg/kg \\ \hline
$P_{GACS}$&Accretion rate of snow by graupel&$Q_{S} \longrightarrow Q_{G}$& \\ \hline
$P_{REVP}$&Evaporation rate of rain&$Q_{R} \longrightarrow q_{v}$& \\ \hline
$P_{SSUB}$&Sublimation rate of snow&$Q_{S} \longrightarrow q_{v}$&$S_{ice} < 1$ \\ \hline
$P_{GSUB}$&Sublimation rate of graupel&$Q_{G} \longrightarrow q_{v}$&$S_{ice} < 1$ \\ \hline
$P_{SDEP}$&Deposition rate of water vapor for snow&$q_{v} \longrightarrow Q_{S}$&$S_{ice} \geq 1$ \\ \hline
$P_{GDEP}$&Deposition rate of water vapor for graupel&$q_{v} \longrightarrow Q_{G}$&$S_{ice} \geq 1$ \\ \hline
$P_{SMLT}$&Melting rate of snow&$Q_{S} \longrightarrow Q_{R}$&$T \geq T_{0}$ \\ \hline
$P_{GMLT}$&Melting rate of graupel&$Q_{G} \longrightarrow Q_{R}$&$T \geq T_{0}$ \\ \hline
$P_{GFRZ}$&Freezing rate of rain to form graupel&$Q_{R} \longrightarrow Q_{G}$&$T < T_{0}$ \\ \hline
$P_{SFW}$&Growth rate of snow by Bergeron process from cloud water&$Q_{W} \longrightarrow Q_{S}$& 243.15 K $\leq T < T_{0}$ \\ \hline
$P_{SFI}$&Growth rate of snow by Bergeron process from cloud ice&$Q_{I} \longrightarrow Q_{S}$& 243.15 K $\leq T < T_{0}$ \\ \hline
\end{tabular}
}
\end{center}
\end{table}

\subsubsection{The saturation adjustment}
Mass exchange among water vapor, cloud water, and cloud ice is controlled by the saturation adjustment.
In the SCALE, the saturation adjustment is calculated after the aforementioned conversion processes of the microphysics.
To calculate the adjustment process, saturated mass concentration of water vapor is defined as follows:
\begin{equation}
  q^{*}_{v}(T) = q^{*}_{vl}(T)+[1-\alpha (T)]q^{*}_{vi}(T)\label{eq:satuqv},
\end{equation}
where $q^{*}_{vl}(T)$ is saturated mass concentration of water vapor against liquid phase, $q^{*}_{vi}(T)$ is that for ice phase, and $\alpha (T)$ is a continuous function which satisfies
\begin{align}
\begin{cases}
  \alpha (T) = 1, & {\rm \ for\ } T \geq {\rm \ 273.15 \ K} ,\\
  \alpha (T) = \frac{T-233.15}{40.0}, & {\rm \ for\ 233.15\ K \ } < T < {\rm \ 273.15 \ K},\\
  \alpha (T) = 0, & {\rm \ for\ } T \leq {\rm \ 233.15 \ K}.
\end{cases}
\end{align}
If it is supersaturated, water vapor is converted into cloud water and cloud ice.
If it is unsaturated, cloud water and cloud ice are converted into water vapor.
As a conserved quantity for the adjustment process, the moist internal energy is also defined as follows:
\begin{align}
  U_{0}&=[q_{d}c_{vd}+q_{v}c_{vv}+(Q_{W}+Q_{R})c_{l}+(Q_{I}+Q_{S}+Q_{G})c_{s}]T \nonumber \\ 
  &+q_{v}L_{v}-(Q_{I}+Q_{S}+Q_{G})L_{f},
\end{align}
where $L_{v}$ is the latent heat between water vapor and liquid water, $L_{f}$ is that between liquid water and solid water. In addition, the sum of the mass concentration of water vapor, cloud water, and cloud ice
\begin{equation}
  q_{sum}=q_{v}+Q_{W}+Q_{I},
\end{equation}
does not change through the saturation adjustment.

First, it is assumed that all of the cloud water $Q_{W}$ and cloud ice $Q_{I}$ evaporate.
In this case, the mass concentration of water vapor becomes equal to $q_{sum}$ and the temperature decreases due to the evaporation.
The moist internal energy can be written as
\begin{align}
  U_{1}&=[q_{d}c_{vd}+q_{sum}c_{vv}+Q_{R}c_{l}+(Q_{S}+Q_{G})c_{s}]T_{1} \nonumber \\ 
  &+q_{sum}L_{v}-(Q_{S}+Q_{G})L_{f}.
\end{align}
Since the moist internal energy does not change through the saturation adjustment, we can obtain the new temperature value $T_{1}$ easily by solving $U_{0}=U_{1}$. Then, if $q_{sum}$ is less than saturated mass concentration of water vapor at this temperature (i.e. $q_{sum}<q^{*}_{v} (T_{1})$), no saturation occurs and the new values of water vapor $q^{\prime}_{v}$, cloud water $Q^{\prime}_{W}$, cloud ice $Q^{\prime}_{I}$, and temperature $T^{\prime}$ are determined as
\begin{align}
\begin{cases}
  q^{\prime}_{v} &= q_{sum}, \\
  Q^{\prime}_{W} &= 0, \\
  Q^{\prime}_{I} &= 0, \\
  T^{\prime} &= T_{1}.
\end{cases}
\end{align}

If $q_{sum}$ exceeds $q^{*}_{v} (T_{1})$, saturation occurs. In this case, new temperature value $T_{2}$ should be determined by satisfying the equations of
\begin{align}
  U_{0}&=[q_{d}c_{vd}+q^{*}_{v}(T_{2})c_{vv}+(Q_{W2}+Q_{R})c_{l}+(Q_{I2}+Q_{S}+Q_{G})c_{s}]T_{2} \nonumber \\ 
  &+q^{*}_{v}(T_{2})L_{v}-(Q_{I2}+Q_{S}+Q_{G})L_{f}\label{eq:newu}, \\
  Q_{W2}&=[q_{sum}-q^{*}_{v}(T_{2})]\alpha (T_{2})\label{eq:newqw}, \\
  Q_{I2}&=[q_{sum}-q^{*}_{v}(T_{2})](1-\alpha (T_{2}))\label{eq:newqi}.
\end{align}
Eqs. (\ref{eq:satuqv}) and (\ref{eq:newu}-\ref{eq:newqi}) are solved numerically and the new values are determined as
\begin{align}
\begin{cases}
  q^{\prime}_{v} &= q^{*}_{v}(T_{2}), \\
  Q^{\prime}_{W} &= Q_{W2}, \\
  Q^{\prime}_{I} &= Q_{I2}, \\
  T^{\prime} &= T_{2}.
\end{cases}
\end{align}

\subsubsection{Fundamental characteristics of precipitation particles}
In the reminder of this subsection, the formulations of all conversion terms listed in Table \ref{table-tomita08-1} are shown.
Before that, however, it is necessary to clarify assumptions about the characteristics of precipitation particles (rain, snow, and graupel).
In \citet{tomita_2008}, it is assumed that seizes of precipitation particles obey the Marshall-Palmer exponential size distribution:
\begin{equation}
  n_{[R,S,G]}(D) = N_{0[R,S,G]}\exp(-\lambda_{[R,S,G]}D)\label{eq:size_dist},
\end{equation}
where $D$ is the diameter of particle, $N_{0}$ is an intercept parameter, and $\lambda$ is a slope parameter. The subscriptions of $R$, $S$, and $G$ denote rain, snow, and graupel, respectively. In the SCALE, each intercept parameter has values of
\begin{align}
\begin{cases}
  N_{0R}&=8.0\times 10^{6}{\rm \ m^{-4}}, \\ 
  N_{0S}&=3.0\times 10^{6}{\rm \ m^{-4}}, \\ 
  N_{0G}&=4.0\times 10^{6}{\rm \ m^{-4}}.
\end{cases}
\end{align}
The mass and terminal velocity of each particle are described as
\begin{align}
  m_{[R,S,G]}(D) &= a_{[R,S,G]}D^{b_{[R,S,G]}}\label{eq:particle_mass} \\
  v_{t[R,S,G]}(D) &= c_{[R,S,G]}D^{d_{[R,S,G]}}\left(\frac{\rho_{0}}{\rho}\right)^{1/2}
\end{align}
where $\rho_{0}$ ($=1.28\rm \ kg/m^{3}$) is a reference density, and $a$, $b$, $c$, and $d$ are coefficients depending on the particle shape. All precipitation particles are treated as spherical, thus
\begin{align}
  &a_{R}=\pi \rho_{w}/6,\quad a_{S}=\pi \rho_{S}/6,\quad a_{G}=\pi \rho_{G}/6, \\
  &b_{R}=b_{S}=b_{G}=3,
\end{align}
where $\rho_{w}=1000 {\rm \ kg/m^{3}}$, $\rho_{S}=100 {\rm \ kg/m^{3}}$, $\rho_{G}=400 {\rm \ kg/m^{3}}$. The coefficients $c$ and $d$ are determined empirically. In the SCALE, their values are
\begin{align}
  c_{R}=130.0,\quad c_{S}=4.84,\quad c_{G}=82.5, \\
  d_{R}=0.5,\quad d_{S}=0.25,\quad d_{G}=0.5,
\end{align}
The slope parameters are determined from Eqs. (\ref{eq:size_dist}) and (\ref{eq:particle_mass}) as
\begin{equation}
  \lambda=\left[\frac{aN_{0}\Gamma (b+1)}{\rho Q}\right]^{1/(b+1)}\label{eq:lambda},
\end{equation}
where $\Gamma$ is the gamma function. The bulk terminal velocities are derived as
\begin{equation}
  V_{T}=c\left(\frac{\rho_{0}}{\rho}\right)^{1/2}\frac{\Gamma(b+d+1)}{\Gamma(b+1)\lambda^{d}}\label{eq:tvelocity}.
\end{equation}
Note that subscription $R$, $S$, and $G$ in Eqs. (\ref{eq:lambda}) and (\ref{eq:tvelocity}) are omitted for simplicity.

\subsubsection{Auto-conversion terms}
The auto-conversion rate of cloud water to form rain ($P_{RAUT}$) is given as
\begin{equation}
  P_{RAUT}=\frac{1}{\rho}\left[16.7\times(\rho Q_{W})^{2}\left(5+\frac{3.6\times10^{-5}N_{d}}{D_{d}\rho Q_{W}}\right)^{-1}\right],
\end{equation}
where $N_{d}$ is the number concentration of cloud water ($N_{d}=50 {\rm \ cm^{-3}}$ in the SCALE) and $D_{d}$ is given as
\begin{equation}
  D_{d}=0.146-5.964\times10^{-2}\ln{\frac{N_{d}}{2000}}.
\end{equation}
The auto-conversion rate of cloud ice to form snow ($P_{SAUT}$) is given as
\begin{equation}
  P_{SAUT}=\beta_{1}(Q_{I}-Q_{I0}),
\end{equation}
where $Q_{I0}$ is set to 0 kg/kg and $\beta_{1}$ is formulated as
\begin{equation}
  \beta_{1}=\beta_{10}\exp[\gamma_{SAUT}(T-T_{0})].
\end{equation}
$\beta_{10}$ and $\gamma_{SAUT}$ are set to 0.001 and 0.025, respectively, in \citet{tomita_2008}. In the SCALE, however, they are set to 0.006 and 0.06 as default settings. The auto-conversion rate of snow to form graupel ($P_{GAUT}$) is given as
\begin{equation}
  P_{GAUT}=\beta_{2}(Q_{S}-Q_{S0}),
\end{equation}
where $Q_{S0}$ is set to $6\times10^{-4}$ kg/kg and $\beta_{2}$ is formulated as
\begin{equation}
  \beta_{2}=\beta_{20}\exp[\gamma_{GAUT}(T-T_{0})].
\end{equation}
$\beta_{20}$ and $\gamma_{GAUT}$ are set to 0.001 and 0.09, respectively, in \citet{tomita_2008}. In the SCALE, however, $\beta_{20} = 0$ as a default setting, which means that auto-conversion of snow to graupel is turned off.

\subsubsection{Accretion terms}
Accretion of cloud particles by precipitation particles ($P_{RACW}$, $P_{SACW}$, $P_{GACW}$, $P_{RACI}$, $P_{SACI}$, and $P_{GACI}$) can be derived as
\begin{align}
  P_{RACW} &=E_{RW}Q_{W}\int_{0}^{\infty}\frac{\pi}{4}D^{2}v_{tR}(D)n_{R}(D)dD \nonumber \\
  &=\frac{\pi E_{RW}N_{0R}c_{R}Q_{W}\Gamma(3+d_{R})}{4\lambda^{3+d_{R}}_{R}}\left(\frac{\rho_{0}}{\rho}\right)^{1/2}, \\
  P_{SACW} &=\frac{\pi E_{SW}N_{0S}c_{S}Q_{W}\Gamma(3+d_{S})}{4\lambda^{3+d_{S}}_{S}}\left(\frac{\rho_{0}}{\rho}\right)^{1/2}, \\
  P_{GACW} &=\frac{\pi E_{GW}N_{0G}c_{G}Q_{W}\Gamma(3+d_{G})}{4\lambda^{3+d_{G}}_{G}}\left(\frac{\rho_{0}}{\rho}\right)^{1/2}, \\
  P_{RACI} &=\frac{\pi E_{RI}N_{0R}c_{R}Q_{I}\Gamma(3+d_{R})}{4\lambda^{3+d_{R}}_{R}}\left(\frac{\rho_{0}}{\rho}\right)^{1/2}, \\
  P_{SACI} &=\frac{\pi E_{SI}N_{0S}c_{S}Q_{I}\Gamma(3+d_{S})}{4\lambda^{3+d_{S}}_{S}}\left(\frac{\rho_{0}}{\rho}\right)^{1/2}, \\
  P_{GACI} &=\frac{\pi E_{GI}N_{0G}c_{G}Q_{I}\Gamma(3+d_{G})}{4\lambda^{3+d_{G}}_{G}}\left(\frac{\rho_{0}}{\rho}\right)^{1/2},
\end{align}
where $E_{RW}$, $E_{SW}$, $E_{GW}$, $E_{RI}$, $E_{SI}$, and $E_{GI}$ are collection efficiency of each accretion process. In the SCALE, $E_{RW} = E_{SW} = E_{GW} = E_{RI} = 1$, $E_{GI} = 0.1$, and $E_{SI}$ is formulated as
\begin{equation}
  E_{SI}=\exp[\gamma_{SACI}(T-T_{0})],
\end{equation}
where $\gamma_{SACI}$ is set to 0.025.
When the accretion of cloud ice by rain occurs, rain freezes to become snow or graupel.
Thus, the conversion term from rain to these particles ($P_{IACR}$) should be considered.
It is derived as
\begin{align}
  P_{IACR}&=\frac{1}{\rho}\int_{0}^{\infty}N_{I}E_{RI}\frac{\pi}{4}D^{2}v_{tR}(D)m_{R}(D)n_{R}(D)dD \nonumber \\
  &=\frac{\pi a_{R}E_{RI}Q_{I}N_{0R}c_{R}\Gamma(6+d_{R})}{4M_{I}\lambda^{6+d_{R}}_{R}}\left(\frac{\rho_{0}}{\rho}\right)^{1/2},
\end{align}
where $N_{I}$ is the number concentration of cloud ice, and $M_{I}$ ($=4.19\times10^{-13}$ kg) is mass of cloud ice particle.

Accretion of precipitation particles ($P_{SACR}$, $P_{GACR}$, and $P_{GACS}$) can be derived as
\begin{align}
  P_{SACR}&= \frac{1}{\rho}\int_{0}^{\infty}E_{SR}m_{R}(D_{R})n_{R}(D_{R})\left[\int_{0}^{\infty}\frac{\pi}{4}(D_{S}+D_{R})^{2}|V_{TS}-V_{TR}|n_{S}(D_{S})dD_{S}\right]dD_{R} \nonumber \\
  &=\frac{\pi a_{R}|V_{TS}-V_{TR}|E_{SR}N_{0S}N_{0R}}{4\rho} \nonumber \\
  &\times\left[\frac{\Gamma(b_{R}+1)\Gamma(3)}{\lambda^{b_{R}+1}_{R}\lambda^{3}_{S}}+2\frac{\Gamma(b_{R}+2)\Gamma(2)}{\lambda^{b_{R}+2}_{R}\lambda^{2}_{S}}+\frac{\Gamma(b_{R}+3)\Gamma(1)}{\lambda^{b_{R}+3}_{R}\lambda_{S}}\right], \\
  P_{GACR}&=\frac{\pi a_{R}|V_{TG}-V_{TR}|E_{GR}N_{0G}N_{0R}}{4\rho} \nonumber \\
  &\times\left[\frac{\Gamma(b_{R}+1)\Gamma(3)}{\lambda^{b_{R}+1}_{R}\lambda^{3}_{G}}+2\frac{\Gamma(b_{R}+2)\Gamma(2)}{\lambda^{b_{R}+2}_{R}\lambda^{2}_{G}}+\frac{\Gamma(b_{R}+3)\Gamma(1)}{\lambda^{b_{R}+3}_{R}\lambda_{G}}\right], \\
  P_{GACS}&=\frac{\pi a_{S}|V_{TG}-V_{TS}|E_{GS}N_{0G}N_{0S}}{4\rho} \nonumber \\
  &\times\left[\frac{\Gamma(b_{S}+1)\Gamma(3)}{\lambda^{b_{S}+1}_{S}\lambda^{3}_{G}}+2\frac{\Gamma(b_{S}+2)\Gamma(2)}{\lambda^{b_{S}+2}_{S}\lambda^{2}_{G}}+\frac{\Gamma(b_{S}+3)\Gamma(1)}{\lambda^{b_{S}+3}_{S}\lambda_{G}}\right],
\end{align}
where $E_{SR}$, $E_{GR}$, and $E_{GS}$ are collection efficiency of each accretion process. Note that terminal velocity for each particle size is approximated by the bulk terminal velocity (e.g. $|v_{tS}(D_{S})-v_{tR}(D_{R})| \simeq |V_{TS}-V_{TR}|$). In the SCALE, $E_{SR} = E_{GR} = 1$, and $E_{GS}$ is formulated as
\begin{equation}
  E_{GS}=\min(1,\exp[\gamma_{GACS}(T-T_{0})]),
\end{equation}
where $\gamma_{GACS}$ is set to 0.09.
When the accretion of rain by snow occurs under the condition of $Q_{R} > 10^{-4}$ kg/kg or $Q_{S} > 10^{-4}$ kg/kg, graupel particle is generated.
Thus, the conversion term from snow to graupel ($P_{RACS}$) should be also considered.
It can be written as
\begin{align}
  P_{RACS}&=\frac{\pi a_{S}|V_{TR}-V_{TS}|E_{SR}N_{0R}N_{0S}}{4\rho} \nonumber \\
  &\times\left[\frac{\Gamma(b_{S}+1)\Gamma(3)}{\lambda^{b_{S}+1}_{S}\lambda^{3}_{R}}+2\frac{\Gamma(b_{S}+2)\Gamma(2)}{\lambda^{b_{S}+2}_{S}\lambda^{2}_{R}}+\frac{\Gamma(b_{S}+3)\Gamma(1)}{\lambda^{b_{S}+3}_{S}\lambda_{R}}\right].
\end{align}

\subsubsection{Evaporation, sublimation, and deposition}
Evaporation rate of rain ($P_{REVP}$) is described as
\begin{align}
  P_{REVP}&=\frac{2\pi N_{0R}(1-\min(S_{liq},1))G_{w}(T)}{\rho} \nonumber \\
  &\times\left[f_{1R}\frac{\Gamma(2)}{\lambda^{2}_{R}}+f_{2R}c^{1/2}_{R}\left(\frac{\rho_{0}}{\rho}\right)^{1/4}\nu^{-1/2}\frac{\Gamma(\frac{5+d_{R}}{2})}{\lambda^{\frac{5+d_{R}}{2}}_{R}}\right],
\end{align}
where $S_{liq}$ is saturation ratio over liquid, coefficients $f_{1R}$ and $f_{2R}$ are 0.78 and 0.27 respectively, $\nu$ is the kinematic viscosity of air, and $G_{w}(T)$ is the thermodynamic function for liquid water given as
\begin{equation}
  G_{w}(T)=\left[\frac{L_{v}}{K_{a}T}\left(\frac{L_{v}}{R_{v}T}-1\right)+\frac{1}{\rho q^{*}_{vl}(T)K_{d}}\right]^{-1},
\end{equation}
where $K_{a}$ is the thermal diffusion coefficient of air and $K_{d}$ is the diffusion coefficient of water vapor in air. $P_{REVP}$ works only when $S_{liq}$ is less than 1 (i.e. unsaturated condition).

Sublimation and deposition rates for snow ($P_{SSUB}$ and $P_{SDEP}$) are described by the same equation:
\begin{align}
  P^{*}_{SSUB,SDEP}&=\frac{2\pi N_{0S}(1-S_{ice})G_{i}(T)}{\rho} \nonumber \\
  &\times\left[f_{1S}\frac{\Gamma(2)}{\lambda^{2}_{S}}+f_{2S}c^{1/2}_{S}\left(\frac{\rho_{0}}{\rho}\right)^{1/4}\nu^{-1/2}\frac{\Gamma(\frac{5+d_{S}}{2})}{\lambda^{\frac{5+d_{S}}{2}}_{S}}\right],
\end{align}
where coefficients $f_{1S}$ and $f_{2S}$ are 0.65 and 0.39 respectively, and $G_{i}(T)$ is the thermodynamic function for ice water given as
\begin{equation}
  G_{i}(T)=\left[\frac{L_{s}}{K_{a}T}\left(\frac{L_{s}}{R_{v}T}-1\right)+\frac{1}{\rho q^{*}_{vi}(T)K_{d}}\right]^{-1},
\end{equation}
where $L_{s}$ is the latent heat between water vapor and solid water. If it is unsaturated ($S_{ice} < 1$), the sublimation rate of snow is given as
\begin{equation}
  P_{SSUB}=P^{*}_{SSUB,SDEP}.
\end{equation}
If it is supersaturated ($S_{ice} \geq 1$), the deposition rate of water vapor for snow is given as
\begin{equation}
  P_{SDEP}=-P^{*}_{SSUB,SDEP}.
\end{equation}
Also sublimation and deposition rates for graupel ($P_{GSUB}$ and $P_{GDEP}$) are described by
\begin{align}
  P^{*}_{GSUB,GDEP}&=\frac{2\pi N_{0G}(1-S_{ice})G_{i}(T)}{\rho} \nonumber \\
  &\times\left[f_{1G}\frac{\Gamma(2)}{\lambda^{2}_{G}}+f_{2G}c^{1/2}_{G}\left(\frac{\rho_{0}}{\rho}\right)^{1/4}\nu^{-1/2}\frac{\Gamma(\frac{5+d_{G}}{2})}{\lambda^{\frac{5+d_{G}}{2}}_{G}}\right],
\end{align}
where coefficients $f_{1G}$ and $f_{2G}$ are 0.78 and 0.27 respectively. If it is unsaturated ($S_{ice} < 1$), the sublimation rate of graupel is given as
\begin{equation}
  P_{GSUB}=P^{*}_{GSUB,GDEP}.
\end{equation}
If it is supersaturated ($S_{ice} \geq 1$), the deposition rate of water vapor for graupel is given as
\begin{equation}
  P_{GDEP}=-P^{*}_{GSUB,GDEP}.
\end{equation}

\subsubsection{Melting and freezing}
When $T \geq T_{0}$, snow particle melts to become rain. The melting rate of snow is described as
\begin{align}
  P_{SMLT}&=\frac{2\pi K_{a}(T-T_{0})N_{0S}}{\rho L_{f}} \nonumber \\
  &\times\left[f_{1S}\frac{\Gamma(2)}{\lambda^{2}_{S}}+f_{2S}c^{1/2}_{S}\left(\frac{\rho_{0}}{\rho}\right)^{1/4}\nu^{-1/2}\frac{\Gamma(\frac{5+d_{S}}{2})}{\lambda^{\frac{5+d_{S}}{2}}_{S}}\right] \nonumber \\
  &+\frac{c_{l}(T-T_{0})}{L_{f}}(P_{SACW}+P_{SACR}).
\end{align}
The last term indicates that the accretions of cloud water and rain promote snow melting. Similarly, the melting rate of graupel is described as
\begin{align}
  P_{GMLT}&=\frac{2\pi K_{a}(T-T_{0})N_{0G}}{\rho L_{f}} \nonumber \\
  &\times\left[f_{1G}\frac{\Gamma(2)}{\lambda^{2}_{G}}+f_{2G}c^{1/2}_{G}\left(\frac{\rho_{0}}{\rho}\right)^{1/4}\nu^{-1/2}\frac{\Gamma(\frac{5+d_{G}}{2})}{\lambda^{\frac{5+d_{G}}{2}}_{G}}\right] \nonumber \\
  &+\frac{c_{l}(T-T_{0})}{L_{f}}(P_{GACW}+P_{GACR}).
\end{align}

When $T < T_{0}$, rain particle freezes to become graupel. The freezing rate of rain is described as
\begin{equation}
  P_{GFRZ}=20\pi^{2}B^{\prime}N_{0R}\frac{\rho_{w}}{\rho}\frac{\exp[A^{\prime}(T_{0}-T)]-1}{\lambda^{7}_{R}},
\end{equation}
where $A^{\prime}=0.66 {\rm \ K^{-1}}$ and $B^{\prime}=100 {\rm \ m^{-3} \ s^{-1}}$.

\subsubsection{Bergeron process}
When cloud water and cloud ice coexist under the condition of $T < T_{0}$, supercooled cloud water evaporates and diffuses to cloud ice because the saturated vapor pressure over liquid water is higher than that for solid water. This process is called Bergeron process. Through this process, cloud ice particle grows to become precipitating snow particle. Thus, mass conversion from cloud water and cloud ice to snow occurs. Conversion rates of cloud water ($P_{SFW}$) and cloud ice ($P_{SFI}$) are formulated as
\begin{align}
  P_{SFW}&=N_{I50}(a_{1}m^{a_{2}}_{I50}+\pi E_{IW}\rho Q_{W}R^2_{I50}U_{I50}), \\
  P_{SFI}&=Q_{I}/\Delta t_{1},
\end{align}
where $m_{I50}$ ($=4.8\times10^{-10} {\rm \ kg}$) and $U_{I50}$ ($=1{\rm \ m/s}$) denotes mass and terminal velocity of ice particle having a radius of 50 ${\rm \mu m}$ ($\equiv R_{I50}$), and $E_{IW}=1$ is collection efficiency of cloud ice for cloud water. The values of $a_{1}$ and $a_{2}$ is determined from a laboratory experiment by \citet{koenig_1971}. $\Delta t_{1}$ is the time during which an ice particle of 40 ${\rm \mu m}$ grows to 50 ${\rm \mu m}$, which is formulated as
\begin{equation}
  \Delta t_{1}=\frac{1}{a_{1}(1-a_{2})}\left[m^{1-a_{2}}_{I50}-m^{1-a_{2}}_{I40}\right],
\end{equation}
where $m_{I40}=2.46\times10^{-10} {\rm \ kg}$. $N_{I50}$ is the number concentration of 50 ${\rm \mu m}$ ice particle which is formulated as
\begin{equation}
  N_{I50}=q_{I50}/m_{I50}=\frac{Q_{I}\Delta t}{m_{I50}\Delta t_{1}}.
\end{equation}
In the SCALE, the Bergeron process occurs only when 243.15 K $\leq T<T_{0}$.

\subsubsection{Optional schemes in the SCALE}
There are some optional schemes which can be applied to the six-class single-moment bulk microphysics. Cloud ice generation can be explicitly solved as the original method of \citet{lin_etal_1983}, instead of the saturation adjustment process. Conversion terms of cloud water to rain ($P_{RAUT}$ and $P_{RACW}$) can be replaced with those used in \citet{khairoutdinov_and_kogan_2000}. Intercept parameters of particle size distribution ($N_{0R}$, $N_{0S}$, $N_{0G}$) can be diagnostically derived by using the equation of \citet{wainwright_etal_2014}, instead of the constant values. Bimodal particle size distribution can be applied to snow following \citet{roh_and_satoh_2007}, instead of the Marshall-Palmer exponential size distribution.
