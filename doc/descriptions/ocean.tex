%\section{Land Physics}
{\bf \Large 
\begin{tabular}{ccc}
\hline
  Corresponding author & : & Tsuyoshi Yamaura\\
\hline
\end{tabular}
}


\subsection{Ocean physics: slab model}

The ocean slab model estimates sea temperature tendencies using a single-layered model.
The temperature tendency equation is estimated from simple heat equation, as follows:
\begin{align}
  \frac{\partial T}{\partial t} = \frac{1}{\rho_{w}c_{l}D} \left( G_{w} + Q \right),
  \label{eq:Ocean-Tdt}
\end{align}
where
$T$ is the sea temperature (K),
$\rho_{w}$ is the water density (kg/m$^3$),
$D$ is the water depth of the slab model,
$c_l$ is the heat capacity of water (J/kg),
$G_{w}$ is the downward heat flux between the sea surface and atmosphere(J/m$^2$/s),
and $Q$ is external heat source (J/m$^2$/s).
The $Q$ contains the negative heat for the melting of the snow precipitation.

Eq. (\ref{eq:Ocean-Tdt}) is discretized as follows:
\begin{align}
  \frac{\Delta T}{\Delta t} = \frac{1}{C_{w}} \left( G_{w} + Q \right),
\end{align}
where
$C_{w}$ is the heat capacity of the slab layer (J/K/m$^2$).

\subsection{Sea surface albedo}
\subsubsection{Nakajima et al. (2000) model}
\citet{nakajima_2000} provided the albedo for the short wave on the sea surface $A$:
\begin{align}
  A = \exp \left[ \Sigma_{i=1}^{3} \Sigma_{j=1}^{5} C_{ij} t^{j-1} \mu_0^{i-1} \right],
\end{align}
where
$C_{ij}$ is the empirical optical parameters,
$t$ is the flux transmissivity for short-wave radiation,
and $\mu_0$ is cosine of the solar zenith angle.


\subsection{Roughness length}
\subsubsection{Miller et al. (1992) model}
\citet{miller_1992} provides the roughness length over the tropical ocean,
based on numerical calculations by combining smooth surface values
with the Charnock relation for aerodynamic roughness length
and constant values for heat and moisture in accordance with \citet{Smith_1988,Smith_1989} suggestions:
\begin{align}
  z_0 &= 0.11u/\nu_* + 0.018u_*^2/g, \label{eq: z_0} \\
  z_t &= 0.40u/\nu_* + 1.4 \times 10^{-5}, \label{eq: z_t} \\
  z_q &= 0.62u/\nu_* + 1.3 \times 10^{-4}, \label{eq: z_q}
\end{align}
where $\nu_*$ is the kinematic viscosity of air ($\sim 1.5 \times 10^{-5}$), and $z_0, z_t$,
and $z_q$ are the roughness length for momentum, heat, and vapor, respectively.

\subsubsection{Moon et al. (2007) model}
\citet{moon_2007} provides the air--sea momentum flux at high wind speeds
based on the coupled wave--wind model simulations for hurricanes.
At first, the wind speed $U$ at 10-m height is estimated from the previous roughness length $z_0$, as follows:
\begin{align}
  U =\frac{u_{*}}{\kappa} \ln \frac{10}{z_0},
\end{align}
where
$u_{*}$ is friction velocity (m/s)
and $\kappa$ is von Kalman constant.
And then, new roughness length $z_0$ is iteratively estimated from the wind speed:
\begin{equation}
  z_0   = \left\{
  \begin{array}{lll}
    \frac{0.0185}{g} u_{*}^2 & \mathrm{for} & U < 12.5, \\
    \left[ 0.085 \left( -0.56 u_{*}^2 + 20.255 u_{*} + 2.458 \right) - 0.58 \right] \times 10^{-3} & \mathrm{for} & U \ge 12.5.
   \end{array} \right.
\end{equation}

Furthermore, \citet{Fairall_2003} provides the roughness length for the heat and vapor
using that for momentum, as follows:
\begin{align}
  z_t &= \frac{ 5.5 \times 10^{-5} }{ ( z_0 u_{*} / \nu_{*} )^{0.6} }, \\
  z_q &= z_t.
\end{align}

