{\bf \Large
\begin{tabular}{ccc}
\hline
  Corresponding author & : & Sachiho A. Adachi\\
\hline
\end{tabular}
}
\\


Urban models calculate energy exchanges between the urban surface (urban canopy) and atmosphere.
In paricular, the model estimates sensible and latent heat fluxes and momentum flux from urban surface to atmosphere.
SCALE-RM has two options for calculation in urban areas: LAND and KUSAKA01.
In LAND option, fluxes over urban subtile are calculated by the land model you chose for your experiment.
Please refer to Section \ref{sec:land}.

\subsection{KUSAKA01: Single layer urban canopy model}

KUSAKA01 is the single layer urban canopy model (UCM) by \citet{kusaka_2001} and \citet{kusaka_2004}.
The UCM of KUSAKA01 assumes street canyons as the urban geometry.

The model has the prognostic variables at the roof, wall, and road:
the surface tempeature and near-surface temparature of artificial construction.
The near-surface temparature is calculated by dividing facets of artificial construction into a number of thickness layers.
The rain amount remaining over the roof, wall, and road are also calculated to evaluate evaporation efficiency.
And then, it also calculates the the heat fluxes from their facets.
The fluxes from urban subtile are estimated by weighted avrage of fluxes from roof, wall, and road.

The model assumes that the urban canopy layer is located lower than the 1st layer of the atmospheric model.
Therefore, there is a restriction that a sum of the displacement height of the canopy and roughness length
must be two meters lower than the center level of the 1st layer of the atmosphere.

In this version, only single urban type can be considered.
Several parameters are prepared to specify the urban geometry and anthropogenic heat. 

