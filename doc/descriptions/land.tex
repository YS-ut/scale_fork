%\section{Land Physics}
{\bf \Large 
\begin{tabular}{ccc}
\hline
  Corresponding author & : & Tsuyoshi Yamaura\\
\hline
\end{tabular}
}


\subsection{Land physics: slab model}

The land slab model estimates the tendency of soil temperature and soil moisture with multi-layered bucket model.
The tendency equiation of soil temperature is estimated from the 1-D vertical diffusion equation.
That is,
\begin{align}
  \frac{\partial T}{\partial t} = \frac{1}{\rho_{L}C_{L}} \left\{ \frac{\partial}{\partial z} \left( \kappa \frac{\partial T}{\partial z} \right) + Q \right\},
  \label{eq:Tdt}
\end{align}
where $T$ is soil temperature ($K$), $\rho_{L}$ is land density ($kg/m^3$), $C_{L}$ is land heat capacity ($J/K/kg$), $\kappa$ is thermal conductivity ($J/K/m/s$), and $Q$ is heat source from the outside ($J/m^3/s$).
The eq. (\ref{eq:Tdt}) is discretized as follows:
\begin{align}
  \frac{\Delta T_{k}}{\Delta t} &= \frac{\nu_{k}}{\Delta z_{k}} \left( \frac{T_{k+1}-T_{k}}{\Delta z_{k+\frac{1}{2}}} - \frac{T_{k}-T_{k-1}}{\Delta z_{k-\frac{1}{2}}} \right) + \frac{Q_{k}}{(\rho_{L}C_{L})_{k}}, \\
                                &= \frac{ 2\nu_{k} }{ \Delta z_{k} (\Delta z_{k+1}+\Delta z_{k}) }(T_{k+1}-T_{k}) - \frac{ 2\nu_{k} }{ \Delta z_{k} (\Delta z_{k}+\Delta z_{k-1}) }(T_{k}-T_{k-1}) +\frac{Q_{k}}{(\rho_{L}C_{L})_{k}},
\end{align}
where
\begin{align}
  \nu_{k} &= \frac{\kappa}{(\rho_{L}C_{L})_{k}}, \\
  (\rho_{L}C_{L})_{k} &= ( 1 - S_{max} ) C_{S} + S_{k} \rho_{W}C_{W},
\end{align}
and $S$ is moisture content at $k$-layer ($m^3/m^3$), $S_{max}$ is maximum moisture content, $C_{S}$ and $C_{W}$ are heat capaticity of soil and water ($J/K/kg$), and $\rho_{W}$ is water density ($kg/m^3$).
The range of $k$ is 1 to $m$.
In this case, $m$ is the number of the lowermost layer.
The tendency equations of soil temperature are implemented as follows:
\begin{align}
  \frac{\Delta T_{1}}{\Delta t} &= - \frac{G_{0}}{(\rho_{L}C_{L})_{1}\Delta z_{1}} + \frac{2\nu_{1}}{\Delta z_{1}(\Delta z_{2}+\Delta z_{1})} (T_{2}-T_{1}), \\
  \frac{\Delta T_{k}}{\Delta t} &= - \frac{2\nu_{k}}{\Delta z_{k}(\Delta z_{k}+\Delta z_{k-1})} (T_{k}-T_{k-1}) + \frac{2\nu_{k}}{\Delta z_{k}(\Delta z_{k+1}+\Delta z_{k})} (T_{k+1}-T_{k}), \\
  \frac{\Delta T_{m}}{\Delta t} &= - \frac{2\nu_{m}}{\Delta z_{m}(\Delta z_{m}+\Delta z_{m-1})} (T_{m}-T_{m-1}),
\end{align}
where $G_{0}$ is upward groud heat flux from atmospheric surface ($J/m^2/s$).
We assume that the ground heat flux under the lowermost layer is zero.

We use implicit scheme for the time integration of the land slab model.
The cofficients in the tendency equation are summarized as follows:
\begin{align}
  a_{1} &= 0, \\
  a_{k} &= - \frac{2\nu_{k}\Delta t}{\Delta z_{k}(\Delta z_{k}+\Delta z_{k-1})}, \\
  b_{k} &= - \frac{2\nu_{k}\Delta t}{\Delta z_{k}(\Delta z_{k}+\Delta z_{k+1})}, \\
  b_{m} &= 0.
\end{align}
Then, the tendency equations of soil temperature are reqritten as follows:
\begin{align}
  T_{1}^{t} &= T_{1}^{t-1} - \frac{G_{0}\Delta t}{\rho_{L}C_{L}\Delta z_{1}} + b_{1} (T_{1}^{t}-T_{2}^{t}), \\
  T_{k}^{t} &= T_{k}^{t-1} + a_{k} (T_{k}^{t}-T_{k-1}^{t}) + b_{k} (T_{k}^{t}-T_{k+1}^{t}), \\
  T_{m}^{t} &= T_{m}^{t-1} + a_{m} (T_{m}^{t}-T_{m-1}^{t}).
\end{align}
This simultaneous equation can be written by using matrix. That is,
\begin{equation}
\begin{pmatrix}
  c_{1}  & b_{1}  &        &        &         &         &         \\
  a_{2}  & c_{2}  & b_{2}  &        &         &         &         \\
         & \ddots & \ddots & \ddots &         &         &         \\
         &        & a_{k}  & c_{k}  & b_{k}   &         &         \\
         &        &        & \ddots & \ddots  & \ddots  &         \\
         &        &        &        & a_{m-1} & c_{m-1} & b_{m-1} \\
         &        &        &        &         & a_{m}   & c_{m}   \\
\end{pmatrix}
\begin{pmatrix}
  T_{1}^{t}   \\
  T_{2}^{t}   \\
  \vdots      \\
  T_{k}^{t}   \\
  \vdots      \\
  T_{m-1}^{t} \\
  T_{m}^{t}   \\
\end{pmatrix}
=
\begin{pmatrix}
  T_{1}^{t-1} - \frac{G_{0}\Delta t}{\rho_{L}C_{L}\Delta z_{1}} \\
  T_{2}^{t-1}   \\
  \vdots        \\
  T_{k}^{t-1}   \\
  \vdots        \\
  T_{m-1}^{t-1} \\
  T_{m}^{t-1}   \\
\end{pmatrix}
,
\end{equation}
where $c_{k} = 1 - a_{k} - b_{k}$.
This matrix can be solved by the Thomas algorithm (tridiagonal matrix algorithm).

Soil moisture is estimated by the similar method.
The tendency equations of soil moisture is defined as the 1-D vertical diffusion euqation.
That is,
\begin{align}
  \frac{\partial W}{\partial t} = \frac{\partial}{\partial z} \left( \nu \frac{\partial W}{\partial z} \right),
  \label{eq:Wdt}
\end{align}
where $W$ is soil moisture ($m^3/m^3$) and $\nu$ is constant water diffusivity for $k$ ($m^2/s$).
The eq. (\ref{eq:Wdt}) is discretized as follows:
\begin{align}
  \frac{\Delta W_{1}}{\Delta t} &= \frac{P - E}{\rho_{W}\Delta z_{1}} + \frac{2\nu}{\Delta z_{1}(\Delta z_{2}+\Delta z_{1})} (W_{2}-W_{1}), \\
  \frac{\Delta W_{k}}{\Delta t} &= - \frac{2\nu}{\Delta z_{k}(\Delta z_{k}+\Delta z_{k-1})} (W_{k}-W_{k-1}) + \frac{2\nu}{\Delta z_{k}(\Delta z_{k+1}+\Delta z_{k})} (W_{k+1}-W_{k}), \\
  \frac{\Delta W_{m}}{\Delta t} &= - \frac{2\nu}{\Delta z_{m}(\Delta z_{m}+\Delta z_{m-1})} (W_{m}-W_{m-1}),
\end{align}
where $P$ is precipitation rate ($kg/m^2/s$) and $E$ is evaporation rate ($kg/m^2/s$).
Same as soil temperature, the tendency equations can be solved by the Thomas algorithm.
