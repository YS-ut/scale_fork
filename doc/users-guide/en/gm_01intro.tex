%###############################################################################

This document is an additional volume of ``SCALE USERS GUIDE''.
This document includes a description of how to use SCALE-Global Model(GM) and a tutrial based on DCMIP2016.
The SCALE-GM is global atmospheric model constructed by using SCALE library.
The overview of SCALE library is described in Chapter 1 of ``SCALE USERS GUIDE''.

In current version, SCALE-GM supports ideal simulations for the dynamical core test.
SCALE-GM is originated from a dynamical core, developed for Nonhydrostatic ICosahedral Atmospheric Model (NICAM).
%The development of NICAM with full physics has been co-developed mainly by
%the Japan Agency for Marine-Earth Science and Technology (JAMSTEC), Atmosphere
%and Ocean Research Institute (AORI) at The University of Tokyo, and RIKEN / Advanced
%Institute for Computational Science (AICS).
A reference paper for NICAM is
Tomita and Satoh (2004), Satoh et al. (2008).
%See also NICAM.jp (\url{http://nicam.jp/}).


%\textcolor{red}{[英語版未対応-------ここから]}

ここで、SCALE-GMの力学コアについて簡潔に説明する。
予報変数は、密度、運動量、全エネルギー(運動エネルギー+内部エネルギー)、及び凝結物等のトレーサーである。
音波は水平方向には陽解法で計算され、鉛直方向には陰解法で計算される(HEVI)。
格子のトポロジーは正二十面体をベースにしており、水平方向にはArakawa A-gridの格子点配置が適用されている。
したがって、全ての予報変数は六角形セルの中心に位置している。
水平方向のオペレーターの離散化には有限体積法を用いている。
鉛直方向にはLorenzタイプのスタッガード格子が利用されている。
正二十面体格子の最適化にはバネ格子法(Tomita et al. 2002)を用いて最適化されており、特定の場所に
格子点を集めて局所的に高解像度化させるストレッチ格子(Tomita et al. 2008)も利用できる。
有限体積法による水平離散化においては、2次精度のdivergenceとgradientが使用されている(Tomita et al. 2001)。
トレーサー移流においては、線形再構築を用いた上流型の移流スキーム(Miura 2007)が用いられている。
時間積分に関しては、3段、もしくは2段のルンゲ・クッタスキームを用いることができる。
また、計算安定性のための数値粘性として、4次の高次粘性スキームが実装されている。

%\textcolor{red}{[英語版未対応-------ここまで]}


\section{Note}
%-------------------------------------------------------------------------------
 \begin{itemize}
   \item Explanations are based on bash environment below.
         In the tcsh environment, use "setenv" command instead of "export".
         (e.g. \verb|> setenv SCALE_SYS "Linux64-gnu-ompi"|)
   \item A symbol of "\verb|>|" means execution of commands in the console.
   \item Gothic means output from standard output.
%   \item \$\{TOP\}   means \verb|scale/scale-gm|
%   \item \$\{ROOT\}  means \verb|scale|
 \end{itemize}

