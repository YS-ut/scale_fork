
%###############################################################################

\section{Preparation}
%-------------------------------------------------------------------------------
Please see the Chapter 2.1 and 2.2 of ``SCALE USERS GUIDE''
as to the details of required system environment, library, and settings of environmental parameters.
Required libraries for SCALE-GM is same as SCALE-RM; HDF5, NetCDF, and MPI.

\section{Compile}
%-------------------------------------------------------------------------------
Please refer the Chapter 2.3.1 of ``SCALE USERS GUIDE'' for how to get the source code
and the Chapter 2.3.2 for the system environmental parameters.
Change directly to test case directly, for example,
Makedef file for SCALE-GM shares the same file with SCALE-RM.
Note that configure the settings for Makedef file before compiling programs.


\noindent To compile SCALE-GM, move to src directory of SCALE-GM.

\begin{verbatim}
  > cd scale-{\version}/scale-gm/src
\end{verbatim}

\noindent Compile the program using make command.
\begin{verbatim}
  > make -j 4
\end{verbatim}
When the compile is finished correctly,
a "\verb|scale-gm|" and below binary files are created in the directly of \texttt{scale-{\version}/bin}.
 \begin{itemize}
   \item \verb|scale-gm| (executable binary of SCALE-GM.)
   \item \verb|gm_fio_cat| (cat commant tool for fio format)
   \item \verb|gm_fio_dump| (dump tool for fio format file)
   \item \verb|gm_fio_ico2ll| (convert tool from icosahedral grid data with fio format to LatLon grid data)
   \item \verb|gm_fio_sel| (sel command tool for fio format)
   \item \verb|gm_mkhgrid| (generation tool of icosahedral horizontal grid applied to spring grid)
   \item \verb|gm_mkllmap| (generation tool of LatLon horizontal grid)
   \item \verb|gm_mkmnginfo| (tool for creating management file of MPI process)
   \item \verb|gm_mkrawgrid| (generation tool of icosahedral horizontal grid)
   \item \verb|gm_mkvlayer| (generation tool of vertical grid)
 \end{itemize}


\begin{itemize}
  \item[*] The number of -j option is a number of parallel compile processes.
   To reduce elapsed time of compile, you can specify the number
   as more than two. We recommend 2 $\sim$ 8 for the -j option.
\end{itemize}

\section{Run experiments}
%-------------------------------------------------------------------------------
\subsection{Test cases}

\noindent Inside the \texttt{scale-{\version}/scale-gm/test/case} directory, you will find various test cases.
For example, the cases based on DCMIP2016 are shown in Table 1.

 \begin{table}[b]
 \begin{center}
 \caption{Corresponding test cases}
 \begin{tabularx}{150mm}{|l|X|} \hline
 \rowcolor[gray]{0.9} test case name in NICAM & test case type \\ \hline
  DCMIP2016-11 & moist baroclinic wave test (161)       \\ \hline
  DCMIP2016-12 & idealized tropical cyclone test (162)  \\ \hline
  DCMIP2016-13 & supercell test (163)                   \\ \hline
 \end{tabularx}
 \end{center}
 \end{table}


\subsection{How to run}

The script for job execution depends on the system.
SCALE-GM has a function, which produces scripts in consideration
of the difference of system environment.
Move to a directory of test case and then type
to create a job script and a script for post-processing;

 \begin{verbatim}
   > make jobshell
 \end{verbatim}

By above command, scripts of ``\verb|run.sh|'' and ``\verb|ico2ll_netcdf.sh|'' are created.
To run the model, type following command;

 \begin{verbatim}
   > make run
 \end{verbatim}

The test experiment will be started.
In DCMIP2016 test experiment, scale-gm generates initial data at the outset of calculation.
The generation process of initial data is not necessary before execution of scale-gm.



\section{Post process}
%-------------------------------------------------------------------------------
 After finish of test run, create the lat-lon grid data from
 the original icosahedral grid data.
 Before submit a job of post process, edit \verb|ico2ll_netcdf.sh|
 following your experimental settings.
 \begin{verbatim}
   > vi ico2ll_netcdf.sh

   [at Line 22]
   # User Settings
   # ---------------------------------------------------------------------

   glev=5          # g-level of original grid
   case=161        # test case number
   out_intev='day' # output interval (format: "1hr", "6hr", "day", "100s")
 \end{verbatim}

 \noindent If the job script is OK, submit a job to the machine.
 \begin{verbatim}
   sh ico2ll_netcdf.sh
 \end{verbatim}

% \begin{verbatim}
%   > bsub < ico2ll_netcdf.sh
% \end{verbatim}

 \noindent The netcdf format data such as "\verb|nicam.161.200.L30.interp_latlon.nc|"
 is created by an "ico2ll" post-process program.

