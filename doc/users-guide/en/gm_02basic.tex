%\textcolor{red}{[英語版未対応-----ここから]}
\section{Preparataion of databases}
%-------------------------------------------------------------------------------

To run the scale-gm, you need horizontal grid databases. We assume that 
we use g-level=5, r-level=0, and 10 MPI processes in this tutorial.
The databases for this setting is included in the tarball and 
thus are not need to be prepared. 
But if you use a different set of settings, you need to use databases 
in one of the following ways.
\subsection{Use preprepared databases}
The databases for some of the typical settings can be downloaded from 
\noindent \url{http://scale.aics.r%iken.jp/ja/download/}

For example, when you download
scale-gm\_database\_gl06rl01pe40.tar.gz\footnote{g-level=6,r-level=1,40
  processes are assumed to be used}
you have 40 boundary (horizontal grid databases) files and 41 llmap (LatLon
grid conversion table) files. 
\begin{verbatim}
  > tar -zxvf scale-gm_database_gl06rl01pe40.tar.gz
\end{verbatim}

\noindent Then you need to move each database to appropriate directories.

\noindent First the boundary files are moved to a new directory under 
\texttt{scale-{\version}/scale-gm/test/data/grid/boundary}
If we suppose we are in the untared directory, you may use the following commands
\\

\verb|  > mkdir scale-|{\version}\verb|/scale-gm/test/data/grid/boundary/gl06rl01pe40|

\verb|  > mv boundary_GL06RL01.* scale-|{\version}\verb|/scale-gm/test/data/grid/boundary/gl06rl01pe40/|
\\

\noindent The llmap files are move to a new directory under 
\texttt{scale-{\version}/scale-gm/test/data/grid/llmap}

\verb|  > mkdir -p scale-|{\version}\verb|/scale-gm/test/data/grid/llmap/gl06/rl01|

\verb|  > mv llmap.* scale-|{\version}\verb|/scale-gm/test/data/grid/boundary/gl06/rl01/| \\


\subsection{Create new databases}
\textcolor{red}{[要検討: ここで言及するのがいいのか、後(例えば3章)の方がいいのか]}

Databases can be made in the following steps.
\subsubsection{1.Creation of horizontal grid databases}
Creating horizontal grid databases consist of the following three steps.
\textcolor{red}{the created databases should be moved to the appropriate directories?}
\footnote{For example\texttt{scale-{\version}/scale-gm/test/data/grid/boundary}}
\renewcommand{\labelenumi}{(\roman{enumi})}
\begin{enumerate}

\item{Creation of process managing database} 

\verb|  > cd scale-|{\version}\verb|/scale-gm/test/framework/mkmnginfo/rl01pe20|
\vspace{-4mm}
\begin{verbatim}
  > make jobshell
  > make run
\end{verbatim}
You need to edit the following file appropriately
 \begin{verbatim}
    > vi nhm_driver.cnf
     &mkmnginfo_cnf
      rlevel       = 1,                      <-- rlevel 
      prc_num      = 20,                     <-- proess number
      output_fname = "rl01-prc20.info",      <-- output filename
     /
 \end{verbatim}

\item{Creation of raw grid database}

\verb|  > cd scale-|{\version}\verb|/scale-gm/test/framework/mkrawgrid/gl05rl01pe20|
\vspace{-4mm}
\begin{verbatim}
  > make jobshell
  > make run
\end{verbatim}
You need to edit the following file appropriately
 \begin{verbatim}
    > vi mkrawgrid.cnf
     &ADMPARAM
       glevel      = 5,                      <-- glevel
       rlevel      = 1,                      <-- rlevel
       vlayer      = 1,                      <-- vertical layer?
       rgnmngfname = "rl01-prc20.info",      <-- input filename 
     /

     &PARAM_MKGRD
       MKGRD_DOSPRING     = .true.,           <-- use spring grid or not
       MKGRD_OUT_BASENAME = "rawgrid_GL05RL01",<-- output filename
       MKGRD_spring_beta  = 1.15D0,           <-- strength of the spring
    /
 \end{verbatim}


\item{水平格子データベースの作成}

\verb|  > cd scale-|{\version}\verb|/scale-gm/test/framework/mkhgrid/gl05rl01pe20|
\vspace{-4mm}
\begin{verbatim}
  > make jobshell
  > make run
\end{verbatim}
You need to edit the following file appropriately
 \begin{verbatim}
    > vi mkhgrid.cnf
     &ADMPARAM
       glevel      = 5,                      <-- glevel
       rlevel      = 1,                      <-- rlevel
       vlayer      = 1,                      <-- vertical layer
       rgnmngfname = "rl01-prc20.info",      <-- input management filename
     /

    &PARAM_MKGRD
       MKGRD_DOPREROTATE      = .false.,     <-- rotate or not
       MKGRD_DOSTRETCH        = .false.,     <-- stretch or not
       MKGRD_DOSHRINK         = .false.,     <-- shrink or not
       MKGRD_DOROTATE         = .false.,     <-- rotate first or not
       MKGRD_IN_BASENAME      = "rawgrid_GL05RL01", <-- input rawgrid filename
       MKGRD_OUT_BASENAME     = "boundary_GL05RL01",<-- outputput bondary filename
   /
\end{verbatim}
\end{enumerate}

\subsubsection{2. Creation of LatLon grid conversion database}

\verb|> cd scale-|{\version}\verb|/scale-gm/test/framework/mkllmap/gl05rl01pe20_t42|
\vspace{-3mm}
\begin{verbatim}
  > make jobshell
  > make run
\end{verbatim}
You need to edit the following file appropriately
 \begin{verbatim}
    > vi mkllmap.cnf
     &ADMPARAM
       glevel      = 5,
       rlevel      = 1,
       vlayer      = 1,
       rgnmngfname = "rl01-prc20.info",
     /

     &GRDPARAM
       hgrid_io_mode = "ADVANCED",
       hgrid_fname   = "boundary_GL05RL01",
       VGRID_fname   = "NONE",
       topo_fname    = "NONE",
     /

     &LATLONPARAM
       latlon_type = "GAUSSIAN",     <-- ``GAUSSIAN'' or ``EQUIDIST'' (equal distance)
       imax        = 128,            <-- number of grid points in x
       jmax        = 64,             <-- number of grid points in y
     /
 \end{verbatim}


%\textcolor{red}{[英語版未対応-----ここまで]}

\section{Run the model}
%-------------------------------------------------------------------------------
\subsection{Test cases}

Several idealized case studies are prepared under the directory 
\noindent \texttt{scale-{\version}/scale-gm/test/case} 
For example, table 1 summarizes the DCMIP2016 experiment cases.
%\textcolor{red}{[英語版未対応-----ここから]}
The details of these test cases can be found at 
\url{https://www.earthsystemcog.org/projects/dcmip-2016/testcases}
or the Test Case Document that is downloadable from the site.
Under these test case directories lie some directories with 
different horizontal grid space and MPI process numbers. 
You can chose a directory depending on your computational resources 
and purposes.
It is also possible to make the databases that are suitable for your needs,
according to the steps addressed in the previous section.
%\textcolor{red}{[英語版未対応-----ここまで]}
 \begin{table}[h]
 \begin{center}
 \caption{Corresponding test cases}
 \begin{tabularx}{150mm}{|l|X|} \hline
 \rowcolor[gray]{0.9} SCALE-GMにおけるテストケースの名称 & 実験内容 \\ \hline
  DCMIP2016-11 & 湿潤傾圧波理想実験  \\ \hline
  DCMIP2016-12 & 全球台風理想実験 \\ \hline
  DCMIP2016-13 & 全球スーパーセル理想実験 \\ \hline
 \end{tabularx}
 \end{center}
 \end{table}


\subsection{計算実行: scale-gm}

Jobcommands depend on the system you use, but we have a system 
that creates scripts depending on your computational environments.
After moving to an arbitrary directory \footnote{for example
  \texttt{scale-{\version}/scale-gm/test/case/DCMIP2016-11/gl06rl01z30pe40}},
execute the following comman to create model run script and post-process script.

 \begin{verbatim}
   > make jobshell
 \end{verbatim}

This commands create ``\verb|run.sh|'' and ``\verb|ico2ll_netcdf.sh|.''
To run the model, execute the following command.

 \begin{verbatim}
   > make run
 \end{verbatim}

This will run the model.
DCMIP2016実験においては、scale-gmを実行したとき、
一番はじめに初期値の作成を行っているため、数値モデルの実行前に初期値を作成する手順はない。

\textcolor{red}{要検討:1〜3の実験をgl04 or gl05くらいで、PE5あたりで実行したときのおおよその所要時間があると
 RMにおけるUGの仕様と合わせることができる。また、正常に実行できた場合にどんなファイルが
 生成されるのか、どれがHistoryファイルで、どれがログファイルなのかくらいの説明があってもよいだろう。}


\section{Post-process: ico2ll}
%-------------------------------------------------------------------------------
To ease drawing and analysis of the outputs, you can convert 
from the original icosahedral horizontal grid itno LatLon horizotnal grid.

Before executing the post-process, you need to edit \verb|ico2ll_netcdf.sh|
according to your experimental setups.
 \begin{verbatim}
   > vi ico2ll_netcdf.sh

   [at Line 22]
   # User Settings
   # ---------------------------------------------------------------------

   glev=5          # g-level of original grid
   case=161        # test case number
   out_intev='day' # output interval (format: "1hr", "6hr", "day", "100s")
 \end{verbatim}

 \noindent The following comman execute the post-process.
 \begin{verbatim}
   sh ico2ll_netcdf.sh
 \end{verbatim}

% \begin{verbatim}
%   > bsub < ico2ll_netcdf.sh
% \end{verbatim}

 \noindent The LatLon grid data created by ico2ll is in netcdf format. 
In this case, the output file name is something like 
``\verb|nicam.161.200.L30.interp_latlon.nc|''
Also by changing the script settings, you can create output data in 
grads format. 
