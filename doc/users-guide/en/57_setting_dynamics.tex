\section{Dynamical Core for Cartesian C-grid} \label{sec:atmos_dyn_cartesC}
%------------------------------------------------------
In this section, the dynamical core for the Cartesian C-grid is described.
The Cartesian C-grid is employed in \scalerm.
In the Cartesian C-grid, scalar quantities, such as density, thermodynamics variable and vapor, is defined at the cell center, while components of vector quantities, such as the momentums and fluxes, are defined at staggered point.
See the description document of \scalerm for more details.



\subsection{Setting Integration Numerical Method}  %\label{subsec:atmos_dyn_sover}
%------------------------------------------------------
The numerical method for time integration in the dynamical process is specified in \nmitem{ATMOS_DYN_TYPE} in \namelist{PARAM_ATMOS} in the configuration file.
\editboxtwo{
\verb|&PARAM_ATMOS  | & \\
\verb| ATMOS_DYN_TYPE    = "HEVE", | & ; Choose from Table \ref{tab:nml_dyn}.\\
\verb|/             | & \\
}

Time step depends on the sound speed in the case of using the explicit method, while it does not in the case of using the implicit method.
In most real atmospheric simulations, vertical grid spacing is much smaller than the horizontal ones.
Thus, fully explicit scheme, that is ``HEVE'', requires a quite small time step, which depends on vertical grid spacing and sound speed.
Therefore, ``HEVI'' is often used for the real atmospheric simulations.



\begin{table}[bth]
\begin{center}
  \caption{Options of methods for time integration in dynamical process}
  \label{tab:nml_dyn}
  \begin{tabularx}{150mm}{llX} \hline
    \rowcolor[gray]{0.9}  Scheme name & Description of scheme & Note\\ \hline
      \verb|HEVE|  & Fully explicit method & \\
      \verb|HEVI|  & Horizontally explicit and vertically implicit methods & Recommended for real experiment\\
    \hline
  \end{tabularx}
\end{center}
\end{table}


\subsection{Setting Temporal and Spatial Schemes} \label{subsec:atmos_dyn_scheme}
%------------------------------------------------------

The temporal and spatial schemes are configured in \namelist{PARAM_ATMOS_DYN}.
An example of setting, which is recommended for real atmospheric simulations,
is shown below.
\editboxtwo{
 \verb|&PARAM_ATMOS_DYN  | & \\
 \verb|ATMOS_DYN_TINTEG_SHORT_TYPE          = RK4,|          & ; Choose from temporal schemes in Table \ref{tab:nml_atm_dyn}\\
 \verb|ATMOS_DYN_TINTEG_TRACER_TYPE         = RK3WS2002,|    & ; Choose from temporal schemes\\
 \verb|ATMOS_DYN_FVM_FLUX_TYPE              = UD3,|          & ; Choose from temporal spatial schemes in Table \ref{tab:nml_atm_dyn}\\
 \verb|ATMOS_DYN_FVM_FLUX_TRACER_TYPE       = UD3KOREN1993,| & ; Choose from spatial schemes\\
 \verb|ATMOS_DYN_FLAG_FCT_TRACER            = .false.,|      & ; Use FCT scheme (.true.) or not (.false.)\\
 \verb|ATMOS_DYN_NUMERICAL_DIFF_COEF        = 0.D0, |        & \\
 \verb|ATMOS_DYN_NUMERICAL_DIFF_COEF_TRACER = 0.D0, |        & \\
 \verb|ATMOS_DYN_coriolis_type              = 'SPHERE',|     & \\
 \verb|ATMOS_DYN_wdamp_height               = 15.D3,|        & ; height [m] of the bottom of sponge layer (for Rayleigh damping)\\
\verb|/             | & \\
}

The other options for temporal and spatial schemes are shown in Table \ref{tab:nml_atm_dyn}.
Note that the time step should be set according to the used schemes in order to ensure numerical stability.
An criteria to determine the time step is described in Section \ref{sec:timeintiv}.


\begin{table}[bth]
\begin{center}
  \caption{Setting temporal and spatial schemes}
  \label{tab:nml_atm_dyn}
  \begin{tabularx}{150mm}{lllX} \hline
    \rowcolor[gray]{0.9} & \multicolumn{1}{l}{Scheme name} & \multicolumn{1}{l}{Description of scheme} & \\ \hline
    \multicolumn{3}{l}{Temporal scheme} &  \\ \hline
    & \multicolumn{1}{l}{\verb|RK3|} & \multicolumn{2}{l}{Heun-type 3 stage and 3rd-order Runge--Kutta scheme} \\
    & \multicolumn{1}{l}{\verb|RK3WS2002|} & \multicolumn{2}{l}{3 stage and generraly 2nd-order Runge--Kutta scheme in \citet{Wicker_2002}} \\
    & \multicolumn{1}{l}{\verb|RK4|} & \multicolumn{2}{l}{4 stage and 4th-order Runge--Kutta scheme} \\
    & \multicolumn{1}{l}{\verb|RK7s6o|} & \multicolumn{2}{l}{7 stage and 6th-order Runge--Kutta scheme in Lawson (1967) (supported only for HEVE)} \\
    \hline
    \multicolumn{3}{l}{Spatial scheme} & Minimum number of halos\\ \hline
    & \multicolumn{1}{l}{\verb|CD2|} & \multicolumn{1}{l}{2nd-order central flux} & \multicolumn{1}{l}{1}\\
    & \multicolumn{1}{l}{\verb|CD4|} & \multicolumn{1}{l}{4th-order central flux} & \multicolumn{1}{l}{2}\\
    & \multicolumn{1}{l}{\verb|CD6|} & \multicolumn{1}{l}{6th-order central flux} & \multicolumn{1}{l}{3}\\
    & \multicolumn{1}{l}{\verb|UD3|} & \multicolumn{1}{l}{3rd-order upwind flux} & \multicolumn{1}{l}{2}\\
    & \multicolumn{1}{l}{\verb|UD5|} & \multicolumn{1}{l}{5th-order upwind flux} & \multicolumn{1}{l}{3}\\
    & \multicolumn{1}{l}{\verb|UD3KOREN1993|} & \multicolumn{1}{l}{3rd-order upwind flux + \citet{Koren_1993}'s filter} & \multicolumn{1}{l}{2}\\
\hline
  \end{tabularx}
\end{center}
\end{table}

The default setting for advection scheme used for the prognostic variables in dynamics, spcified by \nmitem{ATMOS_DYN_FVM_FLUX_TYPE}),
is the 4th-order central flux (\verb|CD4|) in the \scalerm.
When using \verb|CD4| in a simulation with a steep terrain,
an artificial grid-scale vertical flow is often seen at the peak of mountains.
This grid-scale flow may be reduced by using \verb|UD3|.
So, the use of \verb|UD3| is recommended for experiments with steep terrains.


\subsection{Numerical Diffusions}

The numerical stability depends on schemes used in simulations.
Numerical diffusion may improve the stability.
\scalerm has the hyper-diffusion and divergence dumping as the numerical diffusion.
The setting for them is the following:
\editboxtwo{
 \verb|&PARAM_ATMOS_DYN  | & \\
 \verb|ATMOS_DYN_NUMERICAL_DIFF_LAPLACIAN_NUM = 2,    |        & \\
 \verb|ATMOS_DYN_NUMERICAL_DIFF_COEF          = 1.D-4,|        & \\
 \verb|ATMOS_DYN_NUMERICAL_DIFF_COEF_TRACER   = 0.D0, |        & \\
 \verb|ATMOS_DYN_DIVDMP_COEF                  = 0.D0, |        & \\
\verb|/                  | & \\
}


The number of laplacian operator associated with hyper-diffusion is specified by \\
\nmitem{ATMOS_DYN_NUMERICAL_DIFF_LAPLACIAN_NUM}. 
\nmitem{ATMOS_DYN_NUMERICAL_DIFF_COEF} and \\
\nmitem{ATMOS_DYN_NUMERICAL_DIFF_COEF_TRACER} is a non-dimensional coefficient of the hyper-diffusion.
The two-grid scale noise is dumped to $1/e$ in one time step if the coefficient is unity.
The dumping is stronger for larger coefficient.
The hyper-diffusion itself would be numerically unstable if the coefficient is larger than 1.
\nmitem{ATMOS_DYN_NUMERICAL_DIFF_COEF} is for the dynamical prognostic variables, such as density, momentum and potential temperature, and \nmitem{ATMOS_DYN_NUMERICAL_DIFF_COEF_TRACER} is for the tracer variables, such as specific humidity, hydrometeors, and turbulent kinetic energy.
\nmitem{ATMOS_DYN_NUMERICAL_DIFF_COEF} can be set to zero when using the upwind schemes, such as \verb|UD3, UD5|, which has implicit numerical diffusion.


The divergence dumping can also be available to improve numerical stability.
Its coefficient can be set with \nmitem{ATMOS_DYN_DIVDMP_COEF}.


\subsection{Positive Definite}

For tracer advection, guaranteeing a non-negative value is required in most cases.\\
The \verb|UD3KOREN1993| scheme guarantees a non-negative value, whereas other schemes do not.
When schemes other than \verb|UD3KOREN1993| are used the FCT filter can be used to guarantee the non-negative value.
The advection scheme is specified by \nmitem{ATMOS_DYN_FVM_FLUX_TRACER_TYPE}, and switch for the FCT filter is \nmitem{ATMOS_DYN_FLAG_FCT_TRACER}$=$\verb|.true.|.


\subsection{Halos}

The necessary number of halos grid depends on the spatial difference scheme as shown in Table \ref{tab:nml_atm_dyn}.
Set \nmitem{IHALO} and \nmitem{JHALO} in \namelist{PARAM_ATMOS_GRID_CARTESC_INDEX} for the number of halos grid for the x- and y-directions, respectively.
By default, the number of the grid is 2, which is suitable for ``UD3'', ``UD3KOREN1993'', and ``CD4''.
For example, the configuration of the halo for the fifth-order upwind difference scheme is as follows:

\editboxtwo{
 \verb|&PARAM_ATMOS_GRID_CARTESC_INDEX | &  \\
 \verb| IHALO = 3,|   &\\
 \verb| JHALO = 3,|   &\\
 \verb|/ | & \\
}


\subsection{Setting for Coriolis Force} \label{subsec:coriolis}
%----------------------------------------------------------

In this subsection, treatments of the Coriolis force in \scalerm is explained.
The Coriolis parameter is zero as the default, so that you have to set (some) parameter(s) to introduce the Coriolis force in the simulation.
There are two types of setting for the Coriolis parameter: $f$-/$\beta$-plane and sphere.
The type can be specified by \nmitem{ATMOS_DYN_coriolis_type} in \namelist{PARAM_ATMOS_DYN}.

\subsubsection{$f$-/$\beta$-plane}
If \nmitem{ATMOS_DYN_coriolis_type} is set to ``PLANE'', the Coriolis parameter $f$ is $f=f_0 + \beta (y-y_0)$.
When $f_0=0$ and $\beta=0$, which is default, no Coriolis force is taken into account.

The plane for $\beta=0$ is called $f$-plane, otherwise it is called $\beta$-plane.
The parameters of $f_0, \beta$ and $y_0$ is set with the parameters of \namelist{PARAM_ATMOS_DYN} as follows:
\editbox{
  \verb|&PARAM_ATMOS_DYN| \\
  \verb| ATMOS_DYN_coriolis_type = 'PLANE',| \\
  \verb| ATMOS_DYN_coriolis_f0   = 1.0D-5, | ! $f_0$ \\
  \verb| ATMOS_DYN_coriolis_beta = 0.0D0,  | ! $\beta$ \\
  \verb| ATMOS_DYN_coriolis_y0   = 0.0D0,  | ! $y_0$ \\
  \verb| : | \\
  \verb|/| \\
}

The default values of the \nmitem{ATMOS_DYN_coriolis_f0}, \nmitem{ATMOS_DYN_coriolis_beta}, \\
and \nmitem{ATMOS_DYN_coriolis_y0} are 0.0, 0.0, and $y$ at the domain center, respectively.

If you want to add the geostrophic pressure gradient force that is in balance with the Coriolis force accompanied by the geostrophic wind, you need to modify the user specific file \verb|mod_user.f90| (see Section \ref{sec:mod_user}).
The test case of \verb|scale-rm/test/case/inertial_oscillation/20km| is an example of a simulation on the $f$-plane with the geostrophic pressure gradient force.


\subsubsection{Sphere}
On the sphere, the Coriolis parameter depends on the latitude as $f = 2\Omega \sin(\phi)$, where $\Omega$ and $\phi$ are angular velocity of the sphere and latitude, respectively.
In this case, you have to set \nmitem{ATMOS_DYN_coriolis_type} = ``SPHERE''.
The angular velocity of the sphere is set by \nmitem{CONST_OHM} parameter of \namelist{PARAM_CONST} (see Section \ref{subsec:const}).
The latitude of the individual grids is determined depending on the map projection, which is explained in Section \ref{subsec:adv_mapproj}.



\subsection{Lateral Boundary Condition for Coriolis Force}

The lateral boundary in the x-direction for the all the setting (i.e., the $f$-plane, $\beta$-plane, and sphere) can be periodic boundary, and that in the y-direction for the $f$-plane also can be periodic boundary.
On the other hand, the periodic boundary condition cannot be used in the y-direction for the $\beta$-plane or sphere, because the Coriolis parameter differs at the southern and northern boundaries.


The nudge lateral boundary conditions at the south and north boundaries might be used for $f$- and $\beta$-plane experiment.
The test case of \verb|scale-rm/test/case/rossby_wave/beta-plane| is an example of a simulation on the $\beta$-plane with the south and north nudging boundaries.
For the details of the nudging boundary, see Sections \ref{subsec:buffer}.

