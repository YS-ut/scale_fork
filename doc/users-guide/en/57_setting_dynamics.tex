\section{Dynamical core for \scalerm} \label{sec:atmos_dyn}
%------------------------------------------------------

\subsubsection{Setting the numerical method}  %\label{subsec:atmos_dyn_sover}
%------------------------------------------------------
The numerical method for time integration in the dynamical process is specified in \nmitem{ATMOS_DYN_TYPE} in \namelist{PARAM_ATMOS} in the configuration file.
\editboxtwo{
\verb|&PARAM_ATMOS  | & \\
\verb| ATMOS_DYN_TYPE    = "HEVI", | & ; Choose from Table \ref{tab:nml_dyn}.\\
\verb|/             | & \\
}

\begin{table}[bth]
\begin{center}
  \caption{Options of methods for time integration in dynamical process}
  \label{tab:nml_dyn}
  \begin{tabularx}{150mm}{llX} \hline
    \rowcolor[gray]{0.9}  Scheme name & Description of scheme & Note\\ \hline
      \verb|HEVE|  & Fully explicit method & \\
      \verb|HEVI|  & Horizontally explicit and vertically implicit methods & Recommended for real experiment\\
    \hline
  \end{tabularx}
\end{center}
\end{table}


\subsubsection{Setting the temporal and spatial difference schemes} \label{subsec:atmos_dyn_scheme}
%------------------------------------------------------

The temporal integration and spatial difference schemes are configured in \namelist{PARAM_ATMOS_DYN}. According to the spatial integration scheme, the specification of the number of halos is required. The following are recommended along with the other options listed in Table \ref{tab:nml_atm_dyn}. For numerical stability, the time step is changed according to the schemes used. Methods to determine the time step are described in Section \ref{sec:timeintiv}.

\editboxtwo{
 \verb|&PARAM_ATMOS_DYN  | & \\
 \verb|ATMOS_DYN_TINTEG_SHORT_TYPE          = RK4,|          & ; Choose from temporal integration schemes in Table \ref{tab:nml_atm_dyn}\\
 \verb|ATMOS_DYN_TINTEG_TRACER_TYPE         = RK3WS2002,|    & ; Choose from temporal integration schemes\\
 \verb|ATMOS_DYN_FVM_FLUX_TYPE              = UD3,|          & ; Choose from temporal spatial difference schemes\\
 \verb|ATMOS_DYN_FVM_FLUX_TRACER_TYPE       = UD3KOREN1993,| & ; Choose from temporal spatial difference schemes\\
 \verb|ATMOS_DYN_FLAG_FCT_TRACER            = .false.,|      & ; Use FCT scheme (.true.) or not (.false.)\\
 \verb|ATMOS_DYN_NUMERICAL_DIFF_COEF        = 0.D0, |        & \\
 \verb|ATMOS_DYN_NUMERICAL_DIFF_COEF_TRACER = 0.D0, |        & \\
 \verb|ATMOS_DYN_coriolis_type              = 'SPHERE',|     & \\
 \verb|ATMOS_DYN_wdamp_height               = 15.D3,|        & ; height [m] of the bottom of sponge layer (for Rayleigh damping)\\
\verb|/             | & \\
}

\begin{table}[bth]
\begin{center}
  \caption{Setting time integration and spatial difference schemes}
  \label{tab:nml_atm_dyn}
  \begin{tabularx}{150mm}{llXX} \hline
    \rowcolor[gray]{0.9} & \multicolumn{1}{l}{Scheme name} & \multicolumn{1}{l}{Description of scheme} & \\ \hline
    \multicolumn{3}{l}{Temporal integration} &  \\ \hline
    & \multicolumn{1}{l}{\verb|RK3|} & \multicolumn{2}{l}{Heun-type 3rd-order Runge--Kutta scheme} \\
    & \multicolumn{1}{l}{\verb|RK3WS2002|} & \multicolumn{2}{l}{\citet{Wicker_2002}'s 3-step Runge--Kutta scheme} \\
    & \multicolumn{1}{l}{\verb|RK4|} & \multicolumn{2}{l}{4th-order Runge--Kutta scheme} \\
    \hline
    \multicolumn{3}{l}{Spatial difference} & Minimum number of halos\\ \hline
    & \multicolumn{1}{l}{\verb|CD2|} & \multicolumn{1}{l}{2nd-order central difference} & \multicolumn{1}{l}{1}\\
    & \multicolumn{1}{l}{\verb|CD4|} & \multicolumn{1}{l}{4th-order central difference} & \multicolumn{1}{l}{2}\\
    & \multicolumn{1}{l}{\verb|CD6|} & \multicolumn{1}{l}{6th-order central difference} & \multicolumn{1}{l}{3}\\
    & \multicolumn{1}{l}{\verb|UD3|} & \multicolumn{1}{l}{3rd-order upwind difference} & \multicolumn{1}{l}{2}\\
    & \multicolumn{1}{l}{\verb|UD5|} & \multicolumn{1}{l}{5th-order upwind difference} & \multicolumn{1}{l}{3}\\
    & \multicolumn{1}{l}{\verb|UD3KOREN1993|} & \multicolumn{1}{X}{3rd-order upwind scheme + \citet{Koren_1993}'s filter} & \multicolumn{1}{l}{2}\\
\hline
  \end{tabularx}
\end{center}
\end{table}

For advection of prognostic variables in dynamics (\nmitem{ATMOS_DYN_FVM_FLUX_TYPE}),
default setting is the 4th-order central difference (\verb|CD4|) in the \scalerm.
When using \verb|CD4| in an experiment with rough terrain,
it is often confirmed that grid-scale vertical flow is occurred on the peak of mountains.
This grid-scale vertical flow is moderated by using \verb|UD3|.
So, the use of \verb|UD3| is recommended for experiments with rough terrain.
\nmitem{ATMOS_DYN_NUMERICAL_DIFF_COEF} can be set to zero when using \verb|UD3|.

For tracer advection, the use of a scheme for guaranteeing a non-negative value is recommended.
The \verb|UD3KOREN1993| scheme guarantees a non-negative value, whereas other schemes do not. When schemes other than \verb|UD3KOREN1993| are used by being specified in \\
\nmitem{ATMOS_DYN_FVM_FLUX_TRACER_TYPE}, the FCT filter is recommended as \\
\nmitem{ATMOS_DYN_FLAG_FCT_TRACER=}$=$\verb|.true.|

By default, the number of halos is 2. The configuration of the halo is required according to the number shown in Table \ref{tab:nml_atm_dyn}, except for the spatial difference schemes of CD4 and UD3. For example, the configuration of the halo for the fifth-order upwind difference scheme is as follows:

\editboxtwo{
 \verb|&PARAM_ATMOS_GRID_CARTESC_INDEX | &  \\
 \verb| IHALO = 3,|   &\\
 \verb| JHALO = 3,|   &\\
 \verb|/ | & \\
}
