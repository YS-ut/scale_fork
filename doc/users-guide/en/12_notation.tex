
This document assumes an execution in the shell ``bash'' on some Unix system.
If your environment is different, replace the commands
by the relevant commands suitable for your environment.
Unless there is a particular remark, this documentation obeys the following notation:

The command-line symbol for execution is expressed by \verb|$| or \verb|#|.
The difference between the two notations is
in the permission levels of program execution, as shown below:
\begin{verbatim}
 #        <- command by root permission
 $        <- command by user permission
\end{verbatim}

A description enclosed in a rectangle expresses a message generated by the command line, as shown below.
\msgbox{
 -- -- -- -- command-line message\\
 -- -- -- -- -- -- -- -- command-line message\\
 -- -- -- -- -- -- -- -- -- -- -- -- command-line message\\
}

On the other hand, a description enclosed in a polygon with rounded corners shows a part of configure file including namelists, which can be edited as needed.
\editbox{
 -- -- -- -- description in a file\\
 -- -- -- -- -- -- -- -- description in a file\\
 -- -- -- -- -- -- -- -- -- -- -- -- description in a file\\
}

In this documentation, the FORTRAN namelist and its items are denoted by
\namelist{namelist} and \nmitem{item_of_namelist}, respectively.

