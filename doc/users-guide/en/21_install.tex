%%%%%%%%%%%%%%%%%%%%%%%%%%%%%%%%%%%%%%%%%%%%%%%%%%%%%%%%%%%%%%%%%%%%%%%%%%%%%%%%%%%%%%
%  File 21_install.tex
%%%%%%%%%%%%%%%%%%%%%%%%%%%%%%%%%%%%%%%%%%%%%%%%%%%%%%%%%%%%%%%%%%%%%%%%%%%%%%%%%%%%%%
This chapter explains how to compile \scalelib / \scalerm
and the minimum computational requirements for their execution.

\section{Necessary system environment} \label{sec:req_env}
%====================================================================================
\subsubsection{Recommended system environment}

\begin{table}[b]
\begin{center}
\caption{OSs already checked (the ones below are based on x86-64.)}
\begin{tabularx}{150mm}{|l|l|X|} \hline
 \rowcolor[gray]{0.9} Name of OS & Version & Note \\ \hline
 CentOS                & 6.5、6.6、6.8、7.0、7.1、7.2 &  \\ \hline
 openSUSE              & 13.2               &  \\ \hline
 SUSE Enterprise Linux & 11.1、11.3         & GNU compiler is not available in version 11.1. \\ \hline
 fedora                & 20                 &  \\ \hline
 Vine Linux            & 6.2、6.3           &  \\ \hline
 Mac OS X              & 10.10(Yosemite)    &  \\ \hline
\end{tabularx}
\label{tab:compatible_os}
\end{center}
\end{table}


\begin{itemize}
  \item {\bf Hardware system}\\
  Although the necessary hardware depends on the experiment to be performed,
the following are the specifications for conducting a tutorial as in Chapters \ref{chap:tutorial_ideal} and \ref{chap:tutorial_real}.
  \begin{itemize}
    \item {\bf CPU} : 
    The system need to have more than two and more than four physical cores
    for an ideal experiment and a real atmospheric experiment, respectively. 
    \item {\bf Memory} : 
    The system needs 512 MB and 2 GB memory 
    for the ideal experiment and a real atmospheric experiment, respectively.
    Note that this applies to the case involving the use of double-precision floating points.
    \item {\bf HDD} : The system needs to have 3 GB of free disk space in a real atmospheric experiment.
  \end{itemize}
  \item {\bf System software}
  \begin{itemize}
  \item {\bf OS} : Linux OS, Mac OS X\\
  Refer to Table \ref{tab:compatible_os} for other OSs currently supported.
  \item {\bf Compiler} : C, Fortran\\
  The FORTRAN compiler should support FORTRAN 2003 syntax.
  Refer to Table \ref{tab:compatible_compiler} for compilers already confirmed as supported.
  \item {\bf MPI library} : 
  The MPI library should support the MPI 1.0/2.0 protocol.  Refer to Table \ref{tab:compatible_compiler} for MPI libraries already confirmed as supported.
  \item {\bf File I/O library} : 
  The development libraries of gzip, HDF5, and {\netcdf}4 are required.\\
  In stead of the HDF5/{\netcdf}4, {\netcdf}3 is sufficient, but the size of the output file increases in this case.
  \end{itemize}
\end{itemize}


\subsubsection{Useful tools (not always necessary)}
\begin{itemize}
  \item {\bf Data conversion tool}: wgrib, wgrib2, NCL\\
  In the tutorial for the real atmospheric experiment, wgrib is used.
  \item {\bf Viewer tool}:\grads \footnote{\url{http://cola.gmu.edu/grads/}},
  GPhys/Ruby-DCL\footnote{\url{https://www.gfd-dennou.org/arch/ruby/products/gphys/}},
  ncview\footnote{\url{http://meteora.ucsd.edu/~piece/ncview\_home\_page.html}}, and so on.
  \item {\bf Evaluation tool of computational performance}:The PAPI library\footnote{\url{http://icl.utk.edu/papi/}} is available.
\end{itemize}


\begin{table}[tb]
\begin{center}
\caption{Compilers and MPI libraries already checked}
\begin{tabularx}{150mm}{|l|l|X|} \hline
 \rowcolor[gray]{0.9} Name of compiler & Version & Note \\ \hline
 GNU (gcc/gfortran)    & 4.4.x、4.9.x           & Version 4.3 or earlier are not supported. The series with version 4.4.x sometimes issues a warning. \\ \hline
 Intel (icc/ifort)     & 13.0.1、14.0.2、15.0.0  & Version of 2013 or later are recommended \\ \hline
 \rowcolor[gray]{0.9} Name of MPI library &  &  \\ \hline
 openMPI   & 1.8.5、1.10.3            & \\ \hline
 Intel MPI & 4.0 Update 3、4.1.0、5.0 & The version released in 2013 or later is recommended.\\ \hline
 SGI MPT   & 2.05、2.09               & A combined use with the Intel compiler is available. \\ \hline
\end{tabularx}
\label{tab:compatible_compiler}
\end{center}
\end{table}

Since the source code of \scalelib is  written in FORTRAN 2003 standard syntax, the compiler must support it. For example, GNU gfortran version 4.3 or previous versions cannot be used for \scalelib compilation because they do not follow the FORTRAN 2003 standard.

In addition to the above environment, it has been confirmed that \scalelib can be compiled and run on  machines with a SPARC-based instruction set, such as the K Computer and the Fujitsu PRIME-HPC FX10.


\section{Installation of external libraries and drawing tools} \label{sec:inst_env}
%====================================================================================
\subsubsection{Required libraries}
The required external libraries are described below; each library is commonly used.
\begin{itemize}
\item HDF5 Library (\url{https://www.hdfgroup.org/HDF5/})
\item {\netcdf} Library (\url{http://www.unidata.ucar.edu/software/netcdf/})
\item MPI Library (openMPI version、\url{http://www.open-mpi.org/})
\item LAPACK ( \url{http://www.netlib.org/lapack/} ) (only required for \scalegm)
\end{itemize}
The interested reader can refer to the above URLs for the installation information of each library.  Since most series of Linux distributions provide them as packages, their installation is easy. These libraries have been already installed on the K Computer as AICS software. Refer to the description ``AICS software'' in the K portal site or the URL\\ \url{http://www-sys-aics.riken.jp/releasedsoftware/ksoftware/pnetcdf.html}\\ for details. The library paths are described there in ``Makedef.K.'' The tutorials in Chapters \ref{chap:tutorial_ideal} and \ref{chap:tutorial_real} assume that the above library environment has been prepared.


\subsubsection{Drawing tools}
The following are typical drawing tools that can draw
the initial conditions, boundary data, and the simulation results on \scalelib.
The GPhys and \grads are used for a quick view
and the drawing model output in the tutorial in Chapters \ref{chap:tutorial_ideal} and \ref{chap:tutorial_real}, respectively.
Other tools are of course available,
if they can be read in \netcdf format, which is the output format of \scalelib.

\begin{itemize}
\item GPhys / Ruby-DCL by GFD DENNOU Club
 \begin{itemize}
  \item URL: \url{http://ruby.gfd-dennou.org/products/gphys/}
  \item Note: \scalelib outputs the split files 
  in {\netcdf} format according to domain decomposition by the MPI process.
  "gpview" and/or "gpvect" in {\gphys} can directly draw the split file without post-processing.
  \item How to install:
  On the GFD DENNOU Club webpage, the installation is explained for major OSs\\
  \url{https://www.gfd-dennou.org/library/ruby/tutorial/install/index-j.html}\\
   \end{itemize}
\item Grid Analysis and Display System (\grads) by COLA
 \begin{itemize}
  \item URL: \url{http://cola.gmu.edu/grads/}
  \item Note:This is among the most popular drawing tools,
  but the split files with {\netcdf} generated by \scalelib are not directly readable.
  The post-processing tool \verb|netcdf2grads_h| is needed to combine the output of \scalelib in one file that is readable from \grads. Refer to \verb|netcdf2grads_h| for installation instructions in Section \label{sec:source_net2g}, and to Chapters 3 and 4 and Section \label{sec:net2g} for details pertaining to its use.
  \item How to install: Refer to \url{http://cola.gmu.edu/grads/downloads}.
 \end{itemize}
\item Ncview: a {\netcdf} visual browser developed by David W. Pierce
 \begin{itemize}
  \item URL: \url{http://meteora.ucsd.edu/~pierce/ncview_home_page.html}
  \item Note: Ncview is a quick viewer for the {\netcdf} file format.
  Although it cannot combine split files in \scalelib, it is useful in drawing the result file by file.
  \item How to install: Refer to \url{http://meteora.ucsd.edu/~pierce/ncview_home_page.html}
 \end{itemize}
\end{itemize}


\section{Compilation of \scalelib} \label{sec:scale_compile}
%====================================================================================

The environment used in the explanation below is as follows:
\begin{itemize}
\item CPU: Intel Core i5 2410M 2 core /4 thread
\item Memory: DDR3-1333 4GB
\item OS: CentOS 6.6 x86-64、CentOS 7.1 x86-64、openSUSE 13.2 x86-64
\item GNU C/C++、FORTRAN compiler (refer to Appendix \ref{achap:env_setting})
\end{itemize}

\subsubsection{Obtaining the source code} %\label{subsec:get_source_code}
%-----------------------------------------------------------------------------------
The source code for the latest release can be downloaded
from \url{http://scale.aics.riken.jp/ja/download/index.html}.\\
The directory \texttt{scale-{\version}/} can be seen when the tarball file of the source code is extracted. 
\begin{alltt}
 $ tar -zxvf scale-{\version}.tar.gz
 $ ls ./
    scale-{\version}/
\end{alltt}

\subsubsection{Setting Makedef file and environment variables} %\label{subsec:evniromnet}
%-----------------------------------------------------------------------------------

\scalelib is compiled using a Makedef file specified in the environment variable ``\verb|SCALE_SYS|.''  Several variations of the Makedef file corresponding to computer environment are prepared in the directory \texttt{scale-{\version}/sysdep/}.  Choose a Makedef file according to your environment. Table \ref{tab:makedef} shows  environments that support Makedef files that have already been checked. If there is no file for your environment,  create a Makedif file by modifying any existing one.

\begin{table}[htb]
\begin{center}
\caption{Examples of environments and their corresponding Makedef files.}
\begin{tabularx}{150mm}{|l|l|X|l|} \hline
 \rowcolor[gray]{0.9} OS/Computer & Compiler & MPI & Makedef file \\ \hline
                 & gcc/gfortran & openMPI  & Makedef.Linux64-gnu-ompi \\ \cline{2-4}
 Linux OS x86-64 & icc/ifort    & intelMPI & Makedef.Linux64-intel-impi \\ \cline{2-4}
                 & icc/ifort    & SGI-MPT  & Makedef.Linux64-intel-mpt \\ \hline
 Mac OS X        & gcc/gfortran & openMPI  & Makedef.MacOSX-gnu-ompi \\ \hline
 K Computer      & fccpx/frtpx  & mpiccpx/mpifrtpx & Makedef.K \\ \hline
 Fujitsu PRIME-HPC FX10 & fccpx/frtpx & mpiccpx/mpifrtpx & Makedef.FX10 \\ \hline
\end{tabularx}
\label{tab:makedef}
\end{center}
\end{table}

When Linux OS is used, a GNU compiler and openMPI in "Makedef.Linux64-gnu-ompi" are available. If installation is conducted in another environment, change the file name according to Table \ref{tab:makedef}. An environment variable must also be specified as follows:
\begin{verbatim}
 $ export SCALE_SYS="Linux64-gnu-ompi"
\end{verbatim}
If the environment is always the same, it is convenient to explicitly describe the environmental variable in the environmental configuration file, such as \verb|.bashrc|.

The setting of following path is necessary. You must set the environmental variables for HDF5 and {\netcdf}4 as follows: 
\begin{verbatim}
 $ export HDF5="/opt/hdf5"
 $ export NETCDF4="/opt/netcdf"
 $ export NETCDF_INCLUDE="-I/opt/netcdf/include"
 $ export NETCDF_LIBS="-L/opt/hdf5/lib64 -L/opt/netcdf/lib64 \
      -lnetcdff -hdf5_hl -lhdf5 -lm -lz"
\end{verbatim}
The above example shows that
HDF5 is installed in the directory \verb|/opt/hdf5|
and {\netcdf}4 in the directory \verb|/opt/netcdf|
on an Intel compiler.

\subsubsection{Compile} %\label{subsec:compile}
%-----------------------------------------------------------------------------------

Move to the \scalerm source directory 
and compile it by executing the following command:
\begin{verbatim}
 $ cd scale-{\version}/scale-rm/src
 $ make -j 4
\end{verbatim}
The option \verb|"-j 4"| shows the parallel number;
it is four-parallel in this example.
Specify this number according to your environment.
When a compilation is successful, 
the following three executable files are generated under the scale-{\version}/bin directory.
\begin{verbatim}
 scale-rm  scale-rm_init  scale-rm_pp
\end{verbatim}

If you want to compile them again, remove these binary files by the following command:
\begin{verbatim}
 $ make clean
\end{verbatim}
Note that the library already compiled is not deleted in this command. If you want to remove all files, use the following command:
\begin{verbatim}
 $ make allclean
\end{verbatim}
Moreover, if the files are recompiled by changing the compilation environment and options, 
execute ``make allclean.''

{\bf Points to note}
\begin{itemize}
\item  In \scalelib, a compilation and an archive are conducted in directory scale-{\version}/scalelib/ just under the TOP directory scale. 
Object files are placed in a hidden directory called \verb|".lib"| under this directory when you carry out a compilation.
\item  When you want to compile it in debug mode, 
compile it using \verb|"make -j 4 DEBUG=T"|.
\item If you want to change the compilation options in detail, edit \verb|Makedef.***|.
\end{itemize}


\section{Compilation of post-processing tool (net2g)} \label{sec:source_net2g}
%====================================================================================

The output file of \scalerm is divided and stored in every computational node.
\scalelib provides a post-processing tool ``net2g''  to combine these output files
(\verb|history.******.nc|) 
and convert them into a data format direct readable in \grads.
Since it is used in the tutorial Chapters \ref{chap:tutorial_ideal} and \ref{chap:tutorial_real}, the method for the compilation of ``net2g'' is explained here.

Specify the environmental variable for the Makedef file according to your environment,
such as in the compilation of the main body of \scalelib. Then, move to directory  ``net2g'' and execute a command. The parallel executable file using the MPI library is generated by the following command:
\begin{verbatim}
 $ cd scale-{\version}/scale-rm/util/netcdf2grads_h
 $ make -j 2
\end{verbatim}
If there is no MPI library, 
give a compile command to generate the serial executable binary.
\begin{verbatim}
 $ make -j 2 NOMPI=T
\end{verbatim}
If executable file ``net2g'' is generated, the compilation is successful.
When the executable binary is deleted, execute the following command:
\begin{verbatim}
 $ make clean
\end{verbatim}



%%%%%%%%%%%%%%%%%%%%%%%%%%%%%%%%%%%%%%%%%%%%%%%%%%%%%%%%%%%%%%%%%%%%%%%%%%%%%%%%%%%%%%
