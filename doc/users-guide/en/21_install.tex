\section{Download and environmental settings} \label{sec:scale_compile}
%====================================================================================

The environment used in the explanation below is as follows:
\begin{itemize}
\item CPU: Intel Core i5 2410M 2 core /4 thread
\item Memory: DDR3-1333 4GB
\item OS: CentOS 6.6 x86-64、CentOS 7.1 x86-64、openSUSE 13.2 x86-64
\item GNU C/C++、FORTRAN compiler %(refer to Appendix \ref{achap:env_setting})
\end{itemize}

\subsubsection{Obtaining the source code} %\label{subsec:get_source_code}
%-----------------------------------------------------------------------------------
The source code for the latest release can be downloaded
from \url{http://scale.aics.riken.jp/ja/download/index.html}.\\
The directory \texttt{scale-{\version}/} can be seen when the tarball file of the source code is extracted.
\begin{alltt}
 $ tar -zxvf scale-{\version}.tar.gz
 $ ls ./
    scale-{\version}/
\end{alltt}

\subsubsection{Setting Makedef file and environment variables} \label{subsec:environment}
%-----------------------------------------------------------------------------------

\scalelib is compiled using a Makedef file specified in the environment variable ``\verb|SCALE_SYS|.''  Several variations of the Makedef file corresponding to computer environment are prepared in the directory \texttt{scale-{\version}/sysdep/}.  Choose a Makedef file according to your environment. Table \ref{tab:makedef} shows  environments that support Makedef files that have already been checked. If there is no file for your environment,  create a Makedif file by modifying any existing one.

\begin{table}[htb]
\begin{center}
\caption{Examples of environments and their corresponding Makedef files.}
\begin{tabularx}{150mm}{|l|l|X|l|} \hline
 \rowcolor[gray]{0.9} OS/Computer & Compiler & MPI & Makedef file \\ \hline
                 & gcc/gfortran & openMPI  & Makedef.Linux64-gnu-ompi \\ \cline{2-4}
 Linux OS x86-64 & icc/ifort    & intelMPI & Makedef.Linux64-intel-impi \\ \cline{2-4}
                 & icc/ifort    & SGI-MPT  & Makedef.Linux64-intel-mpt \\ \hline
 Mac OS X        & gcc/gfortran & openMPI  & Makedef.MacOSX-gnu-ompi \\ \hline
 K Computer      & fccpx/frtpx  & mpiccpx/mpifrtpx & Makedef.K \\ \hline
 Fujitsu PRIME-HPC FX10 & fccpx/frtpx & mpiccpx/mpifrtpx & Makedef.FX10 \\ \hline
\end{tabularx}
\label{tab:makedef}
\end{center}
\end{table}

When Linux OS is used, a GNU compiler and openMPI in "Makedef.Linux64-gnu-ompi" are available. If installation is conducted in another environment, change the file name according to Table \ref{tab:makedef}. An environment variable must also be specified as follows:
\begin{verbatim}
 $ export SCALE_SYS="Linux64-gnu-ompi"
\end{verbatim}
If the environment is always the same, it is convenient to explicitly describe the environmental variable in the environmental configuration file, such as \verb|.bashrc|.

The setting of following path is necessary. You must set the environmental variables for HDF5 and {\netcdf}4 as follows:
\begin{verbatim}
 $ export SCALE_NETCDF_INCLUDE="-I/opt/netcdf/include"
 $ export SCALE_NETCDF_LIBS=
          "-L/opt/hdf5/lib64 -L/opt/netcdf/lib64 -lnetcdff -lnetcdf -hdf5_hl -lhdf5 -lm -lz"
\end{verbatim}
The above example shows that
HDF5 is installed in the directory \verb|/opt/hdf5|
and {\netcdf}4 in the directory \verb|/opt/netcdf|
on an Intel compiler.

\section{Compile} %\label{subsec:compile}
%-----------------------------------------------------------------------------------

\subsubsection{Compile of \scalerm} %\label{subsec:compile}

Move to the \scalerm source directory
and compile it by executing the following command:
\begin{alltt}
 $ cd scale-{\version}/scale-rm/src
 $ make -j 4
\end{alltt}
The number of \verb|-j| option shows a number of parallel compile processes;
it is four-parallel in this example.
To reduce elapsed time of compile,
it is better to use this option to execute parallel compilation.
Specify this number according to your environment.
When a compilation is successful,
the following three executable files are generated under the \texttt{scale-{\version}/bin} directory.
\begin{verbatim}
 scale-rm  scale-rm_init  scale-rm_pp
\end{verbatim}


\subsubsection{Compile of \scalegm} %\label{subsec:compile}

Move to the \scalegm source directory 
and compile it by executing the following command:
\begin{alltt}
  $  cd scale-{\version}/scale-gm/src
  $  make -j 4
\end{alltt}

The number of \verb|-j| option shows a number of parallel compile processes;
it is four-parallel in this example.
To reduce elapsed time of compile, 
it is better to use this option to execute parallel compilation.
Specify this number according to your environment;
we recommend the use of two $\sim$ eight processes.
When a compilation is successful,
the following three executable files are generated under the scale-{\version}/bin directory.

When a compilation is successful,
the following executable files are created in the \texttt{scale-{\version}/bin} directory.
``fio'' is original format based on binary with header information.
\begin{verbatim}
   scale-gm      (executable binary of \scalegm)
   gm_fio_cat    (cat commant tool for fio format)
   gm_fio_dump   (dump tool for fio format file)
   gm_fio_ico2ll (convert tool from icosahedral grid data with fio format to LatLon grid data)
   gm_fio_sel    (sel command tool for fio format)
   gm_mkhgrid    (generation tool of icosahedral horizontal grid using spring grid)
   gm_mkllmap    (generation tool of LatLon horizontal grid)
   gm_mkmnginfo  (tool for creating management file of MPI process)
   gm_mkrawgrid  (generation tool of icosahedral horizontal grid)
   gm_mkvlayer   (generation tool of vertical grid)
\end{verbatim}


\subsubsection{Points to note}

\noindent If you want to compile them again, remove these binary files by the following command:
\begin{verbatim}
 $ make clean
\end{verbatim}
Note that the library already compiled is not deleted in this command. If you want to remove all files, use the following command:
\begin{verbatim}
 $ make allclean
\end{verbatim}
Moreover, if the files are recompiled by changing the compilation environment and options,
execute \verb|"make allclean"|.\\


In \scalelib, a compilation and an archive are conducted in directory \texttt{scale-{\version}/scalelib/}.
Object files are placed in a hidden directory called \verb|".lib"| under this directory when you carry out a compilation.

When you want to compile it in debug mode, compile it using \verb|"make -j 4 SCALE_DEBUG=T"|.
(All the environment variables applied at compile time are listed in Table \ref{tab:env_var_list}.)
If you want to change the compilation options in detail, edit \verb|Makedef.***|.

\begin{table}[htb]
\begin{center}
\caption{List of environment variables applied at compile time}
\begin{tabularx}{150mm}{|l|X|} \hline
 \rowcolor[gray]{0.9} Environment variable & Description  \\ \hline
 SCALE\_SYS               & Selection of system(necessary)  \\ \hline
 SCALE\_DISABLE\_MPI      & Do not use MPI(only utils)  \\ \hline
 SCALE\_DEBUG             & Compile with options for debug  \\ \hline
 SCALE\_QUICKDEBUG        & Use compile option for quick debug(keep options for speed-up + detect floating point error)  \\ \hline
 SCALE\_USE\_MASSCHECK    & Add calculation for checking mass conservation(only in dynamical process in SCALE-RM)  \\ \hline
 SCALE\_USE\_SINGLEFP     & Use single precision floating-point number(for all sources)  \\ \hline
 SCALE\_USE\_FIXEDINDEX   & Prompt optimization by fixing grid size at compile time  \\ \hline
 SCALE\_ENABLE\_OPENMP    & Enable OpenMP  \\ \hline
 SCALE\_ENABLE\_OPENACC   & Enable OpenACC  \\ \hline
 SCALE\_USE\_AGRESSIVEOPT & Conduct strong optimization with the potentiality of side effect(only with K computer and FX)  \\ \hline
 SCALE\_DISABLE\_INTELVEC & Suppress option for vectorization(only with Intel compiler)  \\ \hline
 SCALE\_NETCDF\_INCLUDE   & Include path of NetCDF library  \\ \hline
 SCALE\_NETCDF\_LIBS      & Library path of NetCDF and specifying libraries  \\ \hline
 SCALE\_ENABLE\_PNETCDF   & Use Parallel NetCDF  \\ \hline
 SCALE\_COMPAT\_NETCDF3   & Limit to NetCDF3-compatible features  \\ \hline
 SCALE\_ENABLE\_MATHLIB   & Use numerical calculation library  \\ \hline
 SCALE\_MATHLIB\_LIBS     & Directory path of numerical library and specifying libraries  \\ \hline
 SCALE\_ENABLE\_PAPI      & Use PAPI  \\ \hline
 SCALE\_PAPI\_INCLUDE     & Include path of PAPI library  \\ \hline
 SCALE\_PAPI\_LIBS        & Directory path of PAPI library and specifying libraries  \\ \hline
 SCALE\_DISABLE\_LOCALBIN & Disable making local binary files in directory for test cases  \\ \hline
 SCALE\_IGNORE\_SRCDEP    & Ignore dependency check at compile time  \\ \hline
 SCALE\_ENABLE\_SDM       & Use SDM (Super Droplet Method) model  \\ \hline
\end{tabularx}
\label{tab:env_var_list}
\end{center}
\end{table}



