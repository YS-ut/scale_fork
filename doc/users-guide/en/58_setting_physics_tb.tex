%Setting the physical process

\section{Sub-Grid Scale Turbulence Scheme} \label{sec:basic_usel_turbulence}
%------------------------------------------------------

The sub-grid scale turbulence model is for representing energy cascade to the sub-grid scale due to the advection terms in Large-eddy simulations (LESs).


The turbulence scheme is specified in \nmitem{ATMOS_PHY_TB_TYPE} in \namelist{PARAM_ATMOS} in files \verb|init.conf| and \verb|run.conf|. The timing of the calling of the turbulence scheme is specified in \namelist{PARAM_TIME}. Refer to Section \ref{sec:timeintiv} for the detailed configuration of the calling timing.

\editboxtwo{
\verb|&PARAM_ATMOS  | & \\
\verb| ATMOS_PHY_TB_TYPE = "SMAGORINSKY", | & ; Select the scheme shown in Table \ref{tab:nml_atm_tb}\\
\verb|/             | & \\
}
\begin{table}[h]
\begin{center}
  \caption{List of turbulence scheme types}
  \label{tab:nml_atm_tb}
  \begin{tabularx}{150mm}{lXX} \hline
    \rowcolor[gray]{0.9}  Schemes & Description of scheme & Reference\\ \hline
      \verb|OFF|          & Do not calculate the turbulence process &  \\
      \verb|SMAGORINSKY|  & Smagorinsky—Lilly-type sub-grid model & \citet{smagorinsky_1963,lilly_1962,Brown_etal_1994,Scotti_1993} \\
      \verb|D1980|        & Deardorff-type sub-grid model & \citet{Deardorff_1980} \\
    \hline
  \end{tabularx}
\end{center}
\end{table}

The \verb|SMAGORINSKY| scheme can be also used for the horizontal eddy viscosity in RANS simulations.
The planetary boundary layer parameterization (Section \ref{sec:basic_usel_pbl}) is a scheme considering only vertical mixing.
If you want to consider the horizontal eddy viscosity in RANS simulations,
use the sub-grid scale turbulence model for horizontal mixing.
Then, \nmitem{ATMOS_PHY_TB_SMG_horizontal} in \namelist{PARAM_ATMOS_PHY_TB_SMG} should be \verb|.true.| as follows:
\editboxtwo{
\verb|&PARAM_ATMOS_PHY_TB_SMG  | & \\
\verb| ATMOS_PHY_TB_SMG_horizontal = .true., | & \\
\verb|/             | & \\
}
