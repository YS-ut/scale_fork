\section{Log file output and how to read log file} \label{sec:log}
%====================================================================================


\scalerm can output log files when you execute \verb|scale-rm|, \verb|scale-rm_init|, and \verb|scale-rm_pp|. On a default setting, log messages from the process of number zero is written into "\verb|LOG.pe000000|" for \verb|scale-rm|, "\verb|init_LOG.pe000000|" for \verb|scale-rm_init|", and \verb|pp_LOG.pe000000|" for \verb|scale-rm_pp|.
Users can modify the settings of log file outputs by editing configuration files of \texttt{run.conf}, \texttt{init.conf}, and \texttt{pp.conf} as follows:

\editboxtwo{
\verb|&PARAM_IO| & \\
\verb| IO_LOG_BASENAME = 'LOG-test1',| & ; output name of log files\\
\verb| IO_LOG_ALLNODE  = .true.,| & ; output log files for all processes or not?\\
\verb|/|\\
}

A name of log file is controled by \nmitem{IO_LOG_BASENAME} in \namelist{PARAM_IO}. In the case of above sample, the name of log files is "\verb|LOG-test1.pe000000|". \nmitem{IO_LOG_ALLNODE} in \namelist{PARAM_IO} controls whether to output log files for all processes or not. When \nmitem{IO_LOG_ALLNODE} is "\verb|.true.|", log files for all processes will be created. The default setting of \nmitem{IO_LOG_ALLNODE} is "\verb|.false.|".
You can find the lines with following format in the log file when you execute \verb|scale-rm|:\\
\msgbox{
  *** TIME: 0000/01/01 00:06:36 + 0.600 STEP:   1984/ 432000 WCLOCK:    2000.2\\
}
This line presents messages about status of computation as follows:
\begin{itemize}
 \item Now, carrying out 6m36.6s of time integration from the initial time of "0000/01/01 00:00:00 + 0.000".
 \item This is 1984th step in whole time steps of 432000.
 \item 2000.2s was taken in a wall clock time (cpu time) .
\end{itemize}
Furthermore, the required time for this computation can be estimated from these information.
In this case, 440000s ( $= 2000.2 \times 432000 \div 1984$ ) is the estimated elapsed time.

\vspace{2ex}
Messages in the log file are output in following format.
\msgbox{
\texttt{{\it type} [{\it subroutine name}] {\it message}} \\
\texttt{\hspace{2em}{\it messages}} \\
\texttt{\hspace{2em} ... }
}
\begin{description}
 \item[{\it type}]: message type that can take one of the following.
   \begin{itemize}
    \item INFO: General information about job execution
    \item WARN: Considerable event about job execution
    \item ERROR: Fatal error that involves stop of execution
   \end{itemize}
 \item[{\it subroutine name}]: the subroutine name writing the message.
 \item[{\it message}]: the main body of the message.
\end{description}


\noindent The sample of error message is as follows.
\msgbox{
\verb|ERROR [ATMOS_PHY_MP_negative_fixer] large negative is found. rank =            1| \\
\verb|      k,i,j,value(QHYD,QV) =           17           8           1   1.7347234759768071E-018   0.0000000000000000| \\
\verb|      k,i,j,value(QHYD,QV) =           19           8           1  -5.4717591620764856E-003   0.0000000000000000| \\
\verb|  ... |
}

