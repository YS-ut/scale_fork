%====================================================================================
\section{How to prepare initial and boundary data} \label{sec:adv_datainput}
%====================================================================================

\begin{table}[tbh]
\begin{center}
\caption{External input data supported in \scalelib}
\begin{tabularx}{150mm}{l|l|X} \hline
 \rowcolor[gray]{0.9} Data type   & \verb|FILETYPE_ORG|  & Note \\ \hline
 SCALE format   & \verb|SCALE-RM|     & History and restart files are supported. The latitude-longitude catalog is needed. \\ \hline
 Binary format  & \verb|GrADS|        & Another namelist for data input is required.    \\ \hline
 WRF format     & \verb|WRF-ARW|      & Both ``wrfout''  and``wrfrst'' are supported.\\ \hline
\end{tabularx}
\label{tab:inputdata_format}
\end{center}
\end{table}

The program \verb|scale-rm_init| converts external data into boundary and initial data by using the configuration file \verb|init.conf|.
It can treat various types of external data as shown in Table \ref{tab:inputdata_format}.
The format of input data is specified by \nmitem{FILETYPE_ORG} in \namelist{PARAM_MKINIT_REAL_(ATMOS|OCEAN|LAND)}.

The ``SCALE format'' is mainly used for offline nesting experiments (Sec. \ref{subsec:nest_offline}).

The ``Binary format'' is defined as binary format with single-precision floating point values that FORTRAN can directly access.
The example of practical usage of binary format data is described in Sec. \ref{sec:tutorial_real_data}.

The ``WRF data format'' is available; WRF model output data can be read directly.
Note that the file should contain all data required to generate the boundary data for \scalerm.

Other format data, such as GRIB/GRIB2 data, can be read in \scale by converting the data to the binary format data.
Note that the format of output files in the latest version is different from that in the version 5.3 or older.
Therefore, the init/boundary files which are made in the version 5.3 or older can't be used in the current version (\scalelib \version).

%%%---------------------------------------------------------------------------------%%%%
\subsubsection{Settings common to SCALE format and Binary format} \label{sec:datainput_common_setting}
%%
The base name of the initial file to be output is set by \nmitem{RESTART_OUT_BASENAME} in \namelist{PARAM_RESTART} as follows:
%%
\editbox{
\verb|&PARAM_RESTART|\\
\verb| RESTART_OUTPUT       = .false.,|  ; whether initial (or restart) files is output. \\
\verb| RESTART_OUT_BASENAME = "",|       ; base name of initial (or restart) files \\
\verb|/|\\
}
To generate initial file, \nmitem{RESTART_OUTPUT} is set to \verb|.true.|. 
The base name of initial file is specified by \nmitem{RESTART_OUT_BASENAME}. 
For example, in the tutorial described in Sec. \ref{sec:tutorial_real_intro}, \nmitem{RESTART_OUT_BASENAME} = ``\verb|init_d01|''.
These settings are used when you want to output restart files in executing \scalerm (see Sec. \ref{sec:restart} for details).
The structure of generated restart files is same as that of initial files.

The settings for input data and boundary files are specified by \namelist{PARAM_MKINIT_REAL_(ATMOS|OCEAN|LAND)} in the file \verb|init.conf|.
%%
\editboxtwo{
\verb|&PARAM_MKINIT_REAL_ATMOS|                                      & \\
\verb| NUMBER_OF_FILES            = 1, |                             & ; Number of input files \\
\verb| NUMBER_OF_TSTEPS           = 1, |                             & ; Number of time step for each input files \\
\verb| FILETYPE_ORG               = '',|                             & ; Select from Table\ref{tab:inputdata_format}\\
\verb| BASENAME_ORG               = '',|                             & ; Information about input files \\
                                                                     & ~~~ (the specified value depends on \verb|FILETYPE_ORG|)\\
\verb| BASENAME_ADD_NUM           = .false.,|                        & ; Switch to number the name of file when \verb|NUMBER_OF_FILES|=1 \\
\verb| BASENAME_BOUNDARY          = '',|                             & ; Base name of boundary files \\
\verb| BOUNDARY_UPDATE_DT         = 0.0,|                            & ; Time interval for input data [s]\\
\verb| USE_FILE_DENSITY           = .false.,|                        & ; Switch to use data of density stored in input files \\
\verb| USE_NONHYDRO_DENS_BOUNDARY = .false.,|                        & ; Switch to use nonhydrostatic density as boundary data \\
\verb| USE_SFC_DIAGNOSES          = .false.,|                        & ; Switch to use diagnostic variables at surface in parent model \\
\verb| USE_DATA_UNDER_SFC         = .true.,|                         & ; Switch to use values below the surface in parent model \\
\verb| SAME_MP_TYPE               = .false.,|                        & ; (For SCALE format) Whether the microphysics scheme is same as that used in parent model \\
\verb| INTRP_TYPE                 = 'LINEAR',|                       & ; Type of horizontal interpolation ("\verb|LINEAR|", "\verb|DIST-WEIGHT|") \\
\verb| SERIAL_PROC_READ = .true.,|                                   & ; Whether only the master process read input data? \\
\verb|/| & \\
\verb|&PARAM_MKINIT_REAL_OCEAN| & \\
\verb| NUMBER_OF_FILES            = 1, |                             & ; Number of input files\\
\verb| NUMBER_OF_TSTEPS           = 1, |                             & ; Number of time steps for each input files\\
\verb| FILETYPE_ORG               = '',|                             & ; Select from Table\ref{tab:inputdata_format}\\
\verb| BASENAME_ORG               = '',|                             & ; Information about input file\\
                                                                     & ~~~ (the specified value depends on \verb|FILETYPE_ORG|)\\
\verb| INTRP_OCEAN_SFC_TEMP       = 'off',|                          & ; (For GrADS format) ("\verb|off|", "\verb|mask|", "\verb|fill|") \\
\verb| INTRP_OCEAN_TEMP           = 'off',|                          & ; (For GrADS format) ("\verb|off|", "\verb|mask|", "\verb|fill|") \\
\verb| SERIAL_PROC_READ = .true.,|                                   & ; Whether only the master process read input data? \\
\verb|/| & \\
\verb|&PARAM_MKINIT_REAL_LAND| & \\
\verb| NUMBER_OF_FILES            = 1, |                             & ; Number of input files\\
\verb| NUMBER_OF_TSTEPS           = 1, |                             & ; Number of time steps for each input files\\
\verb| FILETYPE_ORG               = '',|                             & ; Select from Table\ref{tab:inputdata_format}\\
\verb| BASENAME_ORG               = '',|                             & ; Information about input file\\
                                                                     &  ~~~ (the specified value depends on \verb|FILETYPE_ORG|)\\
\verb| USE_FILE_LANDWATER         = .true.,|                         & ; Switch to use data of soil moisture stored in input file\\
\verb| INTRP_LAND_TEMP            = 'off',|                          & ; (For GrADS format) ("\verb|off|", "\verb|mask|", "\verb|fill|") \\
\verb| INTRP_LAND_WATER           = 'off',|                          & ; (For GrADS format) ("\verb|off|", "\verb|mask|", "\verb|fill|") \\
\verb| INTRP_LAND_SFC_TEMP        = 'off',|                          & ; (For GrADS format) ("\verb|off|", "\verb|mask|", "\verb|fill|") \\
\verb| ELEVATION_CORRECTION       = .true.,|                         & ; Switch to correct the difference of topography elevation from parent model \\
\verb| SERIAL_PROC_READ = .true.,|                                   & ; Whether only the master process read input data? \\
\verb|/| & \\
}

\nmitem{NUMBER_OF_FILES} is the number of input files. 
The program \verb|scale-rm_init| reads these files enumerated from \verb|00000| to the given number \nmitem{NUMBER_OF_FILES}-1. 
For the case of \nmitem{NUMBER_OF_FILES}=1 where the enumeration is not automatically performed, you should set \nmitem{BASENAME_ADD_NUM}=\verb|.true.| to enumerate the files.  
\nmitem{NUMBER_OF_TSTEPS} is the number of time steps stored in each input file.

\nmitem{BOUNDARY_UPDATE_DT} is the time interval of input data; the time interval of boundary data is the same as that of the input data.
\nmitem{BASENAME_BOUNDARY} is the base name of the output boundary files.
If \nmitem{BASENAME_BOUNDARY} is not specified, no boundary files are output.
For example, in the above case, the boundary file(s) are created only for the atmospheric variables; the initial file is prepared for all atmospheric, ocean, and land variables.
In order to perform model integrations, the boundary data for atmospheric variables must have two time steps at least.  
On the other hands, whether boundary data for variables of ocean and land is necessary depends on schemes employed in executing models. 

\nmitem{INTRP_TYPE} is the type of horizontal interpolation;
``\verb|LINEAR|'' and ``\verb|DIST-WEIGHT|'' are available.
If ``\verb|LINEAR|'' is selected, the bi-linear interpolation is used.
This type can not be used when the total number of horizontal gird cells in either the x- or y-direction is 1, i.e., \nmitem{IMAXG} = 1 or \nmitem{JMAXG} = 1, such as a case of 2-dimensional experiment or an unstructured grid data which is stored in one-dimensional array in the horizontal direction.
If ``\verb|DIST-WEIGHT|'' is selected, distance-weighted mean of the nearest $N$-neighbors is used.
In the case of distance-weighted mean, the number of the neighbors is set by \nmitem{COMM_CARTESC_NEST_INTERP_LEVEL} in \namelist{PARAM_COMM_CARTESC_NEST}.


In \scalerm, only the master process reads input data, and then broadcasts the data to each node.
At this time, the algorithm may cause insufficient memory error when reading large input data, especially on a high-performance computing system.
By setting \nmitem{SERIAL_PROC_READ} to \verb|.false.|, each node accesses by itself to only the data it needs, resulting in the reduction of memory usage.
Note that, the computational performance may deteriorate due to locking file access because the algorithm places a load on file I/O.


The above configurations except for \nmitem{BASENAME_BOUNDARY} can be shared between \verb|ATMOS| and \verb|OCEAN| (or \verb|LAND|); 
namely, if the items of namelist are not specified in \namelist{PARAM_MKINIT_REAL_(OCEAN|LAND)},
the values same as that in \namelist{PARAM_MKINIT_REAL_ATMOS} are used.
\\

\noindent\textbf{\underline{Settings for \texttt{PARAM\_MKINIT\_REAL\_ATMOS}}}

The method to calculate density is specified by \nmitem{USE_FILE_DENSITY} and \nmitem{USE_NONHYDRO_DENS_BOUNDARY}. 
For the default setting where both of values are \verb|.false.|, the density in initial and boundary data is calculated with hydrostatic balance, i.e., $\frac{dp}{dz}=-\rho g$ from the input data of temperature and humidity.
If \nmitem{USE_FILE_DENSITY} = \verb|.true.|, the density read from the input file(s) is used as well as the other variables. 
Regardless of the \nmitem{USE_FILE_DENS}, if \nmitem{USE_NONHYDRO_DENS_BOUNDARY} = \verb|.true.|,
the density of boundary data is calculated using the equation of state, i.e., $\rho = p/RT$ and the input temperature, pressure, and humidity data (the initial data is not affected).
This density is generally consistent with that in the parent model. 
% The densities in the initial and boundary data become different if \nmitem{USE_FILE_DENSITY}=\verb|.false.| and \nmitem{USE_NONHYDRO_DENS_BOUNDARY} = \verb|.true.|.
The reason why the option for density boundary data is provided is the following.
In most cases, the density satisfying the hydrostatic balance is preferred for the initial data to reduce initial shock in the simulation.
Therefore, \nmitem{USE_FILE_DENSITY}=\verb|.false.| is recommended. 
On the other hand, since the density constructed with the hydrostatic balance could differ from that in the parent model (or realistic value), this may cause a significant mass bias in the simulation.
In such cases, in terms of representation of the simulation such as pressure field, it could be better to use consistent density with the parent's one using \nmitem{USE_NONHYDRO_DENS_BOUNDARY}=\verb|.true.|.
The vertical acceleration or waves due to the use of hydrostatically inbalanced density as the boundary data is expected to be quickly dumped by the lateral nudging.


\nmitem{USE_SFC_DIAGNOSES} is the switch of method to calculate the values at layers on \scale below the lowermost layer of input data.
If \nmitem{USE_SFC_DIAGNOSES} = \verb|.true.|, the surface quantities, such as T2, RH2, U10, V10, and PSFC are used.
Otherwise, constant potential temperature and hydrostatic balance are assumed.
\nmitem{USE_DATA_UNDER_SFC} is switch whether the input data below its surface is used or ignored.
%The data below the surface may appear on a high-pressure surface at high mountainous region.
\\

\noindent\textbf{\underline{Settings for \texttt{PARAM\_MKINIT\_REAL\_LAND}}}

The soil water is configured by \nmitem{USE_FILE_LANDWATER} of \namelist{PARAM_MKINIT_REAL_LAND} in \verb|init.conf|.
There are two options in preparation of soil moisture: i) the data provided from the input file (\nmitem{USE_FILE_LANDWATER} = \verb|.true.|) and ii) a constant value in the entire region (\nmitem{USE_FILE_LANDWATER} = \verb|.false.|).
In the case of the option i), either fraction of volume of soil moisture (\verb|SMOISVC|) or saturation ratio (\verb|SMOISDS|) is required as the 3D soil moisture data.
Here, the fraction of volume of soil moisture is the ratio of water volume ($V_w$) to soil volume ($V$), i.e., $V_w / V$, 
and the saturation ratio is the ratio of water volume ($V_w$) to void spaces in $V$ ($V_v$), i.e., $V_w / V_v$.
%
In the case of the option ii), the saturation ratio is specified by \nmitem{INIT_LANDWATER_RATIO} whose default value is 0.5.
The porosity of the soil ($V_v/V$) depends on the land use.
\editboxtwo{
\verb|&PARAM_MKINIT_REAL_LAND| &\\
\verb| USE_FILE_LANDWATER   = .false.| & ; Switch to give soil moisture by input files. The default is \verb|.true.| \\
\verb| INIT_LANDWATER_RATIO = 0.5    | & ; (For \verb|USE_FILE_LANDWATER=.false.|) saturation ratio\\
\verb|  ..........                   | & \\
\verb|/| & \\
}

\nmitem{ELEVATION_CORRECTION} in \namelist{PARAM_MKINIT_REAL_LAND} determines
whether soil temperature and land surface temperature for the initial and boundary data are corrected
according to the topographical difference between the parent model and the \scalerm.
In general, topographies of the parent model and the \scalerm are different.
Therefore, if soil temperature and land surface temperature for the initial and boundary data are made by simply interpolating the data of the parent model,
there could be inconsistency of temperature according to the height difference of the topography.
If \nmitem{ELEVATION_CORRECTION} is \verb|.true.|,
soil temperature and land surface temperature for the initial and boundary data are corrected according to the height difference.
For example, if the topography made by the \scalerm is $\Delta h$ higher than that of the parent model,
soil temperature and land surface temperature are uniformly reduced by $\Delta h\Gamma$,
where $\Gamma$ is lapse rate of the international standard atmosphere (ISA): $\Gamma=6.5\times 10^{-3}$ [K/m].
The default value is \nmitem{ELEVATION_CORRECTION} = \verb|.true.|.

%%%---------------------------------------------------------------------------------%%%%
\subsubsection{Input from SCALE format data} \label{sec:datainput_scale}

An example of configurations in \namelist{PARAM_MKINIT_REAL_(ATMOS|OCEAN|LAND)} for the SCALE foramt data is as follows:
\editbox{
\verb|&PARAM_MKINIT_REAL_ATMOS|\\
\verb| NUMBER_OF_FILES            = 2, |             \\
\verb| FILETYPE_ORG               = "SCALE-RM",|     \\
\verb| BASENAME_ORG               = "history_d01",|  \\
\verb| BASENAME_ADD_NUM           = .true.,|         \\
\verb| BASENAME_BOUNDARY          = 'boundary_d01',| \\
\verb| SAME_MP_TYPE               = .false.,|        \\
... \\
\verb|/|\\
\verb|&PARAM_MKINIT_REAL_OCEAN|\\
\verb| FILETYPE_ORG               = "SCALE-RM",|     \\
\verb| BASENAME_ORG               = "history_d01",|  \\
... \\
\verb|/|\\
\verb|&PARAM_MKINIT_REAL_LAND|\\
\verb| FILETYPE_ORG               = "SCALE-RM",|     \\
\verb| BASENAME_ORG               = "history_d01",|  \\
... \\
\verb|/|\\
}


\nmitem{FILETYPE_ORG} is set to \verb|"SCALE-RM"|. 
The base name of input files is specified by \nmitem{BASENAME_ORG}.  
Given \verb|BASENAME_ORG="history_d01"|, the input file should be named \verb|history_d01.nc| when the number of input file is 1. 
When the number of input files is larger than 1 or \nmitem{BASENAME_ADD_NUM} = \verb|.true.| in \namelist{PARAM_RESTART}, 
the name of the input files should be numbered as \verb|"history_d01_XXXXX.nc"|. 

If cloud microphysical scheme used is same as that in the parent model, specify \nmitem{SAME_MP_TYPE}. 


%%%---------------------------------------------------------------------------------%%%%
\subsubsection{Input from binary format data} \label{sec:datainput_grads}

If the binary data is used as input files, 
you need to prepare data according to the format used in \grads.
For the details, see \url{http://cola.gmu.edu/grads/gadoc/aboutgriddeddata.html#structure} .

An example of configurations in \namelist{PARAM_MKINIT_REAL_(ATMOS|OCEAN|LAND)} for the binary format data is as follows:\editbox{
\verb|&PARAM_MKINIT_REAL_ATMOS|\\
\verb| NUMBER_OF_FILES            = 2, |     \\
\verb| FILETYPE_ORG               = "GrADS",|\\
\verb| BASENAME_ORG               = "namelist.grads_boundary.FNL.2005053112-2016051106",|\\
\verb| BASENAME_ADD_NUM           = .true.,| \\
\verb| BASENAME_BOUNDARY          = 'boundary_d01',| \\
\verb| BOUNDARY_UPDATE_DT         = 21600.0,|\\
... \\
\verb|/|\\
\verb|&PARAM_MKINIT_REAL_OCEAN|\\
\verb| FILETYPE_ORG               = "GrADS",|\\
\verb| BASENAME_ORG               = "namelist.grads_boundary.FNL.2005053112-2016051106",|\\
\verb| INTRP_OCEAN_SFC_TEMP       = "mask",|\\
\verb| INTRP_OCEAN_TEMP           = "mask",|\\
... \\
\verb|/|\\
\verb|&PARAM_MKINIT_REAL_LAND|\\
\verb| FILETYPE_ORG               = "GrADS",|\\
\verb| BASENAME_ORG               = "namelist.grads_boundary.FNL.2005053112-2016051106",|\\
\verb| INTRP_LAND_TEMP            = "fill",|\\
\verb| INTRP_LAND_WATER           = "fill",|\\
\verb| INTRP_LAND_SFC_TEMP        = "fill",|\\
... \\
\verb|/|\\
}

\nmitem{FILETYPE_ORG} is set to \verb|"GrADS"|. 
In \scalerm, the file name and the data structure of the \grads formatted binary data are specified in the namelist file specified by \nmitem{BASENAME_ORG} instead of the ``ctl'' file.
The namelist file needs to be prepared beforehand by users.


The following is an example of the namelist file \verb|namelist.grads_boundary*| which provides information of the data file name and data structure.
\editbox{
\verb|#| \\
\verb|# Dimension    |  \\
\verb|#|                \\
\verb|&GrADS_DIMS|  \\
\verb| nx     = 360,|~~~   ; Default value of the number of grids in x direction \\
\verb| ny     = 181,|~~~   ; Default value of the number of grids in y direction \\
\verb| nz     = 26, |~~~~~ ; Default value of the number of layers in z direction \\
\verb|/|                \\
\\
\verb|#              |  \\
\verb|# Variables    |  \\
\verb|#              |  \\
\verb|&GrADS_ITEM  name='lon',     dtype='linear',  swpoint=0.0d0,   dd=1.0d0 /  |  \\
\verb|&GrADS_ITEM  name='lat',     dtype='linear',  swpoint=90.0d0,  dd=-1.0d0 / |  \\
\verb|&GrADS_ITEM  name='plev',    dtype='levels',  lnum=26,| \\
~~~\verb|      lvars=100000,97500,.........,2000,1000, /     |  \\
\verb|&GrADS_ITEM  name='HGT',     dtype='map',     fname='FNLatm', startrec=1,  totalrec=125 / |  \\
\verb|&GrADS_ITEM  name='U',       dtype='map',     fname='FNLatm', startrec=27, totalrec=125 / |  \\
\verb|&GrADS_ITEM  name='V',       dtype='map',     fname='FNLatm', startrec=53, totalrec=125 / |  \\
\verb|&GrADS_ITEM  name='T',       dtype='map',     fname='FNLatm', startrec=79, totalrec=125 / |  \\
\verb|&GrADS_ITEM  name='RH',      dtype='map',     fname='FNLatm', startrec=105,totalrec=125, nz=21 /  |  \\
\verb|&GrADS_ITEM  name='MSLP',    dtype='map',     fname='FNLsfc', startrec=1,  totalrec=9   / |  \\
\verb|&GrADS_ITEM  name='PSFC',    dtype='map',     fname='FNLsfc', startrec=2,  totalrec=9   / |  \\
\verb|&GrADS_ITEM  name='SKINT',   dtype='map',     fname='FNLsfc', startrec=3,  totalrec=9   / |  \\
\verb|&GrADS_ITEM  name='topo',    dtype='map',     fname='FNLsfc', startrec=4,  totalrec=9   / |  \\
\verb|&GrADS_ITEM  name='lsmask',  dtype='map',     fname='FNLsfc', startrec=5,  totalrec=9  /  |  \\
\verb|&GrADS_ITEM  name='U10',     dtype='map',     fname='FNLsfc', startrec=6,  totalrec=9   / |  \\
\verb|&GrADS_ITEM  name='V10',     dtype='map',     fname='FNLsfc', startrec=7,  totalrec=9   / |  \\
\verb|&GrADS_ITEM  name='T2',      dtype='map',     fname='FNLsfc', startrec=8,  totalrec=9   / |  \\
\verb|&GrADS_ITEM  name='RH2',     dtype='map',     fname='FNLsfc', startrec=9,  totalrec=9   / |  \\
\verb|&GrADS_ITEM  name='llev',    dtype='levels',  nz=4, lvars=0.05,0.25,0.70,1.50, /        |  \\
~~~~~~~~\verb| missval=9.999e+20 /|  \\
\verb|&GrADS_ITEM  name='STEMP',   dtype='map',     fname='FNLland', nz=4, startrec=1, totalrec=8,|\\
~~~~~~~~\verb| missval=9.999e+20 /|  \\
\verb|&GrADS_ITEM  name='SMOISVC', dtype='map',     fname='FNLland', nz=4, startrec=5, totalrec=8,|\\
~~~~~~~~\verb| missval=9.999e+20 /|  \\
}

The default value of the number of grids is specified as \verb|nx, ny, nz| in \namelist{GrADS_DIMS}.
Configurations for the variables are individually specified by \namelist{GrADS_ITEM}.
The explanations of \namelist{GrADS_ITEM} is described in Table \ref{tab:namelist_grdvar}.

The base name of input files is set as \verb|fname| in the \namelist{GrADS_ITEM}.
Given the base name is set as \verb|fname="filename"|, the input binary data file should be named  \verb|filename.grd| in the case that the number of the input binary file is one (\nmitem{NUMBER_OF_FILES} = 1).
In the case that there is more than one input files or \nmitem{BASENAME_ADD_NUM} = \verb|.true.| in \namelist{PARAM_RESTART}, 
prepare the files numbered as \verb|"filename_XXXXX.grd"|.

If the number of grids for a variable is different from the default value, the value is specified by \verb|nx, ny, nz| of \namelist{GrADS_ITEM}.
For example, specific humidity (\verb|QV|) and relative humidity (\verb|RH|) in upper levels are not always available.
In such case, the number of layers where the data exist is specified as \verb|nz|.

The method to give the values of QV at layers higher than \verb|nz| is specified by \nmitem{upper_qv_type} in \namelist{PARAM_MKINIT_REAL_GrADS}. 
For \nmitem{upper_qv_type}='\verb|ZERO|', the value of QV is set to zero. 
For \nmitem{upper_qv_type}='\verb|COPY|', the value of RH at layers higher than \verb|nz| is set to that at the top layer where input humidity data exists; it determine the value of QV.
The default value of \verb|upper_qv_type| is "ZERO".
\editboxtwo{
\verb|&PARAM_MKINIT_REAL_GrADS|  & \\
\verb| upper_qv_type = "ZERO"|   & ; Method to give QV at layers higher than \verb|nz|\\
                                 & ~~~("\verb|ZERO|", "\verb|COPY|")\\
\verb|/|\\
}




The variables required for the calculation with \scalerm are listed in Table \ref{tab:grdvar_item}.


{\small
\begin{table}[tbh]
\begin{center}
\caption{Variables of \namelist{GrADS_ITEM}}
\label{tab:namelist_grdvar}
\begin{tabularx}{150mm}{llX} \hline
\rowcolor[gray]{0.9}
item of \verb|GrADS_ITEM|      & Explanation    & Note \\ \hline
name                        & Variable name  & Select from Table \ref{tab:grdvar_item}   \\
dtype                       & Data type      & \verb|"linear"|,\verb|"levels"| or \verb|"map"| \\\hline
\multicolumn{3}{l}{namelist at \nmitem{dtype}\verb|="linear"| (specific use of \verb|"lon", "lat"| )} \\ \hline
fname     & Header name of files           &  \\
swpoint                     & Value of start point &  \\
dd                          & Increment            &  \\ \hline
\multicolumn{3}{l}{namelist at \nmitem{dtype}\verb|"=levels"| (specific use of \verb|"plev", "llev"|)} \\ \hline
lnum      & Number of levels (layers )     &  \\
lvars     & Values of each layer           &  \\ \hline
\multicolumn{3}{l}{namelist at \nmitem{dtype}\verb|="map"|}           \\ \hline
startrec  & Recorded number of variables \nmitem{item}     &  time at t=1\\
totalrec  & Recorded length of all variables per time  &  \\
missval  & Missing value     & (option) \\ \hline
nx       & Number of grids in x direction & (option) \\ \hline
ny       & Number of grids in y direction & (option) \\ \hline
nz       & Number of layers in z direction & (option) \\ \hline
yrev     & If the data is stored from the north to south, set \verb|.true.| & (option) \\ \hline
\end{tabularx}
\end{center}
\end{table}
}

{
\begin{table}[bth]
\begin{center}
\caption{Variable list of \nmitem{name} in \namelist{GrADS_ITEM}. The asterisk means ``it is optional but recommended as possible''. The double-asterisk means ``it is available but not recommended''.}
\label{tab:grdvar_item}
\small
\begin{tabularx}{150mm}{rl|l|l|X} \hline
 \rowcolor[gray]{0.9} & Variable name \nmitem{name} & Explanation & Unit & Data type \nmitem{dtype} \\ \hline
           &\verb|lon|     & longitude data                   & [deg.]         & \verb|linear, map| \\
           &\verb|lat|     & latitude data                    & [deg.]         & \verb|linear, map| \\
           &\verb|plev|    & pressure data                    & [Pa]           & \verb|levels, map| \\
           &\verb|HGT|     & geopotential height data         & [m]            & \verb|map|         \\
    $\ast$ &\verb|DENS|    & air density                      & [kg/m3]        & \verb|map|         \\
           &\verb|U|       & eastward wind speed              & [m/s]          & \verb|map|         \\
           &\verb|V|       & northward wind speed             & [m/s]          & \verb|map|         \\
$\ast\ast$ &\verb|W|       & vertical wind speed              & [m/s]          & \verb|map|         \\
           &\verb|T|       & temperature                      & [K]            & \verb|map|         \\
           &\verb|RH|      & relative humidity                & [\%]           & \verb|map|         \\
           &               & (optional if QV is given)        &                &                    \\
           &\verb|QV|      & specific humidity                & [kg/kg]        & \verb|map|         \\
           &               & (optional if RH is given)        &                &                    \\
$\ast\ast$ &\verb|QC|      & ratio of cloud water mass        & [kg/kg]        & \verb|map|         \\
$\ast\ast$ &\verb|QR|      & ratio of rain water mass         & [kg/kg]        & \verb|map|         \\
$\ast\ast$ &\verb|QI|      & ratio of cloud ice mass          & [kg/kg]        & \verb|map|         \\
$\ast\ast$ &\verb|QS|      & ratio of snow mass               & [kg/kg]        & \verb|map|         \\
$\ast\ast$ &\verb|QG|      & ratio of graupel mass            & [kg/kg]        & \verb|map|         \\
$\ast\ast$ &\verb|MSLP|    & sea level pressure               & [Pa]           & \verb|map|         \\
$\ast\ast$ &\verb|PSFC|    & surface pressure                 & [Pa]           & \verb|map|         \\
$\ast\ast$ &\verb|U10|     & eastward 10m wind speed          & [m/s]          & \verb|map|         \\
$\ast\ast$ &\verb|V10|     & northward 10m wind speed         & [m/s]          & \verb|map|         \\
$\ast\ast$ &\verb|T2|      & 2m temperature                   & [K]            & \verb|map|         \\
$\ast\ast$ &\verb|RH2|     & 2m relative humidity             & [\%]           & \verb|map|         \\
           &               & (optional if Q2 is given)        &                &                    \\
$\ast\ast$ &\verb|Q2|      & 2m specific humidity             & [kg/kg]        & \verb|map|         \\
           &               & (optional if RH2 is given)       &                &                    \\
    $\ast$ &\verb|TOPO|    & topography of GCM                & [m]            & \verb|map|         \\
    $\ast$ &\verb|lsmask|  & ocean--land distribution of GCM  & 0:ocean,1:land & \verb|map|         \\
           &\verb|SKINT|   & surface temperature              & [K]            & \verb|map|         \\
           &\verb|llev|    & soil depth                       & [m]            & \verb|levels|      \\
           &\verb|STEMP|   & soil temperature                 & [K]            & \verb|map|         \\
           &\verb|SMOISVC| & soil moisture (volume fraction)  & [-]            & \verb|map|         \\
           &               & (optional if SMOISDS is given)   &                &                    \\
           &\verb|SMOISDS| & soil moisture (saturation ratio) & [-]            & \verb|map|         \\
           &               & (optional if SMOISVC is given)   &                &                    \\
           &\verb|SST|     & sea surface temperature          & [K]            & \verb|map|         \\
           &               & (optional if SKINT is given)     &                &                    \\\hline
\end{tabularx}
\end{center}
\end{table}
}
