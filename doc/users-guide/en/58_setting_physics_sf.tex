\section{Surface Flux at the Bottom of Atmospheric Boundary } \label{sec:basic_usel_surface}
%------------------------------------------------------
The flux scheme at the bottom atmospheric boundary is configured in \nmitem{ATMOS_PHY_SF_TYPE} in \namelist{PARAM_ATMOS} as follows:

\editboxtwo{
\verb|&PARAM_ATMOS  | & \\
\verb| ATMOS_PHY_SF_TYPE = "COUPLE", | &  ; Select the surface flux scheme shown in Table \ref{tab:nml_atm_sf}\\
\verb|/             | & \\
}
If ocean, land, and urban models are not used, the bottom atmospheric boundary is assumed to be a virtual surface used in an ideal experiment. The timing of the calling of the surface flux scheme is configured in \namelist{PARAM_TIME}. Refer to Section \ref{sec:timeintiv} for the detailed configuration of the calling timing. If ocean, land, and urban models are used, ``COUPLE'' is given to \nmitem{ATMOS_PHY_SF_TYPE}:

\begin{table}[htb]
\begin{center}
  \caption{Choices for the atmospheric bottom boundary }
  \label{tab:nml_atm_sf}
  \begin{tabularx}{150mm}{lX} \hline
    \rowcolor[gray]{0.9}  Value & Description of scheme\\ \hline
      \verb|NONE   | & Do not calculate surface flux, but \verb|"NONE"| is replaced to \verb|"COUPLE"| according to settings of ocean, land, and urban models) \\
      \verb|OFF    | & Do not calculate surface flux\\
      \verb|CONST  | & Fix the constant value of surface flux \\
      \verb|BULK   | & Calculate the surface flux in bulk model \\
      \verb|COUPLE | & Receive surface flux from ocean, land, and urban models \\
    \hline
  \end{tabularx}
\end{center}
\end{table}

%-------------------------------------------------------------------------------
\subsubsection{Configuration of Constant}

If \nmitem{ATMOS_PHY_SF_TYPE} $=$ \verb|"CONST"|, the surface flux can be kept to a value specified in \runconf as follows. The values below are the default ones.

\editboxtwo{
 \verb|&PARAM_ATMOS_PHY_SF_CONST                | & \\
 \verb| ATMOS_PHY_SF_FLG_MOM_FLUX   =    0      | & 0: Bulk coefficient is constant \\
                                                  & 1: Frictional velocity is constant  \\
 \verb| ATMOS_PHY_SF_U_minM         =    0.0E0  | & Lower limit of absolute velocity  [m/s] \\
 \verb| ATMOS_PHY_SF_Const_Cm       = 0.0011E0  | & Constant bulk coefficient for momentum \\
                                                  &  (Active at \verb|ATMOS_PHY_SF_FLG_MOM_FLUX = 0|) \\
 \verb| ATMOS_PHY_SF_CM_min         =    1.0E-5 | & Lower limit of bulk coefficient for momentum \\
                                                  &  (Active at \verb|ATMOS_PHY_SF_FLG_MOM_FLUX = 1|) \\
 \verb| ATMOS_PHY_SF_Const_Ustar    =   0.25E0  | & Constant fictional velocity [m/s] \\
                                                  &  (Active at \verb|ATMOS_PHY_SF_FLG_MOM_FLUX = 1|) \\
 \verb| ATMOS_PHY_SF_Const_SH       =    15.E0  | & Constant sensible heat flux at the surface [W/m2] \\
 \verb| ATMOS_PHY_SF_FLG_SH_DIURNAL =   .false. | & Whether diurnal variation is enabled for sensible heat flux [logical]\\
 \verb| ATMOS_PHY_SF_Const_FREQ     =    24.E0  | & Daily cycle if diurnal variation is enabled [hour]\\
 \verb| ATMOS_PHY_SF_Const_LH       =   115.E0  | & Constant latent heat flux at the surface [W/m2] \\
 \verb|/|            & \\
}

\subsubsection{Bulk Configuration}
%-------------------------------------------------------------------------------
If \nmitem{ATMOS_PHY_SF_TYPE} $=$ \verb|"BULK"|, the surface flux is calculated by the bulk model using the prescribed surface temperature and roughness lengths.
Evaporation efficiency can also be given optionally in a range of 0 to 1.
This flexibility enables the ideal experiment not only for ocean surfaces but also for land.
The evaporation efficiency is specified in \nmitem{ATMOS_PHY_SF_BULK_beta} in \namelist{PARAM_ATMOS_PHY_SF_BULK} in \runconf as follows:

\editboxtwo{
\verb|&PARAM_ATMOS_PHY_SF_BULK  | & \\
\verb| ATMOS_PHY_SF_BULK_beta = 1.0, | & ; Evaporation efficiency (the value must be in a range of 0 to 1). If this value is set as 0, the surface is assumed to be completely dry. If the value is 1, it is assumued to be completely wet (like ocean surface).\\
\verb|/             | & \\
}

The scheme for the bulk coefficient is configured in \nmitem{BULKFLUX_TYPE} in \namelist{PARAM_BULKFLUX} in the file \runconf as follows:

\editboxtwo{
\verb|&PARAM_BULKFLUX  | & \\
\verb| BULKFLUX_TYPE = "B91W01", | & ; Select the bulk coefficient scheme shown in Table \ref{tab:nml_bulk}\\
\verb|/                | & \\
}
\begin{table}[h]
\begin{center}
  \caption{Choices of bulk coefficient scheme}
  \label{tab:nml_bulk}
  \begin{tabularx}{150mm}{llX} \hline
    \rowcolor[gray]{0.9}  Value & Description of scheme & Reference\\ \hline
      \verb|B91W01| & Bulk method by the universal function (Default) & \citet{beljaars_1991,wilson_2001} \\
      \verb|U95|    & Louis-type bulk method  (improved version of Louis (1979) & \citet{uno_1995} \\
    \hline
  \end{tabularx}
\end{center}
\end{table}
