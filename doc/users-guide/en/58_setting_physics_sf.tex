\section{Surface flux scheme} \label{sec:basic_usel_surface}
%------------------------------------------------------
The surface flux scheme calculates momentum, heat and moisture fluxes at the bottom atmospheric boundary.
Type of the scheme is configured in \nmitem{ATMOS_PHY_SF_TYPE} in \namelist{PARAM_ATMOS}.
\editboxtwo{
\verb|&PARAM_ATMOS  | & \\
\verb| ATMOS_PHY_SF_TYPE = "NONE", | &  ; Select the surface flux scheme shown in Table \ref{tab:nml_atm_sf}\\
\verb|/             | & \\
}
If ocean, land, and urban models are used, \nmitem{ATMOS_PHY_SF_TYPE} must be \verb|COUPLE| or \verb|NONE|;
\verb|NONE| is ultimately automatically replaced to \verb|COUPLE|.
Using these models, the surface fluxes are individually calculated in these models, and then, the merged grid value is calculated as area-weighted average of them.

\begin{table}[htb]
\begin{center}
  \caption{List for the atmospheric bottom boundary types}
  \label{tab:nml_atm_sf}
  \begin{tabularx}{150mm}{lX} \hline
    \rowcolor[gray]{0.9}  Schemes & Description of scheme\\ \hline
      \verb|NONE   | & Do not calculate surface flux, but \verb|NONE| is replaced to \verb|COUPLE| according to settings of ocean, land, and urban models \\
      \verb|OFF    | & Do not calculate surface flux \\
      \verb|CONST  | & Use constant values in calculation of surface flux \\
      \verb|BULK   | & Calculate the surface flux by a bulk model \\
      \verb|COUPLE | & Merge surface fluxes from ocean, land, and urban models \\
    \hline
  \end{tabularx}
\end{center}
\end{table}


The time step to call the surface flux scheme is configured in \nmitem{PARAM_DT_ATMOS_PHY_SF} and \nmitem{PARAM_DT_ATMOS_PHY_SF_UNIT} in \namelist{PARAM_TIME}.
Refer to Section \ref{sec:timeintiv} for the detailed configuration of the time step.



%-------------------------------------------------------------------------------
\subsubsection{Constant scheme}

If \nmitem{ATMOS_PHY_SF_TYPE} is \verb|CONST|, the surface flux is determined as configured in \\
\namelist{PARAM_ATMOS_PHY_SF_CONST}.
%
\editboxtwo{
 \verb|&PARAM_ATMOS_PHY_SF_CONST                | & \\
 \verb| ATMOS_PHY_SF_FLG_MOM_FLUX   =    0      | & ; 0: Bulk coefficient is constant \\
                                                  & ; 1: Friction velocity is constant \\
 \verb| ATMOS_PHY_SF_U_minM         =    0.0E0  | & ; Lower limit of absolute velocity [m/s] \\
 \verb| ATMOS_PHY_SF_Const_Cm       = 0.0011E0  | & ; Constant bulk coefficient for momentum \\
                                                  & ;  (valid if \verb|ATMOS_PHY_SF_FLG_MOM_FLUX = 0|) \\
 \verb| ATMOS_PHY_SF_Cm_min         =    1.0E-5 | & ; Lower limit of bulk coefficient for momentum \\
                                                  & ;  (valid if \verb|ATMOS_PHY_SF_FLG_MOM_FLUX = 1|) \\
 \verb| ATMOS_PHY_SF_Const_Ustar    =   0.25E0  | & ; Constant friction velocity [m/s] \\
                                                  & ;  (valid if \verb|ATMOS_PHY_SF_FLG_MOM_FLUX = 1|) \\
 \verb| ATMOS_PHY_SF_Const_SH       =    15.E0  | & ; Constant sensible heat flux at the surface [W/m$^2$] \\
 \verb| ATMOS_PHY_SF_FLG_SH_DIURNAL =   .false. | & ; Whether diurnal variation is considered for sensible heat flux [logical]\\
 \verb| ATMOS_PHY_SF_Const_FREQ     =    24.E0  | & ; Daily cycle if diurnal variation is enabled [hour]\\
 \verb| ATMOS_PHY_SF_Const_LH       =   115.E0  | & ; Constant latent heat flux at the surface [W/m$^2$] \\
 \verb|/|            & \\
}

If \nmitem{ATMOS_PHY_SF_FLAG_MOM_FLUX} is 0, a constant bulk coefficient is used to calculate the surface momentum flux; its value is specified by \nmitem{ATMOS_PHY_SF_Const_Cm}. \\
If \nmitem{ATMOS_PHY_SF_FLAG_MOM_FLUX} is 1, a constant friction velocity is used to calculate the bulk coefficient; its value is specified by \nmitem{ATMOS_PHY_SF_Const_Ustar}.
In the constant friction velocity case, the lower limit of the coefficient is set by \nmitem{ATMOS_PHY_SF_Cm_min}.
The lower limit of absolute velocity at the lowermost layer used in the calculation is set by \nmitem{ATMOS_PHY_SF_U_minM} in both cases.

If \nmitem{ATMOS_PHY_SF_FLG_SH_DIURNAL} = \verb|.false.|, sensible heat flux is constant;
its value is specified by \nmitem{ATMOS_PHY_SF_Const_SH}.
If \nmitem{ATMOS_PHY_SF_FLG_SH_DIURNAL} = \verb|.true.|, the sensible heat flux has temporal sinusoidal variation whose amplitude and time period are \\ \nmitem{ATMOS_PHY_SF_Const_SH} and \nmitem{ATMOS_PHY_SF_Const_FREQ}, respectively, as \\
$\nmitemeq{ATMOS_PHY_SF_Const_SH} \times \sin(2\pi t/3600/\nmitemeq{ATMOS_PHY_SF_Const_FREQ})$,
where $t$ is the integrated time [sec] from start of calculation.

Latent heat flux is constant and its value is specified by \nmitem{ATMOS_PHY_SF_Const_LH}.



\subsubsection{Bulk scheme}
%-------------------------------------------------------------------------------
If \nmitem{ATMOS_PHY_SF_TYPE} is \verb|BULK|, the surface flux is calculated by the bulk model using the prescribed surface state, such as surface temperature, roughness length, and albedo.
The values of the surface state are read from the initial data file.
The surface temperature, roughness length and albedos for short- and long-wave radiation can be set respectively with \nmitem{ATMOS_PHY_SF_DEFAULT_SFC_TEMP}, \\ \nmitem{ATMOS_PHY_SF_DEFAULT_SFC_Z0}, \nmitem{ATMOS_PHY_SF_DEFAULT_SFC_ALBEDO_SW}, \\ and \nmitem{ATMOS_PHY_SF_DEFAULT_SFC_ALBEDO_LW} in \namelist{PARAM_ATMOS_PHY_SF_VARS} at generating the initial data with the \verb|scale-rm_init|.

For calculation of latent heat flux, evaporation efficiency can be given in a range of 0 to 1.
This flexibility enables the ideal experiment not only for ocean surfaces but also for land.
If this value is 0, the surface is assumed to be completely dry, resulting no latent heat flux.
If the value is 1, it is assumed to be completely wet like ocean surface.
The evaporation efficiency is specified in \nmitem{ATMOS_PHY_SF_BULK_beta} in \namelist{PARAM_ATMOS_PHY_SF_BULK}.
\editboxtwo{
\verb|&PARAM_ATMOS_PHY_SF_BULK       | & \\
\verb| ATMOS_PHY_SF_BULK_beta = 1.0, | & ; Evaporation efficiency (range of 0--1) \\
\verb|/                              | & \\
}



\subsubsection{Bulk coefficient}
If \nmitem{ATMOS_PHY_SF_TYPE} is \verb|BULK| or \verb|COUPLE|, the bulk coefficients are calculated based on the Monin-Obukhov similarity.
The parameters for calculation of the coefficients are configured in \namelist{PARAM_BULKFLUX}.
\editboxtwo{
\verb|&PARAM_BULKFLUX                      | & \\
\verb| BULKFLUX_TYPE = "B91W01",           | & ; Select the bulk coefficient scheme shown in Table \ref{tab:nml_bulk} \\
\verb| BULKFLUX_Uabs_min = 1.0D-2,         | & ; Minimum absolute velocity speed in the surface flux calculation [m/s] \\
\verb| BULKFLUX_surfdiag_neutral = .true., | & ; Switch to assume neutral condition in calculation of surface diagnostics \\
\verb| BULKFLUX_WSCF = 1.2D0,              | & ; Empirical scaling factor for w$^{\ast}$ \citep{beljaars_1994} \\
\verb| BULKFLUX_Wstar_min = 1.0E-4,        | & ; Minimum value of w$^{\ast}$ [m/s] \\
\verb| BULKFLUX_NK2018 = .true.,           | & ; Switch to use scheme of \citet{nishizawa_2018} \\
\verb|/                                    | & \\
}

\begin{table}[h]
\begin{center}
  \caption{List of bulk coefficient scheme types}
  \label{tab:nml_bulk}
  \begin{tabularx}{150mm}{llX} \hline
    \rowcolor[gray]{0.9}  Schemes & Description of scheme & Reference\\ \hline
      \verb|B91W01| & Bulk method by the universal function (Default) & \citet{beljaars_1991,wilson_2001} \\
      \verb|U95|    & Louis-type bulk method  (improved version of \citet{louis_1979}) & \citet{uno_1995} \\
    \hline
  \end{tabularx}
\end{center}
\end{table}

The scheme to calculate the bulk coefficients is specified by \nmitem{BULKFLUX_TYPE}.
The supported schemes are listed in Table \ref{tab:nml_bulk}.

The coefficients strongly depend on the wind speed.
Because the velocity in simulation does not contain sub-grid scale component, the surface flux may be underestimated especially under calm condition.
Therefore, a lower limit can be introduced for the absolute velocity at the lowermost layer used in the calculation.
The lower limit is specified by \nmitem{BULKFLUX_Uabs_min}.

The surface diagnostic variables, such as 10-m velocities and 2-m temperature and humidity, are calculated so that they are consistent with the bulk coefficients.
It means that these variables significantly depend on the surface conditions and static stability.
Thus, when grid cells having different surface conditions are adjacently distributed such as near coasts, the values of the diagnostics may have a large gap at the boundaries between these grid cells.
In addition, when the static stability greatly changes around sunrise or sunset, the values of the diagnostics may change drastically in a short time.
These large gaps can be reduced by assuming neutral static stability in the calculation of the surface diagnostic variables.
To use the assumption of neutral stability, set \nmitem{BULKFLUX_surfdiag_neutral} to \verb|.true.|.


For the \verb|B91W01| scheme, a couple of additional parameters are available.
\nmitem{BULKFLUX_WSCF} is an empirical scaling factor for the free convection velocity scale $w^{\ast}$ introduced by \citet{beljaars_1994} ($\beta$ in his paper).
By default, it is determined depending on the horizontal grid spacing $\Delta x$ as $1.2 \times \min(\Delta x/1000, 1)$.
In simulations with the grid spacing larger than 1 km, the factor is 1.2.
For smaller grid spacing, it decreases in proportional to the grid spacing, c.f., $\lim_{\Delta x \to 0} w^{\ast} = 0$.
Note that, the minimum of $w^{\ast}$ is specified by \nmitem{BULKFLUX_Wstar_min} and its default value is 1.0E-4.
\nmitem{BULKFLUX_NK2018} is a switch to use the more appropriate formulation of the similarity for finite volume models proposed by \citet{nishizawa_2018}.
