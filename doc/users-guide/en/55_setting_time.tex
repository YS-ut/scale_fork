\section{Setting integration period and time step} \label{sec:timeintiv}
%------------------------------------------------------

The integration period and time step are configured appropriately according to experimental design. The time step depends on the spatial resolution of the model. A shorter time step is sometimes required to avoid numerical instability. The period of integration and the time step are configured in \namelist{PARAM_TIME} in the configuration file \verb|run.conf|.

\editboxtwo{
\verb|&PARAM_TIME| & \\
\verb| TIME_STARTDATE             = 2014, 8, 10, 0, 0, 0,| & Start date of integration: it is required for the calculation of the radiation process.\\
\verb| TIME_STARTMS               = 0.D0,    | & Start date [mili sec]\\
\verb| TIME_DURATION              = 12.0D0,  | & Integration time [init is defined by \verb|TIME_DURATION_UNIT|]\\
\verb| TIME_DURATION_UNIT         = "HOUR",  | & Unit for \verb|TIME_DURATION|\\
\verb| TIME_DT                    = 60.0D0,  | & Time step for time integration\\
\verb| TIME_DT_UNIT               = "SEC",   | & Unit for \verb|TIME_DT|\\
\verb| TIME_DT_ATMOS_DYN          = 30.0D0,  | & Time step for calculation of dynamical process\\
\verb| TIME_DT_ATMOS_DYN_UNIT     = "SEC",   | & Unit for \verb|TIME_DT_ATMOS_DYN|\\
\verb| TIME_DT_ATMOS_PHY_MP       = 60.0D0,  | & Time step for calculation of microphysics process\\
\verb| TIME_DT_ATMOS_PHY_MP_UNIT  = "SEC",   | & Unit for \verb|TIME_DT_ATMOS_PHY_MP|\\
\verb| TIME_DT_ATMOS_PHY_TB       = 60.0D0,  | & Time step for calculation of turbulence process\\
\verb| TIME_DT_ATMOS_PHY_TB_UNIT  = "SEC",   | & Unit for \verb|TIME_DT_ATMOS_PHY_TB|\\
\verb| TIME_DT_ATMOS_PHY_BL       = 60.0D0,  | & Time step for calculation of boundary layer process\\
\verb| TIME_DT_ATMOS_PHY_BL_UNIT  = "SEC",   | & Unit for \verb|TIME_DT_ATMOS_PHY_BL|\\
\verb| TIME_DT_ATMOS_PHY_RD       = 600.0D0, | & Time step for calculation of radiation process\\
\verb| TIME_DT_ATMOS_PHY_RD_UNIT  = "SEC",   | & Unit for \verb|TIME_DT_ATMOS_PHY_RD|\\
\verb| TIME_DT_ATMOS_PHY_SF       = 60.0D0,  | & Time step for calculation of bottom boundary condition (surface process) for atmosphere\\
\verb| TIME_DT_ATMOS_PHY_SF_UNIT  = "SEC",   | & Unit for \verb|TIME_DT_ATMOS_PHY_SF|\\
\verb| TIME_DT_OCEAN              = 300.0D0, | & Time step for calculation of ocean process\\
\verb| TIME_DT_OCEAN_UNIT         = "SEC",   | & Unit for \verb|TIME_DT_OCEAN|\\
\verb| TIME_DT_LAND               = 300.0D0, | & Time step for calculation of land process\\
\verb| TIME_DT_LAND_UNIT          = "SEC",   | & Unit for \verb|TIME_DT_LAND|\\
\verb| TIME_DT_URBAN              = 300.0D0, | & Time step for calculation of urban process\\
\verb| TIME_DT_URBAN_UNIT         = "SEC",   | & Unit for \verb|TIME_DT_URBAN|\\
\verb| TIME_DT_ATMOS_RESTART      = 21600.D0, | & Output interval of restart files for atmospheric variables\\
\verb| TIME_DT_ATMOS_RESTART_UNIT = "SEC",    | & Unit for \verb|TIME_DT_ATMOS_RESTART|\\
\verb| TIME_DT_OCEAN_RESTART      = 21600.D0, | & Output interval of restart files for ocean variables\\
\verb| TIME_DT_OCEAN_RESTART_UNIT = "SEC",    | & Unit for \verb|TIME_DT_OCEAN_RESTART|\\
\verb| TIME_DT_LAND_RESTART       = 21600.D0, | & Output interval of restart files for land variables\\
\verb| TIME_DT_LAND_RESTART_UNIT  = "SEC",    | & Unit for \verb|TIME_DT_LAND_RESTART|\\
\verb| TIME_DT_URBAN_RESTART      = 21600.D0, | & Output interval of restart files for urban variables\\
\verb| TIME_DT_URBAN_RESTART_UNIT = "SEC",    | & Unit for \verb|TIME_DT_URBAN_RESTART|\\
\verb|/|\\
}
\nmitem{TIME_DT} is the time step for time integration, usually described as $\Delta t$. It is used as time step for tracer advection as well as the basic unit for all physical processes. To avoid numerical instability, \nmitem{TIME_DT} must satisfy the following condition: it is less than the value calculated by dividing grid size by a supposed maximum advection velocity. A shorter time step for dynamical process, i.e. \nmitem{TIME_DT_ATMOS_DYN}, should be given than $Delta t$ because the time integration of dynamic variables is constrained not by advection velocity, but by the speed of the acoustic wave.
\nmitem{TIME_DT_ATMOS_DYN} depends on the time integration scheme in relation to the stability of calculation. As criterion, the standard values of \nmitem{TIME_DT_ATMOS_DYN} are calculated by dividing the minimum grid interval by 420 m/s and 840 m/s, in the case \nmitem{ATMOS_DYN_TINTEG_SHORT_TYPE="RK4, RK3"}, respectively.
Note that \nmitem{TIME_DT_ATMOS_DYN} needs to be a divisor of \nmitem{TIME_DT}.
When the ratio of \nmitem{TIME_DT} to \nmitem{TIME_DT_ATMOS_DYN} is too large,
the numerical instability sometimes occurs.
The ratio of \nmitem{TIME_DT}/\nmitem{TIME_DT_ATMOS_DYN} is recommended to be set two or three.


A time step for the physical process represents the timing of the tendency to update given by the process. Once the model starts, each physical process is called during the setup of the model to obtain the initial tendency. Each tendency is updated at every time step specified process by process. All time steps for the physical process are better as multiples of \nmitem{TIME_DT}.

The surface fluxes are calculated by the surface process for the atmosphere. On the contrary, if ocean, land, and urban models are used, the fluxes are calculated by these models. A model grid contains several types of land use, such as ocean, urban, and land. The grid mean value of fluxes is obtained as the weighted average of fluxes over each instance of land use according to the fraction of land use.

As described above, the initial tendencies of all processes are updated during the setup of the model. Therefore, the output intervals of restart file are required as multiples of time steps for all processes. If not, a restart calculation disagrees with the continuous calculation. When \nmitem{TIME_DT_ATMOS_RESTART}, \nmitem{TIME_DT_OCEAN_RESTART},  \nmitem{TIME_DT_LAND_RESTART}, and\\ \nmitem{TIME_DT_URBAN_RESTART}, are not specified, the restart files are created at the end of the simulation, i.e. at \nmitem{TIME_DURATION}. The details of the restarted simulation are described in Section \ref{sec:restart}.
