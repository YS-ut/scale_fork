\section{Compilation of post-processing tool for \scalerm (net2g)} \label{sec:source_net2g}
%====================================================================================

The output file of \scalerm is divided and stored in every computational node.
\scalelib provides a post-processing tool ``net2g''  to combine these output files
(\verb|history.******.nc|)
and convert them into a data format direct readable in \grads.
Since it is used in the tutorial Chapters \ref{chap:tutorial_ideal} and \ref{chap:tutorial_real}, the method for the compilation of ``net2g'' is explained here.

Specify the environmental variable for the Makedef file according to your environment,
such as in the compilation of the main body of \scalelib. Then, move to directory  ``net2g'' and execute a command. The parallel executable file using the MPI library is generated by the following command:
\begin{alltt}
 $ cd scale-{\version}/scale-rm/util/netcdf2grads_h
 $ make -j 2
\end{alltt}
If there is no MPI library,
give a compile command to generate the serial executable binary.
\begin{verbatim}
 $ make -j 2 SCALE_DISABLE_MPI=T
\end{verbatim}
If executable file ``net2g'' is generated, the compilation is successful.
When the executable binary is deleted, execute the following command:
\begin{verbatim}
 $ make clean
\end{verbatim}


