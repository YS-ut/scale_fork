%-------------------------------------------------------------------------------
\section{Setting the monitor file} \label{sec:monitor}
%-------------------------------------------------------------------------------

A monitor file and output variables are configured at \namelist{PARAM_MONITOR} and \namelist{MONITOR_ITEM} in \verb|run.conf|.
The default format of the monitor file is configured at \namelist{PARAM_MONITOR}.

\editboxtwo{
\verb|&PARAM_MONITOR                      | & \\
\verb| MONITOR_OUT_BASENAME  = "monitor", | & ; Base name of the output file \\
\verb| MONITOR_USEDEVATION   = .true.,    | & ; Use deviation from first step? \\
\verb| MONITOR_STEP_INTERVAL = 1,         | & ; Step interval of monitor output \\
\verb|/                                   | & \\
}

Monitor component outputs global domain total of physical quantity such as dry air mass, water vapor, total energy, surface precipitation flux, and so on.
These output is useful to check mass and energy budget.
The monitor file is ASCII format and the name of the file is set by \nmitem{MONITOR_OUT_BASENAME} \verb|.pe000000|.
The time interval of monitor output is specified in the \nmitem{MONITOR_STEP_INTERVAL} as the multiple of time step ($\Delta t$).

\editboxtwo{
\verb|&MONITOR_ITEM   | & \\
\verb| NAME = "ENGT", | &  Variable name. List of variables is shown in Table \ref{tab:varlist_monitor} \\
\verb|/               | & \\
}

\begin{table}[h]
\begin{center}
  \caption{Variables available for monitor output}
  \label{tab:varlist_monitor}
  \begin{tabularx}{150mm}{|l|X|l|} \hline
    \rowcolor[gray]{0.9}  Values & Description & Unit \\ \hline
      \verb|DENS|         & Air mass                                  & [kg]     \\
      \verb|MOMZ|         & Momentum in z-direction                   & [kg m/s] \\
      \verb|MOMX|         & Momentum in x-direction                   & [kg m/s] \\
      \verb|MOMY|         & Momentum in y-direction                   & [kg m/s] \\
      \verb|RHOT|         & Potential temperature                     & [kg K]   \\
      \verb|TRACER*|      & Tracers in prognostic variable            & [each unit $\times$ kg] \\
      \verb|QDRY|         & Dry air mass                              & [kg] \\
      \verb|QTOT|         & Water mass                                & [kg] \\
      \verb|EVAP|         & Evaporation at the surface                & [kg] \\
      \verb|PRCP|         & Precipitation                             & [kg] \\
      \verb|ENGT|         & Total     energy (\verb|ENGP + ENGK + ENGI|)  & [J] \\
      \verb|ENGP|         & Potential energy ($\rho * g * z$)             & [J] \\
      \verb|ENGK|         & Kinetic   energy ($\rho * (W^2+U^2+V^2) / 2$) & [J] \\
      \verb|ENGI|         & Internal  energy ($\rho * C_v * T$)           & [J] \\
      \verb|ENGFLXT|      & Total energy flux convergenc (\verb|SH + LH + SFC_RD - TOA_RD|) & [J] \\
      \verb|ENGSFC_SH|    & Surface sensible heat flux                & [J] \\
      \verb|ENGSFC_LH|    & Surface latent   heat flux                & [J] \\
      \verb|ENGSFC_RD|    & Surface net radiation flux                & [J] \\
                          & (\verb|SFC_LW_up + SFC_SW_up - SFC_LW_dn - SFC_SW_dn|) & \\
      \verb|ENGTOA_RD|    & Top-of-atmosphere net radiation flux      & [J] \\
                          & (\verb|TOA_LW_up + TOA_SW_up - TOA_LW_dn - TOA_SW_dn|) & \\
      \verb|ENGSFC_LW_up| & Surface longwave  upward   flux           & [J] \\
      \verb|ENGSFC_LW_dn| & Surface longwave  downward flux           & [J] \\
      \verb|ENGSFC_SW_up| & Surface shortwave upward   flux           & [J] \\
      \verb|ENGSFC_SW_dn| & Surface shortwave downward flux           & [J] \\
      \verb|ENGTOA_LW_up| & Top-of-atmosphere longwave  upward   flux & [J] \\
      \verb|ENGTOA_LW_dn| & Top-of-atmosphere longwave  downward flux & [J] \\
      \verb|ENGTOA_SW_up| & Top-of-atmosphere shortwave upward   flux & [J] \\
      \verb|ENGTOA_SW_dn| & Top-of-atmosphere shortwave downward flux & [J] \\
    \hline
  \end{tabularx}
\end{center}
\end{table}

For example, let the below setting for \namelist{MONITOR_ITEM} be added with \nmitem{MONITOR_STEP_INTERVAL} \verb|= 10| and \nmitem{MONITOR_USEDEVATION} \verb|= .false.|,

\editbox{
\verb|&MONITOR_ITEM  NAME="ENGK" /|\\
\verb|&MONITOR_ITEM  NAME="ENGP" /|\\
\verb|&MONITOR_ITEM  NAME="ENGI" /|\\
\verb|&MONITOR_ITEM  NAME="ENGT" /|\\
}

\noindent
The monitor file is output as follows;

\msgbox{
                   ENGT            ENGP            ENGK            ENGI            \\
STEP=      1 (MAIN)  1.18127707E+17  2.92701438E+16  2.40231436E+13  8.88335403E+16\\
STEP=     11 (MAIN)  1.18127712E+17  2.92701415E+16  2.40249223E+13  8.88335453E+16\\
STEP=     21 (MAIN)  1.18127711E+17  2.92701439E+16  2.40223566E+13  8.88335443E+16\\
STEP=     31 (MAIN)  1.18127710E+17  2.92701454E+16  2.40213480E+13  8.88335435E+16\\
STEP=     41 (MAIN)  1.18127710E+17  2.92701495E+16  2.40210662E+13  8.88335392E+16\\
STEP=     51 (MAIN)  1.18127710E+17  2.92701439E+16  2.40205575E+13  8.88335456E+16\\
STEP=     61 (MAIN)  1.18127711E+17  2.92701565E+16  2.40200252E+13  8.88335340E+16\\
STEP=     71 (MAIN)  1.18127711E+17  2.92701457E+16  2.40195927E+13  8.88335455E+16\\
STEP=     81 (MAIN)  1.18127710E+17  2.92701486E+16  2.40193679E+13  8.88335425E+16\\
STEP=     91 (MAIN)  1.18127710E+17  2.92701573E+16  2.40188095E+13  8.88335342E+16\\
STEP=    101 (MAIN)  1.18127710E+17  2.92701404E+16  2.40180752E+13  8.88335517E+16\\
}
