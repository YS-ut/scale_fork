\subsection{How to prepare user-defined data} \label{sec:userdata}

Some schemes requires other input data than initial and boundary data.
In order to use these schemes, the data are necessary be prepared before the simulation.
The input data must be \scalenetcdf file and have the grid information according to the experimental settings.
To prepare the input data, \scalerm provide a system to convert user-defined data to a \scalenetcdf file.
When \nmitem{CNVERT_USER} = \verb|.true.| in \namelist{PARAM_CONVERT}, the data conversion can be executed by \verb|scale-rm_init|.
The supported formats of the user-defined data are shown in Table \ref{tab:userdata_type}.
Since only one variable can be converted in a single execution, the execution must be repeated as for the number of the user-defined variables to be converted.
At this moment, the conversion is available only for spatially two dimensional data.


\begin{table}[tbh]
\begin{center}
\caption{Types of user-defined data supported in \scalelib}
\begin{tabularx}{150mm}{l|X} \hline
 \rowcolor[gray]{0.9} Data type & Description \\ \hline
 \verb|GrADS| & \grads binary data.    \\ \hline
 \verb|TILE|  & horizontally decomposed tiled binary data. \\ \hline
\end{tabularx}
\label{tab:userdata_type}
\end{center}
\end{table}


The configurations for the user-defined data are specified in \namelist{PARAM_CNVUSER} as follows:
\editboxtwo{
\verb|&PARAM_CNVUSER| & \\
\verb| CNVUSER_FILE_TYPE = '',        | & ; Type of the user-defined data. See Table \ref{tab:userdata_type}. \\
\verb| CNVUSER_INTERP_TYPE = 'LINEAR',| & ; Type for the horizontal interpolation. \\
\verb| CNVUSER_INTERP_LEVEL = 5,      | & ; Level of the interpolation. \\
\verb| CNVUSER_OUT_TITLE = 'SCALE-RM 2D Boundary',| & ; Title in the output file. \\
\verb| CNVUSER_OUT_VARNAME = '',      | & ; Variable name in the output file. \\
\verb| CNVUSER_OUT_VARDESC = '',      | & ; Description of the output variable. \\
\verb| CNVUSER_OUT_VARUNIT = '',      | & ; Unit of the output variable. \\
\verb| CNVUSER_OUT_DTYPE 'DEFAULT',   | & ; Data type of the output variable (DEFAULT,INT2,INT4,REAL4,REAL8). \\
\verb| CNVUSER_NSTEPS = 10,           | & ; The number of time steps in the data. \\
\verb| CNVUSER_OUT_DT = -1.0D0,       | & ; Time interval in second of the output variable. \\
\verb| CNVUSER_GrADS_FILENAME = '',   | & ; Name of the namelist for the \grads data. \\
\verb| CNVUSER_GrADS_VARNAME = '',    | & ; Name of the target variable in the namelist. \\
\verb| CNVUSER_GrADS_LATNAME = 'lat', | & ; Name of latitude in the namelist. \\
\verb| CNVUSER_GrADS_LONNAME = 'lon', | & ; Name of longitude in the namelist. \\
\verb| CNVUSER_TILE_DIR = '',         | & ; Directory name where the catalogue and tiled data files exist. \\
\verb| CNVUSER_TILE_CATALOGUE = '',   | & ; Name of the catalogue file. \\
\verb| CNVUSER_TILE_DLAT = 1.0D0,     | & ; Latitude interval in the tiled data. \\
\verb| CNVUSER_TILE_DLON = 1.0D0,     | & ; Longitude interval in the tiled data. \\
\verb| CNVUSER_TILE_DTYPE = 'REAL4',  | & ; Data type of the tiled data (INT2,INT4,REAL4,REAL8). \\
\verb|/|\\
}

The type and level for the horizontal interpolation is specified by \nmitem{CNVUSER_INTERP_TYPE} and \nmitem{CNVUSER_INTERP_LEVEL}, respectively.
See Section \ref{sec:datainput_grads} for the details for the interpolation.

\nmitem{CNVUSE_OUT_TITLE}, \nmitem{CNVUSE_OUT_VARNAME}, \nmitem{CNVUSE_OUT_VARDESC}, and \\
\nmitem{CNVUSE_OUT_VARUNIT} are settings for the title, variable name, variable description, and variable unit in the output file, respectively.
When the data type is \verb|GrADS| and the \nmitem{CNVUSE_OUT_VARNAME} is not set, \nmitem{CNVUSER_GrADS_VARNAME} is used as the output variable name.

If \nmitem{CNVUSER_OUT_DTYPE} is \verb|DEFAULT|, it is the same as the data type used in the simulation;
it is \verb|REAL8| as default, and \verb|REAL4| when \scalerm is compiled with \verb|SCALE_USE_SINGLEFP = "T"| (See Section \ref{sec:compile}).

The number of time steps contained in the user-defined data is specified by \nmitem{CNVUSER_NSTEPS}.
The initial time of the time coordinate in the output file is specified by \nmitem{TIME_STARTDATE} in \namelist{PARAM_TIME} (See Section \ref{sec:timeintiv}), and the time interval in second is specified by \nmitem{CNVUSER_OUT_DT}.



For the \verb|GrADS| type, another namelist file which contains structure of input file is necessary.
\nmitem{USERFILE_GrADS_FILENAME} is used to set the namelist file.
See Section \ref{sec:datainput_grads} for the details of the namelist file.
As the default, the variable name for latitude, and longitude are \verb|lat|, and \verb|lon|, respectively.
If the name is different from the default value, it can be specified with \nmitem{CNVUSER_GrADS_LATNAME}, and \nmitem{CNVUSER_GrADS_LONNAME}, respectively.
Two samples to convert \verb|GrADS|-type use-defined data are provided at \verb|scale/scale-rm/test/framework/cnvuser|
The following is an example of the namelist file.
\editbox{
\verb|#| \\
\verb|# Dimension| \\
\verb|#| \\
\verb|&GrADS_DIMS| \\
\verb| nx = 361,| \\
\verb| ny = 181,| \\
\verb| nz =   1,| \\
\verb|/| \\
\verb|#| \\
\verb|# Variables | \\
\verb|#| \\
\verb|&GrADS_ITEM name='lon', dtype='linear', swpoint=0.0D0, dd=1.0D0 /| \\
\verb|&GrADS_ITEM name='lat', dtype='linear', swpoint=-90.0D0, dd=1.0D0 /| \\
\verb|&GrADS_ITEM name='var', dtype='map', fname='fname_in', startrec=1, totalrec=1,| \textbackslash \ \\
~~~~~~~~~~~~~ \verb|bintype='int1', yrev=.true. /| \\
}



The \verb|TILE|-type data means the horizontally decomposed tiled data.
The individual tiled data should a direct access binary file.
The binary data type is specified by \nmitem{CNVUSER_TILE_DTYPE}.
The data must be structured on a latitude-longitude grid with equal intervals, specified by \nmitem{CNVUSER_TILE_DLAT} and \nmitem{CNVUSER_TILE_DLON} for latitude and longitude, respectively.
A catalogue file is required; the catalogue contains information of the name of individual tiled data files and the area which the tiled data covers.
The name of the catalogue file is specified by \nmitem{CNVUSER_TILE_CATALOGUE}.
The catalogue and tiled data files are located at the directory specified by \nmitem{CNVUSER_TILE_DIR}.
Each line of the catalogue file describes the serial number, the southernmost and the northernmost latitudes, the westernmost and the easternmost longitudes, and the name of the tiled data.
The following is an example of the catalogue file for the global data decomposed into total 4 (2 x 2) files.
\editboxtwo{
\verb|001 -90.0  0.0 -180.0   0.0 TILE_sw.grd| & ; latitude: -90--0, longitude: -180--0, name: \verb|TILE_sw.grd| \\
\verb|002 -90.0  0.0    0.0 180.0 TILE_se.grd| & ; latitude: -90--0, longitude: 0--180,  name: \verb|TILE_se.grd| \\
\verb|003   0.0 90.0 -180.0   0.0 TILE_nw.grd| & ; latitude: 0--90,  longitude: -180--0, name: \verb|TILE_nw.grd| \\
\verb|004   0.0 00.0    0.0 180.0 TILE_ne.grd| & ; latitude: 0--90,  longitude: 0--180,  name: \verb|TILE_ne.grd| \\
}

