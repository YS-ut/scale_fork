%-------------------------------------------------------%
\section{Preparations for input data (boundary data)} \label{sec:tutorial_real_data}
%-------------------------------------------------------%

When a realistic atmospheric experiment is conducted, boundary data need to be provided to \scalerm. Table \ref{tab:real_bnd} shows the items for external input data to create boundary data. The variables denoted by {\color{blue}blue character} in this table  are always needed, whereas the others (black character) are optional.
\begin{table}[tb]
\begin{center}
  \caption{Items of external input data for real atmospheric experiments}
  \label{tab:real_bnd}
  \begin{tabularx}{150mm}{llX} \hline
    \rowcolor[gray]{0.9}
    \multicolumn{3}{l}{Data to create the topography and land-use data (in general, in-situ data)}\\ \hline
    & \multicolumn{2}{l}{\color{blue}{Altitude data}}\\
    & \multicolumn{2}{l}{\color{blue}{Land-use classification data}}\\ \hline
    \rowcolor[gray]{0.9}
    \multicolumn{3}{l}{data to create the initial and boundary data for \scalerm (in general, GCM data)}\\ \hline
    &  \multicolumn{2}{l}{\color{blue}{Information for latitude and longitude of the parent model}}\\
    &  \multicolumn{2}{l}{--- 3D atmospheric data ---}\\
    &  & \multicolumn{1}{l}{\color{blue}{zonal and meridional winds, temperature, specific humidity (relative humidity),}} \\
    &  & \multicolumn{1}{l}{\color{blue}{pressure, geopotential height}} \\
    &  \multicolumn{2}{l}{--- 2D atmospheric data---}\\
    &  & \multicolumn{1}{l}{sea-level pressure, surface pressure, zonal and meridional wind at 10m,} \\
    &  &\multicolumn{1}{l}{temperature and specific humidity (relative humidity) at 2m} \\
    &  \multicolumn{2}{l}{--- 2D land data ---}\\
    & &  \multicolumn{1}{l}{land and sea map in the parent model}\\
    & &  \multicolumn{1}{l}{\color{blue}{surface skin temperature}}\\
    & &  \multicolumn{1}{l}{{\color{blue}{information of depth of soil data in the parent model, soil temperature,}}}\\
    & &  \multicolumn{1}{l}{{{soil moisture(volume content or degree of saturation)}}}\\
    &  \multicolumn{2}{l}{--- 2D ocean data at the surface ---}\\
    & &  \multicolumn{1}{l}{\color{blue}{sea surface temperature (omitted if skin temperature is also used for SST)}}\\ \hline
  \end{tabularx}
\end{center}
\end{table}


\subsubsection{Topography and land-use classification data}

The external topography and land-use classification data are used to obtain
the altitude, the ocean to land ratio, the lake ratio, urban covering and vegetation rates, and the classification of land use at every grid point.
In order to allow users to calculate at any areas over the globe,
the altitude data GTOPO30 from the USGS (U.S. Geological Survey)
and land-use classification data from the GLCCv2 are provided in \scalerm.
These files have already been formatted for \scalerm.
\begin{enumerate}
\item Downloading database\\
Obtain the data for altitude and land-use classification formatted for \scalerm  from \url{https://scale.riken.jp/download/scale_database.tar.gz}, and extract them to any directory:
\begin{verbatim}
  $ tar -zxvf scale_database.tar.gz
  $ ls
    scale_database/topo/    <- altitude data
    scale_database/landuse/ <- land use classification data
\end{verbatim}
\item Setting of the path\\
To prepare the files used in the realistic atmospheric experiment,
the ``making tool for the complete settings of the experiment" is available.
To use the tool, set the directory name including above database
to an environment variable \verb|SCALE_DB| (hereinafter, it denoted as \verb|SCALE_DB|):
\begin{verbatim}
  $ export SCALE_DB="${path_to_directory_of_scale_database}/scale_database"
\end{verbatim}
where \verb|${path_to_directory_of_scale_database}| is the directory name
which the \verb|tar| file including topography and land-use database is extracted in.
%Thus, you should execute above command after substituting the portion of \verb|${path_to_directory_of_scale_database}|.
For example, if the absolute path where you expanded \verb|scale_database.tar.gz|
was \verb|/home/user/scale|, you need to set as follows.
\begin{verbatim}
  $ export SCALE_DB="/home/user/scale/scale_database"
\end{verbatim}

\end{enumerate}


\subsubsection{Data for atmosphere, land, and sea surface temperature}

The initial and boundary data are readable
when they are converted into four-byte binary (\grads form. Hereinafter, it is denoted by ``binary data.'').
The users prepare the ``binary'' data by themselves as mentioned above.
However,
the programs to prepare the ``binary'' data are provided in the directory
\verb|${Tutorial_DIR}/real/tools/| for the execution of this tutorial.
A procedure is explained below.
Note that it is assumed that the installation
of \verb|wgrib|\footnote{\url{http://www.cpc.ncep.noaa.gov/products/wesley/wgrib.html}} is complete
to use NCEP FNL (Final) Operational Global Analysis data with grib1 format.


\begin{enumerate}
\item Downloading data
Download 12-hour data from July 14, 2007 1800UTC from the NCAR website \url{http://rda.ucar.edu/datasets/ds083.2/}
and place them in the directory \\
\verb|${Tutorial_DIR}/real/tools/FNL_input/grib1/2007|.
The following is the data list, formatted by grib1:
\begin{verbatim}
  fnl_20070714_18_00.grib1
  fnl_20070715_00_00.grib1
\end{verbatim}
\item Conversion of data from grib to binary format\\
Execute \verb|convert_FNL-grib2grads.sh| in the directory \verb|${Tutorial_DIR}/real/tools/|:
\begin{verbatim}
 $ cd ${Tutorial_DIR}/real/tools/
 $ sh convert_FNL-grib2grads.sh 2007071418 2007071500 FNL_input FNL_output
\end{verbatim}
The following files are found if it is successful:
\begin{verbatim}
 $ ls FNL_output/*/*
    FNL_output/200707/FNL_ATM_2007071418.grd
    FNL_output/200707/FNL_ATM_2007071500.grd
    FNL_output/200707/FNL_LND_2007071418.grd
    FNL_output/200707/FNL_LND_2007071500.grd
    FNL_output/200707/FNL_SFC_2007071418.grd
    FNL_output/200707/FNL_SFC_2007071500.grd
\end{verbatim}
If the generation of the intended files is failed because of changing data format and variable names in NCEP-FNL data,
\verb|convert_FNL-grib2grads.sh| should be fixed according to the used NCEP-FNL data.
\end{enumerate}
