\section{Setting the history file and output variable} \label{sec:output}
%====================================================================================

A history file and output variables are configured at \namelist{PARAM_HISTORY} and \namelist{HISTITEM} in \verb|run.conf|.
The default format of the history file is configured at \namelist{PARAM_HISTORY}.
\editboxtwo{
\verb|&PARAM_HISTORY| & \\
\verb|  HISTORY_DEFAULT_BASENAME  = "history_d01",| & ; Header name  of files\\
\verb|  HISTORY_DEFAULT_TINTERVAL = 3600.0,|        & ; Output interval of history data \\
\verb|  HISTORY_DEFAULT_TUNIT     = "SEC",|         & ; Unit of \verb|HISTORY_DEFAULT_TINTERVAL| \\
\verb|  HISTORY_DEFAULT_TAVERAGE  = .false.,|       & ; \verb|.false.|: Snapshot value, \verb|.true.|: Average value \\
\verb|  HISTORY_DEFAULT_ZCOORD    = "model",|       & ; Types of vertical interpolation.\\
                                                    & ~ \verb|"model"|: model level (not interpolated)\\
                                                    & ~ \verb|"z"    |: Interpolated value at absolute height\\
                                                    & ~ \verb|"pressure"|: Interpolated value at pressure level\\
\verb|  HISTORY_DEFAULT_DATATYPE  = "REAL4",|       & ; Output data type: \verb|REAL4|, \verb|REAL8|\\
\verb|  HISTORY_OUTPUT_STEP0      = .true.,|        & ; Whether or not data are output at initial time (t=0).\\
                                                    & ~ \verb|.true.|: Output, \verb|.false.|: Not output.\\
\verb|/| & \\
}


\nmitem{HISTORY_DEFAULT_TINTERVAL} is time interval of history output
and its unit is defined by \nmitem{HISTORY_DEFAULT_TUNIT}.
The unit can be selected from \\
\verb|"MSEC", "msec", "SEC", "sec", "s", "MIN", "min", "HOUR", "hour", "h", "DAY", "day"|.\\
%
The time interval of history output must be the same as or a factor of the time interval of its related scheme.
When you want to disable checking this consistency, please set \nmitem{History_ERROR_PUTMISS} to \verb|.false.|.


When the average value output is selected as \verb|HISTORY_DEFAULT_TAVERAGE = .true.|,
the history data averaged over the last period, given as \verb|HISTORY_DEFAULT_TINTERVAL|, is output.

When the absolute height level is selected as \nmitem{HISTORY_DEFAULT_ZCOORD} \verb|= "z"|,
the number of levels in output data is the same as that of the model level.
The height at each level is the same as that of the model level in a grid cell with no topography.
When the pressure level is selected as \nmitem{HISTORY_DEFAULT_ZCOORD} \verb|= "pressure"|,
the configurations of \nmitem{HIST_PRES_nlayer} and \nmitem{HIST_PRES} in \namelist{PARAM_HIST} are required.

When \nmitem{HIST_BND} in \namelist{PARAM_HIST} is \verb|.true.|,
the data in halo, which is located outside of a target domain, is also output except in the case of periodic boundary condition.
The setting of \nmitem{HIST_BND} is applied to all output variables.\\

\editboxtwo{
\verb|&PARAM_HIST| & \\
\verb|  HIST_PRES_nlayer   = -1,         | & ; Number of levels\\
                                           & ~ (OPTION for interpolation at pressure level)\\
\verb|  HIST_PRES          = 0.0,        | & ; Pressure levels in order from lower to upper level. The unit is [hPa].\\
                                           & ~ (OPTION for interpolation at pressure level)\\
\verb|  HIST_BND           = .false.     | & ; Whether output data in halo or not.\\
                                           & ~ \verb|.true.|: Output, \verb|.false.|: Not output.\\
\verb|/| & \\
}

The output variables are set by adding \namelist{HISTITEM}.
The output format follows the default setting specified at \namelist{PARAM_HISTORY}.
By adding namelists with ``(OPTION),''
the format for a specific variable can be changed from the default setting.


\editboxtwo{
\verb|&HISTITEM| &\\
\verb| ITEM     = "RAIN",    | &  Variable name. List of variables is shown in Appendix \ref{achap:namelist}.\\
\verb| BASENAME = "rain_d01",| &  (OPTION) same as \verb|HISTORY_DEFAULT_BASENAME|\\
\verb| TINTERVAL= 600.0,     | &  (OPTION) same as \verb|HISTORY_DEFAULT_TINTERVAL|\\
\verb| TUNIT    = "SEC",     | &  (OPTION) same as \verb|HISTORY_DEFAULT_TINTERVAL|\\
\verb| TAVERAGE = .true.,    | &  (OPTION) same as \verb|HISTORY_DEFAULT_TAVERAGE|\\
\verb| ZCOORD   = "model",   | &  (OPTION) same as \verb|HISTORY_DEFAULT_ZCOORD|\\
\verb| DATATYPE = "REAL4",   | &  (OPTION) same as \verb|HISTORY_DEFAULT_DATATYPE|\\
\verb|/| & \\
}

Namelists with ``(OPTION)'' are applied only to the variable \nmitem{ITEM}. If the default setting is used for the variable, the namelist with ``(OPTION)'' can be omitted. For example, let the below setting for \namelist{HISTITEM} be added, maintaining the above setting of \namelist{PARAM_HISTORY}. The snapshot values of \verb|T, U, V| are stored as four-byte real values at an interval of 3600 s in the file \verb|history_d01.xxxxxx.nc|, whereas the value of \verb|RAIN| averaged over the last 600 seconds is stored each time in the file.

\editbox{
\verb|&HISTITEM  ITEM = "T" /|\\
\verb|&HISTITEM  ITEM = "U" /|\\
\verb|&HISTITEM  ITEM = "V" /|\\
\verb|&HISTITEM  ITEM = "RAIN",  TINTERVAL = 600.0, TAVERAGE = .true. /|\\
}
