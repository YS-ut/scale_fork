\section{Setting the history file and output variable} \label{sec:output}
%====================================================================================

A history file and output variables are configured at \namelist{PARAM_HISTORY} and \namelist{HISTITEM} in \verb|run.conf|.
The default format of the history file is configured at \namelist{PARAM_HISTORY}.
\editboxtwo{
\verb|&PARAM_HISTORY| & \\
\verb|  HISTORY_DEFAULT_BASENAME  = "history_d01",| & ; Header name  of files\\
\verb|  HISTORY_DEFAULT_TINTERVAL = 3600.0,|        & ; Output interval of history data \\
\verb|  HISTORY_DEFAULT_TUNIT     = "SEC",|         & ; Unit of \verb|HISTORY_DEFAULT_TINTERVAL| \\
\verb|  HISTORY_DEFAULT_TAVERAGE  = .false.,|       & ; \verb|.false.|: snapshot value, \verb|.true.|: Average value \\
\verb|  HISTORY_DEFAULT_ZCOORD    = "model",|       & ; Types of vertical interpolation.\\
                                                    & ~ native, AGL, MDL: model level (not interpolated)\\
                                                    & ~ z, Z, HGT: Interpolated value at absolute height\\
                                                    & ~ pz, P, PRES: Interpolated value at pressure level\\
\verb|  HISTORY_DEFAULT_DATATYPE  = "REAL4",|       & ; Output data type: \verb|REAL4|, \verb|REAL8|\\
\verb|  HISTORY_OUTPUT_STEP0      = .true.,|        & ; Whether or not data are output at initial time (t=0).\\
                                                    & ~ \verb|.true.|: output, \verb|.false.|: Not output.\\
\verb|  History_PRES_VALUE        = -1,|            & ; Number of levels\\
                                                    & ~ (OPTION for interpolation at pressure level)\\
\verb|  History_PRES_NLAYER       = 0.0*100,|       & ; Pressure levels in order from lower to upper level. The unit is hPa.\\
                                                    & ~ (OPTION for interpolation at pressure level)\\
\verb|/| & \\
}

The unit for \nmitem{HISTORY_DEFAULT_TUNIT} can be selected from \\
\verb|"MSEC", "msec", "SEC", "sec", "s", "MIN", "min", "HOUR", "hour", "h", "DAY", "day"|.\\
When the average value output is selected as \verb|HISTORY_DEFAULT_TAVERAGE = .true.|, the history data averaged over the last period, given as \verb|HISTORY_DEFAULT_TINTERVAL|, is output.


The output variables are set by adding \namelist{HISTITEM}. The output format follows the default setting specified at \namelist{PARAM_HISTORY}. By adding namelists with ``(OPTION),'' the format for a specific variable can be changed from the default setting.


\editboxtwo{
\verb|&HISTITEM| &\\
\verb| ITEM     = "RAIN",    | &  Variable name. List of variables is shown in Appendix \ref{achap:namelist}.\\
\verb| BASENAME = "rain_d01",| &  (OPTION) same as \verb|HISTORY_DEFAULT_BASENAME|\\
\verb| TINTERVAL= 600.0,     | &  (OPTION) same as \verb|HISTORY_DEFAULT_TINTERVAL|\\
\verb| TUNIT    = "SEC",     | &  (OPTION) same as \verb|HISTORY_DEFAULT_TINTERVAL|\\
\verb| TAVERAGE = .true.,    | &  (OPTION) same as \verb|HISTORY_DEFAULT_TAVERAGE|\\
\verb| ZCOORD   = "model",   | &  (OPTION) same as \verb|HISTORY_DEFAULT_ZCOORD|\\
\verb| DATATYPE = "REAL4",   | &  (OPTION) same as \verb|HISTORY_DEFAULT_DATATYPE|\\
\verb|/| & \\
}

Namelists with ``(OPTION)'' are applied only to the variable \nmitem{ITEM}. If the default setting is used for the variable, the namelist with ``(OPTION)'' can be omitted. For example, let the below setting for \namelist{HISTITEM} be added, maintaining the above setting of \namelist{PARAM_HISTORY}. The snapshot values of \verb|T, U, V| are stored as four-byte real values at an interval of 3600 s in the file \verb|history_d01.xxxxxx.nc|, whereas the value of \verb|RAIN| averaged over the last 600 seconds is stored each time in the file. 

\editbox{
\verb|&HISTITEM  ITEM = "T" /|\\
\verb|&HISTITEM  ITEM = "U" /|\\
\verb|&HISTITEM  ITEM = "V" /|\\
\verb|&HISTITEM  ITEM = "RAIN",  TINTERVAL = 600.0, TAVERAGE = .true. /|\\
}
