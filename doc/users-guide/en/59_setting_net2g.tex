\section{Post-processing} \label{sec:net2g}
%====================================================================================

In this section, the post-processing tool \verb|net2g| is described and its function explained. \verb|net2g| generates the binary format that is readable in \grads by combining history files \verb|history.***.nc| with the \netcdf format.
The conversion of all files to one enable \grads to draw the result.
Thus, formatted data can be easily analyzed, even by a FORTRAN serial program.
Since \verb|net2g| can also be executed as an MPI parallel program,
the elapsed time can be to a greater extent than in the case of serial processing. The following functions are also available
\begin{itemize}
 \item Interpolate data from the model level to arbitrary height coordinates or pressure coordinates.
 \item Output the mean, maximum, and minimum values of the vertical integration of 3D variables.
 \item Output multiple files for 3D variables, layer by layer.
 \item Output multiple files, time-step by time-step.
\end{itemize}


It is convenient to divide files layer by layer or time-step by time-step because such data can be easily handled, particularly large-scale computational data. Refer to Section \ref{sec:source_net2g} for a guide to installing \verb|net2g|.
Note that the current version of \verb|net2g| has the following limitations:
\begin{itemize}
\item The number of MPI processes in \verb|net2g| must be a divisor of the number of MPI processes at execution in \scalerm.
\item The history output format at the execution of \scalerm must be output with   \nmitem{HIST_BND} $=$ \verb|.false. | in \namelist{PARAM_HIST}.
\item Two- and three-dimensional data cannot be simultaneously converted.
\item Only history data can be converted.
\end{itemize}
Beware that if too large a number of MPI processes is set, computational performance is affected. Since most instruction in \verb|net2g| are for the input/output of data, it is possible that the number of requests for storage access becomes too large. In particular, this situation should be attended to if the conversion of large-scale computation results is carried out on small-scale machines.

If MPI parallel is used, \verb|net2g| is executed as follows:
\begin{verbatim}
 $ mpirun  -n  [number of process]  ./net2g  net2g.conf
\end{verbatim}
The last argument \verb|net2g.conf| is the configuration file for \verb|net2g|.
On the contrary, if \verb|net2g| is compiled as a single process version,
\begin{verbatim}
 $ ./net2g  net2g.conf
\end{verbatim}

If only the following message without error can be found, the execution is concluded normally:
\msgbox{
\verb|+++ MPI COMM: Corrective Finalize| \\
}

The following explains how to describe the configuration files in the case of 2D or 3D variables
by using sample files \verb|net2g.3d.conf| and \verb|net2g.2d.conf| in the directory\\
\texttt{scale-\version/scale-rm/util/netcdf2grads\_h/}.
In this section, only the major issues pertaining to its use are treated.
Refer to the sample files \verb|net2g.all.conf| in \\
\texttt{scale-\version/scale-rm/util/netcdf2grads\_h/} for the other options.

\subsubsection{The sample configuration file: Conversion of 3D variables}
%------------------------------------------------------

\editbox{
\verb|&LOGOUT| \\
\verb| LOG_BASENAME   = "LOG_d01_3d",| \\
\verb| LOG_ALL_OUTPUT = .false.,| \\
\verb|/| \\
 \\
\verb|&INFO| \\
\verb| TIME_STARTDATE = 2000, 1, 1, 0, 0, 0,| \\
\verb| START_TSTEP    = 1,| \\
\verb| END_TSTEP      = 25,| \\
\verb| DOMAIN_NUM     = 1,| \\
\verb| CONFFILE       = "../run/run.d01.conf",| \\
\verb| IDIR           = "../run",| \\
\verb| Z_LEV_TYPE     = "plev",| \\
\verb| MAPPROJ_ctl    = .true. | \\
\verb|/| \\
 \\
\verb|&VARI| \\
\verb| VNAME       = "PT","U","V","W","QHYD",| \\
\verb| TARGET_ZLEV = 850,500,200,| \\
\verb|/| \\
}
The above example shows a configuration in the case where 3D variables in a domain are converted into pressure and/or height coordinates. The setting items are as follows:
\begin{itemize}
 \item \namelist{LOGOUT} (The following items are not required.)
 \begin{itemize}
  \item \nmitem{LOG_BASENAME}:If the default LOG file name \verb|LOG| is changed, this item is specified.
  \item \nmitem{LOG_ALL_OUTPUT}:If processes other than the 0th process are output in the LOG files,
\verb|".true."| is assigned to this item.  The default value is \verb|."false".|
 \end{itemize}
 \item \namelist{INFO}
 \begin{itemize}
  \item \nmitem{TIME_STARTDATE} : The start date and time of converted \netcdf data are specified.
  \item \nmitem{START_TSTEP} : The start time step of converted \netcdf data is specified.
    If several first steps are skipped, the appropriate values are assigned to this item.
    The default value is 1.
  \item \nmitem{END_TSTEP}:The end of the time step of converted \netcdf data is specified.
    It is required in all cases.
  \item \nmitem{DOMAIN_NUM}:The domain number is specified. The default value is 1.
  \item \nmitem{CONFFILE}:The path of \verb|run.***.conf| at the execution of \scalerm is specified, including the file name.
  \item \nmitem{IDIR}:The path of history files is specified.
  \item \nmitem{Z_LEV_TYPE}:The type of vertical data conversion is specified. \verb|"original"| represents the model surface,
    \verb|"plev"| the interpolation to the pressure surface,
    and \verb|"zlev"| that to the height surface.
    If \verb|"anal"| is specified, the result with a simple analysis is output. The details for \verb|"anal"| is explained subsequently.
    The default value is \verb|"plev"|.
  \item \nmitem{MAPPROJ_ctl}: It indicates whether the ``ctl'' file corresponding to the map projection using \verb|pdef| is output. Currently, this option is available only for \verb|LC| coordinate.
 \end{itemize}
 \item \namelist{VARI}
 \begin{itemize}
  \item \nmitem{VNAME}:The variables converted are specified.
    As default, \verb|"PT"|,\verb|"PRES"|,\verb|"U"|,\verb|"V"|, \verb|"W"|,\verb|"QHYD"| are given.
  \item \nmitem{TARGET_ZLEV}: The heights corresponding to \nmitem{Z_LEV_TYPE} are specified.
    In the case of \verb|"plev"|, the unit is [$hPa$],
    in the case of \verb|"zlev"|, the unit is [$m$], and
    in the case of \verb|"original"|, the grid numbers are specified.
    As default, 14 layers (1000 hPa, 975 hPa, 950 hPa, 925 hPa, 900 hPa, 850 hPa, 800 hPa, 700 hPa, 600 hPa, 500 hPa, 400 hPa,
        300 hPa, 250 hPa, and 200 hPa ) are given.
 \end{itemize}
\end{itemize}

\subsubsection{Example of configuration file: Vertical integration of 3D variable data}
%------------------------------------------------------
The description below presents an excerpt of a configuration file along with a simple analysis.
The other item settings are the same as in the previous configuration.
\editbox{
\verb|&INFO| \\
\verb|   〜  ...  〜  |\\
\verb| Z_LEV_TYPE  = "anal",| \\
\verb| ZCOUNT      = 1,| \\
\verb|/| \\
 \\
\verb|&ANAL| \\
\verb| ANALYSIS    = "sum",| \\
\verb|/| \\
 \\
\verb|&VARI| \\
\verb| VNAME       = "QC","QI","QG",| \\
\verb|/| \\
}
If \nmitem{Z_LEV_TYPE}$=$\verb|"anal"|, simple analysis is applied to the 3D variable.
This setting enables \namelist{ANAL}.
The specification of \nmitem{TARGET_ZLEV} in \namelist{VARI} is disabled,
and \textcolor{blue}{\nmitem{ZCOUNT} in \namelist{INFO} is necessarily given as "1"}
because of the output of 2D data.
\begin{itemize}
 \item \namelist{ANAL}
 \begin{itemize}
  \item \nmitem{ANALYSIS}:The type of vertical simple analysis is specified. \verb|"max"| and \verb|"min"|
    represent the maximum and minimum value outputs in the vertical column, respectively, whereas \verb|"sum"| and \verb|"ave"| represent
    the vertical integration and the vertical average outputs, respectively. The default value is \verb|"ave"|.
 \end{itemize}
\end{itemize}

\subsubsection{Example of configuration file: Conversion of 2D variables}
%------------------------------------------------------
The example below shows a configuration in the case where 2D variables are converted. Because of the output of 2D data,
\textcolor{blue}{\nmitem{ZCOUNT} in \namelist{INFO} is necessarily given as "1."}
\editbox{
\verb|&LOGOUT| \\
\verb| LOG_BASENAME   = "LOG_d01_2d",| \\
\verb|/| \\
 \\
\verb|&INFO| \\
\verb| TIME_STARTDATE = 2000, 1, 1, 0, 0, 0,| \\
\verb| START_TSTEP = 1,| \\
\verb| END_TSTEP   = 25,| \\
\verb| DOMAIN_NUM  = 1,| \\
\verb| CONFFILE    = "../run/run.d01.conf",| \\
\verb| IDIR        = "../run",| \\
\verb| ZCOUNT      = 1,| \\
\verb| MAPPROJ_ctl    = .true.|\\
\verb|/| \\
 \\
\verb|&VARI| \\
\verb| VNAME       = "T2","MSLP","PREC"| \\
\verb|/| \\
}

\subsubsection{Example of configuration file change: Conversion of irregular data in time}
%------------------------------------------------------

As described in Section \ref{sec:output},
although the output interval is basically defined by \\ \nmitem{HISTORY_DEFAULT_TINTERVAL},
it is possible to change the output interval of particular variables
by giving a value different from \nmitem{HISTORY_DEFAULT_TINTERVAL}  to \nmitem{TINTERVAL} in \namelist{HISTITEM}.
\verb|net2g| supports the data conversion of these variables with different output interval by setting \namelist{EXTRA}.
The below example shows the namelists, which should be added or modified
for the case described in the last paragraph of Section \ref{sec:output}
(only 2D data \verb|"RAIN"| are output at a time interval of 600 s).

The history file can handle data with multiple different time intervals.
Since \verb|net2g| does not support the simultaneous conversion of variables with different output intervals,
it is required to execute \verb|net2g| separately for these variables.

\editbox{
\verb|&EXTRA| \\
\verb| EXTRA_TINTERVAL = 600.0,| \\
\verb| EXTRA_TUNIT     = "SEC",| \\
\verb|/| \\
 \\
\verb|&VARI| \\
\verb| VNAME = "RAIN",| \\
\verb|/| \\
}

\subsubsection{Note for longitude and latitude}
The variable names of longitude and latitude in \scale are ``lon'' and ``lat'', respectively.
These names are used as reserved words in \grads,
so variable names in output from \verb|net2g| are replaced to ``long'' for longitude and ``lati'' for latitude.


\subsubsection{Note for execution on supercomputer}
When a simulation is conducted on a supercomputer such as the K Computer, many output files are generated, where the size of each file is large. In such cases, the local disk space is often not large enough to store them,  and post-processing may take a long time. In this case, both a simulation by \scalerm and its post-processing on the same supercomputer are recommended. On the K Computer, \verb|net2g| can be compiled by the ``make'' command  if the environmental variable is appropriately set as described in Section \ref{subsec:evniromnet}.

