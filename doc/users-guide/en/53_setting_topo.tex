\section{Setting the topography} \label{subsec:basic_usel_topo}
%-----------------------------------------------------------------------

\scalerm employs the terrain-following coordinates to represent topography.
In these coordinates, the bottom face of the lowest grid is given such that it can follow the surface altitude. The allowable maximum angle of inclination, $\theta_{\max}$[radian] is calculated as follows: 
\begin{eqnarray}
  && \theta_{\max} = \arctan( \mathrm{RATIO} \times \mathrm{DZ}/\mathrm{DX} )\nonumber,
\end{eqnarray}
where $\mathrm{DZ}$ and $\mathrm{DX}$ are the horizontal and vertical grid intervals, respectively.  As shown in the above equation, $\theta_{\max}$ depends on spatial resolution. 
In \scalerm, the default value of $\mathrm{RATIO}$ is 1.0.
If $\mathrm{RATIO}$ is greater than unity, the fine topography is expressed, and vice versa. Note that at $\mathrm{RATIO} >1$, the risk of numerical instability increases.

The program \verb|scale-rm_pp| converts external topography data into \scalelib format.  The detailed configurations are specified in \namelist{PARAM_CNVTOPO} in configuration file \verb|pp.conf|. An example is as follows:
\editboxtwo{
\verb|&PARAM_CNVTOPO  |                  & \\
 \verb|CNVTOPO_name                  = "GTOPO30", | & ; Data name used\\
 \verb|CNVTOPO_smooth_maxslope_ratio = 1.0,       | & ; Allowable maximum ratio of inclination to $\mathrm{DZ}$/$\mathrm{DX}$ \\
 \verb|CNVTOPO_smooth_local          = .true.,    | & ; 
Whether smoothened or not, only grids whose angles of inclination exceed the maximum value \\
 \verb|CNVTOPO_copy_parent            = .false.,   | & ; Whether the topography in the buffer region is copied \\
\verb|/|\\
}
\scalerm supports GTOPO30 and DEM50M provided by the Geospatial Information Authority of Japan as external input topography data.

The topographical data is formulated as the area-weighting mean of grid spacing.
This conversion calculates the difference in altitude between them using their neighboring grids and inclinations.
In case $\theta_{\max}$ exceeds the allowable maximum angle of inclination, \\
i.e., \nmitem{CNVTOPO_smooth_maxslope_ratio}, the topography is smoothened by a Laplacian filter with several iterations until the angle is below the allowable maximum angle.
Smoothing is selectively applied either only to grids whose angles of inclination exceed the maximum value, or to the entire domain.
Since the former saves the sharp structure of the topography within the allowable maximum angle, it is selected if the representation of fine structures is desired.
The Gaussian filter is also selected as a smoothing filter. In this case, the topography is smoother than when the Laplacian filter is used.

\nmitem{CNVTOPO_copy_parent} is the item used for the nesting computation.
In general, the topography in the child domain is finer than in the parent domain due to higher spatial resolution.
At this time, problems often occers due to 
an inconsistency between 
the atmospheric data in the buffer region of the child domain and that in the parent domain.
To avoid this problem, the topography of the parent domain can be copied to the buffer region of the child domain by specifying \nmitem{CNVTOPO_copy_parent}$=$\verb|.true.| If there is no parent domain, \nmitem{CNVTOPO_copy_parent} must be \verb|.false.|. Section \ref{subsec:nest_topo} provides a more detailed explanation of the case that involves the use of \nmitem{CNVTOPO_copy_parent}.
