%###############################################################################

This document is an additional volume of ``SCALE USERS GUIDE''.
This document includes a description of how to use SCALE-Global Model(GM) and a tutrial based on DCMIP2016.
The SCALE-GM is global atmospheric model constructed by using SCALE library.
The overview of SCALE library is described in Chapter 1 of ``SCALE USERS GUIDE''.

In current version, SCALE-GM supports ideal simulations for the dynamical core test.
SCALE-GM is originated from a dynamical core, developed for Nonhydrostatic ICosahedral Atmospheric Model (NICAM).
%The development of NICAM with full physics has been co-developed mainly by
%the Japan Agency for Marine-Earth Science and Technology (JAMSTEC), Atmosphere
%and Ocean Research Institute (AORI) at The University of Tokyo, and RIKEN / Advanced
%Institute for Computational Science (AICS).
A reference paper for NICAM is
Tomita and Satoh (2004), Satoh et al. (2008).
%See also NICAM.jp (\url{http://nicam.jp/}).

\section{Note}
%-------------------------------------------------------------------------------
 \begin{itemize}
   \item Explanations are based on bash environment below.
         In the tcsh environment, use "setenv" command instead of "export".
         (e.g. \verb|> setenv SCALE_SYS "Linux64-gnu-ompi"|)
   \item A symbol of "\verb|>|" means execution of commands in the console.
   \item Gothic means output from standard output.
%   \item \$\{TOP\}   means \verb|scale/scale-gm|
%   \item \$\{ROOT\}  means \verb|scale|
 \end{itemize}

