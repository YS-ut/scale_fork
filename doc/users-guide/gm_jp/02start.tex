\section{クイックスタート}
%###############################################################################

\subsection{実行の準備}
%-------------------------------------------------------------------------------
ここではSCALE-GMのコンパイルについて簡潔に説明する。依存関係のあるライブラリの準備や
コンパイル前に必要な環境変数の設定については、``SCALE USERS GUIDE''の第2章の
2.1節、2.2節を参照されたい。
SCALE-GMに必要とされるライブラリは、SCALE-RMと同じく、HDF5、NetCDF、およびMPIである。

\subsection{コンパイル}
%-------------------------------------------------------------------------------
ソースコードの取得方法については ``SCALE USERS GUIDE'' の2.3.1節、
マシン環境やコンパイラ等を指定する環境変数(Makedef)の設定については
 ``SCALE USERS GUIDE'' の2.3.2節を参照されたい。
Makedefファイルは、SCALE-RMと共通のファイルを使用する。
このMakedefファイルの設定をコンパイル前に忘れないようにすること。

\textcolor{red}{[要検討C: ここから]}

DCMIP2016の理想実験を実行するために、DCMIPの組織委員が公開しているソースコードが必要になる。
現行のSCALE-GMのコンパイルには、この追加ソースコードが必須となっている。

\url{https://github.com/ClimateGlobalChange/DCMIP2016}
から、zipパッケージとしてダウンロードするか、gitコマンドを使用して``DCMIP2016''パッケージをダウンロードする。
ダウンロードしたパッケージを展開して、DCMIP2016/interface の下にある
fortranプログラムをすべて\verb|${TOP}/src/atmos/dcmip|へコピーする。
これは、DCMIP2016パッケージを\verb|${ROOT}|下へ展開したことを想定した場合、次のコマンドのようになる。
\begin{verbatim}
  > cp ${ROOT}/DCMIP2016/interface/*.f90 ${TOP}/src/atmos/dcmip/
\end{verbatim}

\textcolor{red}{[ここまで]}

これらの準備が整えば、srcディレクトリへ移動する。
SCALE-GMでは、\verb|${TOP}/src|の下でコンパイルを行い、\verb|${TOP}/bin|の下に
各種実行バイナリが生成され、このバイナリファイルを用いて各種実験や後処理を行う。
各テストケースディレクトリでもmakeできるが、作業効率やバイナリの取り違えを防ぐために、
ここで説明する方法を推奨する。

\begin{verbatim}
  > cd ${TOP}/src
\end{verbatim}

\noindent makeコマンドを用いてコンパイルを行う。
\begin{verbatim}
  > make -j 4
\end{verbatim}
コンパイルが正常に終了したならば、以下のバイナリが\verb|${ROOT}/bin|の下に生成される。
 \begin{itemize}
   \item \verb|scale-gm| (SCALE-GM本体の実行バイナリ)
   \item \verb|gm_fio_cat| (fioフォーマットのcatコマンドツール)
   \item \verb|gm_fio_dump| (fioフォーマットのファイルをダンプするツール)
   \item \verb|gm_fio_ico2ll| (fioフォーマットの二十面体格子データをLatLon格子データに変換するツール)
   \item \verb|gm_fio_sel| (fioフォーマットのselコマンドツール)
   \item \verb|gm_mkhgrid| (バネ格子を適用した二十面体の水平格子を作成するツール)
   \item \verb|gm_mkllmap| (LatLonの水平格子を作成するツール)
   \item \verb|gm_mkmnginfo| (MPIプロセスのマネージメントファイルを作成するツール)
   \item \verb|gm_mkrawgrid| (正二十面体の水平格子を作成するツール)
   \item \verb|gm_mkvlayer| (鉛直格子を作成するツール)
 \end{itemize}

\begin{itemize}
  \item[*] makeコマンドの -j オプションは並列コンパイルに使用するプロセス数をしているオプションである。
   コンパイルにかかる時間を短縮するため、2以上の数を指定することで並列コンパイルを行うことができる。
   SCALE-GMでは、2 $\sim$ 8 プロセスの指定を推奨する。
  \item[*] ``fio'' とは、ヘッダー情報付きバイナリをベースにしたNICAMの独自ファイルフォーマットである。
\end{itemize}


\subsection{データベースの準備}
%-------------------------------------------------------------------------------

SCALE-GMの実行には、水平格子のデータが必要である。チュートリアルで使用する設定は、
g-level=5、r-level=0でMPIプロセスは10を想定している。
この設定のデータベースはソースコードのtarballに含まれているので、
別途ダウンロードする必要はないが、これ以外の設定で
実験を行う場合は以下に説明する要領で追加のダウンロードが必要になる。

\noindent \url{http://scale.aics.riken.jp/ja/download/}
から任意のデータベース(水平格子および正20面体LatLon格子変換テーブル)をダウンロードして
展開しておく。

\noindent 例えば、scale-gm\_database\_gl06rl01pe40.tar.gzをダウンロードした場合、
40個のboundaryファイル(水平格子データベース)と41個のllmapファイル(LatLon格子変換テーブル)
が格納されている。
\begin{verbatim}
  > tar -zxvf scale-gm_database_gl06rl01pe40.tar.gz
\end{verbatim}


\noindent 次にデータベースをそれぞれの格納場所に移動させる。

\noindent まずboundaryファイルは、
\${TOP}/test/data/grid/boundaryの下へ新たにディレクトリを作成して移動させる。
いま、展開したデータベースのディレクトリにいることを想定すると下記のようなコマンド実行になる。

\begin{verbatim}
  > mkdir ${TOP}/test/data/grid/boundary/gl06rl01pe40
  > mkdir boundary_GL06RL01.* ${TOP}/test/data/grid/boundary/gl06rl01pe40/
\end{verbatim}

\noindent 残るllmapファイルは、\${TOP}/test/data/grid/llmapの下へ新たにディレクトリを作成して移動させる。

\begin{verbatim}
  > mkdir -p ${TOP}/test/data/grid/boundary/gl06/rl01
  > mkdir llmap.* ${TOP}/test/data/grid/boundary/gl06/rl01/
\end{verbatim}


\subsection{実験の実行}
%-------------------------------------------------------------------------------
\subsubsection{テストケースの説明}

\noindent \${TOP}/test/case のディレクトリの下にいくつかの理想実験ケースが準備されている。
例えば、DCMIP2016の実験ケースについて表1にまとめた。
テストケースの詳細については\url{https://www.earthsystemcog.org/projects/dcmip-2016/testcases}
を参照するか、同WebページからダウンロードできるTest Case Documentを参照されたい。
これらの実験ケースディレクトリの下には
さらに格子間隔やMPIプロセス数ごとに異なるディレクトリが作成されている。
自分の実行環境や目的に合わせてディレクトリを選択してほしい。
また、各種ツールを用いて、任意の格子間隔やMPIプロセス数の設定を作成することも可能である。

 \begin{table}[b]
 \begin{center}
 \caption{Corresponding test cases}
 \begin{tabularx}{150mm}{|l|X|} \hline
 \rowcolor[gray]{0.9} SCALE-GMにおけるテストケースの名称 & 実験内容 \\ \hline
  DCMIP2016-11 & 湿潤傾圧波理想実験  \\ \hline
  DCMIP2016-12 & 全球台風理想実験 \\ \hline
  DCMIP2016-13 & 全球スーパーセル理想実験 \\ \hline
 \end{tabularx}
 \end{center}
 \end{table}


\subsubsection{計算実行: scale-gm}

計算機上で数値モデルを実行するためのジョブコマンドはシステムによって異なるが、
SCALE-GMでは計算機環境の違いを考慮してスクリプトを作成するシステムが用意されている。
任意の実験ディレクトリへ移動したあと、モデル実行スクリプトと後処理スクリプトを
作成するために下記のコマンドを実行する。

 \begin{verbatim}
   > make jobshell
 \end{verbatim}

このコマンドによって``\verb|run.sh|''と``\verb|ico2ll_netcdf.sh|''のスクリプトが作成される。
モデルを実行するためには、下記のようにコマンドを実行する。

 \begin{verbatim}
   > make run
 \end{verbatim}

これで、実験の計算が開始される。DCMIP2016実験においては、scale-gmを実行したとき、
一番はじめに初期値の作成を行っているため、数値モデルの実行前に初期値を作成する手順はない。

\textcolor{red}{要検討D:1〜3の実験をgl04 or gl05くらいで、PE5あたりで実行したときのおおよその所要時間があると
 RMにおけるUGの仕様と合わせることができる。また、正常に実行できた場合にどんなファイルが
 生成されるのか、どれがHistoryファイルで、どれがログファイルなのかくらいの説明があってもよいだろう。}


\subsection{後処理過程: ico2ll}
%-------------------------------------------------------------------------------
計算結果の描画や解析を容易にするために、実験の実行が終了したあと、
もとの二十面体格子からLatLon格子へ格子変換を行う。

後処理過程を実行するまえに、実施した実験設定に合わせて\verb|ico2ll_netcdf.sh|
の下記の箇所を編集する必要がある。
 \begin{verbatim}
   > vi ico2ll_netcdf.sh

   [at Line 22]
   # User Settings
   # ---------------------------------------------------------------------

   glev=5          # g-level of original grid
   case=161        # test case number
   out_intev='day' # output interval (format: "1hr", "6hr", "day", "100s")
 \end{verbatim}

 \noindent 後処理過程のスクリプトの編集が完了すれば、下記のコマンドによって後処理過程を実行できる。
 \begin{verbatim}
   sh ico2ll_netcdf.sh
 \end{verbatim}

% \begin{verbatim}
%   > bsub < ico2ll_netcdf.sh
% \end{verbatim}

 \noindent ico2llによって作成されるLatLon格子hデータは、現行のスクリプトでは、
netcdfフォーマットで出力されるようになっている。この場合、``\verb|nicam.161.200.L30.interp_latlon.nc|''
といったファイル名になる。また、スクリプトの設定を変更することで、GrADS形式の単純バイナリフォーマットで
データ出力することもできる。

