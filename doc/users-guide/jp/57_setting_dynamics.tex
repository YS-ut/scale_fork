\section{\SecBasicDynamicsSetting} \label{sec:atmos_dyn}
%------------------------------------------------------


\subsection{\SubsecDynsolverSetting}  \label{subsec:atmos_dyn_sover}
%------------------------------------------------------
時間方向の数値解法選択は、
設定ファイルの\namelist{PARAM_ATMOS}の\nmitem{ATMOS_DYN_TYPE}で行う。\\

\noindent {\gt\small
\ovalbox{
\begin{tabularx}{150mm}{ll}
\verb|&PARAM_ATMOS  | & \\
\verb| ATMOS_DYN_TYPE    = "HEVI", | & ; 表\ref{tab:nml_dyn}より選択。\\
\verb|/             | & \\
\end{tabularx}
}}\\

\begin{table}[h]
\begin{center}
  \caption{数値解法の選択肢}
  \label{tab:nml_dyn}
  \begin{tabularx}{150mm}{llX} \hline
    \rowcolor[gray]{0.9}  設定名 & スキームの説明 & 備考\\ \hline
      \verb|HEVE|  & 水平陽解法-鉛直陽解法 & \\
      \verb|HEVI|  & 水平陽解法-鉛直陰解法 & 実大気実験ではこちらを推奨\\
    \hline
  \end{tabularx}
\end{center}
\end{table}


\subsection{\SubsecDynSchemeSetting} \label{subsec:atmos_dyn_scheme}
%------------------------------------------------------
時間・空間差分スキームの設定は、設定ファイルの\namelist{PARAM_ATMOS_DYN}で設定する。
また、空間スキームによってはハロの数の設定も必要である。
下記の設定は推奨設定である。
その他、スキームと各種設定については表\ref{tab:nml_atm_dyn}を参照のこと。
また、時間スキームや空間差分スキーム変更時には、数値安定性が変わるため、
タイムステップも考慮する必要がある。
タイムステップついては、第\ref{sec:timeintiv}節を参照のこと。\\

\noindent {\gt\small
\ovalbox{
\begin{tabularx}{150mm}{ll}
 \verb|&PARAM_ATMOS_DYN  | & \\
 \verb|ATMOS_DYN_TINTEG_SHORT_TYPE          = RK4,|          & ; 時間スキームより選択\\
 \verb|ATMOS_DYN_TINTEG_TRACER_TYPE         = RK3WS2002,|    & ; 時間スキームより選択\\
 \verb|ATMOS_DYN_FVM_FLUX_TYPE              = UD3,|          & ; 空間スキームより選択\\
 \verb|ATMOS_DYN_FVM_FLUX_TRACER_TYPE       = UD3KOREN1993,| & ; 空間スキームより選択\\
 \verb|ATMOS_DYN_FLAG_FCT_TRACER            = .false.,|      & ; FCTスキームを利用するかどうか\\
 \verb|ATMOS_DYN_NUMERICAL_DIFF_COEF        = 0.D0, |        & \\
 \verb|ATMOS_DYN_NUMERICAL_DIFF_COEF_TRACER = 0.D0, |        & \\
 \verb|ATMOS_DYN_enable_coriolis            = .true.,|       & \\
 \verb|ATMOS_DYN_wdamp_height               = 15.D3,|        & ;レイリーダンピングを適用するスポンジ層の\\
                                                             &  ~~下端高度\\
\verb|/             | & \\
\end{tabularx}
}}\\


\begin{table}[h]
\begin{center}
  \caption{力学スキームの設定}
  \label{tab:nml_atm_dyn}
  \begin{tabularx}{150mm}{llXX} \hline
    \rowcolor[gray]{0.9} & \multicolumn{1}{l}{設定名} & \multicolumn{1}{l}{スキーム名} & \\ \hline
    \multicolumn{3}{l}{時間スキーム} &  \\ \hline
    & \multicolumn{1}{l}{\verb|RK3|} & \multicolumn{2}{l}{3次3段ルンゲクッタスキーム(Heun)} \\
    & \multicolumn{1}{l}{\verb|RK3WS2002|} & \multicolumn{2}{l}{Wicker and Skamarock (2002) 3段ルンゲクッタスキーム} \\
    & \multicolumn{1}{l}{\verb|RK4|} & \multicolumn{2}{l}{4次4段ルンゲクッタスキーム} \\
    \hline
    \multicolumn{3}{l}{空間スキーム} & 最小のハロ格子数\\ \hline
    & \multicolumn{1}{l}{\verb|CD2|} & \multicolumn{1}{l}{2次中央差分} & \multicolumn{1}{l}{1}\\
    & \multicolumn{1}{l}{\verb|CD4|} & \multicolumn{1}{l}{4次中央差分} & \multicolumn{1}{l}{2}\\
    & \multicolumn{1}{l}{\verb|CD6|} & \multicolumn{1}{l}{6次中央差分} & \multicolumn{1}{l}{3}\\
    & \multicolumn{1}{l}{\verb|UD3|} & \multicolumn{1}{l}{3次風上差分} & \multicolumn{1}{l}{2}\\
    & \multicolumn{1}{l}{\verb|UD5|} & \multicolumn{1}{l}{5次風上差分} & \multicolumn{1}{l}{3}\\
    & \multicolumn{1}{l}{\verb|UD3KOREN1993|} & \multicolumn{1}{X}{3次風上差分 + Koren(1993)フィルター} & \multicolumn{1}{l}{2}\\
\hline
  \end{tabularx}
\end{center}
\end{table}

力学変数の移流計算スキーム(\nmitem{ATMOS_DYN_FVM_FLUX_TYPE})のデフォルト設定は、
4次中央差分(\verb|CD4|)となっている。
\verb|CD4|では、地形起伏がある場合に、その直上でグリッドスケールに近い鉛直流が発生することが
しばしば確認されており、これは\verb|UD3|を使用することで緩和される。
{\scalerm}では、地形起伏があるような現実大気実験では、\verb|UD3|の使用を推奨する。
\verb|UD3|を使う時は、\nmitem{ATMOS_DYN_NUMERICAL_DIFF_COEF}はゼロでもよい。

トレーサー移流については、何らかの非負保証スキームを使う事が望ましい。
\verb|UD3KOREN1993|は非負保証スキームであるが、
それ以外は非負が保証されない。
つまり \nmitem{ATMOS_DYN_FVM_FLUX_TRACER_TYPE}として \verb|UD3KOREN1993|以外を選択した場合は、
\nmitem{ATMOS_DYN_FLAG_FCT_TRACER} を \verb|.true.| としてFCTを利用するのがよい。

ハロの格子数はデフォルトで2となっている。
従って、空間スキームがCD4、UD3以外の場合、上記の表に記したハロの数の設定が必要である。
設定は、\namelist{PARAM_INDEX}の中の\nmitem{IHALO, JHALO}で行う。

例えば、空間スキームとして5次風上差分を設定した場合、\\

\noindent {\gt\small
\ovalbox{
\begin{tabularx}{150mm}{ll}
 \verb|&PARAM_INDEX | &  \\
 \verb| IHALO = 3,|   &\\
 \verb| JHALO = 3,|   &\\
 \verb|/ | & \\
\end{tabularx}
}}\\

のように設定する。
