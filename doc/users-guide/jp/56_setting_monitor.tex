
%-------------------------------------------------------------------------------
\section{モニターファイル} \label{sec:monitor}
%-------------------------------------------------------------------------------

モニターファイルと出力変数は、\verb|run.conf|中の\namelist{PARAM_MONITOR}と\namelist{MONITOR_ITEM}で設定する。
モニターのデフォルトの形式は、\namelist{PARAM_MONITOR}で設定する。
\editboxtwo{
\verb|&PARAM_MONITOR                      | & \\
\verb| MONITOR_OUT_BASENAME  = "monitor", | & ; 出力ファイルのベース名 \\
\verb| MONITOR_USEDEVATION   = .true.,    | & ; 最初のステップからの偏差を使うか? \\
\verb| MONITOR_STEP_INTERVAL = 1,         | & ; モニター出力ステップ間隔 \\
\verb|/                                   | & \\
}

モニターコンポーネントは、乾燥空気の質量・水蒸気・全エネルギー・表面での降水フラックス等の物理量の領域全体での合計値を出力する。
これらの出力は、質量収支やエネルギー収支の確認に役立つ。
モニターファイルは ASCII 形式であり、ファイル名は \nmitem{MONITOR_OUT_BASENAME} \verb|.pe000000| と設定される。
モニター出力の時間間隔は、時間刻み幅($\Delta t$)の倍数として\nmitem{MONITOR_STEP_INTERVAL}に指定する。

\editboxtwo{
\verb|&MONITOR_ITEM   | & \\
\verb| NAME = "ENGT", | &  変数名。 変数のリストは表\ref{tab:varlist_monitor}に示される。 \\
\verb|/               | & \\
}

\begin{table}[h]
\begin{center}
  \caption{モニターに出力可能な変数}
  \label{tab:varlist_monitor}
  \begin{tabularx}{150mm}{|l|X|l|} \hline
    \rowcolor[gray]{0.9}  Values & Description & Unit \\ \hline
      \verb|DENS|         & 大気の質量                                  & [kg]     \\
      \verb|MOMZ|         & z方向の運動量                   & [kg m/s] \\
      \verb|MOMX|         & x方向の運動量                   & [kg m/s] \\
      \verb|MOMY|         & y方向の運動量                   & [kg m/s] \\
      \verb|RHOT|         & 温位                     & [kg K]   \\
      \verb|TRACER*|      & 予報変数のトレーサー            & [each unit $\times$ kg] \\
      \verb|QDRY|         & 乾燥空気の質量                              & [kg] \\
      \verb|QTOT|         & 水物質の質量                                 & [kg] \\
      \verb|EVAP|         & 表面での蒸発                & [kg] \\
      \verb|PRCP|         & 降水量                             & [kg] \\
      \verb|ENGT|         & 全エネルギー (\verb|ENGP + ENGK + ENGI|)  & [J] \\
      \verb|ENGP|         & ポテンシャルエネルギー ($\rho * g * z$)             & [J] \\
      \verb|ENGK|         & 運動エネルギー ($\rho * (W^2+U^2+V^2) / 2$) & [J] \\
      \verb|ENGI|         & 内部エネルギー ($\rho * C_v * T$)           & [J] \\
      \verb|ENGFLXT|      & 全エネルギーのフラックスの収束  & [J] \\
                          & (\verb|SH + LH + SFC_RD - TOA_RD|) & \\
      \verb|ENGSFC_SH|    & 表面での顕熱フラックス                & [J] \\
      \verb|ENGSFC_LH|    & 表面での潜熱フラックス                & [J] \\
      \verb|ENGSFC_RD|    & 表面での正味の放射フラックス                & [J] \\
                          & (\verb|SFC_LW_up + SFC_SW_up - SFC_LW_dn - SFC_SW_dn|) & \\
      \verb|ENGTOA_RD|    & 大気上端での正味の放射フラックス      & [J] \\
                          & (\verb|TOA_LW_up + TOA_SW_up - TOA_LW_dn - TOA_SW_dn|) & \\
      \verb|ENGSFC_LW_up| & 表面での上向き長波放射フラックス           & [J] \\
      \verb|ENGSFC_LW_dn| & 表面での下向き長波放射フラックス           & [J] \\
      \verb|ENGSFC_SW_up| & 表面での上向き短波放射フラックス           & [J] \\
      \verb|ENGSFC_SW_dn| & 表面での下向き短波放射フラックス           & [J] \\
      \verb|ENGTOA_LW_up| & 大気上端での上向き長波放射フラックス & [J] \\
      \verb|ENGTOA_LW_dn| & 大気上端での下向き長波放射フラックス & [J] \\
      \verb|ENGTOA_SW_up| & 大気上端での上向き短波放射フラックス & [J] \\
      \verb|ENGTOA_SW_dn| & 大気上端での下向き短波放射フラックス & [J] \\
    \hline
  \end{tabularx}
\end{center}
\end{table}

例えば、 \nmitem{MONITOR_STEP_INTERVAL} \verb|= 10| および \nmitem{MONITOR_USEDEVATION}\verb|=.false.|と指定して、
\namelist{MONITOR_ITEM}に以下の設定を付け加えたとする。

\editbox{
\verb|&MONITOR_ITEM  NAME="ENGK" /|\\
\verb|&MONITOR_ITEM  NAME="ENGP" /|\\
\verb|&MONITOR_ITEM  NAME="ENGI" /|\\
\verb|&MONITOR_ITEM  NAME="ENGT" /|\\
}
\noindent
このとき、モニターファイルは以下のように出力される。

\msgbox{
                   ENGT            ENGP            ENGK            ENGI            \\
STEP=      1 (MAIN)  1.18127707E+17  2.92701438E+16  2.40231436E+13  8.88335403E+16\\
STEP=     11 (MAIN)  1.18127712E+17  2.92701415E+16  2.40249223E+13  8.88335453E+16\\
STEP=     21 (MAIN)  1.18127711E+17  2.92701439E+16  2.40223566E+13  8.88335443E+16\\
STEP=     31 (MAIN)  1.18127710E+17  2.92701454E+16  2.40213480E+13  8.88335435E+16\\
STEP=     41 (MAIN)  1.18127710E+17  2.92701495E+16  2.40210662E+13  8.88335392E+16\\
STEP=     51 (MAIN)  1.18127710E+17  2.92701439E+16  2.40205575E+13  8.88335456E+16\\
STEP=     61 (MAIN)  1.18127711E+17  2.92701565E+16  2.40200252E+13  8.88335340E+16\\
STEP=     71 (MAIN)  1.18127711E+17  2.92701457E+16  2.40195927E+13  8.88335455E+16\\
STEP=     81 (MAIN)  1.18127710E+17  2.92701486E+16  2.40193679E+13  8.88335425E+16\\
STEP=     91 (MAIN)  1.18127710E+17  2.92701573E+16  2.40188095E+13  8.88335342E+16\\
STEP=    101 (MAIN)  1.18127710E+17  2.92701404E+16  2.40180752E+13  8.88335517E+16\\
}
