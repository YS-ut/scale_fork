\section{この章の最後に} \label{sec:ideal_exp_last}

本章では、スコールラインの理想実験を例に、SCALEの実行方法について説明した。
次ステップとして、
解像度や計算領域、MPIプロセス数の変更
放射過程や乱流過程、雲微物理スキームといった物理過程の変更を
試すことをお勧めする。
これらの変更方法は、第\ref{chap:basic_usel}章に記載されている。

このスコールラインの理想実験については、
同じディレクトリ下の``sample''ディレクトリ内に、
解像度設定、領域設定、使用する物理スキームについて変更を加えた設定ファイルのサンプルが用意されているので、
これらも参考となる。
また、SCALEには各種理想実験セットが``\verb|scale-rm/test/case|''以下に複数用意されている。
実験設定によってはテストケースに特化したソースコードの入れ替えを行っている場合があるため、
これらの理想実験セットでは設定ファイルのあるディレクトリでmakeコマンドを実行する必要がある。
初期値作成と実行の手順は基本的に本章のチュートリアルと同じである。
