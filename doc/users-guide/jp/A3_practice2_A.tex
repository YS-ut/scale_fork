%\chapter{Q \& A}

\clearpage
\section*{回答}
\begin{enumerate}
\item {\bf 計算領域は変えず、MPI並列数を変更したい}\\
変更箇所は、\namelist{PARAM_INDEX}内の\nmitem{IMAX, JMAX}、\namelist{PARAM_PRC}内の\nmitem{PRC_NUM_X, PRC_NUM_Y}である。下記3つの式を満たしていれば正解である。
\begin{eqnarray}
&& MPI並列数 = (\verb|PRC_NUM_X|) \times (\verb|PRC_NUM_Y|) = 6 \nonumber\\
&& X方向の格子数 = \left(\verb|IMAX| \times \verb|PRC_NUM_X|\right) = 90 \nonumber\\
&& Y方向の格子数 = \left(\verb|JMAX| \times \verb|PRC_NUM_Y|\right) = 90 \nonumber
\end{eqnarray}


\item {\bf MPI並列数は変えず、計算領域を変更したい}\\
MPIプロセスあたりの格子数を$n$倍にすれば、
領域全体の格子数も$n$倍となる。
変更箇所は、\namelist{PARAM_INDEX}内の\nmitem{IMAX, JMAX}のみである。
赤文字の部分がデフォルトからの変更点を意味する。\\

\noindent {\small {\gt
\ovalbox{
\begin{tabularx}{150mm}{ll}
\verb|&PARAM_INDEX| & \\
\verb| KMAX = 36,|  & \\
\textcolor{red}{\verb| IMAX = 60,|}  & (オリジナル設定は \verb|IMAX = 45|)\\
\textcolor{red}{\verb| JMAX = 30,|}  & (オリジナル設定は \verb|JMAX = 45|)\\
\verb|/| & \\
\end{tabularx}
}}}\\

\item {\bf 計算領域は変えず、水平格子間隔を変更したい}\\
MPI並列数を変えない場合、変更箇所は\namelist{PARAM_GRID}の\nmitem{DX, DY}と
\namelist{PARAM_INDEX}内の\nmitem{IMAX,JMAX}である。

\noindent {\small {\gt
\ovalbox{
\begin{tabularx}{150mm}{ll}
\verb|&PARAM_PRC|  & \\
\verb| PRC_NUM_X      = 2,|  & \\
\verb| PRC_NUM_Y      = 2,|  & \\
\\
\verb|&PARAM_INDEX| & \\
\verb| KMAX = 36,|  & \\
\textcolor{red}{\verb| IMAX = 180,|} &  (オリジナル設定は \verb|IMAX = 45|)\\
\textcolor{red}{\verb| JMAX = 180,|} &  (オリジナル設定は \verb|JMAX = 45|)\\
\verb|/| &\\
 \\
\verb|&PARAM_GRID| & \\
\textcolor{red}{\verb| DX = 5000.D0,|} & (オリジナル設定は \verb|DX = 20000.D0|)\\
\textcolor{red}{\verb| DY = 5000.D0,|} & (オリジナル設定は \verb|DY = 20000.D0|)\\
\verb|/| & \\
\end{tabularx}
}}}\\

MPI並列数も変更している場合には、下記の関係を満たしていれば正解である。
\begin{eqnarray}
&& {\XDIR} の格子数 = \left(\verb|IMAX| \times \verb|PRC_NUM_X|\right) = 360 \\\nonumber
&& {\YDIR}の格子数 = \left(\verb|JMAX| \times \verb|PRC_NUM_Y|\right) = 360 \nonumber
\end{eqnarray}

これに加えて、力学変数の時間積分のためのタイムステップ\\
\nmitem{TIME_DT_ATMOS_DYN} と\nmitem{ATMOS_DYN_TINTEG_SHORT_TYPE} の調整も必要である
(第\ref{sec:timeintiv}節参照)。

緩和領域も、格子間隔の20倍から40倍となるよう変更する。

\noindent {\small {\gt
\ovalbox{
\begin{tabularx}{150mm}{ll}
\verb|&PARAM_PRC|  & \\
\verb| BUFFER_DX = 100000.D0, | & (オリジナル設定は\verb|BUFFER_DX = 400000.D0,|) \\
\verb| BUFFER_DY = 100000.D0, | & (オリジナル設定は\verb|BUFFER_DY = 400000.D0,|) \\
\verb|/| &\\
\end{tabularx}
}}}\\


\item {\bf 計算領域の位置を変更したい}\\

計算領域の中心位置の座標を下記の通り、変更すれば良い。
ここで、単位を度で設定する必要があることに注意すること。
例えば、139度45.4分 = 139 + 45.5/60 度。

\noindent {\small {\gt
\ovalbox{
\begin{tabularx}{150mm}{ll}
\verb|&PARAM_MAPPROJ                      | & \\
\textcolor{red}{\verb| MPRJ_basepoint_lon = 139.7567D0,|} & (オリジナル設定は\verb| MPRJ_basepoint_lon = 135.220404D0,|)\\
\textcolor{red}{\verb| MPRJ_basepoint_lat =  35.6883D0,|} & (オリジナル設定は\verb| MPRJ_basepoint_lat =  34.653396D0,|)\\
\verb| MPRJ_type          = 'LC',         | & \\
\verb| MPRJ_LC_lat1       =  30.00D0,     | & \\
\verb| MPRJ_LC_lat2       =  40.00D0,     | & \\
\verb|/| & \\
\end{tabularx}
}}}\\


\item {\bf 積分時間の変更したい}\\

\noindent {\small {\gt
\ovalbox{
\begin{tabularx}{150mm}{ll}
\verb|&PARAM_TIME| & \\
\verb| TIME_STARTDATE             = 2007, 7, 14, 18, 0, 0, | & \\
\verb| TIME_STARTMS               = 0.D0,                  | & \\
\textcolor{red}{\verb| TIME_DURATION = 12.0D0,             |}
                                                     &  (オリジナル設定は\verb| 6.0D0,|) \\
\verb| TIME_DURATION_UNIT         = "HOUR",              | & \\
\verb|/| & \\
\end{tabularx}
}}}\\

また、\verb|scale-rm_init|の境界値を12時間以上用意している必要がある。
第\ref{sec:adv_datainput}節を参照し、\nmitem{NUMBER_OF_FILES}の数を3に変更する。


\item {\bf 出力変数の追加と出力時間間隔の変更したい}\\

例えば、出力時間間隔を30分に変更したい時には、\namelist{PARAM_HISTORY}の中の
\nmitem{HISTORY_DEFAULT_TINTERVAL}を下記のように変更する。
また、出力変数を追加したい時には、\namelist{HISTITEM}の中の\nmitem{ITEM}に追加したい変数を指定する。
変数名に関しては、\ref{achap:namelist}章を参照のこと。
以下では、地表面での上向き、下向き短波フラックスを追加した例である。\\

\noindent {\small {\gt
\ovalbox{
\begin{tabularx}{150mm}{ll}
\verb|&PARAM_HISTORY | & \\
\verb| HISTORY_DEFAULT_BASENAME  = "history_d01", | & \\
\textcolor{red}{\verb| HISTORY_DEFAULT_TINTERVAL = 1800.D0,|} & (オリジナル設定は\verb|3600.D0,|) \\
\verb| HISTORY_DEFAULT_TUNIT     = "SEC",| & \\
\verb|/| & \\
\\
\textcolor{red}{\verb|&HISTITEM item="SFLX_SW_up" /|} & \textcolor{red}{追加}\\
\textcolor{red}{\verb|&HISTITEM item="SFLX_SW_dn" /|} & \textcolor{red}{追加}\\
\\
\end{tabularx}
}}}\\



\item {\bf リスタート計算をしたい}\\


最初の3時間分の計算設定は下記の通りである。3時間目にリスタートファイルを出力する。\\

\noindent {\small {\gt
\ovalbox{
\begin{tabularx}{150mm}{ll}
\verb|&PARAM_TIME| & \\
\verb| TIME_STARTDATE             = 2007, 7, 14, 18, 0, 0, | & \\
\verb| TIME_STARTMS               = 0.D0, | & \\
\textcolor{red}{\verb| TIME_DURATION              = 3.0D0, |} ~~~~~~~~~3時間以上であれば良い。& \\
\verb| TIME_DURATION_UNIT         = "HOUR", | & \\
\verb| ....(省略)....| & \\
\textcolor{red}{\verb| TIME_DT_ATMOS_RESTART      = 10800.D0, |} & \\
\textcolor{red}{\verb| TIME_DT_ATMOS_RESTART_UNIT = "SEC",    |} & \\
\textcolor{red}{\verb| TIME_DT_OCEAN_RESTART      = 10800.D0, |} & \\
\textcolor{red}{\verb| TIME_DT_OCEAN_RESTART_UNIT = "SEC",    |} & \\
\textcolor{red}{\verb| TIME_DT_LAND_RESTART       = 10800.D0, |} & \\
\textcolor{red}{\verb| TIME_DT_LAND_RESTART_UNIT  = "SEC",    |} & \\
\textcolor{red}{\verb| TIME_DT_URBAN_RESTART      = 10800.D0, |} & \\
\textcolor{red}{\verb| TIME_DT_URBAN_RESTART_UNIT = "SEC",    |} & \\
\verb|/| & \\
\\
\verb|&PARAM_RESTART | & \\
\verb| RESTART_RUN          = .false.,               | ~~~~~~~~~~~~~~~追加。なくても良い。& \\
\textcolor{red}{\verb| RESTART_OUTPUT      = .true.,|} ~~~~~~~~~~~~(オリジナル設定は\verb|.false.,|)& \\
\verb| RESTART_IN_BASENAME = "../init/init_d01_20070714-180000.000",| &  \\
\textcolor{red}{\verb| RESTART_OUT_BASENAME = "restart_d01",|} ~~~~~~~~~~~~~~~追加 & \\
\verb|/| & \\
\\
\verb|&PARAM_ATMOS_BOUNDARY| & \\
\verb| ATMOS_BOUNDARY_TYPE           = "REAL",                            | & \\
\verb| ATMOS_BOUNDARY_IN_BASENAME    = "../init/output/boundary_d01",     | & \\
\textcolor{red}{\verb| ATMOS_BOUNDARY_START_DATE     = 2010, 7, 14, 18, 0, 0,|} ~~~~~~~~~~~~~~追加。なくてもよい。 &\\
\verb| ATMOS_BOUNDARY_UPDATE_DT      = 21600.D0,                          | & \\
\verb|/| & \\
\end{tabularx}
}}}\\


\nmitem{TIME_DURATION}が3時間に設定されている場合には、リスタートファイルは積分終了時に出力されるため、
\nmitem{TIME_DT_ATMOS_RESTART}、\nmitem{TIME_DT_OCEAN_RESTART}、\\
\nmitem{TIME_DT_LAND_RESTART}、\nmitem{TIME_DT_URBAN_RESTART}の設定は必要ない。
また、\nmitem{TIME_DURATION}が3時間以上に設定されている場合には、
\nmitem{TIME_DT_ATMOS_RESTART}、\nmitem{TIME_DT_OCEAN_RESTART}、\\
\nmitem{TIME_DT_LAND_RESTART}、\nmitem{TIME_DT_URBAN_RESTART}に3時間、もしくは10800 (sec) の約数かつ\nmitem{TIME_DT}の倍数を指定している必要がある。



3時間目から6時間目までのリスタート計算のための設定は下記の通りである。\\

\noindent {\small {\gt
\ovalbox{
\begin{tabularx}{150mm}{ll}
\verb|&PARAM_TIME| & \\
\textcolor{red}{\verb| TIME_STARTDATE             = 2007, 7, 14, 21, 0, 0, |} & \\
\verb| TIME_STARTMS               = 0.D0, | & \\
\textcolor{red}{\verb| TIME_DURATION              = 3.0D0, |}    & 3時間以上であれば良い。\\
\verb| TIME_DURATION_UNIT         = "HOUR", | & \\
\verb|/| & \\
\\
\verb|&PARAM_RESTART | & \\
\textcolor{red}{\verb| RESTART_RUN          = .true.,               |} & \textcolor{red}{必須}\\
\verb| RESTART_OUTPUT      = .true.,                |                  & あってもなくてもよい。\\
\textcolor{red}{\verb| RESTART_IN_BASENAME = "restart_d01_20070714-210000.000",|} & \textcolor{red}{必須}\\
\verb| RESTART_OUT_BASENAME = "restart2_d01",| & あってもなくてもよい。\\
\verb|/| & \\
\\
\verb|&PARAM_ATMOS_BOUNDARY| & \\
\verb| ATMOS_BOUNDARY_TYPE           = "REAL",                            | & \\
\verb| ATMOS_BOUNDARY_IN_BASENAME    = "../init/output/boundary_d01",     | & \\
\textcolor{red}{\verb| ATMOS_BOUNDARY_START_DATE     = 2010, 7, 14, 18, 0, 0,   |} & \textcolor{red}{必須}\\
\verb| ATMOS_BOUNDARY_UPDATE_DT      = 21600.D0,                          | & \\
\verb|/| & \\
\end{tabularx}
}}}\\




\end{enumerate}

