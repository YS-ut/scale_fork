\section{後処理ツール(net2g)のコンパイル} \label{sec:source_net2g}
%====================================================================================

「net2g」は \scalerm 用の後処理ツールである。
\scalerm の出力ファイルはノードごとに分割されている。
これらの出力ファイル(\verb|history.******.nc|)を結合し、
\grads で直接読み込めるデータ形式へ
変換するための後処理ツール「net2g」を \scalelib は提供している。
第\ref{chap:tutorial_ideal}章、第\ref{chap:tutorial_real}章のチュートリアルで使用するため、
ここでは「net2g」のコンパイル方法を説明する。

%net2gはSCALE本体から独立したツールになっている
%(ただしMakedefファイルを除く)ため、
%任意の場所へコピーしてコンパイルすることができるが、
%コンパイルにはnetCDFライブラリが必要であり、
%また並列実行するためにはMPIライブラリが必要である。
%従って、以降はこれらのライブラリがインストールされている環境であることを想定して進める。\\


まず、SCALE本体のコンパイル時と同様に、
使用環境に合ったMakedefファイルを指定するために環境変数を設定する。
次に、「net2g」のディレクトリに移動し、make コマンドを実行する。
MPIライブラリを使用した並列実行バイナリは、
下記のコマンドによって生成される。
\begin{alltt}
 $ cd scale-{\version}/scale-rm/util/netcdf2grads_h
 $ make -j 2
\end{alltt}
MPI ライブラリが無い場合に、逐次実行バイナリを生成するには、
\begin{alltt}
 $ make -j 2 SCALE_DISABLE_MPI=T
\end{alltt}
のコマンドによってコンパイルする。
「\verb|net2g|」という名前の実行ファイルが生成されていれば、コンパイルは成功である。
作成した実行バイナリを消去する場合は、下記のコマンドを実行する。
\begin{alltt}
 $ make clean
\end{alltt}
