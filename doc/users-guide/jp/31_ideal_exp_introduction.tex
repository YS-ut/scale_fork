%%%%%%%%%%%%%%%%%%%%%%%%%%%%%%%%%%%%%%%%%%%%%%%%%%%%%%%%%%%%%%%%%%%%%
%  File 31_ideal_exp.tex
%%%%%%%%%%%%%%%%%%%%%%%%%%%%%%%%%%%%%%%%%%%%%%%%%%%%%%%%%%%%%%%%%%%%%
\section{概要} \label{sec:ideal_exp_intro}

本章では、数値実験における{\scalerm}の基本的な操作を、
理想実験を題材に説明する。 
第\ref{part:install}部で実行したSCALEのコンパイルが正常に完了しているかどうかの
確認も兼ねているので、このチュートリアルを行うことを強く推奨する。
本章では、既に下記のファイルが生成されているものとして説明を行う。
\begin{alltt}
  scale-{\version}/bin/scale-rm
  scale-{\version}/bin/scale-rm_init
  scale-{\version}/scale-rm/util/netcdf2grads_h/net2g
\end{alltt}
これらに加えて、描画ツールとして\grads を使用する。
結果の確認には、「gpview」も利用することができる。
\grads や「gpview」(\gphys)のインストール方法は、第\ref{sec:inst_env}節を参照のこと。

チュートリアルは、前準備、初期値作成、シミュレーション実行、後処理、そして描画の順番で記述する。





