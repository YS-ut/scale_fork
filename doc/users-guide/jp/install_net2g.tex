\section{後処理ツール(net2g)のコンパイル} \label{sec:source_net2g}
%====================================================================================

「net2g」は、\scalerm 用の後処理ツールである。
\scalerm の出力ファイルは、ノードごとに分割されて出力される。
\scalelib では、これら出力ファイル(\verb|history.******.nc|)を結合し、
\grads で直接読み込めるデータ形式へ
変換する後処理ツール「net2g」を提供している。
第\ref{chap:tutorial_ideal}章、第\ref{chap:tutorial_real}章のチュートリアルでも使用する。
ここでは、net2gのコンパイル方法について説明する。

%net2gはSCALE本体から独立したツールになっている
%(ただしMakedefファイルを除く)ため、
%任意の場所へコピーしてコンパイルすることができるが、
%コンパイルにはnetCDFライブラリが必要であり、
%また並列実行するためにはMPIライブラリが必要である。
%従って、以降はこれらのライブラリがインストールされている環境であることを想定して進める。\\


まず、SCALE本体のコンパイル時と同様に、
使用環境に合ったMakedefファイルを設定するために環境変数を設定する。
次に、net2gのディレクトリに移動し、makeする。
MPIライブラリを用いた並列実行を行うためのバイナリは、
下記のコマンドによって生成される。
\begin{alltt}
 $ cd scale-{\version}/scale-rm/util/netcdf2grads_h
 $ make -j 2
\end{alltt}
MPIライブラリが無い場合などに、逐次実行バイナリを生成するためには、
\begin{alltt}
 $ make -j 2 SCALE_DISABLE_MPI=T
\end{alltt}
としてコンパイルを行う。
\verb|net2g|という名前の実行ファイルが生成されていればコンパイルは成功である。\\


下記のコマンドで作成された実行バイナリを消去できる。
\begin{alltt}
 $ make clean
\end{alltt}
