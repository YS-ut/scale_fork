
%==============================================================%

本書は初めて領域気象気候モデル({\scalerm})
を利用する人に向けた解説書です。
気象気候ライブラリー{\scalelib}  version \version に対応した説明を記載します。
\scalelib の現バージョンには、領域モデル\scalerm と全球モデル\scalegm が含まれます。
本版では、\scalerm の使い方についてのみ詳しく述べています。
\scalegm については、次版で詳しく記載される予定です。
%\scalerm の使い方を通して、\scalelib を他のモデルからの呼び出す方法を
%習得することも可能です。

本書の構成は次の通りです。
第\ref{part:overview}部では SCALE の概要、
第\ref{part:install}部では必要な環境とインストール方法について説明します。
続いて、第\ref{chap:tutorial_ideal}章では\scalerm の基本的な操作方法、
第\ref{chap:tutorial_real}章では現実大気実験の実行方法について、
簡単な例を示しながら説明します。
これらの章はひと繋がりのチュートリアルとなっており、
\scalerm を初めて使うユーザは一通り通読することをお勧めます。
第\ref{part:basic_usel}部以降は、設定の変更、機能やツールの説明が記載されています。

%%%
本書中の不明点やお気づきの点がございましたら、SCALE user's メーリングリスト\\
 \verb|scale-users@ml.riken.jp|までご連絡ください。



\section{\scalelib の特徴} \label{subsec:scale_feature}
%--------------------------------------------------------------%

{\scalelib} (Scalable Computing for Advanced Library and Environment)は
計算機を用いて気象・気候科学計算を行う上で、
研究を進めやすいようにプレ処理から数値シミュレーション、ポスト処理、
解析に至るまですべての過程を網羅する気象・気候数値計算ライブラリを
目指したソフトウェアである。下記に挙げるような特徴を持つ。
\begin{itemize}
\item \scalelib は、「BSD-2ライセンス」のもとオープンソースソフトウェア
として提供されており、商用、非商用に関わらず自由な利用・改変・再配布が可能である。
\item \scalelib には、\scalerm (SCALE-Regional Model)、
%SCALE-GM(SCALE-Global Model)
といった組み上げ済みの数値モデルが含まれている。
\item \scalelib には、次節で説明する様々なコンポーネントが導入されており、
行いたい実験に合わせて選択利用することが可能である。
\item \scalelib で提供されている物理過程は、他の数値モデルへ組み込んで
使用することも可能である。
\end{itemize}

ライセンスの詳細は、トップディレクトリ直下の\texttt{scale-\version/LICENSE}のファイルに記述されている。
\scalelib の使用前に一読しておくこと。またSCALEのWebページにもソフトウェアの説明が記載されているので必要に応じて
参照すること(\url{http://scale.aics.riken.jp/})。

本節の以下では\scalelib の思想とモデルの関係について説明するが、
\scalerm の実行とは直接関係ないため、必要なければ読み飛ばしても構わない。

\Item{\scalelib のライブラリとモデルの関係について}

\begin{figure}[htb]
\begin{center}
  \includegraphics[width=0.9\hsize]{./figure/library_jp.eps}\\
  \caption{\scalelib のねらい}
  \label{fig:scale}
\end{center}
\end{figure}

\scalelib は、理化学研究所(RIKEN)を中心に
開発が進められている気象・気候科学計算向けのライブラリである。
図 \ref{fig:scale}に\scalelib の思想の概念図を示す。
この図に示されるような諸問題に対応することを目指している。
\scalelib は小型PCクラスターから次世代のスーパーコンピュータに至るまで
広く用いられる事を念頭において開発されており、
気候・気象科学を専門とする科学者と計算機科学を専門とする科学者が
共同で開発を行っている。
そのため、スーパーコンピュータ「京」や富士通FX10等の
スーパーコンピュータに加え、インテル機のような汎用計算機でも
計算効率がでることを目指して設計されている。

\scalerm は \scalelib を利用した数値モデルの一つであり、
\scalelib のパッケージに含まれている(図 \ref{fig:scale-rm})。
\scalelib は、並列プロセスの管理、ファイル入出力、プロセス間通信や格子情報の設定を行う。
大気流体の支配方程式を解く部分(流体力学コア、力学過程)と
雲微物理や大気放射のような諸物理過程を解く部分も、\scalelib が提供する。
一方、\scalerm は\scalelib が提供する機能を組み合わせることによって構築されている。
\scalerm 自身は大気の状態の入力データを予報変数として読み込んで保持し、
\scalelib の各コンポーネントを適切に呼び出すことで時間発展を計算する。
ユーザは必要に応じて利用するコンポーネントを選択し、組み合わせてシミュレーションを行うことが出来る。

\begin{figure}[hbt]
\begin{center}
  \includegraphics[width=0.9\hsize]{./figure/scale_jp.eps}\\
  \caption{SCALEとSCALE-RMの関係}
  \label{fig:scale-rm}
\end{center}
\end{figure}



\section{\scalerm の構成}  \label{subsec:sturcture_scale_rm}
%--------------------------------------------------------------%
\scalelib に含まれるコンポーネントとスキームは全て、\scalerm において利用できる。
コンポーネントは三つの部分(フレームワーク、力学コア、物理過程)に分類される。
\scalerm の現バージョンに実装済みのコンポーネントや様々なスキームを以下に列挙する%
\footnote{
詳細なモデル構成や差分化手法については、
\citet{scale_2015}、\citet{satoy_2015b}、
および\citet{nishizawa_2015}を参照されたい。
}。
\\

{\bf フレームワーク}
\begin{itemize}
 \item 距離座標に基づいた三次元カーテシアン格子系
 \item MPI通信を用いた二次元領域分割
 \item 各種地図投影法
 \item ネスティングシステム(1 way:親領域$\to$子領域)
   \begin{itemize}
    \item オンライン実行(複数ドメインの計算を同時に実行)
    \item オフライン実行(外側ドメインの計算終了後に、その結果を用いて内側ドメインの計算を行う)
   \end{itemize}
 \item 複数事例一括実行 システム(バルクジョブシステム)
 \item CF 規約\footnote{\url{http://cfconventions.org/}}に基づく \netcdf ファイル I/O
   \begin{itemize}
   \item {\netcdf}3 または {\netcdf}4 形式
   \end{itemize}
 \item 理想実験のための初期値データ生成
 \item 外部データ読み込みによる標高・土地利用区分データの変換作成
 \item 外部データ読み込みによる初期値・境界値データの変換作成
   \begin{itemize}
    \item
      WRF-ARW\footnote{\url{http://www.wrf-model.org/}}、
%      NICAM\footnote{\url{http://nicam.jp/}}、
      \grads \footnote{\url{http://cola.gmu.edu/grads/}}フォーマットでの入力に対応
   \end{itemize}
\end{itemize}

{\bf 力学コア関係}
\begin{itemize}
 \item 支配方程式系: 3次元完全圧縮非静力学方程式系
 \item 空間離散化: 有限体積法
    \begin{itemize}
      \item 2次, 4次, 6次中央差分
      \item 3次, 5次風上差分
    \end{itemize}
 \item 時間差分: 「完全陽解法」または「水平陽解法-鉛直陰解法」から選択
    \begin{itemize}
      \item Heun型3次ルンゲクッタスキーム
      \item \citet{Wicker_2002}の3段ルンゲクッタスキーム
      \item 4次ルンゲクッタスキーム
    \end{itemize}
 \item 非負保証:
    \begin{itemize}
      \item フラックス修正法 \citep[Flux Corrected Transport, FCT; ][]{zalesak_1979}
      \item \citet{Koren_1993}フィルター  (3次風上差分スキーム使用時のみ)
    \end{itemize}
 \item 数値フィルター: 4次超粘性・拡散
 \item 地形: 地形に沿った座標系を用いて表現
\end{itemize}

{\bf 物理過程}
\begin{itemize}
 \item 乱流過程: 以下から選択可能
   \begin{itemize}
    \item \citet{smagorinsky_1963} \& \citet{lilly_1962}型のサブグリッドスケール乱流モデル (\citet{Brown_etal_1994}と\citet{Scotti_1993}による補正)
    \item \citet{Deardorff_1980} サブグリッドスケール乱流モデル
    \item \citet{my_1982,nakanishi_2004}によるlevel 2.5境界層乱流パラメタリゼーション
   \end{itemize}
 \item 雲微物理: 以下から選択可能
   \begin{itemize}
    \item \citet{kessler_1969}による3-class 1モーメントバルクモデル
    \item \citet{tomita_2008}による6-class 1モーメントバルクモデル
    \item \citet{sn_2014}による6-class 2モーメントバルクモデル
    \item \citet{suzuki_etal_2010}によるビン法モデル
   \end{itemize}
 \item 放射過程: \citet{sekiguchi_2008}による相関k分布法ブロードバンド大気放射伝達モデル
 \item 地表面モデル
  \begin{itemize}
   \item 陸面モデル: 熱拡散・バケツモデル
   \item 海洋モデル: 以下から選択可能
   \begin{itemize}
     \item 初期値固定
     \item 外部データ入力
     \item スラブモデル
   \end{itemize}
   \item 都市モデル: \citet{kusaka_2001}による単層キャノピーモデル
   \item バルク交換係数(陸面および海面): 以下から選択可能
   \begin{itemize}
     \item \citet{beljaars_1991,wilson_2001}による普遍関数によるバルク法
     \item \citet{uno_1995}によるLouis 型バルク法
   \end{itemize}
  \end{itemize}
\end{itemize}
