%%%%%%%%%%%%%%%%%%%%%%%%%%%%%%%%%%%%%%%%%%%%%%%%%%%%%%%%%%%%%%%%%%%%%%%%%%%%%%%%%%%%%%
%  File 21_install.tex
%%%%%%%%%%%%%%%%%%%%%%%%%%%%%%%%%%%%%%%%%%%%%%%%%%%%%%%%%%%%%%%%%%%%%%%%%%%%%%%%%%%%%%

\section{ダウンロードと環境設定} \label{sec:scale_compile}
%====================================================================================

以下の説明で使用した環境は次のとおりである。
\begin{itemize}
\item CPU: Intel Core i5 2410M 2コア/4スレッド
\item Memory: DDR3-1333 4GB
\item OS: CentOS 6.6 x86-64、CentOS 7.1 x86-64、openSUSE 13.2 x86-64
\item GNU C/C++、Fortran compiler (付録\ref{achap:env_setting}章参照)
\end{itemize}

\subsubsection{ソースコードの入手} \label{subsec:get_source_code}
%-----------------------------------------------------------------------------------
最新のリリース版ソースコードは、\\
\url{http://scale.aics.riken.jp/ja/download/index.html}\\
よりダウンロードできる。
ソースコードのtarballファイルを展開すると\texttt{scale-{\version}/} というディレクトリができる。
\begin{alltt}
 $ tar -zxvf scale-{\version}.tar.gz
 $ ls ./
    scale-{\version}/
\end{alltt}

\subsubsection{Makedefファイルと環境変数の設定} \label{subsec:environment}
%-----------------------------------------------------------------------------------

\scalelib はコンパイルするとき、環境変数``\verb|SCALE_SYS|''に設定した
Makedefファイルを使用してコンパイルが行われる。
Makedefファイルは、\texttt{scale-{\version}/sysdep/} 内にいくつかの計算機環境に
対応するファイル(\texttt{Makedef.***})が準備されており、
これらの中から自分の環境にあったものを設定する。
動作確認済みの環境と対応するMakedefファイルを表\ref{tab:makedef}に示す。
自分の環境に合致するものがなければ、既存ファイルを基にして各自作成する。
%%
\begin{table}[htb]
\begin{center}
\caption{環境例と対応するMakedefファイル}
\begin{tabularx}{150mm}{|l|l|X|l|} \hline
 \rowcolor[gray]{0.9} OS/計算機 & コンパイラ & MPI & Makedefファイル \\ \hline
                 & gcc/gfortran & openMPI & Makedef.Linux64-gnu-ompi \\ \cline{2-4}
 Linux OS x86-64 & icc/ifort & intelMPI & Makedef.Linux64-intel-impi \\ \cline{2-4}
                 & icc/ifort    & SGI-MPT & Makedef.Linux64-intel-mpt \\ \hline
 macOS           & gcc/gfortran & openMPI & Makedef.MacOSX-gnu-ompi \\ \hline
 スーパーコンピュータ「京」 & fccpx/frtpx & mpiccpx/mpifrtpx & Makedef.K \\ \hline
 Fujitsu PRIME-HPC FX10 & fccpx/frtpx & mpiccpx/mpifrtpx & Makedef.FX10 \\ \hline
\end{tabularx}
\label{tab:makedef}
\end{center}
\end{table}
%%
Linux x86-64 OS、GNUコンパイラ、openMPIを使用する場合には、
\verb|"Makedef.Linux64-gnu-ompi"|が対応するファイルであり、
下記の通り、環境変数を設定する。
\begin{alltt}
 $ export SCALE_SYS="Linux64-gnu-ompi"
\end{alltt}
実行環境が常に同じであるならば、
環境変数の設定を\verb|.bashrc|などの環境設定ファイルに
記述しておくと便利である。

{\scalelib}は{\netcdf}を必要とする。
多くの場合には、{\netcdf}のPATHは「nc-config」を用いることによって自動的に見つけられる。
もし自動的に見つけることができない場合には、例えば以下のように{\netcdf}に関するPATHを設定しなければならない。
\begin{verbatim}
 $ export SCALE_NETCDF_INCLUDE="-I/opt/netcdf/include"
 $ export SCALE_NETCDF_LIBS= \
        "-L/opt/hdf5/lib64 -L/opt/netcdf/lib64 -lnetcdff -lnetcdf -hdf5_hl -lhdf5 -lm -lz"
\end{verbatim}


\section{コンパイル} %\label{subsec:compile}
%-----------------------------------------------------------------------------------

\subsubsection{\scalerm のコンパイル}

\scalerm のソースディレクトリに移動し、下記のコマンドによってコンパイルを行う。
\begin{alltt}
 $ cd scale-{\version}/scale-rm/src
 $ make -j 4
\end{alltt}
\verb|make|のあとの \verb|"-j 4"| は、
コンパイル時の並列数(例では4並列)を示している。
実行環境によって並列数を指定すれば良く、推奨の並列数は 2$\sim$8 である。
コンパイルが成功すると、下記3つの実行ファイルがscale-{\version}/bin 以下に生成される。
\begin{alltt}
 scale-rm  scale-rm_init  scale-rm_pp
\end{alltt}

\subsubsection{{\scalegm}のコンパイル} %\label{subsec:compile}

{\scalegm}のソースディレクトリに移動し、下記のコマンドによってコンパイルを行う。
\begin{alltt}
  $  cd scale-{\version}/scale-gm/src
  $  make -j 4
\end{alltt}
コンパイルが成功すると、下記の実行ファイルがscale-{\version}/bin 以下に生成される。
「fio」は、ヘッダー情報を含むバイナリに基づく独自の形式である。
\begin{verbatim}
   scale-gm      (\scalegm の実行バイナリ)
   gm_fio_cat    (fio 形式のための cat コマンド)
   gm_fio_dump   (fio 形式のための dump ツール)
   gm_fio_ico2ll (fio 形式の正二十面体格子データから緯度経度格子データへの変換ツール)
   gm_fio_sel    (fio 形式のための sel コマンド)
   gm_mkhgrid    (バネ格子を用いた正二十面体(水平)格子の生成ツール)
   gm_mkllmap    (緯度経度(水平)格子の生成ツール)
   gm_mkmnginfo  (MPI プロセスの割り当てを管理するファイルの生成ツール)
   gm_mkrawgrid  (正二十面体(水平)格子の生成ツール)
   gm_mkvlayer   (鉛直格子の生成ツール)
\end{verbatim}


\subsubsection{注意点}

コンパイルをやり直したい場合には、下記のコマンドで作成された実行バイナリを消去できる。
\begin{alltt}
 $ make clean
\end{alltt}
ただし、コンパイルされたライブラリは消去されないことに注意が必要である。
全てのコンパイル済みファイルを消去したい場合は、
\begin{alltt}
 $ make allclean
\end{alltt}
とする。
コンパイル環境、コンパイルオプションを変更して再コンパイルする場合は、
``allclean''を実行すること。\\

\scalelib では、コンパイルとアーカイブは scale-{\version}/scalelib/ というディレクトリ内で行われる。
オブジェクトファイルは、コンパイルを実行したディレクトリの下の
\verb|".lib"|という名前の隠しディレクトリの中に置かれる。\\

 Debugモードでコンパイルしたい場合は、\verb|"make -j 4 SCALE_DEBUG=T"|としてコンパイルする。
 (コンパイル時に適用される全ての環境変数リストは表\ref{tab:env_var_list}を参照)
細かくコンパイルオプションを変更したい場合は、\verb|Makedef.***|のファイルを編集する。

\begin{table}[htb]
\begin{center}
\caption{コンパイル時の環境変数のリスト}
\begin{tabularx}{150mm}{|l|X|} \hline
 \rowcolor[gray]{0.9} 環境変数 & 説明  \\ \hline
 SCALE\_SYS               & システム選択(必須)  \\ \hline
 SCALE\_DISABLE\_MPI      & MPIを使わない(utilsのみ)  \\ \hline
 SCALE\_DEBUG             & デバッグ用コンパイルオプションでコンパイル  \\ \hline
 SCALE\_QUICKDEBUG        & クイックデバッグ用コンパイルオプション利用(高速化そのまま+浮動小数点エラー検出)  \\ \hline
 SCALE\_USE\_MASSCHECK    & 質量保存チェック用の計算を追加(RM力学過程のみ)  \\ \hline
% SCALE\_USE\_SINGLEFP     & 単精度浮動小数点を使用(原則として全ソース)  \\ \hline
 SCALE\_USE\_FIXEDINDEX   & 格子サイズをコンパイル時に固定して最適化促進  \\ \hline
 SCALE\_ENABLE\_OPENMP    & OpenMP機能を有効にする  \\ \hline
 SCALE\_ENABLE\_OPENACC   & OpenACC機能を有効にする  \\ \hline
 SCALE\_USE\_AGRESSIVEOPT & 副作用が出る可能のある強い最適化まで行う(京・FXのみ)  \\ \hline
 SCALE\_DISABLE\_INTELVEC & ベクトル化オプションの抑制(インテルコンパイラのみ)  \\ \hline
 SCALE\_NETCDF\_INCLUDE   & NetCDFライブラリのincludeディレクトリパス  \\ \hline
 SCALE\_NETCDF\_LIBS      & NetCDFライブラリのディレクトリパスとライブラリ指定  \\ \hline
 SCALE\_ENABLE\_PNETCDF   & parallel NetCDFを利用する  \\ \hline
 SCALE\_COMPAT\_NETCDF3   & NetCDF3互換の機能に限定する  \\ \hline
 SCALE\_ENABLE\_MATHLIB   & 数値計算ライブラリを利用する  \\ \hline
 SCALE\_MATHLIB\_LIBS     & 数値計算ライブラリのディレクトリパスとライブラリ指定  \\ \hline
 SCALE\_ENABLE\_PAPI      & PAPIを利用する  \\ \hline
 SCALE\_PAPI\_INCLUDE     & PAPIライブラリのincludeディレクトリパス  \\ \hline
 SCALE\_PAPI\_LIBS        & PAPIライブラリのディレクトリパスとライブラリ指定  \\ \hline
 SCALE\_DISABLE\_LOCALBIN & テストケースディレクトリに特別版のバイナリが作られないようにする  \\ \hline
 SCALE\_IGNORE\_SRCDEP    & コンパイル時にソースコードの依存関係確認を行わない  \\ \hline
 SCALE\_ENABLE\_SDM       & 超水滴モデルを利用する   \\ \hline
\end{tabularx}
\label{tab:env_var_list}
\end{center}
\end{table}

%%%%%%%%%%%%%%%%%%%%%%%%%%%%%%%%%%%%%%%%%%%%%%%%%%%%%%%%%%%%%%%%%%%%%%%%%%%%%%%%%%%%%%
