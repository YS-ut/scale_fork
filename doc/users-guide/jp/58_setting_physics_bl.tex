%Setting the physical process

\section{惑星境界層スキームの設定} \label{sec:basic_usel_pbl}
%------------------------------------------------------

惑星境界層(PBL)パラメタリゼーションは、惑星境界層における乱流による鉛直混合を表現するためのスキームである。
このスキームは、レイノルズ平均ナビエ-ストークス方程式の計算(RANS)において用いられる。

惑星境界層パラメタリゼーションスキームは、設定ファイル\verb|init.conf|、\verb|run.conf|において\namelist{PARAM_ATMOS}内の\nmitem{ATMOS_PHY_BL_TYPE}で指定する。
このスキームを呼び出す時間間隔は\namelist{PARAM_TIME}で設定する(詳細は第\ref{sec:timeintiv}節を参照)。

\editboxtwo{
\verb|&PARAM_ATMOS  | & \\
\verb| ATMOS_PHY_BL_TYPE = "MYNN", | & ; 表\ref{tab:nml_atm_bl}に示したスキームから選択\\
\verb|/             | & \\
}
\begin{table}[h]
\begin{center}
  \caption{惑星境界層スキームの選択}
  \label{tab:nml_atm_bl}
  \begin{tabularx}{150mm}{lXX} \hline
    \rowcolor[gray]{0.9}  値 & スキームの説明 & 参考文献\\ \hline
      \verb|OFF|          & 惑星境界層の過程を計算しない &  \\
      \verb|MYNN|         & MYNN level 2.5 boundary scheme & \citet{my_1982,nakanishi_2004} \\
    \hline
  \end{tabularx}
\end{center}
\end{table}

惑星境界層スキームは鉛直混合のみを計算する。
RANS シミュレーションにおいて水平渦粘性を考慮したい場合には、
サブグリッドスケール乱流モデルを水平混合を表現するために
用いることができる(第\ref{sec:basic_usel_turbulence}節を参照)。
