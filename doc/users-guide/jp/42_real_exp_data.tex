%-------------------------------------------------------%
\section{入力データ(境界データ)の準備} \label{sec:tutrial_real_data}
%-------------------------------------------------------%

現実大気実験のシミュレーションを行う場合、\scalerm 本体に与える
境界値データが必要になる。境界値作成のための外部入力データとして
表\ref{tab:real_bnd}のデータが必要である。
{\color{blue}青字}は必須の変数、その他は任意である。

\begin{table}[h]
\begin{center}
  \caption{現実大気実験に必要な外部入力データ}
  \label{tab:real_bnd}
  \begin{tabularx}{150mm}{llX} \hline
    \multicolumn{3}{l}{\scalerm の地形と土地利用を作成するための元データ}\\ \hline
    & \multicolumn{2}{l}{\color{blue}{標高データ}}\\
    & \multicolumn{2}{l}{\color{blue}{土地利用区分データ}}\\ \hline
    \multicolumn{3}{l}{\scalerm の初期値境界値を作成するための外部入力データ(一般的にはGCMデータ)}\\ \hline
    &  \multicolumn{2}{l}{\color{blue}{親モデルの緯度・経度情報}}\\
    &  \multicolumn{2}{l}{(3次元大気データ)}\\
    & &  \multicolumn{1}{l}{\color{blue}{東西風速、南北風速、気温、比湿(相対湿度)、気圧、ジオポテンシャル高度}} \\
    &  \multicolumn{2}{l}{(2次元大気データ)}\\
    & & 海面更正気圧、地上気圧、10m東西風速、10m南北風速、2m気温、2m比湿(相対湿度) \\
    &  \multicolumn{2}{l}{(2次元陸面データ)}\\
    & &  \multicolumn{1}{l}{親モデルの海陸マップ}\\
    & &  \multicolumn{1}{l}{\color{blue}{地表面温度(Skin temp)}}\\
    & &  \multicolumn{1}{l}{{\color{blue}{親モデル土壌データの深さ情報、土壌温度}}、土壌水分(体積含水率 or 飽和度)}\\
    &  \multicolumn{2}{l}{(2次元海面データ)}\\
  & &  \multicolumn{1}{l}{\color{blue}{海面水温(Skin tempがある場合は省略可)}}\\ \hline
  \end{tabularx}
\end{center}
\end{table}

\subsubsection{標高データと土地利用区分データ}
標高データと土地利用区分データは実験設定に従って、
\scalerm のそれぞれの格子点における標高、海陸比率、湖比率、都市被覆率、植生比率、土地(植生)利用区分を
作成するために使用する。
ユーザが全球の任意の地域を対象とした計算ができるよう、
フォーマット変換済みの
標高データ USGS(U.S. Geological Survey) のGTOPO30 と、
土地利用区分データ GLCCv2 を提供している。
%これらのデータベースを作成にするにあたり、施された前処理手順の詳細については、〇〇を参照すること(Todo)。

\begin{enumerate}
\item データのダウンロード\\
\scalerm 用の標高・土地利用区分のデータを\\
 \url{http://scale.aics.riken.jp/download/scale_database.tar.gz}\\
より入手し、任意のディレクトリに展開しておく。
展開したディレクトリには、標高データと土地利用区分データが格納されている。
\begin{alltt}
  $ tar -zxvf scale_database.tar.gz
  $ ls
    scale_database/topo/    <- 標高データ
    scale_database/landuse/ <- 土地利用区分データ
\end{alltt}

\item パスの設定\\
現実大気実験の実験に必要なファイル一式の準備には、
makeを用いた「実験セット一式作成ツール」を用いる。
このツールを利用するためには、
標高・土地利用区分データの\verb|tar|ファイルの展開先ディレクトリを、
\verb|SCALE_DB| という環境変数に設定しておくことが必須である (以後、\verb|${SCALE_DB}|と表記)。
\begin{verbatim}
  $ export SCALE_DB="${path_to_directory_of_scale_database}/scale_database"
\end{verbatim}
ここで、\verb|${path_to_directory_of_scale_database}|は、
データベースがあるディレクトリである。
\end{enumerate}



\subsubsection{大気・陸面・海面水温データ}
初期値境界値データは4byteバイナリ(\grads 形式、以降``binary形式''と表記する)に変換すれば、
任意のデータを読み込むことが可能である。
基本的に、バイナリデータはユーザ自身が用意する。
チュートリアルではNCEP FNL(Final) Operational Global Analysis data を使用する方法を示す。
あらかじめ\verb|wgrib|をインストールしておく\footnote{\url{http://www.cpc.ncep.noaa.gov/products/wesley/wgrib.html}}。

\begin{enumerate}
\item データのダウンロード\\
NCARのサイト
\url{http://rda.ucar.edu/datasets/ds083.2/}\\
から、2007年7月14日18時から一日分のgrib1フォーマットのデータ
\begin{alltt}
  fnl_20070714_18_00.grib1
  fnl_20070715_00_00.grib1
  fnl_20070715_06_00.grib1
  fnl_20070715_12_00.grib1
\end{alltt}
を\verb|${Tutorial_DIR}/real/tools/|の下にダウンロードする。

\item データフォーマットをgrib形式からbinary形式に変換\\
 \verb|${Tutorial_DIR}/real/tools/| にある\verb|convert_grib2grads_FNLgrib1.sh|を実行。

\begin{verbatim}
 $ cd ${Tutorial_DIR}/real/tools/
 $ sh convert_grib2grads_FNLgrib1.sh
\end{verbatim}
成功すれば、下記のファイルが作成される。
\begin{alltt}
 $ ls FNL_output/*/*
    FNL_output/200707/FNLatm_2007071418.grd
    FNL_output/200707/FNLatm_2007071500.grd
    FNL_output/200707/FNLatm_2007071506.grd
    FNL_output/200707/FNLatm_2007071512.grd
    FNL_output/200707/FNLland_2007071418.grd
    FNL_output/200707/FNLland_2007071500.grd
    FNL_output/200707/FNLland_2007071506.grd
    FNL_output/200707/FNLland_2007071512.grd
    FNL_output/200707/FNLsfc_2007071418.grd
    FNL_output/200707/FNLsfc_2007071500.grd
    FNL_output/200707/FNLsfc_2007071506.grd
    FNL_output/200707/FNLsfc_2007071512.grd
\end{alltt}
\end{enumerate}

