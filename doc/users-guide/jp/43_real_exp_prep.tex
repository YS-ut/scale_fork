%-------------------------------------------------------%
\section{実験セットの準備} \label{sec:tutrial_real_prep}
%-------------------------------------------------------%

現実大気実験は、理想実験よりも実行手続きや必要なファイルが多く、
プレ処理\verb|pp|、初期値作成\verb|init|、シミュレーション実行\verb|run|
で使用する設定ファイル(\verb|***.conf|)内の実験設定を統一する必要がある。
準備段階でのファイルの不足や設定の不一致はモデルが正常に動かない原因となる。
これを回避するため、現実実験の実行に必要なファイル一式を用意するためのツール
「実験セット一式作成ツール」が用意されている。
まずはじめに、
以下のディレクトリに移動し、
次の手続きにより現実大気実験チュートリアルのためのファイル一式を用意する。
\begin{alltt}
 $ cd ${Tutorial_DIR}/real/
 $ ls
    Makefile : 実験セット一式作成のためのMakefile
    README   : 実験セット一式作成ツールに関するREADME
    USER.sh  : 実験設定の記述
    config/  : 実験セット一式作成つーつのためのファイル(ユーザは基本書き換えない)
    sample/  : サンプル USER.sh スクリプト
    data/    : チュートリアルのためのツール類
    tools/   : チュートリアル用の入力大気データ用意のためのツール(基本各自で準備)
 $ make
 $ ls experiment/    : make により追加
    init/
    net2g/
    pp/
    run/
\end{alltt}
\verb|make|を実行すると、\verb|USER.sh|に記述された設定に従って、
\verb|experiment|ディレクトリの下に実験セットが作成される。
実験セット一式準備ツールに関する詳しい説明については、第\ref{sec:basic_makeconf}節を参照いただきたい。
なお、\verb|sample|ディレクトリにはネスティングの際に利用できるファイルが用意されており、
必要に応じて参考にされたい。

