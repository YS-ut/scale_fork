%-------------------------------------------------------%
\section{実験セットの準備} \label{sec:tutrial_real_prep}
%-------------------------------------------------------%

現実大気実験では、理想実験と比べて多くの実行手続きやファイルが必要である。
加えて、前処理\verb|pp|、初期値作成\verb|init|、シミュレーション実行\verb|run|
で使用する設定ファイル(\verb|***.conf|)内の実験設定を統一する必要がある。
準備段階におけるファイルの不足や設定の不一致は、モデルが正常に動かない原因となる。
このような状況を回避するために、必要なファイル一式を生成するためのツール
「実験セット一式作成ツール」が用意されている。
まずはじめに、
以下のディレクトリに移動し、
次の手続きにより現実大気実験チュートリアルのためのファイル一式を用意する。
\begin{alltt}
 $ cd ${Tutorial_DIR}/real/
 $ ls
    Makefile : 実験セット一式作成のためのMakefile
    README   : スクリプトの使用に関する README
    USER.sh  : 実験設定の記述
    config/  : 一連のファイルの作成に対する各々の設定
              (基本的に、ユーザは書き換える必要はない)
    sample/  : USER.sh のサンプルスクリプト
    data/    : チュートリアルのためのツール類
    tools/   : チュートリアル用の初期条件のためのツール
              (チュートリアルの場合を除いて、基本的に各自で準備する)
 $ make
 $ ls experiment/    : このディレクトリは make により追加される
    init/
    net2g/
    pp/
    run/
\end{alltt}
\verb|make|を実行すると、\verb|USER.sh|に記述された設定に従って
\verb|experiment|ディレクトリの下に実験セットが作成される。
実験セット一式準備ツールに関する詳しい説明については、
第\ref{sec:basic_makeconf}節を参照いただきたい。
%なお、\verb|sample|ディレクトリにはネスティングの際に利用できるファイルが用意されており、
%必要に応じて参考にされたい。

