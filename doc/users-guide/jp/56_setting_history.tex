%-------------------------------------------------------------------------------
\section{ヒストリファイルと出力変数の設定} \label{sec:output}
%-------------------------------------------------------------------------------

ヒストリファイルと出力変数は、\verb|run.conf|内の\namelist{PARAM_FILE_HISTORY_CARTESC}、
\namelist{PARAM_FILE_HISTORY}、\namelist{HISTORY_ITEM}によって設定する。
ヒストリファイルのデフォルトの形式は、\namelist{PARAM_FILE_HISTORY}によって設定する。

\editboxtwo{
\verb|&PARAM_FILE_HISTORY_CARTESC                   | & \\
\verb|  FILE_HISTORY_CARTESC_PRES_nlayer = -1,      | & ; 圧力レベル数 \\
                                                      & ~ (圧力レベルへの補間に関するオプション) \\
\verb|  FILE_HISTORY_CARTESC_PRES        = 0.D0     | & ; 補間を行う圧力レベル(下層から上層の順) [hPa] \\
                                                      & ~ (圧力レベルへの補間に関するオプション) \\
\verb|  FILE_HISTORY_CARTESC_BOUNDARY    = .false., | & ; ハロのデータを出力するか? \\
                                                      & ~ \verb|.true.|: 出力する, \verb|.false.|: 出力しない.\\
\verb|/                                             | & \\
}

\editboxtwo{
\verb|&PARAM_FILE_HISTORY                                      | & \\
\verb| FILE_HISTORY_TITLE                     = "",            | & ; データに関する簡単な説明 (\ref{sec:netcdf}節参照)\\
\verb| FILE_HISTORY_SOURCE                    = "",            | & ; データを作成したソフトウェアの名前  (\ref{sec:netcdf}節参照)\\
\verb| FILE_HISTORY_INSTITUTION               = "",            | & ; データの作成者 (\ref{sec:netcdf}節参照)\\
\verb| FILE_HISTORY_TIME_UNITS                = "seconds",     | & ; \netcdf 中の時間軸の単位\\
\verb| FILE_HISTORY_DEFAULT_BASENAME          = "history_d01", | & ; 出力ファイルのベース名 \\
\verb| FILE_HISTORY_DEFAULT_POSTFIX_TIMELABEL = .false.,       | & ; ファイル名に時間のラベルを加えるか? \\
\verb| FILE_HISTORY_DEFAULT_ZCOORD            = "model",       | & ; 鉛直座標の種類 \\
\verb| FILE_HISTORY_DEFAULT_TINTERVAL         = 3600.D0,       | & ; ヒストリ出力の時間間隔 \\
\verb| FILE_HISTORY_DEFAULT_TUNIT             = "SEC",         | & ; \verb|DEFAULT_TINTERVAL|の単位 \\
\verb| FILE_HISTORY_DEFAULT_TAVERAGE          = .false.,       | & ; 出力の時間間隔中の値を平均するか? \\
\verb| FILE_HISTORY_DEFAULT_DATATYPE          = "REAL4",       | & ; 出力データの種類: \verb|REAL4| or \verb|REAL8| \\
\verb| FILE_HISTORY_OUTPUT_STEP0              = .true.,        | & ; 初期時刻(t=0)のデータを出力するか? \\
\verb| FILE_HISTORY_OUTPUT_WAIT               = 0.D0,          | & ; 出力を抑制する時間 \\
\verb| FILE_HISTORY_OUTPUT_WAIT_TUNIT         = "SEC",         | & ; \verb|OUTPUT_WAIT| の単位 \\
\verb| FILE_HISTORY_OUTPUT_SWITCH_TINTERVAL   = -1.D0,         | & ; ファイルを切り替える時間間隔 \\
\verb| FILE_HISTORY_OUTPUT_SWITCH_TUNIT       = "SEC",         | & ; \verb|OUTPUT_SWITCH_TINTERVAL| の単位 \\
\verb| FILE_HISTORY_ERROR_PUTMISS             = .true.,        | & ; データの準備状況の整合性を確認するか? \\
\verb| FILE_HISTORY_AGGREGATE                 = .false.,       | & ; PnetCDF を用いて単一のファイルに集めるか? \\
\verb|/                                                        | & \\
}

デフォルトでは、各プロセスがヒストリファイルを出力する。
\nmitem{FILE_HISTORY_AGGREGATE}が\verb|.true.|に設定されている場合は, 
分散された出力ファイルは parallel \Netcdf を用いることによって単一のファイルへと集められる。
\nmitem{FILE_HISTORY_AGGREGATE}のデフォルト設定は、\namelist{PARAM_FILE}内の\nmitem{FILE_AGGREGATE}によって決定される。
\ref{sec:netcdf}節を参考にされたい。

\nmitem{FILE_HISTORY_DEFAULT_TINTERVAL}はヒストリ出力の時間間隔であり、その単位は\\
\nmitem{FILE_HISTORY_DEFAULT_TUNIT}によって定義される。単位は、\\
\verb|"MSEC", "msec", "SEC", "sec", "s", "MIN", "min", "HOUR", "hour", "h", "DAY", "day"|から選択することができる。
%
\nmitem{FILE_HISTORY_DEFAULT_TAVERAGE}を\verb|.true.|として平均値の出力が設定されている場合には、
\nmitem{FILE_HISTORY_DEFAULT_TINTERVAL}に指定された直近の期間に渡って平均されたヒストリデータを出力する。

ヒストリ出力の時間間隔は、それと関係したスキームの時間間隔と等しいか倍数でなければならない。
この整合性の確認を無効にしたい場合には、\nmitem{FILE_HISTORY_ERROR_PUTMISS}を\verb|.false.|に設定すれば良い.

\nmitem{FILE_HISTORY_DEFAULT_POSTFIX_TIMELABEL}が\verb|.true.|に設定されている場合には, 
時間に関するラベルが出力ファイル名に付加される。
時間のラベルは、シミュレーションの現時刻から生成され、その形式は\verb|YYYYMMDD-HHMMSS.msec|によって定義される。

\nmitem{FILE_HISTORY_OUTPUT_STEP0}が\verb|.true.|に設定されている場合は, 
時間積分の前の時刻における変数(初期値)をヒストリファイルに出力する。
\nmitem{FILE_HISTORY_OUTPUT_WAIT}と\nmitem{FILE_HISTORY_OUTPUT_WAIT_TUNIT}によって定義されるシミュレーション時間の間、
その値が負であれば、出力の抑制は起こらない。
\nmitem{FILE_HISTORY_OUTPUT_SWITCH_TINTERVAL}は出力ファイルの切り替えの時間間隔であり、
その単位は\nmitem{FILE_HISTORY_OUTPUT_SWITCH_TUNIT}によって定義される。
その値が負であれば、ヒストリ出力のために各プロセス毎に単一ファイルだけが使われる。
このオプションが有効であれば、時間に関するラベルがファイル名に付加される。

大気の3次元変数を出力するために、3種類の鉛直座標が利用可能である。
デフォルトでは\\
\nmitem{FILE_HISTORY_DEFAULT_ZCOORD} \verb|= "model"|と選択される。
この場合には、変数はモデルのもとの座標系({\scalerm}において地形に沿った、z*座標系)を用いて出力される。
\nmitem{FILE_HISTORY_DEFAULT_ZCOORD}が\verb|"z"|に設定されている場合は、 変数は絶対高度へと補間される。
出力データのレベル数は、モデルのレベル数と同じである。
各レベルにおける高度は、地形を伴わない格子セルにおけるモデル高度と同じである。
\nmitem{FILE_HISTORY_DEFAULT_ZCOORD}が\verb|"pressure"|に設定されている場合には、
変数は圧力レベルへと補間される。
この場合には、\namelist{PARAM_FILE_HISTORY_CARTESC}内の\nmitem{FILE_HISTORY_CARTESC_PRES_nlayer}と\nmitem{FILE_HISTORY_CARTESC_PRES}を設定する必要がある。

\namelist{PARAM_FILE_HISTORY_CARTESC}内の\nmitem{FILE_HISTORY_CARTESC_BOUNDARY}が\verb|.true.|である場合には、
周期境界条件の場合を除いて、ハロ(対象領域の外側に位置する)におけるデータも出力される。
\nmitem{FILE_HISTORY_CARTESC_BOUNDARY}の設定は、全ての出力変数に適用される。\\

出力変数は、\namelist{HISTORY_ITEM}を加えることによって設定される。
出力の形式は、\namelist{PARAM_FILE_HISTORY}で指定されたデフォルト設定に従う。
「(オプション)」を伴うネームリストを追加することによって、特定の変数に対する形式をデフォルト設定から変更することができる。

\editboxtwo{
\verb|&HISTORY_ITEM                    | & \\
\verb| NAME              = "RAIN",     | &  変数名. 変数のリストはリファレンスマニュアル内にある(\ref{sec:reference_manual}節を参照) \\
\verb| OUTNAME           = "",         | &  (オプション) \verb|NAME|と同じ \\
\verb| BASENAME          = "rain_d01", | &  (オプション) \verb|FILE_HISTORY_DEFAULT_BASENAME|と同じ \\
\verb| POSTFIX_TIMELABEL = .false.,    | &  (オプション) \verb|FILE_HISTORY_DEFAULT_POSTFIX_TIMELABEL|と同じ \\
\verb| ZCOORD            = "model",    | &  (オプション) \verb|FILE_HISTORY_DEFAULT_ZCOORD|と同じ \\
\verb| TINTERVAL         = 600.D0,     | &  (オプション) \verb|FILE_HISTORY_DEFAULT_TINTERVAL|と同じ \\
\verb| TUNIT             = "SEC",      | &  (オプション) \verb|FILE_HISTORY_DEFAULT_TINTERVAL|と同じ \\
\verb| TAVERAGE          = .true.,     | &  (オプション) \verb|FILE_HISTORY_DEFAULT_TAVERAGE|と同じ \\
\verb| DATATYPE          = "REAL4",    | &  (オプション) \verb|FILE_HISTORY_DEFAULT_DATATYPE|と同じ \\
\verb|/                                | & \\
}

\namelist{HISTORY_ITEM}によって要求された変数がシミュレーションの時間スッテップ中に準備されていない場合には、
実行が停止し、エラーログがログファイルに書かれる。
この状況は、\nmitem{NAME}にスペルミスがある場合や、要求した変数が選択したスキーム内で用いられていない場合に起こりうる。

「(オプション)」と書かれたネームリストは、変数\nmitem{NAME}に対してのみ適用される。
変数に対してデフォルト設定を用いる場合には、「(オプション)」と書かれたネームリストは省略することができる。
例えば、\namelist{PARAM_FILE_HISTORY}の上記の設定を維持して、\namelist{HISTORY_ITEM}に対する以下の設定を付け加えるとしよう。
ファイル\verb|history_d01.xxxxxx.nc|には、\verb|U and V|の瞬間値が 3600 s 間隔で 4バイトの実数値として格納される一方で、
\verb|RAIN|については 600 秒間隔でその期間に渡った平均値がファイルに格納される。
\verb|T|の値は \verb|U|や\verb|V|と同じ規則において\verb|T|として出力され、
圧力座標系に補間した値は\verb|T_pres|として出力される。

\editbox{
\verb|&HISTORY_ITEM  NAME="T" /|\\
\verb|&HISTORY_ITEM  NAME="U" /|\\
\verb|&HISTORY_ITEM  NAME="V" /|\\
\verb|&HISTORY_ITEM  NAME="RAIN", TINTERVAL=600.D0, TAVERAGE=.true. /|\\
\verb|&HISTORY_ITEM  NAME="T", OUTNAME="T_pres", ZCOORD="pressure" /|\\
}

%%%%%%%%%%%%%%%%%%%%%%%%%%%%%%%%%%%%%%%%%%%%%%%%%%%%%%%%%%%%%%%%%%%%%%%%%%%%%%%%%%%%
