\section{\SecAdvanceRestart}\label{sec:restart}
%=======================================================================

計算機のジョブ実行時間に制限がある場合や、
長期積分時に途中で計算が異常終了してしまった場合など、
リスタート計算を行うことで、ひと続きの計算を分割して実行するが可能である。
リスタート計算の手続きは、下記の2ステップに分けられる。
\begin{enumerate}
\item リスタート計算のためのリスタートファイルを出力する。
\item リスタートファイルを読んで、続きの計算(リスタート計算)を行う。
\end{enumerate}
もちろん、リスタート計算時に次の計算のためのリスタートファイルを出力することも可能である。
リスタートファイルの出力とリスタート計算実行のための設定は、設定ファイル\verb|run_***.conf|の
\namelist{PARAM_RESTART}と\namelist{PARAM_TIME}で設定する。\\

\noindent {\small {\gt
\ovalbox{
\begin{tabularx}{150mm}{lX}
\verb|&PARAM_RESTART| & \\
\verb| RESTART_RUN          = .true.,                                      | & リスタート計算かどうかの指定。\\
                                                                             & 明記されていない場合は\verb|.false.|が設定される。\\
                                                                             & \verb|.true.|: リスタート計算、\verb|.false.|: 通常計算。  \\
                                                                             & \\ 
\multicolumn{2}{l}{\verb| RESTART_IN_BASENAME  = "restart1_d01_20070714-180000.000",|}\\
                                                                             & 入力する初期値ファイルの指定。\\
                                                                             & \\ 
\verb| RESTART_OUTPUT       = .true.,                                      | & リスタートファイルを出力するかどうか。\\
                                                                             & \verb|.true.|: 出力する、\verb|.false.|: 出力しない。\\
\verb| RESTART_OUT_BASENAME = "restart2_d01",                              | & リスタートファイルのファイル名の頭。\\
                                                                             & この後ろに、出力時の年月日時刻が追加される。\\
\verb|/| & \\
\\
\verb|&PARAM_TIME| & \\
\verb| TIME_DT_ATMOS_RESTART      = 21600.D0, | & リスタートファイル(大気)の出力間隔\\
\verb| TIME_DT_ATMOS_RESTART_UNIT = "SEC",    | & \verb|TIME_DT_ATMOS_RESTART|の単位\\
\verb| TIME_DT_OCEAN_RESTART      = 21600.D0, | & リスタートファイル(海洋)の出力間隔\\
\verb| TIME_DT_OCEAN_RESTART_UNIT = "SEC",    | & \verb|TIME_DT_OCEAN_RESTART|の単位\\
\verb| TIME_DT_LAND_RESTART       = 21600.D0, | & リスタートファイル(陸面)の出力間隔\\
\verb| TIME_DT_LAND_RESTART_UNIT  = "SEC",    | & \verb|TIME_DT_LAND_RESTART|の単位\\
\verb| TIME_DT_URBAN_RESTART      = 21600.D0, | & リスタートファイル(都市)の出力間隔\\
\verb| TIME_DT_URBAN_RESTART_UNIT = "SEC",    | & \verb|TIME_DT_URBAN_RESTART|の単位\\
\verb|/| & \\
\end{tabularx}
}}}\\

上記の例では、\verb|restart1_***|という入力ファイルからリスタート計算を開始し、
6時間(21600 sec)に1回、リスタートファイルを\verb|restart2_***|という名前で出力する設定となっている。
\nmitem{TIME_DT_ATMOS_RESTART, TIME_DT_OCEAN_RESTART,TIME_DT_LAND_RESTART,TIME_DT_URBAN_RESTART}
が指定されていない場合には、積分時刻の最終時刻\nmitem{TIME_DURATION}にファイルが作成される。
\nmitem{RESTART_RUN}を\verb|.false.|として\nmitem{RESTART_IN_BASENAME}に\verb|scale-rm_init|で作成した初期値ファイル
(\verb|init_***|)を指定すれば通常計算が実行され、
\nmitem{RESTART_RUN}を\verb|.true.|として\nmitem{RESTART_IN_BASENAME}にリスタートファイルを指定すれば
リスタート計算が実行される。
なお、\scalerm は\nmitem{RESTART_IN_BASENAME}で指定されているファイル名でリスタート計算かどうかを判断しているわけではないので、ファイル名はなんでも良い。
チュートリアル、および、本書解説では便宜上、\verb|scale-rm_init|で作成した初期値ファイルの名前を\verb|init_***|とし、リスタートファイルのファイル名を\verb|restart_***|としているが、これに限らない。


現実大気実験の場合、初期値以外に\verb|scale-rm_init|で作成した境界値ファイルが必要である。\\
\namelist{PARAM_ATMOS_BOUNDARY}の\nmitem{ATMOS_BOUNDARY_START_DATE}に
境界値ファイル\verb|boundary_***|の初期時刻を設定しておくことで、リスタート計算時に
適切な時刻のデータを読み込むことが可能である。
\nmitem{ATMOS_BOUNDARY_START_DATE}の設定がない場合には、
境界値ファイル\verb|boundary_***|の最初のデータから読み込まれてしまうので注意が必要である。\\

\noindent {\small {\gt
\ovalbox{
\begin{tabularx}{150mm}{lX}
\verb|&PARAM_ATMOS_BOUNDARY| & \\
\verb| ATMOS_BOUNDARY_TYPE           = "REAL",                            | & 現実実験の場合は\verb|"REAL"|。\\
\verb| ATMOS_BOUNDARY_IN_BASENAME    = "../init/output/boundary_d01",     | & 境界値データのファイル名の頭。\\
\verb| ATMOS_BOUNDARY_START_DATE     = 2010, 7, 14, 18, 0, 0,             | & 境界値データの初期時刻。\\
\verb| ATMOS_BOUNDARY_UPDATE_DT      = 21600.D0,                          | & 境界値データのデータ間隔。\\
\verb|/| & \\
\end{tabularx}
}}}\\

%\nmitem{ATMOS_BOUNDARY_START_DATE}と\nmitem{ATMOS_BOUNDARY_UPDATE_DT}は、
%今後、上記で指定せず、境界値ファイル(\netcdf)から直接読み込むように変更予定である。

\proofcomment{予定はかかなくていいのではないか?要検討}\\
\replycomment{削除しました(足立)}
