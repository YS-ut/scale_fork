\section{\SecBasicOutputSetting} \label{sec:output}
%====================================================================================

\proofcomment{1の手続き方法を書かないのであれば、1は、削除です。ソースコードレベルを
いじるらせるのかどうかは、debatableですが、このままだと、1に書いてある意味が何の情報にもなりません。}\\
\replycomment{削除しました(足立)}


計算結果の出力ファイルと出力形式の設定、及び、出力する変数の追加は、
\namelist{PARAM_HISTORY}と\namelist{HISTITEM}で行う。
まず、出力ファイルと出力形式の設定を、\verb|run.**.conf|の\namelist{PARAM_HISTORY} で行う。\\

\noindent {\small {\gt
\ovalbox{
\begin{tabularx}{150mm}{lX}
\verb|&PARAM_HISTORY| & \\
\verb|  HISTORY_DEFAULT_BASENAME  = "history_d01",| & 出力ファイル名の頭。 \\
\verb|  HISTORY_DEFAULT_TINTERVAL = 3600.0,|       & 出力の時間間隔。 \\
\verb|  HISTORY_DEFAULT_TUNIT     = "SEC",|         & \verb|HISTORY_DEFAULT_TINTERVAL|の単位。 \\
\verb|  HISTORY_DEFAULT_TAVERAGE  = .false.,|       & \verb|.false.|: 瞬間値、\verb|.true.|: 平均値。\\
\verb|  HISTORY_DEFAULT_ZINTERP   = .false.,|       & 鉛直内挿するかどうか。\\
                                                    & \verb|.false.|: モデル面出力。\\
                                                    & \verb|.true.|: Z面に内挿した値を出力。\\
\verb|  HISTORY_DEFAULT_DATATYPE  = "REAL4",|       & 出力データの型。\verb|REAL4|, \verb|REAL8|など。\\
\verb|  HISTORY_OUTPUT_STEP0      = .true.,|        & 初期時刻(t=0)の値を出力するかどうか。\\
                                                    & \verb|.true.|: 出力、\verb|.false.|: 出力しない。\\ 
\verb|/| & \\
\end{tabularx}
}}}\\

\verb|HISTORY_DEFAULT_TAVERAGE = .true.|として、平均値での出力を設定した場合、
出力するタイミング直前の\verb|HISTORY_DEFAULT_TINTERVAL|間の平均値が出力される。
次に、出力する変数の設定を\namelist{HISTITEM}で行う。
\namelist{HISTITEM}は変数毎に設定するため、出力したい変数の数だけ追加することになる。
また、それぞれの変数の出力形式は、基本的に \namelist{PARAM_HISTORY} の設定に準ずるが、
変数毎に設定を変更することも可能である。\\

\noindent {\small {\gt
\ovalbox{
\begin{tabularx}{150mm}{lX}
\verb|&HISTITEM| &\\
\verb| ITEM     = "RAIN",    | &  変数名。 出力可能な変数は付録\ref{achap:histroy_item}を参照 \\
\verb| BASENAME = "rain_d01",| & (オプション) \verb|HISTORY_DEFAULT_BASENAME|に同じ。但し、変数(\verb|ITEM|)のみに適用 \\
\verb| TINTERVAL= "600.0",   | & (オプション) \verb|HISTORY_DEFAULT_TINTERVAL|に同じ。但し、変数(\verb|ITEM|)のみに適用 \\
\verb| TUNIT    = "SEC",     | & (オプション) \verb|HISTORY_DEFAULT_TINTERVAL|に同じ。但し、変数(\verb|ITEM|)のみに適用 \\
\verb| TAVERAGE = .true.,    | & (オプション) \verb|HISTORY_DEFAULT_TAVERAGE|に同じ。但し、変数(\verb|ITEM|)のみに適用\\
\verb| ZINTERP  = .false.,   | & (オプション) \verb|HISTORY_DEFAULT_DATATYPE|に同じ。但し、変数(\verb|ITEM|)のみに適用。\\
\verb| DATATYPE = "REAL4",   | & (オプション) \verb|HISTORY_DEFAULT_ZINTERP|に同じ。但し、変数(\verb|ITEM|)のみに適用\\
\verb|/| & \\
\end{tabularx}
}}}\\

上記では、明示的にすべての設定を書いているが、\nmitem{HISTORY_DEFAULT_***} と同じ設定であれば
それらが適用されるので明記する必要はない。
例えば、上記の\namelist{PARAM_HISTORY}の設定に、下記を組み合わせた場合には、
\verb|history_d01.xxxxxx.nc|に4バイト実数で、3600秒毎に \nmitem|HISTITEM| で指定された変数の瞬間値が出力される。
ただし、\verb|RAIN|については、600秒の出力間隔で、前600秒の平均値として出力されることを意味している。\\

\noindent {\small {\gt
\ovalbox{
\begin{tabularx}{140mm}{l}
\verb|&HISTITEM  ITEM = "T" /|
\verb|&HISTITEM  ITEM = "U" /|
\verb|&HISTITEM  ITEM = "V" /|
\verb|&HISTITEM  ITEM = "RAIN",  TINTERVAL = "600.0", TAVERAGE = .true. /|
\end{tabularx}
}}}\\

%%%%%%%%%%%%%%%%%%%%%%%%%%%%%%%%%%%%%%%%%%%%%%%%%%%%%%%%%%%%%%%%%%%%%%%%%%%%%%%%%%%%
