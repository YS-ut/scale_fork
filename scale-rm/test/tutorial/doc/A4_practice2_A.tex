%\chapter{Q \& A}

\clearpage
\section*{回答}
\begin{enumerate}
\item {\bf 計算領域は変えず、MPI並列数を変更}\\
変更箇所は、\namelist{PARAM_INDEX}内の\nmitem{IMAX}、\nmitem{JMAX}、\namelist{PARAM_PRC}内の\nmitem{PRC_NUM_X}、\nmitem{PRC_NUM_Y}である。下記2つの式を満たしていれば正解である。
\begin{eqnarray}
&& MPI並列数 = (\verb|PRC_NUM_X|) \times (\verb|PRC_NUM_Y|) = 6 \nonumber\\
&& 領域内の総格子数 = \left(\verb|IMAX| \times \verb|PRC_NUM_X|\right)
   \times (\verb|JMAX| \times \verb|PRC_NUM_Y|) = 8100 \nonumber
\end{eqnarray}


\item {\bf MPI並列数は変えず、計算領域を変更}\\
MPIプロセスあたりの格子数を$n$倍にすれば、
領域全体の格子数も$n$倍となる。
変更箇所は、\namelist{PARAM_INDEX}内の\nmitem{IMAX}、\nmitem{JMAX}のみである。
赤文字の部分がデフォルトからの変更点を意味する。\\

\noindent {\small {\gt
\ovalbox{
\begin{tabularx}{140mm}{ll}
\verb|&PARAM_INDEX| & \\
\verb| KMAX = 36,|  & \\
\textcolor{red}{\verb| IMAX = 60,|}  & (オリジナル設定は \verb|IMAX = 45|)\\
\textcolor{red}{\verb| JMAX = 30,|}  & (オリジナル設定は \verb|JMAX = 45|)\\
\verb|/| & \\
\end{tabularx}
}}}\\

\item {\bf 計算領域は変えず、水平格子間隔を変更}\\
MPI並列数を変えない場合、変更箇所は\namelist{PARAM_GRID}の\nmitem{DX, DY}と
\namelist{PARAM_INDEX}内の\nmitem{IMAX}、\nmitem{JMAX}である。

\noindent {\small {\gt
\ovalbox{
\begin{tabularx}{140mm}{ll}
\verb|&PARAM_PRC|  & \\
\verb| PRC_NUM_X      = 2,|  & \\
\verb| PRC_NUM_Y      = 2,|  & \\
\\
\verb|&PARAM_INDEX| & \\
\verb| KMAX = 36,|  & \\
\textcolor{red}{\verb| IMAX = 180,|} &  (オリジナル設定は \verb|IMAX = 45|)\\
\textcolor{red}{\verb| JMAX = 180,|} &  (オリジナル設定は \verb|JMAX = 45|)\\
\verb|/| &\\
 \\
\verb|&PARAM_GRID| & \\
\textcolor{red}{\verb| DX = 5000.D0,|} & (オリジナル設定は \verb|DX = 20000.D0|)\\
\textcolor{red}{\verb| DY = 5000.D0,|} & (オリジナル設定は \verb|DY = 20000.D0|)\\
\verb|/| & \\
\end{tabularx}
}}}\\

MPI並列数も変更している場合には、下記の関係を満たしていれば正解である。
\begin{eqnarray}
&& {\XDIR} の格子数 = \left(\verb|IMAX| \times \verb|PRC_NUM_X|\right) = 360 \\\nonumber
&& {\YDIR}の格子数 = \left(\verb|JMAX| \times \verb|PRC_NUM_Y|\right) = 360 \nonumber
\end{eqnarray}

これに加えて、力学変数の時間積分のためのタイムステップ
\nmitem{TIME_DT_ATMOS_DYN} と\nmitem{ATMOS_DYN_TINTEG_SHORT_TYPE} の調整も必要である
(第\ref{sec:timeintiv}節参照)。

緩和領域も、格子間隔の20倍から40倍となるよう、変更してもよい。。。。。

\noindent {\small {\gt
\ovalbox{
\begin{tabularx}{140mm}{ll}
\verb|&PARAM_PRC|  & \\
\verb| BUFFER_DX = 300000.D0, | & \\
\verb| BUFFER_DY = 300000.D0, | & \\
\verb|/| &\\
\end{tabularx}
}}}\\

\begin{itemize}
\item {\bf 計算領域の位置を変更}
\item {\bf 積分時間の変更}
\item init.confの説明も必要
\item {\bf 出力変数の追加と出力時間間隔の変更}
\item {\bf リスタート計算をする}
\end{itemize}


\end{enumerate}

