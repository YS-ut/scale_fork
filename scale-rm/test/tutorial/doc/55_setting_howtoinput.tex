\section{\SecAdvanceInputDataSetting} \label{sec:adv_datainput}
%====================================================================================

%\subsection{初期値・境界値データ} \label{sec:adv_bnddata}
%------------------------------------------------------
現在、\scalerm では下記のデータの読み込みとそれらに基づく初期値・境界値作成に対応している。

\begin{table}[htb]
\begin{center}
\caption{\scalelib が読込に対応する外部入力データフォーマット}
\begin{tabularx}{150mm}{|l|l|X|} \hline
 \rowcolor[gray]{0.9} データ形式 & \verb|FILETYPE_ORG| & 備考 \\ \hline
 バイナリデータ & \verb|GrADS| & データ読み込み用のnamelistを別途必要とする。 \\ \hline
 NICAMデータ &  \verb|NICAM-NETCDF| & NetCDF形式のLatLonデータに対応する。 \\ \hline
 WRFデータ &  \verb|WRF-ARW| & ``wrfout''、``wrfrst''の両方に対応する。 \\ \hline
 SCALEデータ &  \verb|SCALE-RM| & historyデータのみ対応;latlonカタログを必要とする。 \\ \hline
\end{tabularx}
\label{tab:inputdata_format}
\end{center}
\end{table}

これらの使い分けは、初期値・境界値作成時、すなわち``scale-rm\_init''の実行時の設定ファイルの
\namelist{PARAM_MKINIT_REAL}の項目中の\nmitem{FILETYPE_ORG}に表\ref{tab:inputdata_format}に示した設定値を指定することで使い分ける。

バイナリデータとは、「4バイト単精度
浮動小数点のダイレクトアクセス方式、Fortran型バイナリデータ」を指す。その主な使用方法は、
第\ref{chap:tutorial_real}章の現実大気実験チュートリアルで説明したとおりである。\textcolor{red}{GRIB/GRIB2のデータ形式は、
チュートリアルで説明した方法に基づいて、バイナリデータ形式を経由してSCALEに読み込ませることができる。}
その他に任意のデータを境界値に使用したい場合は、バイナリデータ形式に変換することで読み込ませることができる。

SCALEデータ形式は主にオフライン・ネスティング実験で使用される。詳細については、\ref{subsec:nest_offline}節を
参照されたい。NICAMデータは、正20面体格子データではなく、緯度・経度座標に変換されたデータの
み読み込みに対応している。WRFデータについてはモデル出力データをそのまま使用することができる。
%これらの読み込み方法に関しては随時説明を加えていく予定。

%%%%%%%%%%%%%%%%%%%%%%%%%%%%%%%%%%%%%%%%%%%%%%%%%%%%%%%%%%%%%%%%%%%%%%%%%%%%%%%%%%%%%%
