\section{力学スキームの設定}
\label{sec:atmos_dyn}
%------------------------------------------------------


\subsection{数値解法}
%------------------------------------------------------
設定は、confファイルの\verb|PARAM_ATMOS|の\verb|ATMOS_DYN_TYPE|で設定する。\\

\noindent {\gt\small
\ovalbox{
\begin{tabularx}{140mm}{ll}
\verb|&PARAM_ATMOS  | & \\
\verb| ATMOS_DYN_TYPE    = "HEVI", | & ; 表\ref{tab:nml_dyn}より選択。\\
\verb|/             | & \\
\end{tabularx}
}}\\

\begin{table}[h]
\begin{center}
  \caption{数値解法の選択肢}
  \label{tab:nml_dyn}
  \begin{tabularx}{150mm}{llX} \hline
    \rowcolor[gray]{0.9}  設定名 & スキームの説明 & 備考\\ \hline
      \verb|HEVE|  & 水平陽解法-鉛直陽解法 & \\
      \verb|HEVI|  & 水平陽解法-鉛直陰解法 & 推奨(特に実大気実験)\\
    \hline
  \end{tabularx}
\end{center}
\end{table}


\subsection{時間・空間差分スキーム}
\label{sec:atmos_dyn_scheme}
%------------------------------------------------------
設定は、confファイルの\verb|PARAM_ATMOS_DYN|で設定する。
また、空間スキームによってはHaloの設定も必要である。
下記の設定は推奨設定である。
その他、スキームと各種設定については表\ref{tab:nml_atm_dyn}を参照のこと。
また、時間スキーム変更時には、タイムステップも考慮する必要があるが、
それについては、第\ref{sec:timeintiv}節をを参照いただきたい。\\

\noindent {\gt\small
\ovalbox{
\begin{tabularx}{140mm}{ll}
 \verb|&PARAM_ATMOS_DYN  | & \\
 \verb|ATMOS_DYN_TINTEG_SHORT_TYPE          = RK4,|       & ; 時間スキームより選択\\
 \verb|ATMOS_DYN_TINTEG_TRACER_TYPE         = RK3WS2002,| & ; 時間スキームより選択\\
 \verb|ATMOS_DYN_FVM_FLUX_TYPE              = CD4,|       & ; 空間スキームより選択\\
 \verb|ATMOS_DYN_FVM_FLUX_TRACER_TYPE       = UD3KOREN1993,| & ; 空間スキームより選択\\
 \verb|ATMOS_DYN_FLAG_FCT_TRACER            = .false.,|   & ; FCTスキームを利用するかどうか\\
 \verb|ATMOS_DYN_NUMERICAL_DIFF_COEF        = 1.D-2, |    & \\
 \verb|ATMOS_DYN_NUMERICAL_DIFF_COEF_TRACER = 0.D0, |     & \\
 \verb|ATMOS_DYN_enable_coriolis            = .true.,|    & \\
\verb|/             | & \\
\end{tabularx}
}}\\



\begin{table}[h]
\begin{center}
  \caption{力学スキームの設定}
  \label{tab:nml_atm_dyn}
  \begin{tabularx}{150mm}{llXX} \hline
    \rowcolor[gray]{0.9} & \multicolumn{1}{l}{設定名} & \multicolumn{1}{l}{スキーム名} & \\ \hline
    \multicolumn{3}{l}{時間スキーム} &  \\ \hline
    & \multicolumn{1}{l}{\verb|RK3|} & \multicolumn{1}{l}{Heun's 3rd order Runge-Kutta scheme} & \\
    & \multicolumn{1}{l}{\verb|RK3WS2002|} & \multicolumn{1}{X}{Wicker and Skamarock (2002) 3段ルンゲクッタスキーム} & \\
    & \multicolumn{1}{l}{\verb|RK4|} & \multicolumn{1}{l}{4th order Runge-Kutta scheme} & \\
    \hline
    \multicolumn{3}{l}{空間スキーム} & 最小の設定格子数(Haloの数)\\ \hline
    & \multicolumn{1}{l}{\verb|CD2|} & \multicolumn{1}{l}{2-nd order central difference schemes} & \multicolumn{1}{l}{1}\\
    & \multicolumn{1}{l}{\verb|CD4|} & \multicolumn{1}{l}{4-th order central difference schemes} & \multicolumn{1}{l}{2}\\
    & \multicolumn{1}{l}{\verb|CD6|} & \multicolumn{1}{l}{6-th order central difference schemes} & \multicolumn{1}{l}{3}\\
    & \multicolumn{1}{l}{\verb|UD3|} & \multicolumn{1}{l}{3-rd order upwind difference schemes} & \multicolumn{1}{l}{2}\\
    & \multicolumn{1}{l}{\verb|UD5|} & \multicolumn{1}{l}{5-th order upwind difference schemes} & \multicolumn{1}{l}{3}\\
    & \multicolumn{1}{l}{\verb|UD3KOREN1993|} & \multicolumn{1}{X}{3-rd order upwind difference schemes with Koren (1993) filter} & \\
\hline
  \end{tabularx}
\end{center}
\end{table}

トレーサー移流については、何らかの非負保証スキームを使う事が望ましい。\\
つまり \verb|ATMOS_DYN_FVM_FLUX_TRACER_TYPE| として \verb|UD3KOREN1993| 以外を選択した場合は、\\
\verb|ATMOS_DYN_FLAG_FCT_TRACER| を \verb|.true.| としてFCTを利用するのがよい。


空間スキームがCD4, UD3以外の場合(Haloの数が2以外の場合)、Haloの設定も必要である。
Haloの数は、n次の空間スキームに対して$\lceil n/2 \rceil$である。
ここで、$\lceil$ $\rceil$ は天井関数である。\\

\noindent {\gt\small
\ovalbox{
\begin{tabularx}{140mm}{ll}
 \verb|&PARAM_INDEX | &  \\
 \verb| IHALO = 2,|   & ; $\lceil n/2 \rceil$を設定\\
 \verb| JHALO = 2,|   & ; $\lceil n/2 \rceil$を設定\\
 \verb|/ | & \\
\end{tabularx}
}}\\
