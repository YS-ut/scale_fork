%%%%%%%%%%%%%%%%%%%%%%%%%%%%%%%%%%%%%%%%%%%%%%%%%%%%%%%%%%%%%%%%%%%%%%%%%%%%%%%%%%%%%%%%%%%%

SCALEのインストールに必要なコンパイラやライブラリ環境のインストール方法について説明する。
ここでの記載内容は、こちらのテスト環境でのインストールプロセスを示しているものであって
必ずしも全く同じとは限らない。
うまくいかない場合には、それぞれのツール・ライブラリの開発元に直接問い合わせること。


Linuxをインストール後、各種プログラムのインストールはコマンドライン端末にて行う。
本書で説明するライブラリ環境のインストールでは、root権限が必要になる。
したがって、想定する環境は、ユーザーがroot権限を所持しているかサーバやデスクトップマシンである。
別途サーバー管理者が存在し、root権限を取得できない場合等は、必要な環境条件が整っているか
サーバー管理者に問い合わせること。

本節では、HDF5, NetCDF, MPIについてGNU compilerでコンパイルされたライブラリの説明を行う。
GNU compiler以外のIntel compilerなどを利用する場合は、各自でインストール方法を調べてインストールすること。\\

\noindent ここでインストールするコンパイラおよびライブラリ環境は、主に下記の4点である。
\begin{itemize}
\item GNU C/C++, fortran compiler
\item HDF5 Library (\url{https://www.hdfgroup.org/HDF5/})
\item NetCDF Library (\url{http://www.unidata.ucar.edu/software/netcdf/})
\item Message Passing Interface (MPI) Library (openMPI版、\url{http://www.open-mpi.org/})
\end{itemize}
これらのインストール方法について、本書では下記の5種類のOperating System (OS)について説明する。
\begin{itemize}
\item Linux CentOS 6.6 x86-64
\item Linux CentOS 7.1 x86-64
\item Linux openSUSE 13.2 x86-64
\item Apple Mac OS X 10.10 Yosemite
\item スーパーコンピュータ「京」
\end{itemize}
他のOSディストリビューション(下記参照)でもSCALEを利用可能だが、
本書でサポートするのは上記の範囲とする。\\

\noindent{\bf 動作確認済みの他のOSディストリビューション}
\begin{itemize}
\item Linux SUSE Enterprise Linux 11.1, 11.3 x86-64
\item Linux Vine Linux 6.3 x86-64
\item Linux Fedora 16 x86-64
\end{itemize}


\section{インストール方法 (Linux - CentOS 6.6-6.8 編)} \label{chap:install_centos}
%==========================================================================================

以下の説明で使用した環境は次のとおりである。
\begin{itemize}
\item CPU: Intel Core i5 2410M (sandybridge)
\item Memory: DDR3-1333 4GB
\item OS: CentOS 6.6 (kernel: 2.6.32-504.23.4.el6.x86\_64)\\
{\small *インストール時、"日本語"、 "Desktop"、"Kdump有り"を選択}
\end{itemize}

\subsubsection{ライブラリのインストール}

CentOS 6.6では、一部のライブラリをエンタープライズLinux用の拡張パッケージ(EPEL)リポジトリからインストールする。
そこで、はじめにEPELリポジトリをシステムにインストールし登録する。
CentOS 6.6では、ソフトウェアのインストールに"yum"コマンドを利用する。
すべての作業を行うまえに、下記のコマンドにてパッケージをアップデートしておくことをおすすめする。
\begin{verbatim}
 # yum update
\end{verbatim}

ルート権限で、下記のコマンドを実行することでリポジトリの登録が可能である。
\begin{verbatim}
 # yum install epel-release
\end{verbatim}
実行時のコマンドラインの様子は以下のようになる。
インストール対象がリストされるので、確認して"y"をタイプして先へ進める。\\

\noindent {\small {\gt
\fbox{
\begin{tabularx}{140mm}{l}
読み込んだプラグイン:fastestmirror, refresh-packagekit, security\\
インストール処理の設定をしています\\
Loading mirror speeds from cached hostfile\\
 * base: ftp.***.**.jp\\
 * extras: ftp.***.**.jp\\
 * updates: ftp.***.**.jp\\
依存性の解決をしています\\
-- トランザクションの確認を実行しています。\\
--- パッケージ epel-release.noarch 0:7-5 を インストール\\
-- 依存性解決を終了しました。\\
\\
依存性を解決しました\\
\\
======================================\\
 Package                アーキテクチャー バージョン      リポジトリー      容量\\
======================================\\
インストール中:\\
 epel-release           noarch           6-8             extras            14 k\\
\\
トランザクションの要約\\
======================================\\
インストール  1 パッケージ\\
\\
総ダウンロード容量: 14 k\\
インストール容量: 24 k\\
Is this ok (y/N): y\\
パッケージをダウンロードしています:
epel-release-6-8.noarch.rpm                                   14 kB     00:00\\
rpm\_check\_debug を実行しています\\
トランザクションのテストを実行しています\\
トランザクションのテストを成功しました\\
トランザクションを実行しています\\
  インストールしています  : epel-release-6-8.noarch                            1/1\\
  Verifying               : epel-release-6-8.noarch                            1/1\\
\\
インストール:\\
  epel-release.noarch 0:6-8\\
\\
完了しました!\\
\end{tabularx}
}}}\\

\noindent {\small *この時点で、yumによるインストールに失敗する場合は、
プロキシ設定等を含めた通信環境、yumリポジトリの登録状況等を再確認すること。}

\noindent yumのグループインストール機能を用いて,開発ツール
(ここでの対象は主にGNU compilerとmakeシステム)をまとめてインストールする。
\begin{verbatim}
 # yum groupinstall "development tools"
\end{verbatim}

\noindent つづいて、グループインストールではインストールされないライブラリを個別に追加する。
\begin{verbatim}
 # yum install zlib-devel
 # yum install hdf5-devel hdf5-static
 # yum install netcdf-devel netcdf-static
 # yum install openmpi-devel
 # yum install wgrib wgrib2
\end{verbatim}

SCALEは陰解法計算の部分で、数値計算ライブラリ Lapack
\footnote{\url{http://www.netlib.org/lapack/}}
を利用するオプションがある。
もし必要ならば、Lapack もインストールすること。
\begin{verbatim}
 # yum install lapack lapack-devel
\end{verbatim}

\noindent \textcolor{blue}{\small *wgrib、wgrib2は、第\ref{chap:tutorial_real}章:Tutorial: Real case で
外部入力データのプレ処理を行うために使用する。}

\noindent {\small *"yum -y install package name" のように ``-y'' オプションをつけて実行することで、インストール前の再確認をスキップできる。}


\subsubsection{環境変数の設定}

ローカルシステムでMPI並列プログラムを実行するために、OpenMPIライブラリの環境変数設定を行う。
ユーザ権限に移動して.bashrcをエディタで開き,
\begin{verbatim}
 $ vi ~/.bashrc
\end{verbatim}
下記をファイルの最後に追加して,環境変数の設定を記述する。\\

\noindent {\gt
\ovalbox{
\begin{tabularx}{140mm}{l}
 \\
 \verb|// ---------------- Add to end of the file ----------------|\\
 \verb|# OpenMPI|\\
 \verb|export MPI="/usr/lib64/openmpi"|\\
 \verb|export PATH="$PATH:$MPI/bin"|\\
 \verb|export LD_LIBRARY_PATH="$LD_LIBRARY_PATH:$MPI/lib"|\\
 \\
\end{tabularx}
}}\\

編集が終わったら、環境設定を有効にする。
\begin{verbatim}
 $ . ~/.bashrc
\end{verbatim}


%\subsubsection{Installation of GPhys}
%CentOSの場合、yumリポジトリに地球電脳倶楽部のGFD-Dennouリポジトリを登録することで、
%簡単にGPhysをインストールできる。
%root権限で、GFD-Dennouリポジトリを次のような内容で登録する。
%
%\begin{verbatim}
% # vi /etc/yum.repos.d/GFD-Dennou.repo
%\end{verbatim}
%
%\begin{verbatim}
% // ---------------- Edit the file ----------------
% [gfd-dennou]
% name=GFD DENNOU Club RPMS for CentOS $releasever - $basearch
% baseurl=http://www.gfd-dennou.org/library/cc-env/rpm-dennou/CentOS/$releasever/$basearch/
% enabled=1
% gpgcheck=0
%\end{verbatim}
%編集が終わったら、yumでGPhysをインストールする。
%\begin{verbatim}
% # yum install gphys
%\end{verbatim}


\section{インストール方法 (Linux - CentOS 7.1-7.2 編)} \label{chap:install_centos71}
%==========================================================================================

以下の説明で使用した環境は次のとおりである。
\begin{itemize}
\item CPU: Intel Core i5 2410M (sandybridge)
\item Memory: DDR3-1333 4GB
\item OS: CentOS 7.1 (kernel: 3.10.0-229.7.2.el7.x86\_64)\\
{\small *インストール時、"日本語"、 "Gnome デスクトップ"、"Kdump有り"を選択}
\end{itemize}

\subsubsection{ライブラリのインストール}

CentOS 7.1では、一部のライブラリをエンタープライズLinux用の拡張パッケージ(EPEL)リポジトリからインストールする。
そこで、はじめにEPELリポジトリをシステムにインストールし登録する。
CentOS 7.1では、ソフトウェアのインストールに"yum"コマンドを利用する。
すべての作業を行うまえに、下記のコマンドにてパッケージをアップデートしておくことをおすすめする。
\begin{verbatim}
 # yum update
\end{verbatim}

ルート権限で、下記のコマンドを実行することでリポジトリの登録が可能である。
\begin{verbatim}
 # yum install epel-release
\end{verbatim}
実行時のコマンドラインの様子は以下のようになる。
インストール対象がリストされるので、確認して"y"をタイプして先へ進める。\\

\noindent {\small {\gt
\fbox{
\begin{tabularx}{140mm}{l}
読み込んだプラグイン:fastestmirror, langpacks\\
base                                                      3.6 kB     00:00\\
extras                                                    3.4 kB     00:00\\
updates                                                   3.4 kB     00:00\\
Loading mirror speeds from cached hostfile\\
 * base: ftp.***.**.jp\\
 * extras: ftp.***.**.jp\\
 * updates: ftp.***.**.jp\\
依存性の解決をしています\\
-- トランザクションの確認を実行しています。\\
--- パッケージ epel-release.noarch 0:7-5 を インストール\\
-- 依存性解決を終了しました。\\
\\
依存性を解決しました\\
\\
======================================\\
 Package                アーキテクチャー バージョン      リポジトリー      容量\\
======================================\\
インストール中:\\
 epel-release           noarch           7-5             extras            14 k\\
\\
トランザクションの要約\\
======================================\\
インストール  1 パッケージ\\
\\
総ダウンロード容量: 14 k\\
インストール容量: 24 k\\
Is this ok (y/d/N): y\\
Downloading packages:\\
extras/7/x86\_64/prestodelta                                 7.6 kB   00:00\\
epel-release-7-5.noarch.rpm                                  14 kB   00:00\\
Running transaction check\\
Running transaction test\\
Transaction test succeeded\\
Running transaction\\
  インストール中          : epel-release-7-5.noarch                         1/1\\
  検証中                  : epel-release-7-5.noarch                         1/1\\
\\
インストール:\\
  epel-release.noarch 0:7-5\\
\\
完了しました!\\
\end{tabularx}
}}}\\

\noindent {\small *この時点で、yumによるインストールに失敗する場合は、
プロキシ設定等を含めた通信環境、yumリポジトリの登録状況等を再確認すること。}

\noindent yumのグループインストール機能を用いて,開発ツール
(ここでの対象は主にGNU compilerとmakeシステム)をまとめてインストールする。
\begin{verbatim}
 # yum groupinstall "development tools"
\end{verbatim}

\noindent つづいて、グループインストールではインストールされないライブラリを個別に追加する。
\begin{verbatim}
 # yum install hdf5-devel hdf5-static
 # yum install netcdf-devel netcdf-static
 # yum install netcdf-fortran-devel
 # yum install openmpi-devel
 # yum install wgrib wgrib2
\end{verbatim}

SCALEは陰解法計算の部分で、数値計算ライブラリ Lapack
\footnote{\url{http://www.netlib.org/lapack/}}
を利用するオプションがある。
もし必要ならば、Lapack もインストールすること。
\begin{verbatim}
 # yum install lapack lapack-devel
\end{verbatim}

\noindent \textcolor{red}{\small *fortran用のモジュールファイルは別パッケージになっている。
"netcdf-fortran-devel"のインストールを忘れないこと。}

\noindent \textcolor{blue}{\small *wgrib、wgrib2は、第\ref{chap:tutorial_real}章:Tutorial: Real case で
外部入力データのプレ処理を行うために使用する。}

\noindent {\small *"yum -y install package name"として実行することで、インストール前の再確認をスキップできる。}

\subsubsection{環境変数の設定}

ローカルシステムでMPI並列プログラムを実行するために、OpenMPIライブラリの環境変数設定を行う。
ユーザ権限に移動して.bashrcをエディタで開き,
\begin{verbatim}
 $ vi ~/.bashrc
\end{verbatim}
下記をファイルの最後に追加して,環境変数の設定を記述する。\\

\noindent {\gt
\ovalbox{
\begin{tabularx}{140mm}{l}
 \\
 \verb|// ---------------- Add to end of the file ----------------|\\
 \verb|# OpenMPI|\\
 \verb|export MPI="/usr/lib64/openmpi"|\\
 \verb|export PATH="$PATH:$MPI/bin"|\\
 \verb|export LD_LIBRARY_PATH="$LD_LIBRARY_PATH:$MPI/lib"|\\
 \\
\end{tabularx}
}}\\

編集が終わったら、環境設定を有効にする。
\begin{verbatim}
 $ . ~/.bashrc
\end{verbatim}


\section{インストール方法 (Linux - openSUSE 13.2 編)} \label{chap:install_opensuse}
%==========================================================================================

以下の説明で使用した環境は次のとおりである。
\begin{itemize}
\item CPU: Intel Core i5 2410M (sandybridge)
\item Memory: DDR3-1333 4GB
\item OS: openSUSE 13.2 (kernel: 3.16.7-21-desktop x86\_64)\\
{\small *インストール時、"日本語"、 "Gnome Desktop"を選択}
\end{itemize}

\subsubsection{ライブラリのインストール}

openSUSE 13.2では、一部のライブラリを外部リポジトリ
(ocefpaf's Home Project; \url{https://build.opensuse.org/project/show/home:ocefpaf})からインストールする。
このため、まずhome\_ocefpafリポジトリをシステムにインストールし登録する。
このリポジトリには、grads、ncview、GMT、ncl、そしてcdoといったツール群も含まれており便利である。

openSUSE 13.2では、ソフトウェアのインストールに"zypper"コマンドを利用する。
openSUSEでは一般にユーザーがrootユーザーにスイッチすることを推奨しておらず、
デフォルトのままOSをインストールすると"su"コマンドによってrootユーザーに
スイッチすることはできないので、"sudo"コマンドを利用してインストール作業を行う。
すべての作業を行うまえに、下記のコマンドにてパッケージをアップデートしておくことをおすすめする。
\begin{verbatim}
 # sudo zypper update
\end{verbatim}

下記のコマンドを実行することでリポジトリの登録が可能である。
\begin{verbatim}
 $ sudo zypper ar \\
   http://download.opensuse.org/repositories/home:/ocefpaf/openSUSE_13.2/ \\
   home_ocefpaf
\end{verbatim}
{\small *上記コマンド中の"\verb|\\|"は、組版上の改行であることを意味する。
実際は改行も"\verb|\\|"の記述も必要ない。}
実行時のコマンドラインの様子は以下のようになる。\\

\noindent {\small {\gt
\fbox{
\begin{tabularx}{140mm}{l}
 リポジトリ 'home\_ocefpaf' を追加しています ...............................完了 \\
 リポジトリ 'home\_ocefpaf' を正常に追加しました\\
 有効         : はい (Y)\\
 自動更新     : いいえ (N)\\
 GPG チェック : はい (Y)\\
 URI          : \url{http://download.opensuse.org/repositories/home:/ocefpaf/openSUSE_13.2/}
\end{tabularx}
}}}\\

{\small *この時点で、zypperによるインストールに失敗する場合は、
プロキシ設定等を含めた通信環境、zypperリポジトリの登録状況等を再確認すること。}

\noindent zypperのパターンインストール機能を用いて,基本開発ツール
(ここでの対象は主にGNU compilerとmakeシステム)をまとめてインストールする。
\begin{verbatim}
 $ sudo zypper install --type pattern devel_basis
\end{verbatim}

home\_ocefpafリポジトリを登録して最初のインストールの場合、
下記のようにパッケージの署名鍵の信頼について問われることがある。
"a"の「ずっと信頼」を選択して作業を進める。
その後、インストール対象がリストされるので、確認して"y"をタイプして先へ進める。\\

\noindent {\small {\gt
\fbox{
\begin{tabularx}{140mm}{l}
 鍵を拒否しますか (R)? 一時的に信頼しますか (T)? \\
 それとも今後ずっと信頼しますか (A)? [r/t/a/? 全てのオプションを表示] (r): a
\end{tabularx}
}}}\\

\noindent つづいて、devel\_basisパッケージに含まれないライブラリを個別に追加する。
\begin{verbatim}
 $ sudo zypper install gcc-fortran
 $ sudo zypper install hdf5-devel hdf5-devel-static
 $ sudo zypper install netcdf-devel netcdf-devel-static
 $ sudo zypper install netcdf-fortran-devel netcdf-fortran-static
 $ sudo zypper install openmpi-devel openmpi-devel-static
 $ sudo zypper install wgrib wgrib2
\end{verbatim}

SCALEは陰解法計算の部分で、数値計算ライブラリ Lapack
\footnote{\url{http://www.netlib.org/lapack/}}
を利用するオプションがある。
もし必要ならば、Lapack もインストールすること。
\begin{verbatim}
 $ sudo zypper install lapack-devel lapack-devel-static
\end{verbatim}

\noindent \textcolor{blue}{\small *wgrib、wgrib2は、第\ref{chap:tutorial_real}章:Tutorial: Real case で
外部入力データのプレ処理を行うために使用する。}


\subsubsection{環境変数の設定}

ローカルシステムでMPI並列プログラムを実行するために、OpenMPIライブラリの環境変数設定を行う。
ユーザ権限に移動して.bashrcをエディタで開き,
\begin{verbatim}
 $ vi ~/.bashrc
\end{verbatim}
下記をファイルの最後に追加して,環境変数の設定を記述する。\\

\noindent {\gt
\ovalbox{
\begin{tabularx}{140mm}{l}
 \\
 \verb|// ---------------- Add to end of the file ----------------|\\
 \verb|# OpenMPI|\\
 \verb|export MPI="/usr/lib64/mpi/gcc/openmpi"|\\
 \verb|export PATH="$PATH:$MPI/bin"|\\
 \verb|export LD_LIBRARY_PATH="$LD_LIBRARY_PATH:$MPI/lib64"|\\
 \\
\end{tabularx}
}}\\

編集が終わったら、環境設定を有効にする。
\begin{verbatim}
 $ . ~/.bashrc
\end{verbatim}


\section{インストール方法(Mac OS X 編)} \label{chap:install_mac}
%==========================================================================================

\subsubsection{macportsを用いたインストール}

Apple Mac OS XでのSCALE実行環境を整備する方法について説明する。
ここではMac OS Xのパッケージマネージャの一つであるmacportsを用いる方法を紹介する。
その他の主要なパッケージマネージャとしては、homebrewが挙げられる。homebrewを利用しても環境は手軽に揃えられるので、
興味のある方は利用してもらいたい。

まずはAppleの開発ツールであるXcodeをインストールする。
大元のgccコンパイラを導入するために、必ずインストールする必要がある。
最近のOSのバージョンのものは、App Store経由で入手できる(無料)。
古いOSでは、インストールディスクから追加することが出来る。
最近のOSのXcodeの場合、最初に以下の様な設定をターミナルから行う必要がある。
\begin{verbatim}
 コマンドラインツールのインストール
 # xcode-select --install
\end{verbatim}
\begin{verbatim}
 ライセンス条項の承認(root権限必要)
 # sudo xcodebuild -license
\end{verbatim}

次にmacports本体をインストールする。
\url{https://www.macports.org/}
必要なパッケージインストーラーをダウンロードし、インストールを進める。\\
macportsとmacportsが管理するパッケージは/opt/local以下に配置される。
インストール時に\verb|.bash_profile|に、/opt/local/binへのパスが張られているので確認されたい。
macportsはコマンドラインから操作する。主要なコマンドは以下の通り。

\begin{verbatim}
 インストール可能なソフトウェアを検索する
 $ port search <検索文字>
\end{verbatim}
\begin{verbatim}
 ソフトウェアのインストール時に選択可能なオプション(variants)を確認する
 $ port variants <アプリ名>
\end{verbatim}
\begin{verbatim}
 ソフトウェアのインストール(root権限必要)
 $ sudo port install <アプリ名> [variants]
\end{verbatim}
\begin{verbatim}
 ソフトウェアのアンインストール(root権限必要)
 $ sudo port uninstall <アプリ名> [variants]
\end{verbatim}
\begin{verbatim}
 macports本体とパッケージカタログの更新(root権限必要)
 $ sudo port selfupdate
\end{verbatim}
\begin{verbatim}
 パッケージの更新(root権限必要)
 $ sudo port upgrade outdated
\end{verbatim}
\begin{verbatim}
 不要なパッケージ(activateされていない過去のバージョン等)の削除
 $ sudo port -u uninstall
\end{verbatim}

\subsubsection{gccからNetCDFまでのインストール}

macportsはパッケージの依存関係を解決してくれるが、必要なvariantsを備えたセットを作るには、
順番にインストールしていく方が問題が少ない。以下にsudo port installしていく順番とvariantsの設定を示す。
この例ではgcc4.9を利用する。
\begin{verbatim}
 $ gcc49
 $ openmpi-gcc49 +threads
 $ hdf4 +gcc49 +szip
 $ hdf5 +gcc49 +szip +fortran +cxx +openmpi +threadsafe
 $ netcdf +gcc49 +openmpi +netcdf4 +hdf4
 $ netcdf-fortran +gcc49 +openmpi
\end{verbatim}

macportsでは複数のコンパイラとMPIライブラリをインストール出来るため、
その中で利用するものを選択する必要がある。
今回の場合、gccとMPIライブラリが該当する。
この操作を行うと、gfortran等の一般的な名前でエイリアスが作られてPATHが通るようになる。
\begin{verbatim}
 $ sudo port select --set gcc mp-gcc49
 $ sudo port select --set mpi openmpi-gcc49-fortran
\end{verbatim}

インストールしたNetCDFライブラリを用いるときのPATHの設定は以下の通りである。
\verb|.bash_profile| をエディタで開き、
\begin{verbatim}
 $ emacs ~/.bash_profile
\end{verbatim}
下記をファイルに追加して、環境変数の設定を記述する。

\noindent {\gt
\ovalbox{
\begin{tabularx}{140mm}{l}
 \\
 \verb|export NETCDF_INCLUDE="-I/opt/local/include"|\\
 \verb|export NETCDF_LIBS="-L/opt/local/lib -lnetcdff -L/opt/local/lib -Wl,-headerpad_max_install_names -lnetcdf"|\\
 \\
\end{tabularx}
}}\\

SCALEは陰解法計算の部分で、数値計算ライブラリを利用するオプションがある。
もし必要ならば、macportsからATLASをインストールすることが出来る。
\begin{verbatim}
 $ atlas +gcc49
\end{verbatim}

追加する環境変数の設定は以下の通りである。

\noindent {\gt
\ovalbox{
\begin{tabularx}{140mm}{l}
 \\
 \verb|export LAPACK_LIBS="-L/opt/local/lib -llapack -lcblas -lf77blas -latlas"|\\
 \\
\end{tabularx}
}}\\

\subsubsection{Mac OS XにおけるGPhys / Ruby-DCLのインストール}

GphysはRubyのパッケージ管理システムRubyGemsを通してインストールできる。
詳細な情報については、(\url{https://www.hdfgroup.org/HDF5/})を参照されたい。
まず必要であればrubyのインストールをmacportsを用いて行う。この例ではRuby2.1を利用することにする。

\begin{verbatim}
 $ sudo port install ruby21
 $ sudo port select --set ruby ruby21
\end{verbatim}

次にmacportsを用いて、Gphysに必要なライブラリをインストールする。

\begin{verbatim}
 $ sudo port install fftw-3
 $ sudo port install gsl
 $ sudo port install C-DCL6
\end{verbatim}

最後にRubyGemsを用いて、Gphysをインストールする。

\begin{verbatim}
 $ sudo gem install gphys
\end{verbatim}

%\subsubsection{Mac OS XにおけるGrADSのインストール}(Todo)

\subsubsection{Mac OS Xでの実行時の注意点}

Mac OS Xを用いて\scalerm プログラムを実行すると、実行時に
「アプリケーション"scale-rm"へのネットワーク受信接続を許可しますか?」
というダイアログが出ることがあります。
これはコンパイルしたバイナリがマシンをまたいだMPI通信をするかファイアウォール機能が確認するためです。
コンパイルし直すたびにMPI並列数の分だけダイアログが出てきてしまいますが、今のところ表示を回避するためには
「システム環境設定」の「セキュリティとプライバシー」項目で「ファイアウォール」のタブを選択し、
プログラムの実行時にファイアウォールを切る方法しかありません。



\section{インストール方法 (スーパーコンピュータ「京」 編)} \label{chap:install_supercom}
%==========================================================================================

以下の説明で使用した環境は次のとおりである。
\begin{itemize}
\item 計算機: スーパーコンピュータ「京」
\item 言語環境: K-1.2.0-18
\end{itemize}

\subsubsection{ライブラリについて}
スーパーコンピュータ「京」では、SCALEのコンパイルに
必要なライブラリがAICSソフトウェアとして準備されている。
詳細は、京ポータルサイトの「AICSソフトウェア等」の項目、もしくは下記のWebページを参照のこと。\\
\noindent \url{http://www-sys-aics.riken.jp/releasedsoftware/ksoftware/pnetcdf.html}

一般に、スーパーコンピュータ「京」におけるSCALEのコンパイルには、\\
\noindent "\verb|/opt/aics/netcdf/k-serial-noszip/|"下にあるHDF5、NetCDFライブラリを用いる。
コンパイラやMPIライブラリについてもスーパーコンピュータ「京」専用のコンパイラとライブラリを用いるため、
特別にライブラリ環境を準備する必要はない。

\noindent {\small *コンパイル時に参照するライブラリのPATHは、
SCALEコンパイル時に使用する"Makedef.K"に記述されているため、
環境変数について特に設定する必要はない。}


\section{描画ツールのインストール} \label{chap:install_drawtool}
\label{sec:env_vis_tools}
%==========================================================================================

SCALEの計算結果や、初期値/境界値データなどを描画するのに利用可能である描画ツールの例を挙げる。
個人の好みでどのツールを使ってもよいし、出力形式を理解していれば、
ここに挙げた以外のツールで解析・描画することももちろん可能である。

\begin{itemize}
\item GPhys / Ruby-DCL by 地球電脳倶楽部\\
 \begin{itemize}
  \item URL: \url{http://ruby.gfd-dennou.org/products/gphys/}
  \item 概略:SCALEの出力ファイルは、MPI並列の計算領域分割に従ってMPIプロセスごとに
              NetCDF形式の分割ファイルとして出力される。GPhysの"gpview"や"gpvect"といった
              描画ツールを使えば、分割ファイルを後処理なしに直接開いて描画することができる。
  \item インストール方法:
  地球電脳倶楽部のWebページに、主なOSでのインストール方法についての解説がある。\\
  \url{https://www.gfd-dennou.org/library/ruby/tutorial/install/index-j.html}\\
  本書で使用したCentOS6、CentOS7については、下記のWebページにインストール方法が記載されている。\\
  \url{http://www.gfd-dennou.org/library/cc-env/rpm-dennou/index.html.ja}\\
   Mac OS Xにおけるインストール方法は第\ref{chap:install_mac}節で説明している他、
   \url{https://www.gfd-dennou.org/library/ruby/products/macports/index-j.html}
   でも解説されている。
   \end{itemize}
\item Grid Analysis and Display System (GrADS) by COLA\\
 \begin{itemize}
  \item URL: \url{http://iges.org/grads/}
  \item 概略:言わずと知れた描画ツール。SCALEのNetCDF形式の分割ファイルをそのまま読むことはできない
             ため、SCALEで提供している出力データの後処理ツール"\verb|netcdf2grads_h|"を使用して分割ファイルを結合し、
             GrADSで読み込めるファイル形式に変換する必要ある。"\verb|netcdf2grads_h|"のインストール方法は、
本書の第\ref{sec:inst_env}章、使用方法は第3章、および第4章を参照のこと。
  \item インストール方法:\url{http://iges.org/grads/downloads.html}を参照のこと。
                        CentOS6、CentOS7ではEPELリポジトリを登録していればyumコマンドによって、
                        openSUSE 13ではhome\_ocefpafリポジトリを登録していればzypperコマンドによって
                        インストールできる。
 \end{itemize}
\item Ncview: a NetCDF visual browser by David W. Pierce\\
 \begin{itemize}
  \item URL: \url{http://meteora.ucsd.edu/~pierce/ncview_home_page.html}
  \item 概略:NetCDF形式ファイルのクイックビューアーである。SCALEの分割ファイルを結合して描画することは
             できないが、分割ファイルを1つずつ描画してチェックすることはできる。
  \item インストール方法:\url{http://meteora.ucsd.edu/~pierce/ncview_home_page.html}を参照のこと。
                        CentOS6、CentOS7ではEPELリポジトリを登録していればyumコマンドによって、
                        openSUSE 13ではhome\_ocefpafリポジトリを登録していればzypperコマンドによって
                        インストールできる。
 \end{itemize}
\end{itemize}





