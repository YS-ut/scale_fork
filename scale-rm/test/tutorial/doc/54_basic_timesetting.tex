\newpage
\section{積分時間と積分時間間隔の設定} \label{sec:timeintiv}
%------------------------------------------------------
積分時間やタイムステップは、実験の目的や設定によって適切に設定する必要がある。
空間解像度を変えた場合はそれに応じたタイムステップを設定する必要があり、
同じ解像度でも計算不安定を防ぐためにタイムステップを短くすることもある。

積分時間とタイムステップの設定は、
設定ファイル\verb|run_***.conf|の\namelist{PARAM_PRC}の項目を編集することで設定できる。\\

\noindent {\small {\gt
\ovalbox{
\begin{tabularx}{140mm}{lX}
\verb|&PARAM_TIME| & \\
\verb| TIME_STARTDATE             = 2014, 8, 10, 0, 0, 0,| & 計算開始の日付:放射過程を用いる実験等で必要\\
\verb| TIME_STARTMS               = 0.D0,  | & 計算開始時刻[mili sec]\\
\verb| TIME_DURATION              = 12.0D0,| & 積分時間[単位は\verb|TIME_DURATION_UNIT|で設定]\\
\verb| TIME_DURATION_UNIT         = "HOUR",| & \verb|TIME_DURATION|の単位\\
\verb| TIME_DT                    = 60.0D0,| & トレーサー移流のタイムステップ\\
\verb| TIME_DT_UNIT               = "SEC", | & \verb|TIME_DT|の単位 \\
\verb| TIME_DT_ATMOS_DYN          = 30.0D0,| & 力学過程計算のタイムステップ\\
\verb| TIME_DT_ATMOS_DYN_UNIT     = "SEC", | & \verb|TIME_DT_ATMOS_DYN|の単位\\
\verb| TIME_DT_ATMOS_PHY_MP       = 60.0D0,| & 雲物理過程のタイムステップ \\
\verb| TIME_DT_ATMOS_PHY_MP_UNIT  = "SEC", | & \verb|TIME_DT_ATMOS_PHY_MP|の単位\\
\verb| TIME_DT_ATMOS_PHY_TB       = 60.0D0,| & 乱流スキームのタイムステップ \\
\verb| TIME_DT_ATMOS_PHY_TB_UNIT  = "SEC", | & \verb|TIME_DT_ATMOS_PHY_TB|の単位\\
\verb| TIME_DT_ATMOS_PHY_RD       = 600.0D0, | & 放射スキームのタイムステップ \\
\verb| TIME_DT_ATMOS_PHY_RD_UNIT  = "SEC",  | & \verb|TIME_DT_ATMOS_PHY_RD|の単位\\
\verb| TIME_DT_ATMOS_PHY_SF       = 60.0D0, | & 大気下端境界(フラックス計算)のタイムステップ\\
\verb| TIME_DT_ATMOS_PHY_SF_UNIT  = "SEC",  | & \verb|TIME_DT_ATMOS_PHY_SF|の単位\\
\verb| TIME_DT_OCEAN              = 300.0D0,| & 海面・海洋スキームのタイムステップ\\
\verb| TIME_DT_OCEAN_UNIT         = "SEC",  | & \verb|TIME_DT_OCEAN|の単位\\
\verb| TIME_DT_LAND               = 300.0D0,| & 陸面スキームのタイムステップ\\
\verb| TIME_DT_LAND_UNIT          = "SEC",  | & \verb|TIME_DT_LAND|の単位\\
\verb| TIME_DT_URBAN              = 300.0D0,| & 都市スキームのタイムステップ\\
\verb| TIME_DT_URBAN_UNIT         = "SEC",  | & \verb|TIME_DT_URBAN|の単位\\
\verb| TIME_DT_ATMOS_RESTART      = 21600.D0, | & リスタートファイル(大気)の出力間隔\\
\verb| TIME_DT_ATMOS_RESTART_UNIT = "SEC",    | & \verb|TIME_DT_ATMOS_RESTART|の単位\\
\verb| TIME_DT_OCEAN_RESTART      = 21600.D0, | & リスタートファイル(海洋)の出力間隔\\
\verb| TIME_DT_OCEAN_RESTART_UNIT = "SEC",    | & \verb|TIME_DT_OCEAN_RESTART|の単位\\
\verb| TIME_DT_LAND_RESTART       = 21600.D0, | & リスタートファイル(陸面)の出力間隔\\
\verb| TIME_DT_LAND_RESTART_UNIT  = "SEC",    | & \verb|TIME_DT_LAND_RESTART|の単位\\
\verb| TIME_DT_URBAN_RESTART      = 21600.D0, | & リスタートファイル(都市)の出力間隔\\
\verb| TIME_DT_URBAN_RESTART_UNIT = "SEC",    | & \verb|TIME_DT_URBAN_RESTART|の単位\\
\verb|/|\\
\end{tabularx}
}}}\\


\nmitem{TIME_DT} は、トレーサー移流のためのタイムステップであり、
格子間隔と移流速度から計算不安定を起こさないように
格子間隔を移流速度で割った値が取りうる最少値よりも小さな値を設定する。
\nmitem{TIME_DT_ATMOS_DYN} は、力学変数の時間積分のためのタイムステップであり、音速で制約される。
計算安定性のためには、\nmitem{ATMOS_DYN_TINTEG_SHORT_TYPE} が \verb|RK4| の場合には
最少格子間隔(HE-VI利用時には水平の最少格子間隔)を 420 m/s で割った値が、
\verb|RK3| の場合には 840 m/s で割った値が目安となる。



