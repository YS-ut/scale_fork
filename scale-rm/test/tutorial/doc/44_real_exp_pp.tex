%-------------------------------------------------------%
\section{地形・土地利用データの作成:pp} \label{sec:tutrial_real_pp}
%-------------------------------------------------------%

ppディレクトリへ移動し、現実実験のための地形データ、土地利用データを作成する。
\begin{verbatim}
 $ cd ${Tutorial_DIR}/real/experiment/pp/
 $ ls 
  pp.d01.conf scale-rm_pp
\end{verbatim}
ppディレクトリの中には、\verb|pp.d01.conf|という名前の
設定ファイルが準備されている。
ドメインの位置や格子点数など、実験設定に合わせて、
適宜\verb|pp.d01.conf|を編集する必要があるが、
チュートリアルではすでに表\ref{tab:grids}の設定に
従って編集済みの\verb|pp.d01.conf|が用意されているため、
そのまま利用する。
\verb|pp.d01.conf|の設定の中で、
\namelist{PARAM_CONVERT}の中の以下の項目を確認してほしい。\\

\noindent {\gt
\ovalbox{
\begin{tabularx}{140mm}{l}
\verb|&PARAM_CONVERT| \\
\verb|  CONVERT_TOPO = .true.,| \\
\verb|  CONVERT_LANDUSE = .true.,| \\
\verb|/| \\
\end{tabularx}
}}\\

\noindent 上記のように\nmitem{CONVERT_TOPO}と\nmitem{CONVERT_LANDUSE}が
\verb|.true.|となっていることが、
それぞれ地形と土地利用の処理を行うことを意味している。
%詳細な設定ファイルの内容については、付録\ref{achap:namelist}を参照されたい。

また、
\namelist{PARAM_CNVTOPO_GTOPO30}の中の\nmitem{GTOPO30_IN_DIR}と
\namelist{PARAM_CNVLANDUSE_GLCCv2}の中の\nmitem{GLCCv2_IN_DIR}は、
地形データと土地利用データの場所を示しており、
環境変数\verb|${SCALE_DB}|で設定された場所が指定されている。\\

\proofcomment{上記意味不明。環境変数で競ってしているのに、
なぜ、設定ファイルで設定しなおさねばならないのか???}\\
\replycomment{修正しました(足立)}\\


\noindent {\gt
\ovalbox{
\begin{tabularx}{140mm}{l}
\verb|&PARAM_CNVTOPO_GTOPO30| \\
\verb| GTOPO30_IN_CATALOGUE = "GTOPO30_catalogue.txt",|\\
\verb| GTOPO30_IN_DIR       = "./topo/GTOPO30/Products",|\\
\verb|/|\\
\\
\verb|&PARAM_CNVLANDUSE_GLCCv2|\\
\verb| GLCCv2_IN_CATALOGUE = "GLCCv2_catalogue.txt",|\\
\verb| GLCCv2_IN_DIR       = "./landuse/GLCCv2/Products",|\\
\verb| limit_urban_fraction = 0.3D0,|\\
\verb|/|\\
\end{tabularx}
}}\\


今回は、表\ref{tab:grids}に示されているように。
4つのMPIプロセスを使用する設定なので次のように実行する。
\begin{verbatim}
 $ mpirun -n 4 ./scale-rm_pp pp.d01.conf
\end{verbatim}
%本節使用した環境において、実行にはおおよそ15秒を要する。
ジョブが正常に終了すれば、\verb|topo_d01.pe######.nc|と\\
\verb|landuse_d01.pe######.nc|と
いうファイルがMPIプロセス数だけ、つまり4つずつ生成される(
\verb|######|にはMPIプロセスの番号が入る)。
それぞれ、各格子点における地形と土地利用の情報が入っている。
実行時のログは、\verb|pp_LOG_d01.pe000000|に出力されるので内容を確かめておくこと。


%% サポート外
%% \vspace{1cm}
%% \noindent {\Large\em OPTION} \hrulefill \\
%% gpviewがインストールされている場合、次のコマンドによって、
%% 作成された地形と土地利用データが正しく作成されているかどうか
%% 確認することが出来る.正しく作成されていれば,図 \ref{fig:tutrial_real_domain}と同様の図ができる.
%% \begin{verbatim}
%%   $ gpview topo_d01.pe00000*@TOPO --aspect=1 --nocont
%%   $ gpview landuse_d01.pe00000*@FRAC_LAND --aspect=1 --nocont
%% \end{verbatim}

