\section{実行方法} \label{sec:ideal_exp_run}
%====================================================================================

実験の実行は、前準備、初期値作成、シミュレーション実行、後処理、そして描画の順番で作業を進める。

\Item{実験設定}
%====================================================================================

チュートリアル理想実験として、積雲対流の理想実験を用意している。
この実験では、積乱雲が発生するときの
典型的な大気の鉛直プロファイルと対流圏下層に初期擾乱を与え、
積乱雲が発達する様子を準2次元モデルで実験する内容となっている。
実験設定を表\ref{tab:setting_ideal}に示す。

\begin{table}[htb]
\begin{minipage}{150mm}
\begin{center}
\caption{本章での理想実験の実験設定}
\begin{tabularx}{150mm}{|l|X|X|} \hline
 \rowcolor[gray]{0.9} 項目 & 設定内容 & 備考 \\ \hline
 MPIプロセス数 & 東西:2、南北:1 & 計2プロセスでの並列計算を行う \\ \hline
 水平格子間隔 & 東西:500 m、南北:1000 m & 東西-鉛直の面を切り取った準2次元実験である \\ \hline
 水平格子点数 & 東西:40、南北:2\footnote{現在は2次元実験を行うための枠組みは用意されていないが、{\YDIR}に同じ値をもつ初期値を与える事で2次元実験に相当する実験を行うことが可能である。この場合、ハロの格子数と同じ数の格子数を{\YDIR}に設定する必要がある。ハロの必要格子数については\ref{subsec:atmos_dyn_scheme}参照。} &  \\ \hline
 鉛直層数     & 97層(トップ:20 km)& 下層ほど細かい層厚をもつストレッチ設定である \\ \hline
 側面境界条件 & 周期境界 & 東西、南北境界とも \\ \hline
 積分時間間隔 & 5 sec      & 雲微物理スキームは10 sec毎 \\ \hline
 積分期間     & 3,600 sec  & 720 steps \\ \hline
 データ出力間隔 & 300 sec  &  \\ \hline
 物理スキーム & 雲微物理モデルのみ使用 &
 6-class single moment bulk model \citep{tomita_2008} \\ \hline
 初期鉛直プロファイル & GCSS Case1 squall-line \citep{Redelsperger2000}&
 風のプロファイルは、\citet{Ooyama_2001}に基づいた鉛直シアを与える \\ \hline
 初期擾乱 & ウォームバブル & 水平半径4 km、
 鉛直半径3 kmの大きさを持つ最大プラス3Kの強度のウォームバブルを置く\\ \hline
\end{tabularx}
\label{tab:setting_ideal}
\end{center}
%\footnotetext{現在は2次元実験を行うための枠組みは用意されていないが、{\YDIR}に同じ値をもつ初期値を与える事で2次元実験に相当する実験を行うことが可能である。この場合、ハロの格子数と同じ数の格子数を{\YDIR}に設定する必要がある。ハロの必要格子数については\ref{subsec:atmos_dyn_scheme}参照。}
\end{minipage}
\end{table}


\subsection{前準備} \label{subsec:ideal_exp_prepare}
%------------------------------------------------------
チュートリアル理想実験は、\verb|scale-rm/test/tutorial/ideal|の
ディレクトリにて実行する。
このディレクトリに移動し、
scale-{\version}/binにある実行バイナリへの静的リンクを張る。
\begin{alltt}
  $ cd scale-rm/test/tutorial/ideal
  $ ln -s ../../../../bin/scale-rm      ./
  $ ln -s ../../../../bin/scale-rm_init ./
\end{alltt}
``\verb|scale-rm|''はモデル本体、
``\verb|scale-rm_init|''は初期値・境界値作成ツールである。

\subsection{初期値作成} \label{subsec:ideal_exp_init}
%------------------------------------------------------
初期値の作成は、\verb|scale-rm_init|に
設定ファイルを与えて実行する。
\verb|init_R20kmDX500m.conf| には、
表\ref{tab:setting_ideal} に対応した実験設定が書き込まれている。
この設定ファイルを\verb|scale-rm_init|に与えることで、
設定ファイルの指示に従って大気の成層構造を計算し、
初期擾乱が設定される。


SCALEの基本的な実行コマンドは下記のとおりである。
\begin{alltt}
  $ mpirun  -n [プロセス数]  [実行バイナリ名]  [設定ファイル]
\end{alltt}
[プロセス数]の部分にはMPI並列で使用したいプロセス数を記述する。
[実行バイナリ]には、\verb|scale-rm|や\verb|scale-rm_init|が入る。
そして、実験設定を記述した設定ファイルを[設定ファイル]の部分に指定する。
%
例えば、
\verb|init_R20kmDX500m.conf|は2-MPI並列(2つのMPIプロセス)を使って計算するように指定されている。
\verb|scale-rm_init|を実行するコマンドは次のように記述する。
\begin{alltt}
  $ mpirun  -n 2  ./scale-rm_init  init_R20kmDX500m.conf
\end{alltt}
%
\noindent 実行が成功した場合には、コマンドラインのメッセージは
下記のように表示される。\\

\noindent {\small {\gt
\fbox{
\begin{tabularx}{140mm}{l}
 *** Start Launch System for SCALE-RM\\
 *** Execute preprocess? :  T\\
 *** Execute model?      :  F\\
 *** a single comunicator\\
 *** a single comunicator\\
 *** End   Launch System for SCALE-RM\\
\end{tabularx}
}}}\\


\noindent この実行によって、
\begin{alltt}
  init_LOG.pe000000
  init_00000101-000000.000.pe000000.nc
  init_00000101-000000.000.pe000001.nc
\end{alltt}
の3つのファイルが、現在のディレクトリ下に作成される。
``init\_LOG.pe000000''には、
コマンドラインには表示されない詳しい実行ログが記録されている。
この例では2つのMPIプロセスを利用しているが、
デフォルト設定では0番目のプロセス(マスターランク)のログファイルだけが
出力される。すべてのプロセスの実行ログを出力するよう設定を変更することも出来る。
実行が正常に終了している場合、このLOGファイルの最後に\\

\noindent {\small {\gt
\fbox{
\begin{tabularx}{140mm}{l}
 ++++++ Finalize MPI\\
 ++++++ MPI is peacefully finalized\\
\end{tabularx}
}}}\\

\noindent と記述される。

\verb|init_00000101-000000.000.pe000000.nc|と\verb|init_00000101-000000.000.pe000001.nc|の
2つのファイルは初期値ファイルであり、それぞれおおよそ100KBのファイルサイズになる。
計算領域全体を2つのMPIプロセスで分割し担当するため、
2つのファイルが生成される。
もし、4-MPI並列で実行すれば、4つの初期値ファイルが生成される。
これらのファイル名の末尾が``.nc''で終わるファイルは
NetCDF形式のファイルであり、
GPhys/Ruby-DCLやncviewといったツールで直接読むことができる。



\subsection{モデル本体の実行} \label{subsec:ideal_exp_run}
%------------------------------------------------------
プロセス並列数は、初期値作成のときと同じ数を指定する。
設定ファイルには``\verb|run_R20kmDX500m.conf|''を指定する。
\begin{alltt}
  $ mpirun  -n  2  ./scale-rm  run_R20kmDX500m.conf
\end{alltt}

本書の必要要件にあった計算機であれば、2分程度で計算が終わる。
\noindent この実行によって、\\
\begin{alltt}
  LOG.pe000000
  history.pe000000.nc
  history.pe000001.nc
\end{alltt}
の3つのファイルが、現在のディレクトリ下に作成されているはずである。
``LOG.pe000000''には、
コマンドラインには表示されない詳しい実行ログが記録されている。
実行が正常に終了している場合、このLOGファイルの最後に\\

\noindent {\small {\gt
\fbox{
\begin{tabularx}{140mm}{l}
 ++++++ Finalize MPI\\
 ++++++ MPI is peacefully finalized\\
\end{tabularx}
}}}\\

\noindent と記述される。
``history.pe000000.nc''と``history.pe000001.nc''
の2つのファイルが計算結果が記録されたhistoryファイルであり、それぞれおおよそ5.4MBのファイルサイズになる。
2-MPI並列で実行したため、2つのファイルが生成されており、
ファイル形式はNetCDFである。
%``monitor.pe000000''は、計算中にモニタリングしている
%物理変数の時間変化を記録したテキストファイルである。

