\newcommand{\namelist}[1]{{\color{magenta}\texttt{[\detokenize{#1}]}}}
\newcommand{\nmitem}[1]{{\color{magenta}\texttt{(\detokenize{#1})}}}
%数式モード
\newcommand{\nmitemeq}[1]{{\texttt{\detokenize{#1}}}}
\newcommand{\XDIR}{X方向}
\newcommand{\YDIR}{Y方向}
\newcommand{\ZDIR}{Z方向}

%%%%
%%%% for chapter 5
\newcommand{\SecBasicDomainSetting}{{対象計算領域の設定}}
\newcommand{\SubsecRelationOfResoGridProcess}{{計算領域と解像度、格子点数、MPIプロセスの関係}}
\newcommand{\SubsecDomainSetting}{{計算領域の設定}}
\newcommand{\SubsecMPIProcess}{{MPIプロセス数の設定}}
\newcommand{\SubsecGridNumSettng}{{水平・鉛直格子数の設定}}
\newcommand{\SubsecGridIntvSettng}{{水平・鉛直格子間隔の設定}}
\newcommand{\SecBasicBufferSetting}{{緩和領域の設定}}
\newcommand{\SecBasicTopoSetting}{{地形の設定}}
\newcommand{\SecBasicIntegrationSetting}{{積分時間と積分時間間隔の設定}}
\newcommand{\SecBasicOutputSetting}{{出力変数の追加・変更方法}}
\newcommand{\SecBasicDynamicsSetting}{{力学スキームの設定}}
\newcommand{\SubsecDynsolverSetting}{{数値解法の設定}}
\newcommand{\SubsecDynSchemeSetting}{{時間・空間差分スキームの設定}}
\newcommand{\SecBasicPhysicsSetting}{{物理スキームの設定}}
\newcommand{\SubsecMicrophysicsSetting}{{雲微物理スキームの設定}}
\newcommand{\SubsecTurbulenceSetting}{{乱流スキームの設定}}
\newcommand{\SubsecRadiationSetting}{{放射スキームの設定}}
\newcommand{\SubsecSurfaceSetting}{{地表面(大気下端境界)の設定}}
\newcommand{\SubsecOceanSetting}{{海洋モデルの設定}}
\newcommand{\SubsecLandSetting}{{陸面モデルの設定}}
\newcommand{\SubsecUrbanSetting}{{都市モデル(大気-都市面フラックス)の設定}}
\newcommand{\SecMakeconfTool}{{設定ファイルと実験に必要なファイル一式を用意する}}

%%%%
%%%% for chapter 6
\newcommand{\SecAdvanceMapprojectionSetting}{{地図投影法と計算領域の位置の設定}}
\newcommand{\SecAdvanceInputDataSetting}{{様々な初期値・境界値データの作成方法}}
\newcommand{\SecAdvanceRestart}{{リスタート計算の方法}}
\newcommand{\SecAdvancePostprosess}{{ポスト処理}}
\newcommand{\SecAdvanceNesting}{{領域ネスティング実験の方法}}
\newcommand{\SubsecOflineNesting}{{オフライン・ネスティング実験の方法}}
\newcommand{\SubsecOnlineNesting}{{オンライン・ネスティング実験の方法}}
\newcommand{\SecAdvanceBulkjob}{{複数の実験を一括実行するバルクジョブの設定}}

%%%
\newcommand{\proofcomment}[1]{{\color{red} \Large 校正コメント: #1}}
\newcommand{\replycomment}[1]{{\color{blue} \Large 回答: #1}}

\newcommand{\netcdf}{{netCDF}}
\newcommand{\grads}{{GrADS}}

\newcommand{\scalerm}{{SCALE-RM}}
\newcommand{\scalegm}{{SCALE-GM}}
\newcommand{\scalelib}{{SCALE}}

\newcommand{\Item}[1]{~\\\noindent{\bf \large \underline{#1}}\\}

