\newcommand{\namelist}[1]{{\color{magenta}\texttt{[\detokenize{#1}]}}}
\newcommand{\nmitem}[1]{{\color{magenta}\texttt{(\detokenize{#1})}}}
\newcommand{\XDIR}{X方向}
\newcommand{\YDIR}{Y方向}
\newcommand{\ZDIR}{Z方向}
%%%%
%%%% for chapter 5
\newcommand{\SecBasicDomainSetting}{{対象計算領域の設定}}
\newcommand{\SubsecRelationOfResoGridProcess}{{計算領域と解像度、格子点数、MPIプロセスの関係}}
\newcommand{\SubsecDomainSetting}{{計算領域の設定}}
\newcommand{\SubsecMPIProcess}{{MPIプロセス数}}
\newcommand{\SubsecGridNumSettng}{{水平・鉛直格子数の設定}}
\newcommand{\SubsecGridIntvSettng}{{水平・鉛直格子間隔の設定}}
\newcommand{\SecBasicBufferSetting}{{緩和領域の設定}}
\newcommand{\SecBasicTopoSetting}{{地形の設定}}
\newcommand{\SecBasicIntegrationSetting}{{積分時間と積分時間間隔の設定}}
\newcommand{\SecBasicOutputSetting}{{出力変数の追加・変更方法}}
\newcommand{\SecBasicDynamicsSetting}{{力学スキームの設定}}
\newcommand{\SubsecDynsolverSetting}{{数値解法の設定}}
\newcommand{\SubsecDynSchemeSetting}{{時間・空間差分スキームの設定}}
\newcommand{\SecBasicPhysicsSetting}{{物理スキームの設定}}
\newcommand{\SubsecMicrophysicsSetting}{{雲微物理スキームの設定}}
\newcommand{\SubsecTurbulenceSetting}{{乱流スキームの設定}}
\newcommand{\SubsecRadiationSetting}{{放射スキームの設定}}
\newcommand{\SubsecSurfaceSetting}{{地表面(大気下端境界)の設定}}
\newcommand{\SubsecOceanSetting}{{海洋モデルの設定}}
\newcommand{\SubsecLandSetting}{{陸面モデルの設定}}
\newcommand{\SubsecUrbanSetting}{{都市モデル(大気-都市面フラックス)の設定}}

%%%%
%%%% for chapter 6
\newcommand{\SecAdvanceMapprojectionSetting}{{地図投影法と計算領域の位置の設定}}
\newcommand{\SecAdvanceInputDataSetting}{{様々な初期値・境界値データを使用する}}
\newcommand{\SecAdvanceRestart}{{リスタート計算の方法}}
\newcommand{\SecAdvancePostprosess}{{ポスト処理 : net2g}}
