%%%%%%%%%%%%%%%%%%%%%%%%%%%%%%%%%%%%%%%%%%%%%%%%%%%%%%%%%%%%%%%%%%%%%%%%%%%%%%%%%%%%%%
%  File 21_install.tex
%%%%%%%%%%%%%%%%%%%%%%%%%%%%%%%%%%%%%%%%%%%%%%%%%%%%%%%%%%%%%%%%%%%%%%%%%%%%%%%%%%%%%%

本章では、SCALEのコンパイル及び、実行に必要な環境について説明する。

\section{必要なシステム環境}
\label{sec:req_env}
%====================================================================================

以下に説明するもの以外に、SCALEは、
理化学研究所 計算科学研究機構のスーパーコンピュータ「京」、
及びFujitsu PRIME-HPC FX10の環境でも動作確認されている。\\

\noindent {\bf 推奨のシステム環境}
\begin{itemize}
  \item {\bf システムハードウェア構成}\\
必要なハードウェアは実験設定に依存するが、
ここでは第\ref{sec:tuto_ideal}章と第\ref{sec:tuto_real}章の
チュートリアル実行に必要なスベックを示す。
  \begin{itemize}
    \item {\bf CPU} : 理想実験は物理コアが2コア以上、現実大気実験は4コア以上が望ましい。
    \item {\bf Memory} : 理想実験は512MB以上、現実大気実験は2GB以上のメモリ容量が必要 (倍精度浮動小数点使用時)。
    \item {\bf HDD} : 現実大気実験には約3GBのディスク空き容量が必要。
  \end{itemize}

  \item {\bf システムソフトウェア構成}
  \begin{itemize}
  \item {\bf OS} : Linux OS、Mac OS X\\
        対応確認済みOSについては、表\ref{tab:compatible_os}を参照のこと。
  \item {\bf コンパイラ} : Cコンパイラ、Fortran\\
        FortranコンパイラはFortran2003をサポートするコンパイラを必要とする。
        対応確認済みコンパイラについては、表\ref{tab:compatible_compiler}を参照のこと。
  \item {\bf MPIライブラリ} : MPI1.0/2.0 に対応するMPIライブラリを必要とする。
        対応確認済みMPIライブラリについては、表\ref{tab:compatible_mpi}を参照のこと。
  \item {\bf ファイルI/Oライブラリ} : gzip、HDF5、NetCDF4を必要とする。\\
        HDF5/NetCDF4の代わりに、NetCDF3でも動作するが、出力ファイルサイズが大きくなる。
  \end{itemize}
\end{itemize}


\noindent {\bf あると便利なシステム環境}
\begin{itemize}
  \item {\bf データ変換ツール}:wgrib、wgrib2やNCLにより、SCALEで読込可能な入力データの作成が可能である。Tutorialの現実大気実験では、wgrib2を使用する。
  \item {\bf 描画環境}:GrADS、GPhys/Ruby-DCL、ncviewなど。
  \item {\bf 演算性能評価}:PAPIライブラリが使用可能。
\end{itemize}


\begin{table}[htb]
\begin{center}
\caption{対応確認済みOS(全てx86-64の64bit版)}
\begin{tabularx}{150mm}{|l|l|X|} \hline
 \rowcolor[gray]{0.9} OS名 & 確認済みバージョン & 備考 \\ \hline
 CentOS                & 6.5、6.6、7.0、7.1 &  \\ \hline
 openSUSE              & 13.2               &  \\ \hline
 SUSE Enterprise Linux & 11.1、11.3         & 11.1ではGNUコンパイラは使用不可 \\ \hline
 fedora                & 20                 &  \\ \hline
 Vine Linux            & 6.2、6.3           &  \\ \hline
 Mac OS X              & 10.10 Yosemite     &  \\ \hline
\end{tabularx}
\label{tab:compatible_os}
\end{center}
\end{table}

\begin{table}[htb]
\begin{center}
\caption{対応確認済みコンパイラ}
\begin{tabularx}{150mm}{|l|l|X|} \hline
 \rowcolor[gray]{0.9} コンパイラ名 & 確認済みバージョン & 備考 \\ \hline
 GNU (gcc/gfortran)    & 4.4.x, 4.8.x           & 4.4.xではコンパイル時にWarningが出ることがある \\ \hline
 Intel (icc/ifort)     & 13.0.1、14.0.2、15.0.0 & 2013年以降のバージョンを推奨 \\ \hline
 PGI (gcc/pgfortran)   & 14.3                   &  \\ \hline
\end{tabularx}
\label{tab:compatible_compiler}
\end{center}
\end{table}

\begin{table}[htb]
\begin{center}
\caption{対応確認済みMPIライブラリ}
\begin{tabularx}{150mm}{|l|l|X|} \hline
 \rowcolor[gray]{0.9} MPIライブラリ名 & 確認済みバージョン & 備考 \\ \hline
 openMPI   & 1.8.5                    & \\ \hline
 Intel MPI & 4.0 Update 3、4.1.0、5.0 & 2013年以降のバージョンを推奨 \\ \hline
 SGI MPT   & 2.05、2.09               & Intel Compilerとの組み合わせで確認済み \\ \hline
\end{tabularx}
\label{tab:compatible_mpi}
\end{center}
\end{table}



\section{ライブラリ環境のインストール}
\label{sec:inst_env}
%====================================================================================
SCALEライブラリに必要な各種ライブラリ(C, Fortranコンパイラ、NetCDF, HDF5, MPI 等)
のインストールを行う。
詳細は、付録 \ref{sec:env_setting}章を参照のこと。
第\ref{sec:tuto_ideal}章、第\ref{sec:tuto_real}章のチュートリアルは、
それらのライブラリ環境がインストールされていることを前提として進める。

描画ツールとして、
第\ref{sec:tuto_ideal}章、第\ref{sec:tuto_real}章のチュートリアルでは
クイックビューのためGPhysを使用する。
また、GrADSを使った結果の描画方法についても紹介する。
これらの描画ツールの詳細とインストール方法については
付録 \ref{sec:env_vis_tools}を参照のこと。



\section{SCALEのコンパイル} \label{sec:source_code}
%====================================================================================

以下の説明で使用した環境は次のとおりである。
\begin{itemize}
\item CPU: Intel Core i5 2410M 2コア/4スレッド
\item Memory: DDR3-1333 4GB
\item OS: CentOS 6.6 x86-64, CentOS 7.1 x86-64, openSUSE 13.2 x86-64
\item GNU C/C++, Fortran compiler (付録\ref{sec:env_setting}章参照)
\end{itemize}

\subsection{ソースコードの入手}
%-----------------------------------------------------------------------------------
安定版ソースコードは、\\
 \url{http://scale.aics.riken.jp/ja/download/index.html}\\
よりダウンロードできる。
ソースコードのtarballファイルを展開すると\verb|scale/|というディレクトリができる。
\verb|{ver}|には、``\version''等のバーション番号が入る。
\begin{verbatim}
 $ tar -zxvf scale-{ver}.tar.gz
 $ ls ./
   scale/
\end{verbatim}



\subsection{Makedefファイルと環境変数の設定}
%-----------------------------------------------------------------------------------

SCALEはコンパイルするとき、環境変数"\verb|SCALE_SYS|"に設定した
Makedefファイルを使用してコンパイルが行われる。
Makdefファイルは、\verb|scale/sysdep/|内にいくつかの計算機環境に
対応するファイル(\verb|Makedef.***|)が準備されており、
これらの中から自分の環境にあったものを設定する。
動作確認済みの環境と対応するMakedefファイルを表\ref{tab:makedef}に示す。

\begin{table}[htb]
\begin{center}
\caption{環境例と対応するMakedefファイル}
\begin{tabularx}{150mm}{|l|l|X|l|} \hline
 \rowcolor[gray]{0.9} OS/計算機 & コンパイラ & MPI & Makedefファイル \\ \hline
              & gcc/gfortran & openMPI & Makedef.Linux64-gnu-ompi \\ \cline{2-4}
 Linux OS x86-64 & icc/ifort & intelMPI & Makedef.Linux64-intel-impi \\ \cline{2-4}
              & icc/ifort    & SGI-MPT & Makedef.Linux64-intel-mpt \\ \hline
 Mac OS X     & gcc/gfortran & openMPI & Makedef.MacOSX-gnu-ompi \\ \hline
 スーパーコンピュータ「京」 & fccpx/frtpx & mpiccpx/mpifrtpx & Makedef.K \\ \hline
 Fujitsu PRIME-HPC FX10   & fccpx/frtpx & mpiccpx/mpifrtpx & Makedef.FX10 \\ \hline
\end{tabularx}
\label{tab:makedef}
\end{center}
\end{table}



Linux OS、GNUコンパイラ、openMPIを使用する場合には、
\verb|"Makedef.Linux64-gnu-ompi"|が対応するファイルとなる。
下記の通り、環境変数を設定する。
\begin{verbatim}
 $ export SCALE_SYS="Linux64-gnu-ompi"
\end{verbatim}
自分の環境に合致するものがなければ、既存ファイルをベースに各自作成する。
実行環境が常に同じであるならば、
環境変数の設定を\verb|.bashrc|などの環境設定ファイルに
記述しておくと便利である。\\


各種ライブラリのインストールを
付録\ref{sec:env_setting}章に従って行った場合を除き、
下記のPATHの設定が必要となる。
HDF5とNetCDF4については、下記のように環境変数を設定する。
例えば、Intel コンパイラを利用して、
HDF5を\verb"/opt/hdf5"、NetCDF4を\verb|/opt/netcdf|に
それぞれインストールした場合の例を示す。
適宜、各自の環境に応じて読み替えて設定すること。
\begin{verbatim}
 $ export HDF5="/opt/hdf5"
 $ export NETCDF4="/opt/netcdf"
 $ export NETCDF_INCLUDE="-I/opt/netcdf/include"
 $ export NETCDF_LIBS="-L/opt/hdf5/lib64 -L/opt/netcdf/lib64 -lnetcdff -hdf5_hl -lhdf5 -lm -lz"
\end{verbatim}


\subsection{コンパイル}
\label{sec:compile}
%-----------------------------------------------------------------------------------

チュートリアル用のディレクトリに移動して、makeコマンドによってコンパイルを行う。
\begin{verbatim}
 $ cd scale/scale-rm/test/tutorial/bin
 $ make -j 4
\end{verbatim}
\verb|make|のあとの \verb|"-j 4"| は、
コンパイル時の並列数(例では4並列)を示しており、
実行環境によって並列数を指定すれば良い。
コンパイルが成功すると下記3つの実行ファイルが生成される。
\begin{verbatim}
 scale-rm  scale-rm_init  scale-rm_pp
\end{verbatim}


下記のコマンドで作成された実行バイナリを消去できる。
\begin{verbatim}
 $ make clean
\end{verbatim}
ただし、この場合、コンパイルされたライブラリは消去されないため、
全てのコンパイル済みファイルを消去したい場合は、
\begin{verbatim}
 $ make allclean
\end{verbatim}
とする。
コンパイル環境、コンパイルオプションを変更して再コンパイルする場合は、
``allclean''を実行すること。


{\bf 注意点}
\begin{itemize}
\item SCALEは、scaleのTOPディレクトリ直下の
 \verb|scale/scalelib/|というディレクトリ内でコンパイルとアーカイブが行われる。
 SCALE-RM は、コンパイルを実行したディレクトリの下の
 \verb|".lib"|という名前の隠しディレクトリの中にコンパイルされたオブジェクトファイルが置かれ、実行バイナリファイルはカレントディレクトリにコピーされる。
\item Debugモードでコンパイルしたい場合は、\verb|"make -j 4 DEBUG=T"|としてコンパイルする。
\item 細かくコンパイルオプションを変更したい場合は、\verb|Makedef.***|のファイルを編集する。
%\item 開発版ソースコードをコンパイルしている場合,
% 一部のコンパイラバージョンにおいて、コンパイルが正常に終了しないケースがある.
% そのような場合はSCALE開発チームまでご報告ください.
\end{itemize}



\section{後処理ツール(net2g)のコンパイル} \label{sec:source_net2g}
%====================================================================================

SCALE-RMの出力ファイルは、ノードごとに分割されて出力される。
SCALEでは、これら出力ファイル(\verb|history.******.nc|)を結合し、
GrADSで直接読み込めるデータ形式へ
変換する後処理ツール「net2g」を提供している。
第\ref{sec:tuto_ideal}章、第\ref{sec:tuto_real}章のチュートリアルでも使用する。
ここでは、net2gのコンパイル方法について説明する。

%net2gはSCALE本体から独立したツールになっている
%(ただしMakedefファイルを除く)ため、
%任意の場所へコピーしてコンパイルすることができるが、
%コンパイルにはNetCDFライブラリが必要であり、
%また並列実行するためにはMPIライブラリが必要である。
%従って、以降はこれらのライブラリがインストールされている環境であることを想定して進める。\\


まず、SCALE本体のコンパイル時と同様に
使用する環境にあったMakedefファイル設定のための環境変数を設定する
(\ref{sec:compile}節コンパイル参照)。
\begin{verbatim}
 $ export SCALE_SYS="Linux64-gnu-ompi"  //GNUコンパイラ、openMPIを使用する場合
\end{verbatim}

次に、net2gのディレクトリに移動し、makeする。
MPIライブラリを用いた並列実行を行うためのバイナリは、
下記のコマンドによって生成される。
\begin{verbatim}
 $ cd scale/scale-rm/util/netcdf2grads_h
 $ make -j 2
\end{verbatim}
MPIライブラリが無い場合など、逐次実行バイナリを生成するためには、\\
\begin{verbatim}
 $ make -j 2 NOMPI=T
\end{verbatim}
としてコンパイルを行う。
"net2g"という名前の実行ファイルが生成されていればコンパイルは成功である.\\

下記のコマンドで作成された実行バイナリを消去できる。
\begin{verbatim}
 $ make clean
\end{verbatim}

%%%%%%%%%%%%%%%%%%%%%%%%%%%%%%%%%%%%%%%%%%%%%%%%%%%%%%%%%%%%%%%%%%%%%%%%%%%%%%%%%%%%%%

