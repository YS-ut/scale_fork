%%%%%%%%%%%%%%%%%%%%%%%%%%%%%%%%%%%%%%%%%%%%%%%%%%%%%%%%%%%%%%%%%%%%%%%%%%%%%%%%%%%%
%  File 54_namelist_history.tex
%%%%%%%%%%%%%%%%%%%%%%%%%%%%%%%%%%%%%%%%%%%%%%%%%%%%%%%%%%%%%%%%%%%%%%%%%%%%%%%%%%%%


\section{出力変数の追加・変更}
\label{sec:output}
%====================================================================================
ここでは、``run.***.conf''ファイルの中の出力変数の設定方法について説明する。
history出力ファイルへ新たに出力変数を追加するには、正式には次の2段階の手続きが必要である。

\begin{enumerate}
\item ソースファイル内の設定。対象の変数をhistory出力するための設定で、この手続きにより、出力のための準備が行われる。
\item run.conf内の設定。実際にhistoryに出力するかどうかを指定。コンパイルし直すことなく、実験毎に変更可能。
\end{enumerate}
頻繁に使用する予報変数や主要な変数については、すでに1の手続きは行われているため、2の手続きだけを行えばよい。
2の手続きだけで出力可能な変数については、Appendix\ref{app:vari_hist}にリストアップしてあるので、そちらを参照すること。
これ以外の変数をユーザーが書き出したい場合は、1の手続きが必要であるがここでは割愛する。


history出力ファイルへの変数の追加は、下記のフォーマットに従ってrun.confに追記すればよい。\\

\noindent {\small {\gt
\ovalbox{
\begin{tabularx}{140mm}{l}
\verb|&HISTITEM ITEM     = "[character]",| \\
\textcolor{cyan}{\verb|         BASENAME = "[character]",|} \\
\textcolor{cyan}{\verb|         TINTERVAL= [real],|} \\
\textcolor{cyan}{\verb|         TUNIT    = "[character]", |} \\
\textcolor{cyan}{\verb|         TAVERAGE = [logical],|} \\
\textcolor{cyan}{\verb|         ZINTERP  = [logical], |} \\
\textcolor{cyan}{\verb|         DATATYPE = "[character]",|} \\
\verb|/| \\
\end{tabularx}
}}}\\

\textcolor{cyan}{水色文字}の部分はオプションである。\verb|[character]|は任意の文字列、
\verb|[real]|は任意の浮動小数点値、\verb|[logical]|は任意の論理変数(\verb|.true.| or \verb|.false.|)
をそれぞれ指定することを意味する。デフォルト設定は、
つぎに説明する\verb|&PARAM_HISTORY|の項目によって定められる。\\

\noindent {\small {\gt
\ovalbox{
\begin{tabularx}{140mm}{l}
\verb|&PARAM_HISTORY| \\
\verb|  HISTORY_DEFAULT_BASENAME  = "[character]",| \\
\verb|  HISTORY_DEFAULT_TINTERVAL = [real],| \\
\verb|  HISTORY_DEFAULT_TUNIT     = "[character]",| \\
\verb|  HISTORY_DEFAULT_TAVERAGE  = [logical],| \\
\verb|  HISTORY_DEFAULT_ZINTERP   = [logical],| \\
\verb|  HISTORY_DEFAULT_DATATYPE  = "[character]",| \\
\verb|  HISTORY_OUTPUT_STEP0      = [logical],| \\
\verb|/| \\
\end{tabularx}
}}}\\

上記のとおり、変数毎の個別の設定群とデフォルト設定群の違いはあるが、各設定変数の意味は同じである。
それらの各設定変数の説明は以下の通りである。\\
%%\begin{table}[h]
%{\renewcommand\arraystretch{1.2}
%\begin{tabular}{ll}
%\hline

\begin{table}[htb]
\begin{center}
\caption{出力ファイル設定変数の説明}
\begin{tabularx}{150mm}{|l|X|} \hline
 \rowcolor[gray]{0.9} 設定変数 & 説明 \\ \hline
 ITEM                   & 変数名。 Appendix \ref{app:vari_hist}を参照\\ \hline
 BASENAME               & 出力ファイル名 BASENAME\_xxxxxx.ncとなる。xxxxxxはノード番号\\ \hline
 TINTERVAL              & 出力間隔\\ \hline
 TUNIT                  & TINTERVALで指定した出力間隔の単位\\ \hline
 TAVERAGE               & .false.=瞬間値、.true.=平均値として出力する;平均値の場合、
 出力タイミングの直前のTINTERVAL間の平均値\\ \hline
 DATATYPE               & 出力値の型;''REAL4'',''REAL8''など\\ \hline
 ZINTERP                & .false.=モデル面、.true.=Z面(FZ面$?$、CZ面$?$)の値として出力\\ \hline
 HISTORY\_OUTPUT\_STEP0 & .false.=初期時刻(t=0)の値を出力、.true.=出力しない;\verb|&PARAM_HISTORY|でのみ有効\\ \hline
\end{tabularx}
\label{tab:history_settings}
\end{center}
\end{table}

\verb|&PARAM_HISTORY|の\verb|HISTORY_DEFAULT_TINTERVAL|は、デフォルトのhistory出力間隔を設定する変数だが、
出力アイテム毎に出力間隔を設定することもできる。一部の変数でhistory出力間隔を変更したい場合の記述例を次に挙げる。\\

\noindent {\small {\gt
\ovalbox{
\begin{tabularx}{140mm}{l}
\verb|&PARAM_HISTORY| \\
\verb|  HISTORY_DEFAULT_BASENAME  = "history_d03",| \\
\verb|  HISTORY_DEFAULT_TINTERVAL = 3600.D0,| \\
\verb|  HISTORY_DEFAULT_TUNIT     = "SEC",| \\
\verb|  HISTORY_DEFAULT_TAVERAGE  = .false.,| \\
\verb|  HISTORY_DEFAULT_DATATYPE  = "REAL4",| \\
\verb|  HISTORY_DEFAULT_ZINTERP   = .false.,| \\
\verb|  HISTORY_OUTPUT_STEP0      = .true.,| \\
\verb|/| \\
 \\
\verb|&HISTITEM item="T"    /| \\
\verb|&HISTITEM item="PRES" /| \\
\verb|&HISTITEM item="U"    /| \\
\verb|&HISTITEM item="V"    /| \\
\verb|     〜 中略 〜|\\
 \\
\verb|&HISTITEM item="RAIN", taverage=.true., tinterval=600.D0 /| \\
\end{tabularx}
}}}\\

上記のように設定すると、``RAIN''は600秒の出力間隔で、それ以外は3600秒の出力間隔となる。
以上で、出力変数の追加・変更についての説明を終える。


%%%%%%%%%%%%%%%%%%%%%%%%%%%%%%%%%%%%%%%%%%%%%%%%%%%%%%%%%%%%%%%%%%%%%%%%%%%%%%%%%%%%
