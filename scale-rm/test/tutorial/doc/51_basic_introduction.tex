\section{概要} \label{chap:basic_usel_intro}

この章では、チュートリアルから発展して、基本的な様々な設定が出きるように、
各種設定を網羅的に記述している。
各節で閉じており、以下の項目に記述されており、辞書代わりに使ってほしい。

{%\small
\begin{center}
\begin{tabular}[h]{ll}\hline
対象計算領域の設定 & 第\ref{sec:domain} 節 \\
計算領域と解像度、格子点数、MPIプロセスの関係 & 第\ref{subsec:relation_dom_reso} 節 \\
計算領域の設定 & 第\ref{subsec:relation_dom_reso2} 節 \\
MPIプロセス数 & 第\ref{subsec:relation_dom_reso3} 節 \\
水平・鉛直格子数 & 第\ref{subsec:relation_dom_reso4} 節 \\
水平・鉛直格子間隔 & 第\ref{subsec:gridinterv} 節 \\
緩和領域の設定 & 第\ref{sec:buffer} 節 \\
積分時間と積分時間間隔の設定 & 第\ref{sec:timeintiv} 節 \\
出力変数の追加・変更 & 第\ref{sec:output} 節\\
力学スキームの設定 & 第\ref{sec:atmos_dyn} 節 \\
数値解法  & 第\ref{subsec:atmos_dyn_sover} 節 \\
時間・空間差分スキーム & 第\ref{subsec:atmos_dyn_scheme} 節 \\
物理スキームの設定 & 第\ref{chap:basic_usel_physics} 節 \\
雲微物理スキーム & 第\ref{chap:basic_usel_microphys} 節 \\
乱流スキーム & 第\ref{chap:basic_usel_turbulence} 節 \\
放射スキーム & 第\ref{chap:basic_usel_radiation} 節 \\
地表面(大気下端境界) & 第\ref{chap:basic_usel_surface} 節 \\
海洋モデル & 第\ref{chap:basic_usel_ocean} 節 \\
陸面モデル & 第\ref{chap:basic_usel_land} 節 \\
都市モデル(大気-都市面フラックス) & 第\ref{chap:basic_usel_urban} 節 \\
\hline
\end{tabular}
\end{center}
}
