\section{\SecBasicTopoSetting} \label{subsec:basic_usel_topo}
%-----------------------------------------------------------------------

\scalerm では地形に沿った座標系を採用している。この座標系はモデル下端の格子面を
与えられた標高に対して沿うように与える。
許容される最大の地形傾斜角度$\theta_{\max}$[radian]は次の式で計算される。
\[
  \theta_{\max} = \arctan( \mathrm{RATIO} \times \mathrm{DZ}/\mathrm{DX} )
\]
ここで、$\mathrm{DZ}$、$\mathrm{DX}$ は、それぞれ鉛直と水平の格子間隔である。
上記の計算式から分かるように、許容される最大傾斜角度は空間解像度に応じて変わる。

\scalerm では$\mathrm{RATIO}$のデフォルト値を1.0に設定している。
$\mathrm{RATIO}$の設定を、1.0よりも大きくすれば地形がより細かく、1.0よりも小さくすれば地形がより粗く表現される。
ただし$\mathrm{RATIO}$を1.0よりも大きくした場合、計算が途中で破綻する危険性が高くなる。

外部入力する標高データを実験設定に合わせて変換する作業は、\verb|scale-rm_pp|プログラムで行ない、
詳細は実行時に使用する設定ファイル(\verb|pp.conf|)の\namelist{PARAM_CNVTOPO}の中で設定する。
以下に例を示す。\\

\noindent {\small {\gt
\ovalbox{
\begin{tabularx}{150mm}{lX}
\verb|&PARAM_CNVTOPO  |                  & \\
 \verb|CNVTOPO_name                  = "GTOPO30", | & ; 使用する地形データ名\\
 \verb|CNVTOPO_smooth_maxslope_ratio = 1.0,       | & ; 許容する傾斜の$\mathrm{DZ}$/$\mathrm{DX}$に対する倍率 \\
 \verb|CNVTOPO_smooth_local          = .true.,    | & ; 最大傾斜角度を超えた格子のみ平滑化を行うかどうか \\
 \verb|CNVTOPO_copyparent            = .false.,   | & ; 緩和領域に親ドメインの地形をコピーするかどうか \\
\verb|/|\\
\end{tabularx}
}}}\\

\scalerm では変換に使用する外部入力標高データベースとして、GTOPO30、
または国土地理院による高精度地形データ(DEM50M)をサポートしている。


標高データは設定した水平格子に合わせて面積重み付け平均される。この変換された標高データの時点で、
隣り合う格子との標高差を傾斜角として計算する。最大傾斜角度を超える傾斜角があった場合、
それを最大傾斜角度以下になるように、反復計算を用いてラプラシアンフィルターでの平滑化を実行する。\\
このとき、最大傾斜角度を超えた格子のみ平滑化を行うか、計算領域全体で行うかを選択することができる。
前者は、最大傾斜角度以内のシャープな地形構造を残すことができるので、細かな地形表現を望む場合に選択する。
平滑化フィルターは、ガウシアンフィルターを選択することも出来る。こちらを用いた場合はラプラシアンフィルタより平滑化された地形になる。


\nmitem{CNVTOPO_copyparent}は、ネスティング計算のための設定項目である。
一般的に、多段ネスティング計算を行う場合、
子ドメインのほうが空間解像度が細かいため、地形もシャープに表現される。
このとき、親ドメインと子ドメインの地形表現が異なるために、
子ドメインの緩和領域で用いられる親ドメインの大気データと不整合を起こすことがある。
これを回避するために、
\nmitem{CNVTOPO_copyparent}を\verb|.true.|とすることで、親ドメインの地形を子ドメインの緩和領域にコピーすることができる。
親ドメインが存在しない場合は\nmitem{CNVTOPO_copyparent}を必ず\verb|.false.|に設定しなければならない。
\nmitem{CNVTOPO_copyparent}を利用する場合の設定については、第\ref{subsec:nest_topo}節で詳しく説明する。
