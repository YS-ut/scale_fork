%-------------------------------------------------------%
\section{時間積分を行う:run} \label{sec:tutrial_real_run}
%-------------------------------------------------------%
\subsubsection{run.confの準備}
runディレクトリへ移動する。
\begin{verbatim}
 $ cd ${Tutorial_DIR}/real/experiment/run
\end{verbatim}
%
runディレクトリの中には、\verb|run.d01.conf|という名前の
設定ファイルが準備されており、
ドメインの位置や格子点数など、
チュートリアル用の設定(表\ref{tab:grids})に合わせて設定されている。\\

\noindent {\small {\gt
\ovalbox{
\begin{tabularx}{140mm}{l}
\verb|&PARAM_PRC| \\
\verb| PRC_NUM_X      = 2,| \\
\verb| PRC_NUM_Y      = 2,| \\
\verb| PRC_PERIODIC_X = .false.,| \\
\verb| PRC_PERIODIC_Y = .false.,| \\
\verb|/| \\
 \\
\verb|&PARAM_INDEX| \\
\verb| KMAX = 36,| \\
\verb| IMAX = 45,| \\
\verb| JMAX = 45,| \\
\verb|/| \\
\end{tabularx}
}}}\\


{\XDIR} 、{\YDIR}ともに2分割されており、
総計として4つのMPIプロセスを使用する設定となっている。
1つのMPIプロセスあたりの格子点数については、
\nmitem{IMAX = 45}、\nmitem{JMAX = 45}と指定されているため、
{\XDIR} 、{\YDIR}の総格子点数は、ともに$2 \times 45$ で90である。
計算ドメインの大きさは、
\namelist{PARAM_GRID}の\nmitem{DX,DY}はともに20000 m(20 km)
と指定されているため、
一辺の長さが90 $\times$ 20 km より、1800 km $\times$ 1800 km の正方形の計算領域
が設定されていることがわかる。


モデル本体の実行には
事前に作成した地形・土地利用データや初期値・境界値データを利用する。
これらのファイルの指定は、
\verb|run.d01.conf|の下記部分で設定する。\\

\noindent {\gt
\ovalbox{
\begin{tabularx}{140mm}{l}
\verb|&PARAM_TOPO| \\
\verb|   TOPO_IN_BASENAME = "../pp/topo_d01",| \\
\verb|/| \\
 \\
\verb|&PARAM_LANDUSE| \\
\verb|   LANDUSE_IN_BASENAME  = "../pp/landuse_d01",| \\
\verb|/| \\
 \\
\verb|&PARAM_RESTART| \\
\verb| RESTART_RUN          = .false.,|\\
\verb| RESTART_OUTPUT       = .true., |\\
\verb| RESTART_OUT_BASENAME = "restart_d01",|\\
\verb| RESTART_IN_BASENAME  = "../init/init_d01_20070714-180000.000",|\\
\verb|/| \\
 \\
\verb|&PARAM_ATMOS_BOUNDARY| \\
\verb| ATMOS_BOUNDARY_TYPE           = "REAL",                |\\
\verb| ATMOS_BOUNDARY_IN_BASENAME    = "../init/boundary_d01",|\\
\verb| ATMOS_BOUNDARY_START_DATE     = 2007, 7, 14, 18, 0, 0, |\\
\verb| ATMOS_BOUNDARY_UPDATE_DT      = 21600.0,               |\\
\verb| ATMOS_BOUNDARY_USE_DENS       = .true.,     |\\
\verb| ATMOS_BOUNDARY_USE_VELZ       = .true.,     |\\
\verb| ATMOS_BOUNDARY_USE_QHYD       = .false.,    |\\
\verb| ATMOS_BOUNDARY_VALUE_VELZ     = 0.0,        |\\
\verb| ATMOS_BOUNDARY_ALPHAFACT_DENS = 1.0,        |\\
\verb| ATMOS_BOUNDARY_LINEAR_H       = .false.,    |\\
\verb| ATMOS_BOUNDARY_EXP_H          = 2.0,        |\\
\verb|/| \\
\end{tabularx}
}}\\


\verb|run.d01.conf|の設定の中で時間積分に関する設定は、
\namelist{PARAM_TIME}の項目にある。\\

\noindent {\gt\small
\ovalbox{
\begin{tabularx}{150mm}{ll}
\verb|&PARAM_TIME| & \\
\verb| TIME_STARTDATE     = 2007, 7, 14, 18, 0, 0,| &← 時間積分を開始する時刻 \\
\verb| TIME_STARTMS       = 0.D0,| &\\
\verb| TIME_DURATION      = 6.0D0,| &← 積分期間 \\
\verb| TIME_DURATION_UNIT = "HOUR",| &← \verb|TIME_DURATION|の単位\\
\verb| TIME_DT            = 90.0D0,| &← トレーサー移流計算の時間ステップ\\
\verb| TIME_DT_UNIT       = "SEC",|  &← \verb|TIME_DT|の単位\\
\verb| TIME_DT_ATMOS_DYN  = 45.0D0,| &← トレーサー移流計算以外の力学過程の時間ステップ\\
\verb| TIME_DT_ATMOS_DYN_UNIT = "SEC",| &← \verb|TIME_DT_ATMOS_DYN|の単位\\
 \\
\verb| ~~中略~~| &\\
 \\
\verb|/| &\\
\end{tabularx}
}}\\

\noindent 初期時刻は、\nmitem{TIME_STARTDATE}においてUTCで指定する。
チュートリアルでは2007年7月14日18時UTCに設定している。
積分のための時間ステップは、上記の他、
それぞれの物理スキーム毎に設定できるようになっている。


計算結果の出力に関する設定は\nmitem{PARAM_HISTORY}で行う。\\

\noindent {\gt
\ovalbox{
\begin{tabularx}{140mm}{l}
\verb|&PARAM_HISTORY| \\
\verb|   HISTORY_DEFAULT_BASENAME  = "history_d01",|  ← 出力するファイル名\\
\verb|   HISTORY_DEFAULT_TINTERVAL = 3600.D0,|      ← 出力時間間隔\\
\verb|   HISTORY_DEFAULT_TUNIT     = "SEC",|          ← 出力時間間隔の単位\\
\verb|   HISTORY_DEFAULT_TAVERAGE  = .false.,| \\
\verb|   HISTORY_DEFAULT_DATATYPE  = "REAL4",| \\
\verb|   HISTORY_DEFAULT_ZINTERP   = .false.,|  ← 出力時に高さ面へ内挿するかどうか\\
\verb|   HISTORY_OUTPUT_STEP0      = .true.,|  ← 初期時刻(t=0)の値を出力するかどうか\\
\verb|/| \\
\end{tabularx}
}}\\

\noindent 上記の設定に従って、下記の\nmitem{HISTITEM}に羅列された変数が出力される。
\nmitem{HISTITEM}ではオプション変数を加えることで、変数毎に、出力間隔を変更したり、
平均値を出力したりすることも可能である。
これらの説明は\ref{sec:output}を参照されたい。\\

\noindent {{\gt
\ovalbox{
\begin{tabularx}{140mm}{ll}
\verb|&HISTITEM item="MSLP" /|              & 海面更正気圧 \\
\verb|&HISTITEM item="PREC" /|              & 降水強度 (2次元) \\
\verb|&HISTITEM item="OLR"  /|               & 外向き赤外放射(2次元) \\
\verb|&HISTITEM item="U10" / |               & 地表10mでのX方向水平速度成分(2次元) \\
\verb|&HISTITEM item="V10" / |               & 地表10mでのY方向水平速度成分(2次元) \\
\verb|&HISTITEM item="T2"  / |              & 地表2mでの温度 (2次元) \\
\verb|&HISTITEM item="Q2"  / |              &  地表2mでの水蒸気混合比 (2次元) \\
\verb|&HISTITEM item="SFC_PRES"   /|         & 地表気圧 (2次元) \\
\verb|&HISTITEM item="SFC_TEMP"   /|        & バルクの地表面温度 (2次元) \\
\verb|&HISTITEM item="DENS" /|              & 密度 (3次元) \\
\verb|&HISTITEM item="QV"   /|              & 水蒸気の混合比 (3次元) \\
\verb|&HISTITEM item="QHYD" /|              & 凝結水物質の混合比 (3次元) \\
\verb|&HISTITEM item="PRES" /|              & 圧力 (3次元) \\
\verb|&HISTITEM item="U"    /|              & X方向水平速度成分 (3次元) \\
\verb|&HISTITEM item="V"    /|              & Y方向水平速度成分 (3次元) \\
\verb|&HISTITEM item="T"    /|              & 温度 (3次元) \\
\verb|&HISTITEM item="W"    /|              & 鉛直方向速度成分 (3次元) \\
\verb|&HISTITEM item="Uabs" /|              & 風速 (3次元) \\
\verb|&HISTITEM item="PT"   /|              & 温位 (3次元) \\
\verb|&HISTITEM item="RH"   /|              & 相対湿度 (3次元) \\
\end{tabularx}
}}}\\

\noindent その他に実験で使用される物理過程の設定は、
\namelist{PARAM_TRACER,PARAM_ATMOS,PARAM_OCEAN,PARAM_LAND,PARAM_URBAN}に
記述されている。
詳細な設定ファイルの内容については、付録\ref{achap:namelist}を参照されたい。

%
\subsubsection{シミュレーションの実行}

実行に必要なファイルとして、下記が用意されている。
\begin{alltt}
 $ ls 
  MIPAS  PARAG.29  PARAPC.29  VARDATA.RM29  cira.nc  : 放射スキーム用のパラメータファイル
  run.d01.conf       : 設定ファイル
  param.bucket.conf  : 陸面スキーム用のパラメータファイル
  scale-rm           : \scalerm 実行バイナリ
  run.launch.conf    : ネスティング計算用のlaunchファイル(チュートリアルでは使用しない)
\end{alltt}
%
準備が整ったら、4-MPI並列により\scalerm を実行する。
\begin{verbatim}
  $ mpirun -n 4 ./scale-rm run.d01.conf >& log &
\end{verbatim}


実行にはある程度時間を要するため、
上記のように標準出力をファイルへ書き出すようにして
バックグラウンドで実行すると便利である。
計算が開始されれば,処理内容のログとして、
\verb|"LOG_d01.pe000000"|が生成される。
さらに、ジョブが正常に終了すると、
\begin{verbatim}
 $ ls
  history_d01.pe000000.nc
  history_d01.pe000001.nc
  history_d01.pe000002.nc
  history_d01.pe000003.nc
\end{verbatim}
が作成される。
\verb|history_d01.pe######.nc|は
\nmitem{HISTITEM}に指定した出力変数が書き出される。
\verb|######|はMPIプロセス番号を表している。


%####################################################################################

