%%%%%%%%%%%%%%%%%%%%%%%%%%%%%%%%%%%%%%%%%%%%%%%%%%%%%%%%%%%%%%%%%%%%%
%  File 31_ideal_exp.tex
%%%%%%%%%%%%%%%%%%%%%%%%%%%%%%%%%%%%%%%%%%%%%%%%%%%%%%%%%%%%%%%%%%%%%
\section{概要} \label{sec:ideal_exp_intro}

本章では、SCALEを使った理想実験の実行方法を説明する。
第\ref{chap:install}章で実行したSCALEのコンパイルが
正常に完了しているかどうかのチェックも含めてぜひ実施してもらいたい。


\subsection{チュートリアル実行のための推奨環境} \label{subsec:assumed_env}

本章のチュートリアルの説明は、下記の環境を前提として記述している。
コンパイラ、ライブラリについては、適宜、使用環境に合わせて読み替えること。

\begin{itemize}
 \item {\bf CPU} : 物理コアが2コア以上 %[Intel Core i5 2410M 2.3GHz 2コア/4スレッド] %、第\ref{chap:tutorial_reall}章は4コアを搭載
 \item {\bf Memory} : 512MB以上 %[DDR3-1333 4GB]     %、第\ref{chap:tutorial_reall}章は2GB
 \item {\bf OS} : Linux OS x86-64  %[CentOS 6.6、CentOS 7.1、openSUSE 13.2のいずれか]
 \item {\bf コンパイラ} : GNU コンパイラ(gcc/gfortran)
 \item {\bf MPIライブラリ} : openMPI(リポジトリ経由でのインストール)
\end{itemize}

本章では、SCALEのコンパイルが正常に終了し、
すでに下記のファイルが生成されているものとして説明を行う。
\begin{alltt} 
  scale-{\version}/scale-rm/test/tutorial/bin/scale-rm
  scale-{\version}/scale-rm/test/tutorial/bin/scale-rm_init
  scale-{\version}/scale-rm/util/netcdf2grads_h/net2g
\end{alltt}
これらに加えて、本章のチュートリアルでは、
描画ツールとしてGrADSを使用する。
gpviewは、結果の確認用に利用することができる。
GrADSおよびgpview(GPhys)との詳細やインストール方法については、
付録 \ref{sec:env_vis_tools}節を参照のこと。


\subsection{実験設定} \label{subsec:idealexpsetup}
%====================================================================================

実験設定として、積雲対流の理想実験を例にあげる。
この実験では、積乱雲が発生するときの
典型的な大気の鉛直プロファイルと対流圏下層に初期擾乱を与え、
積乱雲が発達する様子を準2次元モデルで実験する内容となっている。
実験設定は表\ref{tab:setting_ideal}に示す通りである。

\begin{table}[htb]
\begin{center}
\caption{チュートリアル理想実験の実験設定}
\begin{tabularx}{150mm}{|l|X|X|} \hline
 \rowcolor[gray]{0.9} 項目 & 設定内容 & 備考 \\ \hline
 水平格子間隔 & 東西:500 m、南北:1000 m & 東西-鉛直の面を切り取った準2次元実験である \\ \hline
 水平格子点数 & 東西:40、南北:2\footnotemark &  \\ \hline
 鉛直層数     & 97層(トップ:20 km)& 下層ほど細かい層間隔をとったストレッチ設定である \\ \hline
 側面境界条件 & 周期境界 & 東西、南北とも \\ \hline
 積分時間間隔 & 5 sec      & 雲微物理スキームは10 sec毎 \\ \hline
 積分期間     & 3,600 sec  & 720 steps \\ \hline
 データ出力間隔 & 300 sec  &  \\ \hline
 物理スキーム & 雲微物理モデルのみ使用 &
 6-class single moment bulk model \citep{tomita_2008} \\ \hline
 初期鉛直プロファイル & GCSS Case1 squall-line \citep{Redelsperger2000}&
 風のプロファイルは、\citet{Ooyama_2001}に基づいた鉛直シアを与える \\ \hline
 初期擾乱 & ウォームバブル & 水平半径4 km、
 鉛直半径3 kmの大きさを持つ最大プラス3Kの強度のウォームバブルを置く\\ \hline
\end{tabularx}
\label{tab:setting_ideal}
\end{center}
\end{table}
\footnotetext{現在は2次元実験を行うための枠組みは用意されていないが、{\YDIR}に同じ値をもつ初期値を与える事で2次元実験に相当する実験を行うことが可能である。この場合、ハロの格子数と同じ数の格子数を{\YDIR}に設定する必要がある。ハロの必要格子数については\ref{subsec:atmos_dyn_scheme}参照。}



