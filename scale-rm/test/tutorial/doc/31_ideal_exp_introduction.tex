%%%%%%%%%%%%%%%%%%%%%%%%%%%%%%%%%%%%%%%%%%%%%%%%%%%%%%%%%%%%%%%%%%%%%
%  File 31_ideal_exp.tex
%%%%%%%%%%%%%%%%%%%%%%%%%%%%%%%%%%%%%%%%%%%%%%%%%%%%%%%%%%%%%%%%%%%%%
\section{概要} \label{sec:ideal_exp_intro}

本章では、まずSCALE-RMを使ってみるための基本的な操作について説明する。
第\ref{chap:install}章で実行したSCALEのコンパイルが正常に完了しているか
どうかのチェックも兼ねているのでぜひ実施してもらいたい。

%%  ====2章に書いてあるので削除した。
%% \subsubsection{本章を実行するための推奨環境}

%% 本章の説明は、下記の環境を前提として記述している。
%% コンパイラ、ライブラリについては、適宜、使用環境に合わせて読み替えること。

%% \begin{itemize}
%%  \item {\bf CPU} : 物理コアが2コア以上 %[Intel Core i5 2410M 2.3GHz 2コア/4スレッド] %、第\ref{chap:tutorial_real}章は4コアを搭載
%%  \item {\bf Memory} : 512MB以上 %[DDR3-1333 4GB]     %、第\ref{chap:tutorial_real}章は2GB
%%  \item {\bf OS} : Linux OS x86-64  %[CentOS 6.6、CentOS 7.1、openSUSE 13.2のいずれか]
%%  \item {\bf コンパイラ} : GNU コンパイラ(gcc/gfortran)
%%  \item {\bf MPIライブラリ} : openMPI(リポジトリ経由でのインストール)
%% \end{itemize}

本章では、SCALEのコンパイルが正常に終了し、
すでに下記のファイルが生成されているものとして説明を行う。
\begin{alltt} 
  scale-{\version}/scale-rm/test/tutorial/bin/scale-rm
  scale-{\version}/scale-rm/test/tutorial/bin/scale-rm_init
  scale-{\version}/scale-rm/util/netcdf2grads_h/net2g
\end{alltt}
これらに加えて、描画ツールとして\grads を使用する。
gpviewは、結果の確認用に利用することができる。
\grads およびgpview(GPhys)との詳細やインストール方法については、
付録 \ref{sec:env_vis_tools}節を参照のこと。




