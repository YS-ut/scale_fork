\section{出力変数の追加・変更} \label{sec:output}
%====================================================================================
新たな変数をhistoryファイルへ追加するには、正式には次の2段階の手続きが必要である。

\begin{enumerate}
\item ソースファイルに、対象の変数をhistory出力するための設定をする。
未設定の場合、run.confに\namelist{HISTITEM}を追加しても出力されない。
\item run.conf内の設定。各変数をhistoryファイルに出力するかどうかを指定。
コンパイルし直すことなく、実験毎に変更可能。
\end{enumerate}
主要な変数についてはすでに1の手続きは行われているため、2の手続きだけ行えばよい。
2の手続きだけで出力可能な変数については、
付録\ref{achap:histroy_item}にリストアップしてあるので、そちらを参照すること。
以下では、2の手続きについて、説明する。\\

まず、出力ファイルと出力形式について、run.**.confの
\namelist{PARAM_HISTORY}に以下の項目を設定する。\\

\noindent {\small {\gt
\ovalbox{
\begin{tabularx}{140mm}{ll}
\verb|&PARAM_HISTORY| \\
\verb|  HISTORY_DEFAULT_BASENAME  |& \verb| = "[character]",| \\
\verb|  HISTORY_DEFAULT_TINTERVAL |& \verb| = [real],| \\
\verb|  HISTORY_DEFAULT_TUNIT     |& \verb| = "[character]",| \\
\verb|  HISTORY_DEFAULT_TAVERAGE  |& \verb| = [logical],| \\
\verb|  HISTORY_DEFAULT_ZINTERP   |& \verb| = [logical],| \\
\verb|  HISTORY_DEFAULT_DATATYPE  |& \verb| = "[character]",| \\
\verb|  HISTORY_OUTPUT_STEP0      |& \verb| = [logical],| \\
\verb|/| \\
\end{tabularx}
}}}\\

\begin{table}[htb]
\begin{center}
\caption{出力ファイルと出力形式の説明}
\begin{tabularx}{150mm}{|l|X|} \hline
 \rowcolor[gray]{0.9} 設定変数 & 説明 \\ \hline
 \verb|HISTORY_DEFAULT_BASENAME  | & 出力ファイル名。\verb|BASENAME_xxxxxx.nc|に出力される。\verb|xxxxxx|はノード番号。\\ \hline
 \verb|HISTORY_DEFAULT_TINTERVAL | & 出力の時間間隔 \\ \hline
 \verb|HISTORY_DEFAULT_TUNIT     | & \verb|HISTORY_DEFAULT_TINTERVAL|の単位\\ \hline
 \verb|HISTORY_DEFAULT_TAVERAGE  | & \verb|.false.| : 瞬間値、\verb|.true.| : 平均値。平均値の場合、
 出力タイミングの直前の\verb|HISTORY_DEFAULT_TINTERVAL|間の平均値が出力される。\\ \hline
 \verb|HISTORY_DEFAULT_DATATYPE  | & 出力値の型。``REAL4'',``REAL8''など。\\ \hline
 \verb|HISTORY_DEFAULT_ZINTERP   | & 鉛直内挿するかどうか。\verb|.false.| : モデル面出力、\verb|.true.| : Z面に内挿した値として出力。\\ \hline
 \verb|HISTORY_OUTPUT_STEP0      | & 初期時刻(t=0)の値を出力するかどうか。\verb|.true.| :出力、\verb|.false.| : 出力しない。\\ \hline
\end{tabularx}
\label{tab:history_settings}
\end{center}
\end{table}


次に、どの変数を出力するか、また各出力変数毎の出力形式の設定を、
run.**.confの\namelist{HISTITEM}に追加する。
\nmitem{ITEM}に設定された変数が出力される。
\textcolor{blue}{青色文字}の部分はオプションで、
オプション変数が特に指定されていない場合は 
\namelist{PARAM_HISTORY}内の設定が適用される。
つまり、出力は、基本的には \namelist{PARAM_HISTORY}に従って行われるが、
\nmitem{HISTITEM}のオプション設定をすることにより、
各変数ごとに出力形式を指定することが可能である。\\


\noindent {\small {\gt
\ovalbox{
\begin{tabularx}{140mm}{l}
\verb|&HISTITEM|\\
                 \verb| ITEM     = "[character]",| \\
\textcolor{blue}{\verb| BASENAME = "[character]",|} \\
\textcolor{blue}{\verb| TINTERVAL= [real],|} \\
\textcolor{blue}{\verb| TUNIT    = "[character]", |} \\
\textcolor{blue}{\verb| TAVERAGE = [logical],|} \\
\textcolor{blue}{\verb| ZINTERP  = [logical], |} \\
\textcolor{blue}{\verb| DATATYPE = "[character]",|} \\
\verb|/| \\
\end{tabularx}
}}}\\


\begin{table}[htb]
\begin{center}
\caption{出力ファイルと出力形式の説明}
\begin{tabularx}{150mm}{|l|X|} \hline
 \rowcolor[gray]{0.9} 設定変数 & 説明 \\ \hline
 \verb|ITEM      | & 変数名。 付録\ref{achap:histroy_item}を参照\\ \hline
 \verb|BASENAME  | & \verb|HISTORY_DEFAULT_BASENAME|に同じ。ただし、変数(\verb|ITEM|)のみに適用。\\ \hline
 \verb|TINTERVAL | & \verb|HISTORY_DEFAULT_TINTERVAL|に同じ。ただし、変数(\verb|ITEM|)のみに適用。\\ \hline
 \verb|TUNIT     | & \verb|TINTERVAL|で指定した出力間隔の単位。\\ \hline
 \verb|TAVERAGE  | & \verb|HISTORY_DEFAULT_TAVERAGE|に同じ。ただし、変数(\verb|ITEM|)のみに適用。\\ \hline
 \verb|DATATYPE  | & \verb|HISTORY_DEFAULT_DATATYPE|に同じ。ただし、変数(\verb|ITEM|)のみに適用。\\ \hline
 \verb|ZINTERP   | & \verb|HISTORY_DEFAULT_ZINTERP|に同じ。ただし、変数(\verb|ITEM|)のみに適用。\\ \hline
\end{tabularx}
\label{tab:histitem}
\end{center}
\end{table}



下記に記述例を示す。
下記の設定では、\verb|history_d03.xxxxxx.nc|に4バイト実数で、
3600秒毎に\verb|HISTITEM|に指定されている変数が出力される。
ただし、``RAIN''については、600秒の出力間隔で、
前600秒の平均値として出力されることを意味している。\\

\noindent {\small {\gt
\ovalbox{
\begin{tabularx}{140mm}{l}
\verb|&PARAM_HISTORY| \\
\verb|  HISTORY_DEFAULT_BASENAME  = "history_d03",| \\
\verb|  HISTORY_DEFAULT_TINTERVAL = 3600.D0,| \\
\verb|  HISTORY_DEFAULT_TUNIT     = "SEC",| \\
\verb|  HISTORY_DEFAULT_TAVERAGE  = .false.,| \\
\verb|  HISTORY_DEFAULT_DATATYPE  = "REAL4",| \\
\verb|  HISTORY_DEFAULT_ZINTERP   = .false.,| \\
\verb|  HISTORY_OUTPUT_STEP0      = .true.,| \\
\verb|/| \\
 \\
\verb|&HISTITEM item="T"    /| \\
\verb|&HISTITEM item="PRES" /| \\
\verb|&HISTITEM item="U"    /| \\
\verb|&HISTITEM item="V"    /| \\
\verb|     〜 中略 〜|\\
\verb|&HISTITEM item="RAIN", taverage=.true., tinterval=600.D0 /| \\
\end{tabularx}
}}}\\


%%%%%%%%%%%%%%%%%%%%%%%%%%%%%%%%%%%%%%%%%%%%%%%%%%%%%%%%%%%%%%%%%%%%%%%%%%%%%%%%%%%%
