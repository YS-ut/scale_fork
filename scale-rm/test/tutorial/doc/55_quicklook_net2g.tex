\section{Postprocess : netcdf2grads}
\label{sec:net2g}
%====================================================================================

本節では、SCALEの領域分割されたhistoryデータ(\verb|history.***.nc|)をGrADSで
読み込める形式に変換する後処理ツール``net2g''の実行方法について説明する。
net2gのインストール方法については、\ref{sec:source_net2g}節を参照すること。net2gの実行に
おいて注意するべき点は下記のとおりである。

\begin{itemize}
 \item 使用するMPIプロセス数は、SCALEの本体の実行で使用したMPIプロセス数の約数でなければならない。
 \item history出力形式は、``\verb|HIST_BND = .false.|''でなければならない(今後対応予定)。
 \item 2次元データと3次元データは同時に変換できない(今後対応予定)。
 \item 変換できるデータはhistoryデータのみ(今後対応予定)。
\end{itemize}

``HIST\_BND''の対応、2次元データと3次元データの同時実行、そしてhistoryデータ以外のデータへの対応は
今後早急に進める予定ではあるが、どうしてもgpview等の他のツールが使えず、データの描画に支障が出る場合は、
「旧型netcdf2grads」を使用することができる。旧型netcdf2gradsは、シングルプロセス実行しかできないが、
上記の制限はない。旧型netcdf2gradsの概要については本節の最後に紹介する。

その他に注意するべき点は、net2g実行時のMPI並列数である。net2gはMPI並列実行できるが、プログラム中の
作業のほとんどがhistoryファイルからのデータ読み込みである。したがって、並列ファイルシステム上でない限り、
一般に単一のディスク、もしくは単一のディスクアレイからデータを読み込む速度は、MPIプロセス数を増やしても
向上しない。むしろ読み込み要求が立て込んで、作業効率が下がり結果として読み込み速度が落ちることもある。
したがって、新しい環境でnet2gを実行する場合、少しMPIプロセス数を変えて実行してみて最適なMPIプロセス数を
予め知っておくことも必要である。

net2gの実行方法は、MPI並列版としてコンパイルした場合は、
\begin{verbatim}
 $ mpirun  -n  [procs]  ./net2g  net2g.conf
\end{verbatim}
として実行する。\verb|[procs]|には任意のMPIプロセス数を指定する。最後の``net2g.conf''はnet2gの実行設定が
記述されたconfigファイルである。一方、シングルプロセス版としてコンパイルした場合はmpiコマンドを使わずに、
\begin{verbatim}
 $ ./net2g  net2g.conf
\end{verbatim}
とする。次にconfigファイルの記述方法と設定内容について説明する。ここでは主要な設定項目を取り上げて説明する。
ここで取り上げなかったオプションについては、``scale/scale-rm/util/netcdf2grads\_h/''のディレクトリに
入っている``net2g.all.conf''のサンプルconfigファイルを参照してもらいたい。

\subsubsection{configファイルサンプル:3次元データの変換}
%------------------------------------------------------

\noindent {\small {\gt
\ovalbox{
\begin{tabularx}{140mm}{l}
\verb|&LOGOUT| \\
\verb| LOG_BASENAME   = "LOG_d01",| \\
\verb| LOG_ALL_OUTPUT = .false.,| \\
\verb|/| \\
 \\
\verb|&INFO| \\
\verb| START_TSTEP = 1,| \\
\verb| END_TSTEP   = 121,| \\
\verb| INC_TSTEP   = 1,| \\
\verb| DOMAIN_NUM  = 1,| \\
\verb| CONFFILE    = "../run/run.d01.conf",| \\
\verb| IDIR        = "../run",| \\
\verb| Z_LEV_TYPE  = "plev",| \\
\verb|/| \\
 \\
\verb|&VARI| \\
\verb| VNAME       = "PT","U","V","W",| \\
\verb| TARGET_ZLEV = 850, 500, 200,| \\
\verb|/| \\
\end{tabularx}
}}}\\

\noindent 上記はあるdomain 1のデータの3次元変数を、指定した気圧高度面へ変換して出力する場合のconfigファイルの
全記述内容の例を示している。各々の設定項目は次の説明のとおりである。
\begin{itemize}
 \item \verb|LOGOUT|の項目(この項目は必須ではない)
 \begin{itemize}
  \item \verb|LOG_BASENAME|:デフォルトのLOGファイル名``LOG''を変更したいときに指定する。
  \item \verb|LOG_ALL_OUTPUT|:プロセス番号0番以外もLOGファイルを出力させたい場合に``true''にする。
        デフォルト値は``false''である。
 \end{itemize}
 \item \verb|INFO|の項目
 \begin{itemize}
  \item \verb|START_TSTEP|:変換を開始するnetCDFデータの最初のTime Stepを指定する。最初のいくつかの
        ステップを飛ばして変換したい場合に任意の値を指定する。デフォルト値は1である。
  \item \verb|END_TSTEP|:変換を終了するnetCDFデータのTime Stepを指定する。必ず指定する。
  \item \verb|INC_TSTEP|:netCDFデータを読み込む時の時間軸の増加量(インクリメント)を指定する。
        例えば1つ飛ばしで変換したい場合に任意の値を指定する。デフォルト値は1である。
  \item \verb|DOMAIN_NUM|:ドメイン番号を指定する。デフォルト値は1である。
  \item \verb|CONFFILE|:SCALE本体を実行したときの\verb|run.***.conf|ファイルのPATHを指定する
        (ファイル名を含む)。
  \item \verb|IDIR|:SCALE本体のhistory出力ファイルのPATHを指定する。
  \item \verb|Z_LEV_TYPE|:鉛直方向のデータ変換の種類を指定する。``original''はhistoryデータのまま
        出力、``plev''は気圧高度面に変換して出力、そして``zlev''は高度面データに変換する。
        ``anal''を指定すると別途指定の簡易解析を行なって出力する(次項で説明)。デフォルト値は``plev''である。
 \end{itemize}
 \item \verb|VARI|の項目
 \begin{itemize}
  \item \verb|VNAME|:変換したい変数名を指定する。デフォルトでは、``PT''、``PRES''、``U''、``V''、
        ``W''、``QHYD''が指定される。
  \item \verb|TARGET_ZLEV|:\verb|Z_LEV_TYPE|に応じた変換高度を指定する。plevの場合は``hPa''、
        zlevの場合は``m''、そしてoriginalの場合は格子点番号で指定する。デフォルトでは、
        1000hPa、975hPa、950hPa、925hPa、900hPa、850hPa、800hPa、700hPa、600hPa、500hPa、400hPa、
        300hPa、250hPa、200hPaの14層が指定される。 
 \end{itemize}
\end{itemize}

\subsubsection{configファイルサンプル:3次元データの変換:鉛直積算値を出す}
%------------------------------------------------------

\noindent {\small {\gt
\ovalbox{
\begin{tabularx}{140mm}{l}
\verb|&INFO| \\
\verb|    〜 中略 〜|\\
\verb| Z_LEV_TYPE  = "anal",| \\
\verb|/| \\
 \\
\verb|&ANAL| \\
\verb| ANALYSIS    = "sum"| \\
\verb|/| \\
 \\
\verb|&VARI| \\
\verb| VNAME       = "QC","QI","QG"| \\
\verb|/| \\
\end{tabularx}
}}}\\

\noindent 上記はconfigファイル記述例の抜粋である。
\verb|Z_LEV_TYPE|を``anal''と指定すると\verb|ANAL|項目を指定できるようになる。これは、鉛直次元に
大して何からの簡易解析を施してデータ出力する機能である。この場合は、変換後のデータは必ず水平2次元データとなるため
、\verb|VARI|の項目の\verb|TARGET_ZLEV|は指定できない。下記に説明する以外の設定変数については、
先の3次元変数の変換の場合と同じである。
\begin{itemize}
 \item \verb|ANAL|の項目
 \begin{itemize}
  \item \verb|ANALYSIS|:鉛直次元のの簡易解析の種類を指定する。``max''を指定すると鉛直カラム中の最大値、
        ``min''を指定するすると最小値、``sum''を指定すると鉛直カラム積算値、そして``ave''を指定すると
        鉛直カラム平均値を算出する。デフォルト値は``ave''である。
 \end{itemize}
\end{itemize}


\subsubsection{configファイルサンプル:2次元データの変換}
%------------------------------------------------------

\noindent {\small {\gt
\ovalbox{
\begin{tabularx}{140mm}{l}
\verb|&INFO| \\
\verb|    〜 中略 〜|\\
\verb| Z_LEV_TYPE  = "original",| \\
\verb|/| \\
 \\
\verb|&VARI| \\
\verb| VNAME       = "T2","MSLP","PREC"| \\
\verb|/| \\
\end{tabularx}
}}}\\

\noindent 上記はconfigファイル記述例の抜粋である。\textcolor{red}{2次元データを変換する場合は、
{\bf Z\_LEV\_TYPE}を必ず``original''と指定する。}また、2次元データなので\verb|VARI|の項目の
\verb|TARGET_ZLEV|は指定できない。その他の設定項目は3次元データ変換の場合と変わりない。


\subsubsection{configファイルサンプル:特殊な時間軸を持つデータの変換}
%------------------------------------------------------

\noindent {\small {\gt
\ovalbox{
\begin{tabularx}{140mm}{l}
\verb|&EXTRA| \\
\verb| EXTRA_TINTERVAL = 600.0,| \\
\verb| EXTRA_TUNIT     = "SEC",| \\
\verb|/| \\
 \\
\verb|&VARI| \\
\verb| VNAME = "RAIN",| \\
\verb|/| \\
\end{tabularx}
}}}\\

\noindent 上記はconfigファイル記述例の抜粋である。\ref{sec:output}節の最後で説明した出力アイテム毎の
設定で、\verb|HISTITEM|の\verb|tinterval|を設定し、一部のデータ変数が\verb|HISTORY_DEFAULT_TINTERVAL|に
従わない場合は、net2gでは\verb|EXTRA|の設定項目を上記のように設定することで対応できる。ここでは、
\ref{sec:output}節の最後で説明した設定に合わせて、2次元データの``RAIN''だけが600秒間隔で出力されていた
場合のconfigファイル設定例を示している。


\subsection{スーパーコンピュータ「京」での実行方法}
%------------------------------------------------------

スーパーコンピュータ「京」などの大型計算機で並列計算を行った場合、出力ファイルの数が多く、
それぞれのファイルのデータ容量も大きい。そのような場合には、手元のマシンへhistoryデータをコピーするだけの
ディスク空き容量がなかったり、並列ファイルシステムではないために後処理に膨大な時間がかかり、現実的な
時間スケールで解析が行えない可能性がある。こういった場合には、SCALE本体の計算を行ったスーパーコンピュータで
後処理も行なってしまうことをおすすめする。この目的のためにここでは、スーパーコンピュータ「京」でnet2gを
使用する方法を簡単に説明する。

スーパーコンピュータ「京」には、JOBのサイズや実行形態に応じたいくつかのJOBグループがあるが、
ここでは、もっとも簡単に実行を行える``micro JOB''を対象に実行方法を説明する。micro JOBでは実行バイナリが
``/scratch''ディレクトリの下に存在する必要があることに注意されたい。

\verb|/scratch/GROUP/USER/|の下に、作業ディレクトリを用意する。そこに、
\begin{verbatim}
  net2g         : copy executive file
  net2g.conf    : copy configure file
  output/       : link to directory with scale history files
  bindata/      : create directory for grads file output
  job.sh        : job script for K (option)
\end{verbatim}
を用意する。\verb|net2g.conf|の設定と\verb|job.sh|の設定をする。\verb|net2g.conf|の\verb|INFO|項目の
\verb|IDIR|と\verb|ODIR|は、上記設定に合わせて適切に記述しなければならない。
使用するノード数は、SCALE本体の計算に使用したノード数の約数である必要がある。\verb|job.sh|の中身は
下記のような内容を記述する。各パラメタの詳細等は、スーパーコンピュータ「京」のユーザーズマニュアル等を参照されたい。
準備が整えば、\verb|job.sh|を使ってpjsubコマンドにてJOBを投入すればよい。実行時間が30分の制限時間に間に合わない
場合は、一度に変換する変数の数を減らすか、鉛直層数を減らして実行する。もしくはLustreのstriping countの調整等を
行なってもよい。\\

\noindent {\gt \verb|job.sh|の記述例} \\
\noindent {\small {\gt
\ovalbox{
\begin{tabularx}{140mm}{l}
\verb|#! /bin/bash -x| \\
\verb|#PJM --rsc-list "rscgrp=micro"| \\
\verb|#PJM --rsc-list "node=12"| \\
\verb|#PJM --rsc-list "elapse=00:30:00"| \\
\verb|#PJM -j| \\
\verb|#PJM -s| \\
\verb|#| \\
\verb|. /work/system/Env_base| \\
\verb|#| \\
\verb|export PARALLEL=8| \\
\verb|export OMP_NUM_THREADS=8| \\
\verb|#| \\
\verb|mpiexec -n 12 ./net2g  n2g_d01-3d.conf| \\
\end{tabularx}
}}}\\


\subsection{バルクジョブ対応版の使用方法}
%------------------------------------------------------

\ref{sec:bulkjob}節で説明した「複数の実験を一括実行するバルクジョブ機能」を用いてSCALEを走らせた場合は、
バルクジョブ対応版のnet2gを利用するのが便利である。コンパイルは、通常版のnet2gと同じである。
``scale/scale-rm/util/netcdf2grads\_bulk''の下で\ref{sec:source_net2g}節で説明したとおりの方法で
コンパイルすれば、バルクジョブ対応版のnet2gが生成される。

基本的な使用方法や制限事項も通常版のnet2gと同じである。net2gに渡すconfigファイルなどの記述も通常版と
同じように記述すればよい。ただし、実行にあたっては``launch.conf''が必要になることと、\ref{sec:bulkjob}節で
説明したバルクジョブ実行時のディレクトリ構造を準備する必要がある。SCALE本体をバルクジョブ機能で実行した
場合にはディレクトリ構造はすでに準備されているため新たに用意する必要はない。net2gに渡すconfigファイルだけ、
各バルク番号のディレクトリ下に設置すればよい。以下に、launch.confファイルの記述例を挙げておく。\\

\noindent {\small {\gt
\ovalbox{
\begin{tabularx}{140mm}{l}
\verb|&PARAM_LAUNCHER| \\
\verb| NUM_BULKJOB = 31,| \\
\verb| NUM_DOMAIN  = 2,| \\
\verb| PRC_DOMAINS = 12,36,| \\
\verb| CONF_FILES  = net2g.3d.d01.conf,net2g.3d.d02.conf,| \\
\verb|/| \\
\end{tabularx}
}}}\\

\noindent この例の場合、一度に31個のジョブを実行している。また1つのジョブは2段オンライン・ネスティング実験
となっており、net2gの実行にあたってはdomain 1は12-MPI並列、domain 2は36-MPI並列で実行される。ここで
指定するMPIプロセス数は、SCALE本体の実行時に使用したMPIプロセス数の約数でなければならない。

それぞれのドメインについて実行するnet2gのconfigファイルは、
それぞれ``net2g.3d.d01.conf''と``net2g.3d.d02.conf''と指定されている。
このconfigファイルは31個のバルク番号ディレクトリの中に
収められてことを想定している。

この例では、1つのジョブあたり、$12 + 36 = 48$プロセスを使用し、全体で31ジョブあるので総計で1488プロセスを
必要とする。下記のコマンドのように実行する。

\begin{verbatim}
 $ mpirun  -n  1488  ./net2g  launch.conf
\end{verbatim}



\subsection{旧型netcdf2gradsの使用方法}
%------------------------------------------------------
ここでは、シングルプロセス実行しかできないが、``HIST\_BND''の対応、そしてhistoryデータ以外のデータにも対応
する「旧型netcdf2grads」の実行方法概略について説明する。このプログラムは、並列版/シングル版net2gの整備が
済み次第、サポート外となるため注意して欲しい。

netcdf2gradsのソースファイルは \verb|scale/scale-rm/util/netcdf2grads/|にある。
SCALE本体とは独立なので、ディレクトリを任意の場所に移動/コピーして使用することが可能である。

\subsubsection{コンパイル}
\begin{description}
\item[Intel compiler]\mbox{}\\
 \begin{verbatim}
 $ ifort -convert big_endian -assume byterecl -I${NETCDF4}/include 
    -L${NETCDF4}/lib -lnetcdff -lnetcdf make_grads_file.f90 -o convine
  \end{verbatim}
\item[gfortran]\mbox{}\\
\begin{verbatim}
 $ gfortran -frecord-marker=4 --convert=big-endian -I${NETCDF4}/include
    -L${NETCDF4}/lib -lnetcdff -lnetcdf make_grads_file1.f90 -o convine
\end{verbatim}
\end{description}


\subsubsection{使用方法}
実行時に,インタラクティブモードかサイレントモードかを選択することが出来る.
\begin{verbatim}
   Interactive mode :  ``./convine -i''
   Silent mode      :  ``./convine -s''
\end{verbatim}
\begin{description}
\item[インタラクティブモード]\mbox{}\\
\begin{verbatim}
 $ cd netcdf2grads/
 $ ./convine -i
\end{verbatim}
と実行すると、下記のメッセージが出るので、指示通りに必要なファイルのパスを打ち込む。\\

\noindent {\small {\gt
\fbox{
\begin{tabularx}{140mm}{l}
\verb|path to configure file for run with the quotation mark| \\
\textcolor{blue}{\verb| `${path to directory of configure file}/run.conf' <- configureファイルへのパス|} \\
\verb|path to directory of history files with the quotation mark| \\
\textcolor{blue}{\verb| `${path to directory of history files}/' <- SCALE-RMの出力ファイルのパス|} \\
\verb|path to directory of output files with the quotation mark| \\
\textcolor{blue}{\verb| `./grads/'  <- gradsファイルの出力先|} \\
\verb|start time of convert data| \\
\textcolor{blue}{\verb| 1           <- 任意の番号の時間から変換可能 |} \\
\verb|end time of convert data| \\
\textcolor{blue}{\verb| 10          <- 変換を終了する時間|} \\
\verb|Imput number of variable| \\
\verb|0 -> all variable output from model| \\
\textcolor{blue}{\verb| X           <- 変換したい変数の数|} \\
\verb|Imput variable| \\
\textcolor{blue}{\verb| VARIABLE(PRECなど)  <- 変換したい変数名|} \\
\end{tabularx}
}}}\\

\noindent \textcolor{blue}{青色文字}の行は、ユーザーが入力する行である。
うまく実行されれば、指定した出力先にctlファイルとgrdファイルが作成される。
\item[サイレントモード]\mbox{}\\
サイレントモードの場合は、あらかじめ、\verb|namelist.in|に必要な情報を書き入れておく.
\begin{verbatim}
 $ cd netcdf2grads/
 $ ./convine -s
\end{verbatim}
と実行すれば変換が始まる。
\end{description}


%####################################################################################

