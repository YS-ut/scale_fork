
%-------------------------------------------------------%
\section{地形・土地利用データの作成:pp}
%-------------------------------------------------------%

ppディレクトリへ移動し、現実実験のための地形データ、土地利用データを作成する。
\begin{verbatim}
 $ cd ${Tutrial_DIR}/real/pp
\end{verbatim}
ppディレクトリの中には、\verb|pp.conf|という名前の
コンフィグファイルが準備されている。
ドメインの位置や格子点数など、実験設定に合わせて、
適宜\verb|pp.conf|を編集する必要があるが、
チュートリアルではすでにTable \ref{tab:grids}の設定に
従って編集済みの\verb|pp.conf|が用意されているためそのまま利用する。
\verb|pp.conf|の設定の中で特に注意するべき項目は、\verb|PARAM_CONVERT|である。\\

\noindent {\gt
\ovalbox{
\begin{tabularx}{140mm}{l}
\verb|&PARAM_CONVERT| \\
\verb|  CONVERT_TOPO = .true.,| \\
\verb|  CONVERT_LANDUSE = .true.,| \\
\verb|/| \\
\end{tabularx}
}}\\

\noindent 上記のように\verb|CONVERT_TOPO|と\verb|CONVERT_LANDUSE|が
\verb|.true.|となっていることが、
それぞれ地形と土地利用の処理を行うことを意味している。
詳細なコンフィグファイルの内容については、
Appendix \ref{app:namelist}を参照されたい。

次に、コンパイル済みのバイナリと入力データをppディレクトリへリンクする。
\begin{verbatim}
 $ ln -s ../../bin/scale-rm_pp ./
 $ ln -s ${SCALE_DB}/topo    ./
 $ ln -s ${SCALE_DB}/landuse ./
\end{verbatim}
今回は、Table \ref{tab:grids}に示されているように。
4つのMPIプロセスを使用する設定なので次のように実行する。
\begin{verbatim}
 $ mpirun -n 4 ./scale-rm_pp pp.conf
\end{verbatim}
実行にはおおよそ15秒を要する。ジョブが正常に終了すれば、\verb|topo_d01.pe######.nc|と\\
\verb|landuse_d01.pe######.nc|と
いうファイルがMPIプロセス数だけ、つまり4つずつ生成される(\verb|######|にはMPIプロセスの番号が入る)。
それぞれ、ドメインの格子点に内挿された地形と土地利用の情報が入っている。
実行時のログは、\verb|pp_LOG_d01.pe000000|に出力されるので内容を確かめておくこと。


\vspace{1cm}
\noindent {\Large\em OPTION} \hrulefill \\
gpviewがインストールされている場合、次のコマンドによって、
作成された地形と土地利用データが正しく作成されているかどうか
確認することが出来る.正しく作成されていれば,図 \ref{fig:domain}と同様の図ができる.
\begin{verbatim}
  $ gpview topo_d01.pe00000*@TOPO --aspect=1 --nocont
  $ gpview landuse_d01.pe00000*@FRAC_LAND --aspect=1 --nocont
\end{verbatim}

