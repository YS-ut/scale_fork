%====================================================================================

SCALEライブラリには5種類の雲モデルが実装されている。すなわち、氷雲の微物理過程を
含まない1モーメントバルクモデル(\cite{kessler_1969})、氷雲の微物理過程を
含む1モーメントバルクモデル(\cite{tomita_2008})、氷雲を含む2モーメントバルクモデル
(\cite{sn_2014})、1モーメントビン法(\cite{suzuki_etal_2010})、
及び、超水滴法(\cite{Shima_etal_2009})である。
1モーメントビン法はオプションにより氷を含む場合とそうでない場合に切り替えられる。
また超水滴法に関しては著作権の関係から公開されていない。
使用したい場合はSCALEの開発者に連絡をされたい。\\
これらの雲微物理モデルの違いは雲粒の粒径分布関数の表現方法にある。
ここでは超水滴法を除く4種類の雲モデルの利用法について記述する。

\begin{enumerate}
\item {\bf 氷を含まない1モーメントバルク法\cite{kessler_1969}}\\
1モーメントバルク法は粒径分布関数を3次のモーメント(質量)のみで表現する。
このスキームでは雲粒と雨粒のカテゴリを考慮し、
雲粒混合比(QC[kg/kg])と雨粒混合比(QR[kg/kg])を予報する。\\
粒径分布はデルタ関数、すなわち雲粒のサイズは全て同じと仮定することでで表現する。
雲粒と雨粒の半径はそれぞれ8µm、100µmと与えられている。\\
考慮する雲粒の成長過程は、雲粒生成(飽和調整)、凝結・蒸発、衝突併合、落下である。
\item {\bf 氷を含む1モーメントバルク法\cite{tomita_2008}}\\
このスキームでは\cite{kessler_1969}と同じく、粒径分布関数を3次のモーメント(質量)のみで表現する。
雲粒、雨粒、氷粒、雪片、あられの5カテゴリを考慮し、
それぞれの質量混合比(QC、QR、QI、QS、QG[kg/kg])を予報する。\\
粒径分布については雲粒と氷粒をデルタ分布(それぞれ半径8µm、40µm)、
その他はMarshall-Palmer分布を仮定して表現する。\\
考慮する成長過程は雲粒生成(飽和調整)、凝結・蒸発、衝突併合、落下である。
\item {\bf 氷を含む2モーメントバルク法\cite{sn_2014}}\\
氷を含む2モーメントバルク法は粒径分布関数を質量に加えて、0次のモーメント(個数)で表現する。
雲粒、雨粒、氷粒、雪片、あられの5カテゴリを考慮し、
それぞれの質量混合比(QC、QR、QI、QS、QG)と数密度(NC、NR、NI、NS、NG)を予報する。\\
粒径分布は一般ガンマ関数で近似して表現する。\\
考慮する成長過程は雲粒生成、凝結・蒸発、衝突併合、分裂、落下である。
\item {\bf 1モーメントビン法\cite{suzuki_etal_2010}}\\
1モーメントビン法は粒径分布関数を差分化して陽に表現する。
差分化された各粒径分布関数をビンと呼ぶためビン法と呼ばれる。
雲粒、雨粒、氷粒、雪片、あられ、ひょうの6カテゴリを考慮し、
各粒径ビンの質量混合比を予報する。\\
粒径分布は陽に与えられ、ビン数のとりかたによって分布の表現精度が異なる。\\
考慮する成長過程は雲粒生成、凝結・蒸発、衝突併合、落下である。
\end{enumerate}

上記の4種類の雲微物理は1→4の順に高精度になるが、その分計算コストも高くなる。

\proofcomment{(八代)今となってはこれをAppendixにしておく必要が感じられません。5.7にしていいと思います。} \\
