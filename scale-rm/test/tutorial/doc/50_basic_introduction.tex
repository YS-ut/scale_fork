\section{概要}

この章では、チュートリアルから発展して、基本的な様々な設定が出きるように、
各種設定を網羅的に記述している。
各節で閉じており、以下の項目に付いてされており、辞書代わりに使ってほしい。

{\small
\begin{description}
\item[対象計算領域の設定] \ref{sec:domain} 節
\item[計算領域と解像度、格子点数、MPIプロセスの関係] \ref{subsec:relation_dom_reso} 節
\item[計算領域の設定] \ref{subsec:relation_dom_reso2} 節
\item[MPIプロセス数] \ref{subsec:relation_dom_reso3} 節
\item[水平・鉛直格子数] \ref{subsec:relation_dom_reso4} 節
\item[水平・鉛直格子間隔] \ref{sec:gridinterv} 節
\item[緩和領域の設定] \ref{sec:buffer} 節
\item[積分時間と積分時間間隔の設定] \ref{sec:timeintiv} 節
\item[出力変数の追加・変更] \label{sec:output}
\item[力学スキームの設定] \ref{sec:atmos_dyn} 節
\item[数値解法]  \ref{subsec:atmos_dyn_sover} 節
\item[時間・空間差分スキーム] \ref{subsec:atmos_dyn_scheme} 節
\item[物理スキームの設定] \ref{sec:basic_physics} 節
\item[雲微物理スキーム] \ref{sec:basic_microphys} 節
\item[乱流スキーム] \ref{sec:basic_turbulence} 節
\item[放射スキーム] \ref{sec:basic_radiation} 節
\item[地表面(大気下端境界)] \ref{sec:basic_surface} 節
\item[海洋モデル] \ref{sec:basic_ocean} 節
\item[陸面モデル] \ref{sec:basic_land} 節
\item[都市モデル(大気-都市面フラックス)] \ref{sec:basic_urban} 節
\end{description}
}
