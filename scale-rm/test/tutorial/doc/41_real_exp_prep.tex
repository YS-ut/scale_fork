%-------------------------------------------------------%
\section{境界データの準備}
%-------------------------------------------------------%

現実大気実験のシミュレーションを行う場合、SCALE本体に加えて
境界値データが必要になる。境界値データとしては表\ref{tab:real_bnd}
が必要である。{\color{blue}青字}は必須の変数、その他は任意である。

\begin{table}[h]
\begin{center}
  \caption{現実大気実験に必要な初期値境界値データ}
  \label{tab:real_bnd}
  \begin{tabularx}{150mm}{llX} \hline
    \multicolumn{3}{l}{地形データ(SCALE-RMの地形を用意する)}\\ \hline
    & \multicolumn{2}{l}{\color{blue}{標高データ}}\\
    & \multicolumn{2}{l}{\color{blue}{土地利用データ}}\\ \hline
    \multicolumn{3}{l}{初期値境界値データ}\\ \hline
    &  \multicolumn{2}{l}{\color{blue}{親モデルの緯度・経度}}\\
    &  \multicolumn{2}{l}{(3次元大気データ)}\\
    & &  \multicolumn{1}{l}{\color{blue}{東西風速, 南北風速, 気温, 比湿(相対湿度), 気圧, ジオポテンシャル高度}} \\
    &  \multicolumn{2}{l}{(2次元大気データ)}\\
    & & 海面更正気圧, 地上気圧, 10m東西風速, 10m南北風速, 2m気温, 2m比湿(相対湿度) \\
    &  \multicolumn{2}{l}{(2次元陸面データ)}\\
    & &  \multicolumn{1}{l}{親モデルの海陸マップ}\\
    & &  \multicolumn{1}{l}{\color{blue}{地表面温度(Skin temp)}}\\
    & &  \multicolumn{1}{l}{{\color{blue}{親モデル土壌データの深さ情報, 土壌温度}}, 土壌水分(体積含水率 or 飽和度)}\\
    &  \multicolumn{2}{l}{(2次元海面データ)}\\
  & &  \multicolumn{1}{l}{\color{blue}{海面水温(Skin tempがある場合は省略可)}}\\ \hline
  \end{tabularx}
\end{center}
\end{table}


\subsubsection{地形データと土地利用データ}
標高データと土地利用データは実験設定に従って、
SCALEのそれぞれの格子点における地形、海陸分布、土地利用を
作成するために使用する。
ユーザーが全球の任意の地域を対象とした計算ができるよう、
フォーマット変換済みの
標高データ USGS(U.S. Geological Survey) のGTOPO30 と、
土地利用データ GLCCv2 を提供している。

\begin{enumerate}
\item データのダウンロード\\
SCALE用の地形・土地利用のデータを\\
 \url{http://scale.aics.riken.jp/download/scale_database.tar.gz}\\
より入手し、任意のディレクトリに展開しておく。
\begin{alltt}
  $ tar -zxvf scale_database.tar.gz
\end{alltt}
展開したディレクトリには、地形データと土地利用データが格納されている。
\begin{alltt}
  scale_database/topo/    <- 地形データ
  scale_database/landuse/ <- 土地利用データ
\end{alltt}

\item パスの設定\\
makeを使ったジョブスクリプトを使用する場合には、
展開先のディレクトリを \verb|SCALE_DB| という環境変数に設定しておくことが必須である。
(以後、\verb|${SCALE_DB}|と表記)。
\begin{alltt}
  $ export SCALE_DB=${path_to_directory_of_scale_database}/scale_database
\end{alltt}
\end{enumerate}

\subsubsection{大気・陸面・海面水温データ}
\label{sec:real_prep}
初期値境界値データは4-byte バイナリー(grads format)に変換すれば、
任意のデータを読み込むことが可能である。
チュートリアルではNCEP FNL(Final) Operational Global Analysis dataを使用する。
\begin{enumerate}
\item データのダウンロード\\
NCARのサイト
 \url{http://rda.ucar.edu/datasets/ds083.2/}\\
から、2014年8月10日のgrib2フォーマットのデータ
\begin{alltt}
  fnl_20140810_00_00.grib2
  fnl_20140810_06_00.grib2
  fnl_20140810_12_00.grib2
  fnl_20140810_18_00.grib2
\end{alltt}
を\verb|${Tutorial_DIR}/real/tools/|の下にダウンロードする。

\item データフォーマットをgribからbinaryに変換\\
 SCALEは4byte バイナリー(grads format)の境界値データを読み込むことができる。\\
\verb|${Tutorial_DIR}/real/tools/| の中にある \verb|convert_grib2grads.sh|を実行。
ただし、あらかじめ\verb|wgrib2|がインストールされている必要がある.
\begin{alltt}
  $ sh convert_grib2grads.sh
\end{alltt}
成功すれば、下記のファイルが作成される。
\begin{alltt}
  $ ls FNL_output/*/*
     FNL_output/201408/FNLatm_2014081000.grd
     FNL_output/201408/FNLatm_2014081006.grd
     FNL_output/201408/FNLatm_2014081012.grd
     FNL_output/201408/FNLatm_2014081018.grd
     FNL_output/201408/FNLland_2014081000.grd
     FNL_output/201408/FNLland_2014081006.grd
     FNL_output/201408/FNLland_2014081012.grd
     FNL_output/201408/FNLland_2014081018.grd
     FNL_output/201408/FNLsfc_2014081000.grd
     FNL_output/201408/FNLsfc_2014081006.grd
     FNL_output/201408/FNLsfc_2014081012.grd
     FNL_output/201408/FNLsfc_2014081018.grd
\end{alltt}
\end{enumerate}

