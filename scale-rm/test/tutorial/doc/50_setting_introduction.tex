%\section*{概要}

この章では、チュートリアルから発展して、基本的な様々な設定が出きるように、
各種設定を網羅的に記述している。
各節で閉じており、辞書代わりに使ってほしい。

%% {
%% \begin{center}
%% \begin{tabular}[h]{ll}\hline
%% \SecBasicDomainSetting & 第\ref{sec:domain} 節 \\
%%%%%% \SubsecDomainSetting & 第\ref{subsec:relation_dom_reso2} 節 \\
%% \SubsecMPIProcess & 第\ref{subsec:relation_dom_reso2} 節 \\
%% \SubsecGridNumSettng & 第\ref{subsec:relation_dom_reso3} 節 \\
%% \SubsecGridIntvSettng & 第\ref{subsec:gridinterv} 節 \\
%% \SecBasicBufferSetting & 第\ref{subsec:buffer} 節 \\
%% \SecBasicTopoSetting   & 第\ref{subsec:basic_usel_topo} 節 \\
%% \SecBasicIntegrationSetting & 第\ref{sec:timeintiv} 節 \\
%% \SecBasicOutputSetting & 第\ref{sec:output} 節\\
%% \SecBasicDynamicsSetting & 第\ref{sec:atmos_dyn} 節 \\
%% \SubsecDynsolverSetting  & 第\ref{subsec:atmos_dyn_sover} 節 \\
%% \SubsecDynSchemeSetting & 第\ref{subsec:atmos_dyn_scheme} 節 \\
%% \SecBasicPhysicsSetting & 第\ref{sec:basic_usel_physics} 節 \\
%% \SubsecMicrophysicsSetting & 第\ref{subsec:basic_usel_microphys} 節 \\
%% \SubsecTurbulenceSetting & 第\ref{subsec:basic_usel_turbulence} 節 \\
%% \SubsecRadiationSetting & 第\ref{subsec:basic_usel_radiation} 節 \\
%% \SubsecSurfaceSetting & 第\ref{subsec:basic_usel_surface} 節 \\
%% \SubsecOceanSetting & 第\ref{subsecp:basic_usel_ocean} 節 \\
%% \SubsecLandSetting & 第\ref{subsec:basic_usel_land} 節 \\
%% \SubsecUrbanSetting & 第\ref{subsec:basic_usel_urban} 節 \\
%% \SecMakeconfTool & 第\ref{sec:basic_makeconf} 節 \\
%% \SecAdvanceMapprojectionSetting & 第\ref{subsec:adv_mapproj}節 \\
%% \SecAdvanceInputDataSetting & 第\ref{sec:adv_datainput}節\\
%% \SecAdvanceRestart & 第\ref{sec:restart}節 \\
%% \SecAdvancePostprosess & 第\ref{sec:net2g}節 \\
%% \SecAdvanceNesting & 第\ref{sec:nest_exp}節 \\
%% \SubsecOflineNesting & 第\ref{subsec:nest_offline}節\\
%% \SubsecOnlineNesting & 第\ref{subsec:nest_online}節\\
%% \SecAdvanceBulkjob & 第\ref{sec:bulkjob}節\\
%% \hline
%% \end{tabular}
%% \end{center}
%% }

\proofcomment{(八代)順番入れ替えに合わせて、最後にファイル名を修正して下さい(迷子になった)}\\
\replycomment{(足立)直しました。ごめんなさいですー。}
