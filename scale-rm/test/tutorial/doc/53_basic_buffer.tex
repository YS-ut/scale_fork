\section{緩和領域の設定} \label{sec:buffer}
%-----------------------------------------------------------------------
モデル最上層では重力波の反射、
側面境界では現実大気実験/ネスティング実験を行う際に親領域と対象領域の間の不一致が起こる。
この問題を解決するため、「緩和領域」を設ける。
SCALE-RMでは計算領域の境界のすぐ内側に緩和領域を設定することができる。
緩和領域の格子では、指定された値(境界値データ、親領域のデータなど)に対して
ある時定数で緩和される。以下これをナッジングと呼ぶ。
緩和領域の幅は、設定ファイルの\namelist{PARAM_GRID}の中で設定する。
以下に例を示す。
設定はすべての設定ファイルにおいて共通していなければならない。\\

\noindent {\small {\gt
\ovalbox{
\begin{tabularx}{150mm}{lX}
\verb|&PARAM_GRID  |            & \\
 \verb|BUFFER_DZ = 5000.D0,   | & ; z方向(モデルトップから下向き方向)の緩和領域の幅 [m]\\
 \verb|BUFFER_DX = 300000.D0, | & ; x方向(東西方向)の緩和領域の幅 [m]\\
 \verb|BUFFER_DY = 300000.D0, | & ; y方向(南北方向)の緩和領域の幅 [m]\\
 \verb|BUFFFACT  = 1.D0,      | & ; 緩和領域内の格子間隔に対するストレッチ係数(デフォルトは1.0)\\
\verb|/|\\
\end{tabularx}
}}}\\

水平方向には東西南北の四方境界に緩和領域が設定されるが、
鉛直方向には計算領域の上端にのみ緩和領域が設定され、下端には設定されない。
%
緩和領域は、計算領域内に設定されるため、
ナッジングの影響を受けない領域(緩和領域を除いた範囲)は
計算領域よりも狭くなることに注意が必要である。

\subsubsection{緩和領域の格子間隔をストレッチさせる}
緩和領域の格子間隔は、基本的に 
\namelist{PARAM_GRID}の中の\nmitem{DX, DY, DZ}で指定した通りであるが、
\nmitem{BUFFFACT}に1以上の値を設定することで、ストレッチさせることも可能である。
ただし、格子間隔を等間隔で指定した場合、
この\nmitem{BUFFFACT}の設定は、x, y, z方向すべてに適用されることの注意されたい。
z方向の層レベルを任意の格子点位置に指定する場合、
すなわち、\nmitem{FZ(:)}を与える場合(第\ref{sec:gridinterv}節参照のこと)には適用されない。

緩和領域内の格子間隔 (\verb|BDX|) は次の通り決定される。
\begin{eqnarray}
 \verb|BDX(|n\verb|)| &=& \verb|DX| \times \verb|BUFFFACT|^n \nonumber
\end{eqnarray}
ここで、$n$は緩和領域内の格子点番号を表し、計算領域の内側から外側へ向かって番号が振られる。
緩和領域の格子間隔は、
\verb|BUFFFACT=1.0|ならば内部領域と同じであり、
\verb|BUFFFACT=1.2|ならば内側から外側(境界)に向かって1.2倍の割合で広がっていく。
\verb|BUFFFACT|はいくつに設定しても良いが、計算の安定性を考慮すると 1.0から1.2 が推奨である。

緩和領域の格子数\verb|ibuff|は、
\begin{eqnarray}
\sum_{n=1}^{\verb|ibuff|} \verb|BDX|(n) \ge \verb|BUFFER_DX| \nonumber
\end{eqnarray}
の関係を満たす最小の整数で自動的に計算される。
%
緩和領域の幅(\nmitem{BUFFER_DX})が同じでも、
\nmitem{BUFFFACT}の値を大きくすると緩和領域に用意される格子数は少なくなる。
ここでは、x方向の説明をしたが、y方向、z方向も同様である。


一般に、緩和領域の大きさ、緩和格子点の数については、解く問題により、明確な指標はない。
SCALE-RMでは、鉛直方向(計算領域トップ)の緩和格子点は5点以上、
水平方向(側面境界付近)の緩和格子点は20〜40点程度を推奨している。
実験設定や事例によっては、さらに緩和格子点を増やしたり、
ストレッチ係数を用いて緩和領域を広げたり、
ここでは説明しないが
\namelist{PARAM_ATMOS_BOUNDARY}の中の
\nmitem{ATMOS_BOUNDARY_taux,ATMOS_BOUNDARY_tauy}を調整して
緩和領域のナッジング強度を調整したりする必要があるだろう。

