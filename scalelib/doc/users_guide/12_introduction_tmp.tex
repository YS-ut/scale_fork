\section{tmp}
実際にモデルを動かす場合のSCALE-LESモデルの動作形態を下記の図に示している.

{\Huge 図}

つまり,これは最も外側のDriverの内容の概観である.
まず,モデルを構成する様々な要素を起動する.
例えば格子点やそれらの点が持つ変数に対応する配列を用意,各パラメタの初期化,またMPI通信の準備などが行なわれる.
このあと,各種実験において必要になるデータ,たとえば初期値・境界値を読み込む.
そのあと,いよいよ時間積分を開始し,設定された最終時刻に到達した時点で時間積分Loopから抜けてモデルを終了させる.
これが基本的なSCALE-LESモデルの実行過程である.
この構造は,プリプロセス・プログラム(scale-les\_initやscale-les\_pp)でも時間積分がない点以外は同様である.
理想実験や実事例実験によって,走らせ方が少々異なる部分があるが,基本的な作業は下図のとおりである.
それぞれのツールやモデルの実行時には実行設定が記述されたconfigurationファイル(namelist)が存在し,これを編集することでさまざまな実験を行うことが出来るようになる.
configurationファイルの設定項目についての説明は5章に譲ることとし,ここではモデル実行のための工程について説明する.


まず実験に必要な初期値・境界値を作成する.地形や土地利用といった下端境界条件は,scale-les\_ppによって作成される.
scale-lesでは日本領域については国土地理院のデータをもとにした地形,土地利用に関するデータベースを別途提供している(2.2節を参照).
このデータベースをもとにして実験ドメイン,格子設定にあった下端境界条件を作成する.
地形は元データよりも鈍したデータにすることも可能である.
つづいて,初期値・境界値を作成する.
理想実験の場合は,scale-les\_initが各実験ケースにあった初期値・境界値を作成してくれる.
実事例の場合は外部データが必要となる.
現在,scale-lesではWRF-ARWモデル,NICAM,およびscale-lesのデータを外部データとして初期値・境界値を作成する事ができる.
ここまでがモデル実行の準備段階となる.
最後に,モデル本体のscale-lesを実行することで,モデルの実行,つまり時間積分を行う.
現在,scale-lesでは,単一ドメインの計算と複数ドメイン,つまりNesting計算をサポートしている.
Nesting計算は1-way(親ドメインから娘ドメインへのデータ受け渡し)のみをサポートしている.
こういった実行時の設定は,第5章を参照されたい.scale-lesは通常のoutputとしてhistoryファイル,リスタート用のoutputとしてrestartファイル,モデル監視用変数のoutputとしてmonitorファイル,そして,実行時のログを吐くファイルとしてLOGファイルをoutputする.
historyファイルとrestartファイルをHDF5を介してデータ圧縮を行ったnetcdf4フォーマットのファイルである.
scale-lesでは,各MPIプロセスが個別にファイルをoutputする.
以下,理想実験の実行過程,実事例の実行過程についてコマンドの実行手順を示す.
