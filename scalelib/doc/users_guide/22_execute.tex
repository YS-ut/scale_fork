\section{Running the SCALE-LES}

\subsection{モデル実行方法の概要}
SCALE-LESモデルの実行過程は,基本的に
\begin{enumerate}
\item pp : 地形・海陸分布データの作成.
\item init : 初期値・境界値データの作成.
\item run : 時間積分を行う
\end{enumerate}
という手順である.


\subsection{現実事例の実行方法}
\subsubsection{pp}
理想実験の場合との違いは地形,土地利用といった下準備が必要になる点が違いである.
まず,地形データを作成するためにtopo/のもとでpp.confを目的の実験に合わせて編集する.

Preprocessプログラムは、地形および土地利用データの生成を行う。
地形にはGTOPO30またはDEM50M、土地利用にはGLCCv2またはLU100Mのデータセットを用いる。
pp.confのサンプルは付録参照。

\begin{enumerate}
\item 下記のコマンドで,事前処理(地形作成)を実行する.ここの例ではMPI並列として
6プロセスを使用している.
\begin{verbatim}
$ mpirun -n 6 ./scale-les_pp pp.conf
\end{verbatim}
正常にJOBが終了すれば,topo.pe\verb|#####|.ncというファイルがMPIプロセス数だけ生成される(\verb|#####|はMPIプロセスの番号).

\item 同様に土地利用データを作成するためにlanduse/のもとでpp.confを適宜編集する.
MPIプロセスの設定や,格子,領域中心(basepointの緯度・経度)の設定はtopoを作成したときのpp.confと揃えておくこと.

\item 下記のコマンドで,事前処理(土地利用作成)を実行する.
\begin{verbatim}
$ mpirun -n 6 ./scale-les_pp pp.conf
\end{verbatim}
正常にJOBが終了すれば,landuse.pe\verb|#####|.ncというファイルがMPIプロセス数だけ生成される.
\end{enumerate}


\subsubsection{init}

Initialプログラムは、初期値および境界値を生成する。
データソースとして、SCALE、WRF、NICAM、一般バイナリを想定している。
init.confのサンプルは付録参照。
任意のデータを使用する場合のバイナリデータの作成手順については\ref{sec:adv_bnddata}を参照。

\begin{enumerate}
\item 次に入力する外部データに合わせてinit.confを適宜編集し,事前処理(初期値・境界値作成)を実行する.
ここでも,MPIプロセスの設定や,格子,領域中心(basepointの緯度・経度)の設定はここまでの作業で指定したものと揃えておくこと.
\begin{verbatim}
$ mpirun -n 6 ./scale-les_init init.conf
\end{verbatim}
\end{enumerate}



\subsubsection{run}
SCALE-LESプログラムは、実行プログラム本体となる。
run.confのサンプルは付録参照。


\begin{enumerate}
\item run.confを適宜編集する.MPIプロセスの設定や,格子,領域中心(basepointの緯度・経度)の設定,およびinit.confで指定した雲微物理モデルの指定などを揃えておくこと.
また,init.conf中にある,外部データとして読み込ませる親モデルデータのOutput時間間隔を正しく設定すること.

\item 以上の設定,確認を終えたら下記のコマンドによってモデルを実行する.
\begin{verbatim}
$ mpirun -n 6 ./scale-les run.conf
\end{verbatim}
configurationファイルの設定によるが,history.pe\verb|#####|.ncという名前のファイルが作成され,この中に出力変数が含まれている.
\end{enumerate}



