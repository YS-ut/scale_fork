%Appendix
\chapter{Namelist in run.conf}

\subsubsection{PARAM\_IO}
\begin{tabularx}{150mm}{|l|c|c|X|} \hline
 \rowcolor[gray]{0.9} 名称 & 種類 & 初期値 & 説明 \\ \hline
 \verb|IO_LOG_BASENAME| & 文字列 & "LOG" & ログファイルの接頭辞。 \\ \hline
 \verb|IO_LOG_ALLNODE| & 論理値 & .false. & 全ノードログ出力するかどうか。 \\ \hline
\end{tabularx}

%\subsubsection{PARAM\_PROF}
%\begin{tabularx}{150mm}{|l|c|c|X|} \hline
% \rowcolor[gray]{0.9} 名称 & 種類 & 初期値 & 説明 \\ \hline
% \verb|PROF_RAP_LEVEL| & 論理値 & .false. & 全ノードログ出力するかどうか。 \\ \hline
%\end{tabularx}

\subsubsection{PARAM\_CONST}
\begin{tabularx}{150mm}{|l|c|c|X|} \hline
 \rowcolor[gray]{0.9} 名称 & 種類 & 初期値 & 説明 \\ \hline
 \verb|CONST_RADIUS| & 実数 & 6.37122d+6 & 惑星半径(m)(デフォルトは地球) \\ \hline
 \verb|CONST_OHM| & 実数 & 7.2920D-5 & 惑星の角速度(1/s) \\ \hline
 \verb|CONST_GRAV| & 実数 & 9.80665D0 & 重力加速度($m/s^{2}$) \\ \hline
 \verb|CONST_RDRY| & 実数 & 287.04D0 & 乾燥気体の気体定数 (J/kg/K)\\ \hline
 \verb|CONST_CPDRY| & 実数 & 1004.64D0 & 乾燥気体の定圧比熱 (J/kg/K) \\ \hline
 \verb|CONST_LAPS| & 実数 & 6.5D-3 & International Standard Atmosphere (ISA)の気温減率(K/m) \\ \hline
 \verb|CONST_PSTD| & 実数 & 101325.D0 & 標準気圧(Pa) \\ \hline
 \verb|CONST_PRE00| & 実数 & 100000.D0 & 基準気圧(Pa) \\ \hline
 \verb|CONST_TSTD| & 実数 & 288.15D0 & 基準温度(K) \\ \hline
 \verb|CONST_THERMODYN_TYPE| & 文字列 & "EXACT" & 内部エネルギーの定義種類。SIMPLEは定数。EXACTは温度依存。 \\ \hline
\end{tabularx}

\subsubsection{PARAM\_TIME}
\begin{tabularx}{150mm}{|l|c|c|X|} \hline
 \rowcolor[gray]{0.9} 名称 & 種類 & 初期値 & 説明 \\ \hline
 \verb|TIME_STARTDATE| & 整数配列 & 0000, 1, 1, 0, 0, 0 & 積分実行時の初期時刻。 \\ \hline
 \verb|TIME_STARTMS| & 実数 & 0.0D0 & 初期時刻マイクロ秒。 \\ \hline
 \verb|TIME_DURATION| & 実数 & 0.0D0 & 実行する積分時間。 \\ \hline
 \verb|TIME_DT| & 実数 & 0.0D0 & 積分1STEPに要する時間。 \\ \hline
 \verb|TIME_DT_ATMOS_DYN| & 実数 & \verb|TIME_DT| & 力学スキームの時間差分値。\verb|TIME_DT|の約数である必要がある。 \\ \hline
 \verb|TIME_DT_ATMOS_PHY_MP| & 実数 & \verb|TIME_DT| & 雲微物理スキームの時間差分値。\verb|TIME_DT|の倍数である必要がある。 \\ \hline
 \verb|TIME_DT_ATMOS_PHY_RD| & 実数 & \verb|TIME_DT| & 放射スキームの時間差分値。\verb|TIME_DT|の倍数である必要がある。 \\ \hline
 \verb|TIME_DT_ATMOS_PHY_SF| & 実数 & \verb|TIME_DT| & 地表面スキームの時間差分値。\verb|TIME_DT|の倍数である必要がある。 \\ \hline
 \verb|TIME_DT_ATMOS_PHY_TB| & 実数 & \verb|TIME_DT| & 乱流スキームの時間差分値。\verb|TIME_DT|の倍数である必要がある。 \\ \hline
 \verb|TIME_DT_OCEAN| & 実数 & \verb|TIME_DT| & 海洋スキームの時間差分値。\verb|TIME_DT|の倍数である必要がある。 \\ \hline
 \verb|TIME_DT_LAND| & 実数 & \verb|TIME_DT| & 陸面スキームの時間差分値。\verb|TIME_DT|の倍数である必要がある。 \\ \hline
 \verb|TIME_DT_URBAN| & 実数 & \verb|TIME_DT| & 都市スキームの時間差分値。\verb|TIME_DT|の倍数である必要がある。 \\ \hline
 \verb|TIME_DURATION_UNIT| & 文字列 & "SEC" & 積分時間単位。 \\ \hline
 \verb|TIME_DT_UNIT| & 文字列 & "SEC" & 積分1STEPの時間単位。 \\ \hline
 \verb|TIME_DT_ATMOS_DYN_UNIT| & 文字列 & \verb|TIME_DT_UNIT| & 力学スキームの時間単位。 \\ \hline
 \verb|TIME_DT_ATMOS_PHY_MP_UNIT| & 文字列 & \verb|TIME_DT_UNIT| & 雲微物理スキームの時間単位。 \\ \hline
 \verb|TIME_DT_ATMOS_PHY_RD_UNIT| & 文字列 & \verb|TIME_DT_UNIT| & 放射スキームの時間単位。 \\ \hline
 \verb|TIME_DT_ATMOS_PHY_SF_UNIT| & 文字列 & \verb|TIME_DT_UNIT| & 地表面スキームの時間単位。 \\ \hline
 \verb|TIME_DT_ATMOS_PHY_TB_UNIT| & 文字列 & \verb|TIME_DT_UNIT| & 乱流スキームの時間単位。 \\ \hline
 \verb|TIME_DT_OCEAN_UNIT| & 文字列 & \verb|TIME_DT_UNIT| & 海洋スキームの時間単位。 \\ \hline
 \verb|TIME_DT_LAND_UNIT| & 文字列 & \verb|TIME_DT_UNIT| & 陸面スキームの時間単位。 \\ \hline
 \verb|TIME_DT_URBAN_UNIT| & 文字列 & \verb|TIME_DT_UNIT| & 都市スキームの時間単位。 \\ \hline
 \verb|TIME_DT_ATMOS_RESTART| & 実数 & \verb|TIME_DURATION| & 大気のリスタートファイルを出力する時間間隔。 \\ \hline
 \verb|TIME_DT_ATMOS_RESTART_UNIT| & 文字列 & \verb|TIME_DT_UNIT| & 大気のリスタートファイルを出力する時間間隔の時間単位。 \\ \hline
 \verb|TIME_DT_OCEAN_RESTART| & 実数 & \verb|TIME_DURATION| & 海洋のリスタートファイルを出力する時間間隔。 \\ \hline
 \verb|TIME_DT_OCEAN_RESTART_UNIT| & 文字列 & \verb|TIME_DT_UNIT| & 海洋のリスタートファイルを出力する時間間隔の時間単位。 \\ \hline
 \verb|TIME_DT_LAND_RESTART| & 実数 & \verb|TIME_DURATION| & 地表面スキーム関係のリスタートファイルを出力する時間間隔。 \\ \hline
 \verb|TIME_DT_LAND_RESTART_UNIT| & 文字列 & \verb|TIME_DT_UNIT| & 地表面スキーム関係のリスタートファイルを出力する時間間隔の時間単位。 \\ \hline
 \verb|TIME_DT_URBAN_RESTART| & 実数 & \verb|TIME_DURATION| & 都市スキーム関係のリスタートファイルを出力する時間間隔。 \\ \hline
 \verb|TIME_DT_URBAN_RESTART_UNIT| & 文字列 & \verb|TIME_DT_UNIT| & 都市スキームのリスタートファイルを出力する時間間隔の時間単位。 \\ \hline
\end{tabularx}


\subsubsection{PARAM\_GRID}
\begin{tabularx}{150mm}{|l|c|c|X|} \hline
 \rowcolor[gray]{0.9} 名称 & 種類 & 初期値 & 説明 \\ \hline
 \verb|GRID_IN_BASENAME| & 文字列 &  & Gridを外部から与えるときの入力ファイル(省略時はDX, DY, DZ, BAFFFACTなどから生成する) \\ \hline
 \verb|GRID_OUT_BASENAME| & 文字列 &  & Grid情報の出力ファイル名 \\ \hline
% \verb|GRID_OFFSET_X| & 論理値 & .false. & 全ノードログ出力するかどうか。 \\ \hline
% \verb|GRID_OFFSET_Y| & 論理値 & .false. & 全ノードログ出力するかどうか。 \\ \hline
% \verb|FZ| & 論理値 & .false. & 全ノードログ出力するかどうか。 \\ \hline
 \verb|DX| & 実数 & 500.D0 & X方向のGridの間隔(m) \\ \hline
 \verb|DY| & 実数 & 500.D0 & Y方向のGridの間隔(m) \\ \hline
 \verb|DZ| & 実数 & 500.D0 & Z方向のGridの間隔(m) \\ \hline
 \verb|BUFFER_DZ| & 実数 & 0.D0 & Z方向のダンピング層(スポンジ層)の暑さ(m) \\ \hline
 \verb|BUFFER_DX| & 実数 & 0.D0 & X方向のダンピング層(スポンジ層)の暑さ(m) \\ \hline
 \verb|BUFFER_DY| & 実数 & 0.D0 & Y方向のダンピング層(スポンジ層)の暑さ(m) \\ \hline
 \verb|BUFFFACT| & 実数 & 1.D0 & ダンピング層でのGridの引き伸ばし度合い($(dx)_{i+1}=(dx)^{BUFFFACT}_{i}$) \\ \hline
\end{tabularx}

%\subsubsection{PARAM\_COMM}
%\begin{tabularx}{150mm}{|l|c|c|X|} \hline
% \rowcolor[gray]{0.9} 名称 & 種類 & 初期値 & 説明 \\ \hline
% \verb|COMM_VSIZE_MAX| & 論理値 & .false. & 全ノードログ出力するかどうか。 \\ \hline
% \verb|COMM_VSIZE_MAX_PC| & 論理値 & .false. & 全ノードログ出力するかどうか。 \\ \hline
% \verb|COMM_USE_MPI_PC| & 論理値 & .false. & 全ノードログ出力するかどうか。 \\ \hline
%\end{tabularx}

%\subsubsection{PARAM\_LANDUSE}
%\begin{tabularx}{150mm}{|l|c|c|X|} \hline
% \rowcolor[gray]{0.9} 名称 & 種類 & 初期値 & 説明 \\ \hline
% \verb|LANDUSE_IN_BASENAME| & 論理値 & .false. & 全ノードログ出力するかどうか。 \\ \hline
% \verb|LANDUSE_OUT_BASENAME| & 論理値 & .false. & 全ノードログ出力するかどうか。 \\ \hline
% \verb|LANDUSE_OUT_DTYPE| & 論理値 & .false. & 全ノードログ出力するかどうか。 \\ \hline
% \verb|LANDUSE_PFT_MOSAIC| & 論理値 & .false. & 全ノードログ出力するかどうか。 \\ \hline
% \verb|LANDUSE_PFT_NMAX| & 論理値 & .false. & 全ノードログ出力するかどうか。 \\ \hline
% \verb|LANDUSE_ALLLAND| & 論理値 & .false. & 全ノードログ出力するかどうか。 \\ \hline
% \verb|LANDUSE_ALLURBAN| & 論理値 & .false. & 全ノードログ出力するかどうか。 \\ \hline
% \verb|LANDUSE_MOSAICWORLD| & 論理値 & .false. & 全ノードログ出力するかどうか。 \\ \hline
%\end{tabularx}

%\subsubsection{PARAM\_DOMAIN\_CATALOGUE}
%\begin{tabularx}{150mm}{|l|c|c|X|} \hline
% \rowcolor[gray]{0.9} 名称 & 種類 & 初期値 & 説明 \\ \hline
% \verb|DOMAIN_CATALOGUE_FNAME| & 論理値 & .false. & 全ノードログ出力するかどうか。 \\ \hline
% \verb|DOMAIN_CATALOGUE_OUTPUT| & 論理値 & .false. & 全ノードログ出力するかどうか。 \\ \hline
%\end{tabularx}

%\subsubsection{PARAM\_GTRANS}
%\begin{tabularx}{150mm}{|l|c|c|X|} \hline
% \rowcolor[gray]{0.9} 名称 & 種類 & 初期値 & 説明 \\ \hline
% \verb|GTRANS_OUT_BASENAME| & 論理値 & .false. & 全ノードログ出力するかどうか。 \\ \hline
% \verb|GTRANS_OUT_DTYPE| & 論理値 & .false. & 全ノードログ出力するかどうか。 \\ \hline
%\end{tabularx}

\subsubsection{PARAM\_NEST}
\begin{tabularx}{150mm}{|l|c|c|X|} \hline
 \rowcolor[gray]{0.9} 名称 & 種類 & 初期値 & 説明 \\ \hline
 \verb|USE_NESTING| & 論理値 & .false. & Nestingを使うかどうか。 \\ \hline
 \verb|OFFLINE| & 論理値 & .true. & Online Nestingかどうか。\verb|USE_NESTING|が真のときのみ有効。 \\ \hline
 \verb|ONLINE_DOMAIN_NUM| & 整数 &  & ドメイン番号。\verb|USE_NESTING|が真, \verb|OFFLINE|が偽のときのみ有効。 \\ \hline
 \verb|ONLINE_IAM_PARENT| & 論理値 &  & 親ドメインをもつかどうか。\verb|USE_NESTING|が真, \verb|OFFLINE|が偽のときのみ有効。 \\ \hline
 \verb|ONLINE_IAM_DAUGHTER| & 論理値 &  & 娘ドメインをもつかどうか。\verb|USE_NESTING|が真, \verb|OFFLINE|が偽のときのみ有効。 \\ \hline
 \verb|ONLINE_BOUNDARY_USE_QHYD| & 論理値 & .false. & 娘ドメインにQHYDを渡すかどうか。\verb|USE_NESTING|が真, \verb|OFFLINE|が偽のときのみ有効。 \\ \hline
 \verb|ONLINE_AGGRESSIVE_COMM| & 論理値 & .false. & 安全な同期通信を行うかどうか。\verb|USE_NESTING|が真, \verb|OFFLINE|が偽のときのみ有効。 \\ \hline
% \verb|ONLINE_SPECIFIED_MAXRQ| & 論理値 & .false. & わかりません。\verb|USE_NESTING|が真, \verb|OFFLINE|が偽のときのみ有効。 \\ \hline
\end{tabularx}


\subsubsection{PARAM\_STATISTICS}
\begin{tabularx}{150mm}{|l|c|c|X|} \hline
 \rowcolor[gray]{0.9} 名称 & 種類 & 初期値 & 説明 \\ \hline
 \verb|STATISTICS_checktotal| & 論理値 & .false. & 値のチェックを行うかどうか。 \\ \hline
 \verb|STATISTICS_use_globalcomm| & 論理値 & .false. & 全ノード通信を行うかどうか。 \\ \hline
\end{tabularx}


\subsubsection{PARAM\_RESTRAT}
\begin{tabularx}{150mm}{|l|c|c|X|} \hline
 \rowcolor[gray]{0.9} 名称 & 種類 & 初期値 & 説明 \\ \hline
 \verb|RESTART_OUTPUT| & 論理値 & .false. & restartファイルを出力するかどうか。 \\ \hline
 \verb|RESTART_OUT_BASENAME| & 文字列 &  & 書き出すrestartファイルの接頭辞。\verb|RESTART_OUTPUT|が真のときに有効。 \\ \hline
 \verb|RESTART_IN_BASENAME| & 文字列 &  & 読み込むrestartファイルの接頭辞。 \\ \hline
\end{tabularx}


\subsubsection{PARAM\_TOPO}
\begin{tabularx}{150mm}{|l|c|c|X|} \hline
 \rowcolor[gray]{0.9} 名称 & 種類 & 初期値 & 説明 \\ \hline
 \verb|TOPO_IN_BASENAME| & 文字列 &  & 読み込む地形ファイルの接頭辞。 \\ \hline
\end{tabularx}


\subsubsection{PARAM\_LANDUSE}
\begin{tabularx}{150mm}{|l|c|c|X|} \hline
 \rowcolor[gray]{0.9} 名称 & 種類 & 初期値 & 説明 \\ \hline
 \verb|LANDUSE_IN_BASENAME| & 文字列 &  & 読み込む土地利用ファイルの接頭辞。 \\ \hline
\end{tabularx}


\subsubsection{PARAM\_LAND\_PROPERTY}
\begin{tabularx}{150mm}{|l|c|c|X|} \hline
 \rowcolor[gray]{0.9} 名称 & 種類 & 初期値 & 説明 \\ \hline
 \verb|LAND_PROPERTY_IN_FILENAME| & 文字列 &  & 読み込む土壌パラメータファイル名。 \\ \hline
\end{tabularx}


\subsubsection{PARAM\_PRC}
\begin{tabularx}{150mm}{|l|c|c|X|} \hline
 \rowcolor[gray]{0.9} 名称 & 種類 & 初期値 & 説明 \\ \hline
 \verb|PRC_NUM_X| & 整数 &  & X方向に割り当てるプロセス数。 \\ \hline
 \verb|PRC_NUM_Y| & 整数 &  & Y方向に割り当てるプロセス数。 \\ \hline
 \verb|PRC_PERIODIC_X| & 論理値 &  & X方向に周期境界とするかどうか。 \\ \hline
 \verb|PRC_PERIODIC_Y| & 論理値 &  & Y方向に周期境界とするかどうか。 \\ \hline
% \verb|PRC_CART_REORDER| & 論理値 & .false. & 全ノードログ出力するかどうか。 \\ \hline
\end{tabularx}


\subsubsection{PARAM\_INDEX}
\begin{tabularx}{150mm}{|l|c|c|X|} \hline
 \rowcolor[gray]{0.9} 名称 & 種類 & 初期値 & 説明 \\ \hline
 \verb|KMAX| & 整数 &  & 大気の鉛直層数。 \\ \hline
 \verb|IMAX| & 整数 &  & プロセスあたりのX方向の格子数。 \\ \hline
 \verb|JMAX| & 整数 &  & プロセスあたりのY方向の格子数。 \\ \hline
% \verb|IBLOCK| & 整数 &  & 全ノードログ出力するかどうか。 \\ \hline
% \verb|JBLOCK| & 整数 &  & 全ノードログ出力するかどうか。 \\ \hline
\end{tabularx}


\subsubsection{PARAM\_LAND\_INDEX}
\begin{tabularx}{150mm}{|l|c|c|X|} \hline
 \rowcolor[gray]{0.9} 名称 & 種類 & 初期値 & 説明 \\ \hline
 \verb|LKMAX| & 整数 &  & 陸面の鉛直層数。 \\ \hline
\end{tabularx}


\subsubsection{PARAM\_URBAN\_INDEX}
\begin{tabularx}{150mm}{|l|c|c|X|} \hline
 \rowcolor[gray]{0.9} 名称 & 種類 & 初期値 & 説明 \\ \hline
 \verb|UKMAX| & 整数 &  & 都市の鉛直層数。 \\ \hline
\end{tabularx}


\subsubsection{PARAM\_LAND\_GRID}
\begin{tabularx}{150mm}{|l|c|c|X|} \hline
 \rowcolor[gray]{0.9} 名称 & 種類 & 初期値 & 説明 \\ \hline
 \verb|LDZ| & 実数配列 &  & 陸面の鉛直層の層厚。鉛直層数分の設定が必要。 \\ \hline
% \verb|LAND_GRID_IN_BASENAME| & 論理値 & .false. & 全ノードログ出力するかどうか。 \\ \hline
% \verb|LAND_GRID_OUT_BASENAME| & 論理値 & .false. & 全ノードログ出力するかどうか。 \\ \hline
\end{tabularx}


\subsubsection{PARAM\_URBAN\_GRID}
\begin{tabularx}{150mm}{|l|c|c|X|} \hline
 \rowcolor[gray]{0.9} 名称 & 種類 & 初期値 & 説明 \\ \hline
 \verb|UDZ| & 実数配列 &  & 都市の鉛直層の層厚。鉛直層数分の設定が必要。 \\ \hline
% \verb|URBAN_GRID_IN_BASENAME| & 論理値 & .false. & 全ノードログ出力するかどうか。 \\ \hline
% \verb|URBAN_GRID_OUT_BASENAME| & 論理値 & .false. & 全ノードログ出力するかどうか。 \\ \hline
\end{tabularx}


%\subsubsection{PARAM\_GRID}
%\begin{tabularx}{150mm}{|l|c|c|X|} \hline
% \rowcolor[gray]{0.9} 名称 & 種類 & 初期値 & 説明 \\ \hline
% \verb|DZ| & 実数 &  & 大気の鉛直層の層厚。FZと排他的設定。 \\ \hline
% \verb|DX| & 実数 &  & 大気のX方向の格子間隔。\\ \hline
% \verb|DY| & 実数 &  & 大気のY方向の格子間隔。\\ \hline
% \verb|FZ| & 実数配列 &  & 大気の鉛直層の面高度。鉛直層数分の設定が必要。DZと排他的設定。 \\ \hline
% \verb|BUFFER_DZ| & 実数 &  & 大気の最上層の緩和領域幅。 \\ \hline
% \verb|BUFFER_DX| & 実数 &  & 大気のX方向の緩和領域幅。\\ \hline
% \verb|BUFFER_DY| & 実数 &  & 大気のY方向の緩和領域幅。\\ \hline
%\end{tabularx}


\subsubsection{PARAM\_MAPPROJ}
\begin{tabularx}{150mm}{|l|c|c|X|} \hline
 \rowcolor[gray]{0.9} 名称 & 種類 & 初期値 & 説明 \\ \hline
 \verb|MPRJ_basepoint_lon| & 実数 &  & 計算領域の中心経度。 \\ \hline
 \verb|MPRJ_basepoint_lat| & 実数 &  & 計算領域の中心緯度。 \\ \hline
 \verb|MPRJ_type| & 文字列 &  & 計算領域の投影図法。 \\ \hline
 \verb|MPRJ_LC_lat1| & 実数 &  & 投影図法がLCの場合の参照緯度1。 \\ \hline
 \verb|MPRJ_LC_lat2| & 実数 &  & 投影図法がLCの場合の参照緯度2。 \\ \hline
% \verb|MPRJ_PS_LAT| & 論理値 & .false. & 全ノードログ出力するかどうか。 \\ \hline
% \verb|MPRJ_M_LAT| & 論理値 & .false. & 全ノードログ出力するかどうか。 \\ \hline
% \verb|MPRJ_EC_LAT| & 論理値 & .false. & 全ノードログ出力するかどうか。 \\ \hline
\end{tabularx}


\subsubsection{PARAM\_TRACER}
\begin{tabularx}{150mm}{|l|c|c|X|} \hline
 \rowcolor[gray]{0.9} 名称 & 種類 & 初期値 & 説明 \\ \hline
 \verb|TRACER_TYPE| & 文字列 & "OFF" & トレーサーの種類。通常、\verb|ATMOS_PHY_MP_TYPE|と同じ。 \\ \hline
\end{tabularx}


\subsubsection{PARAM\_ATMOS}
\begin{tabularx}{150mm}{|l|c|c|X|} \hline
 \rowcolor[gray]{0.9} 名称 & 種類 & 初期値 & 説明 \\ \hline
 \verb|ATMOS_DYN_TYPE| & 文字列 & "OFF" & 力学スキームの種類。(OFF、HEVE、HEVI、HIVIが選択可能) \\ \hline
 \verb|ATMOS_PHY_MP_TYPE| & 文字列 & "OFF" & 雲微物理スキームの種類。(OFF、KESSLER、TOMITA08、SN14、SUZUKI10が選択可能)\\ \hline
 \verb|ATMOS_PHY_RD_TYPE| & 文字列 & "OFF" & 放射スキームの種類。 (OFF、MSTRNXが選択可能)\\ \hline
 \verb|ATMOS_PHY_SF_TYPE| & 文字列 & "OFF" & 地表面スキームの種類。(OFF、CONST、COUPLEが選択可能) \\ \hline
 \verb|ATMOS_PHY_TB_TYPE| & 文字列 & "OFF" & 乱流スキームの種類。 (OFF、SMAGORINSKY、MYNNが選択可能)\\ \hline
\end{tabularx}


\subsubsection{PARAM\_OCEAN}
\begin{tabularx}{150mm}{|l|c|c|X|} \hline
 \rowcolor[gray]{0.9} 名称 & 種類 & 初期値 & 説明 \\ \hline
 \verb|OCEAN_TYPE| & 文字列 & "OFF" & 海洋スキームの種類。(OFF、CONSTが選択可能) \\ \hline
\end{tabularx}


\subsubsection{PARAM\_LAND}
\begin{tabularx}{150mm}{|l|c|c|X|} \hline
 \rowcolor[gray]{0.9} 名称 & 種類 & 初期値 & 説明 \\ \hline
 \verb|LAND_TYPE| & 文字列 & "OFF" & 陸面スキームの種類。 (OFF、SLUBが選択可能)\\ \hline
\end{tabularx}


\subsubsection{PARAM\_URBAN}
\begin{tabularx}{150mm}{|l|c|c|X|} \hline
 \rowcolor[gray]{0.9} 名称 & 種類 & 初期値 & 説明 \\ \hline
 \verb|URBAN_TYPE| & 文字列 & "OFF" & 都市スキームの種類。(OFF、SLCが選択可能)\\ \hline
\end{tabularx}

*
\subsubsection{PARAM\_HIST}
\begin{tabularx}{150mm}{|l|c|c|X|} \hline
 \rowcolor[gray]{0.9} 名称 & 種類 & 初期値 & 説明 \\ \hline
 \verb|HIST_BND| & 論理値 & .true. & Nesting時にダンピング層をHistoryファイルに出力するか。 \\ \hline
\end{tabularx}


\subsubsection{PARAM\_HISTORY}
\begin{tabularx}{150mm}{|l|c|c|X|} \hline
 \rowcolor[gray]{0.9} 名称 & 種類 & 初期値 & 説明 \\ \hline
 \verb|HISTORY_TITLE| & 文字列 & SCALE-LES HISTORY OUTPUT & NetCDF形式のHistoryファイルのタイトル \\ \hline
% \verb|HISTORY_SOURCE| & 文字列 & "OFF" & 都市スキームの種類。 \\ \hline
% \verb|HISTORY_INSTITUTION| & 文字列 & "OFF" & 都市スキームの種類。 \\ \hline
% \verb|HISTORY_TIME_SINCE| & 文字列 & "OFF" & 都市スキームの種類。 \\ \hline
 \verb|HISTORY_DEFAULT_BASENAME| & 文字列 &  & Historyファイルのファイル名(相対パス) \\ \hline
 \verb|HISTORY_DEFAULT_TINTERVAL| & 実数 &  & Historyを出力する時間間隔。\verb|HISTORY_DEFAULT_TUNIT|が単位となる。 \\ \hline
 \verb|HISTORY_DEFAULT_TUNIT| & 文字列 & "SEC" & \verb|HISTORY_DEFAULT_TINTERVAL|の単位(SEC、HOUR、DAYが利用可能) \\ \hline
 \verb|HISTORY_DEFAULT_TAVERAGE| & 論理値 & .false. & Historyに出力する際\verb|HISTORY_DEFAULT_TINTERVAL|で指定した時間間隔ごとの時間平均を出力するか。 \\ \hline
% \verb|HISTORY_DEFAULT_ZINTERP| & 文字列 & "OFF" & 都市スキームの種類。 \\ \hline
% \verb|HISTORY_DEFAULT_DATATYPE| & 文字列 & "OFF" & 都市スキームの種類。 \\ \hline
% \verb|HISTORY_OUTPUT_STEP0| & 文字列 & "OFF" & 都市スキームの種類。 \\ \hline
% \verb|HISTORY_ERROR_PUTMISS| & 文字列 & "OFF" & 都市スキームの種類。 \\ \hline
\end{tabularx}

%\subsubsection{PARAM\_MONITOR}
%\begin{tabularx}{150mm}{|l|c|c|X|} \hline
% \rowcolor[gray]{0.9} 名称 & 種類 & 初期値 & 説明 \\ \hline
% \verb|MONITOR_OUT_BASENAME| & 文字列 & "OFF" & 都市スキームの種類。 \\ \hline
% \verb|MONITOR_USEDEVATION| & 文字列 & "OFF" & 都市スキームの種類。 \\ \hline
% \verb|MONITOR_STEP_INTERVAL| & 文字列 & "OFF" & 都市スキームの種類。 \\ \hline
%\end{tabularx}

%\subsubsection{PARAM\_NEST}
%\begin{tabularx}{150mm}{|l|c|c|X|} \hline
% \rowcolor[gray]{0.9} 名称 & 種類 & 初期値 & 説明 \\ \hline
% \verb|USE_NESTING| & 論理値 & .false. & 安全な同期通信を行うかどうか。\verb|USE_NESTING|が真, \verb|OFFLINE|が偽のときのみ有効。 \\ \hline
% \verb|LATLON_CATALOGUE_FNAME| & 論理値 & .false. & 安全な同期通信を行うかどうか。\verb|USE_NESTING|が真, \verb|OFFLINE|が偽のときのみ有効。 \\ \hline
% \verb|OFFLINE_PARENT_PRC_NUM_X| & 論理値 & .false. & 安全な同期通信を行うかどうか。\verb|USE_NESTING|が真, \verb|OFFLINE|が偽のときのみ有効。 \\ \hline
% \verb|OFFLINE_PARENT_PRC_NUM_Y| & 論理値 & .false. & 安全な同期通信を行うかどうか。\verb|USE_NESTING|が真, \verb|OFFLINE|が偽のときのみ有効。 \\ \hline
% \verb|OFFLINE_PARENT_KMAX| & 論理値 & .false. & 安全な同期通信を行うかどうか。\verb|USE_NESTING|が真, \verb|OFFLINE|が偽のときのみ有効。 \\ \hline
% \verb|OFFLINE_PARENT_IMAX| & 論理値 & .false. & 安全な同期通信を行うかどうか。\verb|USE_NESTING|が真, \verb|OFFLINE|が偽のときのみ有効。 \\ \hline
% \verb|OFFLINE_PARENT_JMAX| & 論理値 & .false. & 安全な同期通信を行うかどうか。\verb|USE_NESTING|が真, \verb|OFFLINE|が偽のときのみ有効。 \\ \hline
% \verb|OFFLINE_PARENT_LKMAX| & 論理値 & .false. & 安全な同期通信を行うかどうか。\verb|USE_NESTING|が真, \verb|OFFLINE|が偽のときのみ有効。 \\ \hline
% \verb|OFFLINE| & 論理値 & .false. & 安全な同期通信を行うかどうか。\verb|USE_NESTING|が真, \verb|OFFLINE|が偽のときのみ有効。 \\ \hline
% \verb|ONLINE_DOMAIN_NUM| & 論理値 & .false. & 安全な同期通信を行うかどうか。\verb|USE_NESTING|が真, \verb|OFFLINE|が偽のときのみ有効。 \\ \hline
% \verb|ONLINE_IAM_PARENT| & 論理値 & .false. & 安全な同期通信を行うかどうか。\verb|USE_NESTING|が真, \verb|OFFLINE|が偽のときのみ有効。 \\ \hline
% \verb|ONLINE_IAM_DAUGHTER| & 論理値 & .false. & 安全な同期通信を行うかどうか。\verb|USE_NESTING|が真, \verb|OFFLINE|が偽のときのみ有効。 \\ \hline
% \verb|ONLINE_USE_VELZ| & 論理値 & .false. & 安全な同期通信を行うかどうか。\verb|USE_NESTING|が真, \verb|OFFLINE|が偽のときのみ有効。 \\ \hline
% \verb|ONLINE_NO_ROTATE| & 論理値 & .false. & 安全な同期通信を行うかどうか。\verb|USE_NESTING|が真, \verb|OFFLINE|が偽のときのみ有効。 \\ \hline
% \verb|ONLINE_BOUNDARY_USE_QHYD| & 論理値 & .false. & 安全な同期通信を行うかどうか。\verb|USE_NESTING|が真, \verb|OFFLINE|が偽のときのみ有効。 \\ \hline
% \verb|ONLINE_AGGRESSIVE_COMM| & 論理値 & .false. & 安全な同期通信を行うかどうか。\verb|USE_NESTING|が真, \verb|OFFLINE|が偽のときのみ有効。 \\ \hline
% \verb|ONLINE_WAIT_LIMIT| & 論理値 & .false. & 安全な同期通信を行うかどうか。\verb|USE_NESTING|が真, \verb|OFFLINE|が偽のときのみ有効。 \\ \hline
% \verb|ONLINE_SPECIFIED_MAXRQ| & 論理値 & .false. & 安全な同期通信を行うかどうか。\verb|USE_NESTING|が真, \verb|OFFLINE|が偽のときのみ有効。 \\ \hline
% \verb|NEST_INTERP_LEVEL| & 論理値 & .false. & 安全な同期通信を行うかどうか。\verb|USE_NESTING|が真, \verb|OFFLINE|が偽のときのみ有効。 \\ \hline
%\end{tabularx}

%\subsubsection{PARAM\_ATMOS\_SATURATION}
%\begin{tabularx}{150mm}{|l|c|c|X|} \hline
% \rowcolor[gray]{0.9} 名称 & 種類 & 初期値 & 説明 \\ \hline
% \verb|ATMOS_SATURATION_ULIMIT_TEMP| & 論理値 & .false. & 安全な同期通信を行うかどうか。\verb|USE_NESTING|が真, \verb|OFFLINE|が偽のときのみ有効。 \\ \hline
% \verb|ATMOS_SATURATION_LLIMIT_TEMP| & 論理値 & .false. & 安全な同期通信を行うかどうか。\verb|USE_NESTING|が真, \verb|OFFLINE|が偽のときのみ有効。 \\ \hline
%\end{tabularx}


%\subsubsection{PARAM\_BULKFLUX}
%\begin{tabularx}{150mm}{|l|c|c|X|} \hline
% \rowcolor[gray]{0.9} 名称 & 種類 & 初期値 & 説明 \\ \hline
% \verb|BULKFLUX_TYPE| & 論理値 & .false. & 安全な同期通信を行うかどうか。\verb|USE_NESTING|が真, \verb|OFFLINE|が偽のときのみ有効。 \\ \hline
% \verb|BULKFLUX_WSCF| & 論理値 & .false. & 安全な同期通信を行うかどうか。\verb|USE_NESTING|が真, \verb|OFFLINE|が偽のときのみ有効。 \\ \hline
% \verb|BULKFLUX_UABS_MIN| & 論理値 & .false. & 安全な同期通信を行うかどうか。\verb|USE_NESTING|が真, \verb|OFFLINE|が偽のときのみ有効。 \\ \hline
% \verb|BULKFLUX_RIB_MIN| & 論理値 & .false. & 安全な同期通信を行うかどうか。\verb|USE_NESTING|が真, \verb|OFFLINE|が偽のときのみ有効。 \\ \hline
% \verb|BULKFLUX_WSTAR_MIN| & 論理値 & .false. & 安全な同期通信を行うかどうか。\verb|USE_NESTING|が真, \verb|OFFLINE|が偽のときのみ有効。 \\ \hline
%\end{tabularx}

%\subsubsection{PARAM\_ROUGHNESS}
%\begin{tabularx}{150mm}{|l|c|c|X|} \hline
% \rowcolor[gray]{0.9} 名称 & 種類 & 初期値 & 説明 \\ \hline
% \verb|ROUGHNESS_TYPE| & 論理値 & .false. & 安全な同期通信を行うかどうか。\verb|USE_NESTING|が真, \verb|OFFLINE|が偽のときのみ有効。 \\ \hline
% \verb|ROUGHNESS_VISCK| & 論理値 & .false. & 安全な同期通信を行うかどうか。\verb|USE_NESTING|が真, \verb|OFFLINE|が偽のときのみ有効。 \\ \hline
% \verb|ROUGHNESS_USTAR_MIN| & 論理値 & .false. & 安全な同期通信を行うかどうか。\verb|USE_NESTING|が真, \verb|OFFLINE|が偽のときのみ有効。 \\ \hline
% \verb|ROUGHNESS_Z0M_MIN| & 論理値 & .false. & 安全な同期通信を行うかどうか。\verb|USE_NESTING|が真, \verb|OFFLINE|が偽のときのみ有効。 \\ \hline
% \verb|ROUGHNESS_Z0H_MIN| & 論理値 & .false. & 安全な同期通信を行うかどうか。\verb|USE_NESTING|が真, \verb|OFFLINE|が偽のときのみ有効。 \\ \hline
% \verb|ROUGHNESS_Z0E_MIN| & 論理値 & .false. & 安全な同期通信を行うかどうか。\verb|USE_NESTING|が真, \verb|OFFLINE|が偽のときのみ有効。 \\ \hline
%\end{tabularx}

%\subsubsection{PARAM\_ROUGHNESS\_MOON07}
%\begin{tabularx}{150mm}{|l|c|c|X|} \hline
% \rowcolor[gray]{0.9} 名称 & 種類 & 初期値 & 説明 \\ \hline
% \verb|ROUGHNESS_MOON07_ITELIM| & 論理値 & .false. & 安全な同期通信を行うかどうか。\verb|USE_NESTING|が真, \verb|OFFLINE|が偽のときのみ有効。 \\ \hline
%\end{tabularx}


\subsubsection{PARAM\_ATMOS\_VARS}
\begin{tabularx}{150mm}{|l|c|c|X|} \hline
 \rowcolor[gray]{0.9} 名称 & 種類 & 初期値 & 説明 \\ \hline
% \verb|ATMOS_RESTART_IN_BASENAME| & 論理値 & .false. & 安全な同期通信を行うかどうか。\verb|USE_NESTING|が真, \verb|OFFLINE|が偽のときのみ有効。 \\ \hline
% \verb|ATMOS_RESTART_IN_ALLOWMISSINGQ| & 論理値 & .false. & 安全な同期通信を行うかどうか。\verb|USE_NESTING|が真, \verb|OFFLINE|が偽のときのみ有効。 \\ \hline
% \verb|ATMOS_RESTART_OUTPUT| & 論理値 & .false. & 安全な同期通信を行うかどうか。\verb|USE_NESTING|が真, \verb|OFFLINE|が偽のときのみ有効。 \\ \hline
% \verb|ATMOS_RESTART_OUT_BASENAME| & 論理値 & .false. & 安全な同期通信を行うかどうか。\verb|USE_NESTING|が真, \verb|OFFLINE|が偽のときのみ有効。 \\ \hline
% \verb|ATMOS_RESTART_OUT_TITLE| & 論理値 & .false. & 安全な同期通信を行うかどうか。\verb|USE_NESTING|が真, \verb|OFFLINE|が偽のときのみ有効。 \\ \hline
% \verb|ATMOS_RESTART_OUT_DTYPE| & 論理値 & .false. & 安全な同期通信を行うかどうか。\verb|USE_NESTING|が真, \verb|OFFLINE|が偽のときのみ有効。 \\ \hline
% \verb|ATMOS_RESTART_CHECK| & 論理値 & .false. & 安全な同期通信を行うかどうか。\verb|USE_NESTING|が真, \verb|OFFLINE|が偽のときのみ有効。 \\ \hline
% \verb|ATMOS_RESTART_CHECK_BASENAME| & 論理値 & .false. & 安全な同期通信を行うかどうか。\verb|USE_NESTING|が真, \verb|OFFLINE|が偽のときのみ有効。 \\ \hline
% \verb|ATMOS_RESTART_CHECK_CRITERION| & 論理値 & .false. & 安全な同期通信を行うかどうか。\verb|USE_NESTING|が真, \verb|OFFLINE|が偽のときのみ有効。 \\ \hline
 \verb|ATMOS_VARS_CHECKRANGE| & 論理値 & .false. & わかりません。 \\ \hline
\end{tabularx}

%*
%\subsubsection{PARAM\_ATMOS\_DYN\_VARS}
%\begin{tabularx}{150mm}{|l|c|c|X|} \hline
% \rowcolor[gray]{0.9} 名称 & 種類 & 初期値 & 説明 \\ \hline
% \verb|ATMOS_DYN_RESTART_IN_BASENAME| & 論理値 & .false. & 安全な同期通信を行うかどうか。\verb|USE_NESTING|が真, \verb|OFFLINE|が偽のときのみ有効。 \\ \hline
% \verb|ATMOS_DYN_RESTART_OUTPUT| & 論理値 & .false. & 安全な同期通信を行うかどうか。\verb|USE_NESTING|が真, \verb|OFFLINE|が偽のときのみ有効。 \\ \hline
% \verb|ATMOS_DYN_RESTART_OUT_BASENAME| & 論理値 & .false. & 安全な同期通信を行うかどうか。\verb|USE_NESTING|が真, \verb|OFFLINE|が偽のときのみ有効。 \\ \hline
% \verb|ATMOS_DYN_RESTART_OUT_DTYPE| & 論理値 & .false. & 安全な同期通信を行うかどうか。\verb|USE_NESTING|が真, \verb|OFFLINE|が偽のときのみ有効。 \\ \hline
%\end{tabularx}
%
%*
%\subsubsection{PARAM\_ATMOS\_PHY\_MP\_VARS}
%\begin{tabularx}{150mm}{|l|c|c|X|} \hline
% \rowcolor[gray]{0.9} 名称 & 種類 & 初期値 & 説明 \\ \hline
% \verb|ATMOS_PHY_MP_RESTART_IN_BASENAME| & 論理値 & .false. & 安全な同期通信を行うかどうか。\verb|USE_NESTING|が真, \verb|OFFLINE|が偽のときのみ有効。 \\ \hline
% \verb|ATMOS_PHY_MP_RESTART_OUTPUT| & 論理値 & .false. & 安全な同期通信を行うかどうか。\verb|USE_NESTING|が真, \verb|OFFLINE|が偽のときのみ有効。 \\ \hline
% \verb|ATMOS_PHY_MP_RESTART_OUT_BASENAME| & 論理値 & .false. & 安全な同期通信を行うかどうか。\verb|USE_NESTING|が真, \verb|OFFLINE|が偽のときのみ有効。 \\ \hline
% \verb|ATMOS_PHY_MP_RESTART_OUT_TITLE| & 論理値 & .false. & 安全な同期通信を行うかどうか。\verb|USE_NESTING|が真, \verb|OFFLINE|が偽のときのみ有効。 \\ \hline
% \verb|ATMOS_PHY_MP_RESTART_OUT_DTYPE| & 論理値 & .false. & 安全な同期通信を行うかどうか。\verb|USE_NESTING|が真, \verb|OFFLINE|が偽のときのみ有効。 \\ \hline
%\end{tabularx}
%
%*
%\subsubsection{PARAM\_ATMOS\_PHY\_AE\_VARS}
%\begin{tabularx}{150mm}{|l|c|c|X|} \hline
% \rowcolor[gray]{0.9} 名称 & 種類 & 初期値 & 説明 \\ \hline
% \verb|ATMOS_PHY_AE_RESTART_IN_BASENAME| & 論理値 & .false. & 安全な同期通信を行うかどうか。\verb|USE_NESTING|が真, \verb|OFFLINE|が偽のときのみ有効。 \\ \hline
% \verb|ATMOS_PHY_AE_RESTART_OUTPUT| & 論理値 & .false. & 安全な同期通信を行うかどうか。\verb|USE_NESTING|が真, \verb|OFFLINE|が偽のときのみ有効。 \\ \hline
% \verb|ATMOS_PHY_AE_RESTART_OUT_BASENAME| & 論理値 & .false. & 安全な同期通信を行うかどうか。\verb|USE_NESTING|が真, \verb|OFFLINE|が偽のときのみ有効。 \\ \hline
% \verb|ATMOS_PHY_AE_RESTART_OUT_TITLE| & 論理値 & .false. & 安全な同期通信を行うかどうか。\verb|USE_NESTING|が真, \verb|OFFLINE|が偽のときのみ有効。 \\ \hline
% \verb|ATMOS_PHY_AE_RESTART_OUT_DTYPE| & 論理値 & .false. & 安全な同期通信を行うかどうか。\verb|USE_NESTING|が真, \verb|OFFLINE|が偽のときのみ有効。 \\ \hline
%\end{tabularx}
%
%*
%\subsubsection{PARAM\_ATMOS\_PHY\_CP\_VARS}
%\begin{tabularx}{150mm}{|l|c|c|X|} \hline
% \rowcolor[gray]{0.9} 名称 & 種類 & 初期値 & 説明 \\ \hline
% \verb|ATMOS_PHY_CP_RESTART_IN_BASENAME| & 論理値 & .false. & 安全な同期通信を行うかどうか。\verb|USE_NESTING|が真, \verb|OFFLINE|が偽のときのみ有効。 \\ \hline
% \verb|ATMOS_PHY_CP_RESTART_OUTPUT| & 論理値 & .false. & 安全な同期通信を行うかどうか。\verb|USE_NESTING|が真, \verb|OFFLINE|が偽のときのみ有効。 \\ \hline
% \verb|ATMOS_PHY_CP_RESTART_OUT_BASENAME| & 論理値 & .false. & 安全な同期通信を行うかどうか。\verb|USE_NESTING|が真, \verb|OFFLINE|が偽のときのみ有効。 \\ \hline
% \verb|ATMOS_PHY_CP_RESTART_OUT_TITLE| & 論理値 & .false. & 安全な同期通信を行うかどうか。\verb|USE_NESTING|が真, \verb|OFFLINE|が偽のときのみ有効。 \\ \hline
% \verb|ATMOS_PHY_CP_RESTART_OUT_DTYPE| & 論理値 & .false. & 安全な同期通信を行うかどうか。\verb|USE_NESTING|が真, \verb|OFFLINE|が偽のときのみ有効。 \\ \hline
%\end{tabularx}
%
%*
%\subsubsection{PARAM\_ATMOS\_PHY\_CH\_VARS}
%\begin{tabularx}{150mm}{|l|c|c|X|} \hline
% \rowcolor[gray]{0.9} 名称 & 種類 & 初期値 & 説明 \\ \hline
% \verb|ATMOS_PHY_CH_RESTART_IN_BASENAME| & 論理値 & .false. & 安全な同期通信を行うかどうか。\verb|USE_NESTING|が真, \verb|OFFLINE|が偽のときのみ有効。 \\ \hline
% \verb|ATMOS_PHY_CH_RESTART_OUTPUT| & 論理値 & .false. & 安全な同期通信を行うかどうか。\verb|USE_NESTING|が真, \verb|OFFLINE|が偽のときのみ有効。 \\ \hline
% \verb|ATMOS_PHY_CH_RESTART_OUT_BASENAME| & 論理値 & .false. & 安全な同期通信を行うかどうか。\verb|USE_NESTING|が真, \verb|OFFLINE|が偽のときのみ有効。 \\ \hline
% \verb|ATMOS_PHY_CH_RESTART_OUT_TITLE| & 論理値 & .false. & 安全な同期通信を行うかどうか。\verb|USE_NESTING|が真, \verb|OFFLINE|が偽のときのみ有効。 \\ \hline
% \verb|ATMOS_PHY_CH_RESTART_OUT_DTYPE| & 論理値 & .false. & 安全な同期通信を行うかどうか。\verb|USE_NESTING|が真, \verb|OFFLINE|が偽のときのみ有効。 \\ \hline
%\end{tabularx}
%
%*
%\subsubsection{PARAM\_ATMOS\_PHY\_RD\_VARS}
%\begin{tabularx}{150mm}{|l|c|c|X|} \hline
% \rowcolor[gray]{0.9} 名称 & 種類 & 初期値 & 説明 \\ \hline
% \verb|ATMOS_PHY_RD_RESTART_IN_BASENAME| & 論理値 & .false. & 安全な同期通信を行うかどうか。\verb|USE_NESTING|が真, \verb|OFFLINE|が偽のときのみ有効。 \\ \hline
% \verb|ATMOS_PHY_RD_RESTART_OUTPUT| & 論理値 & .false. & 安全な同期通信を行うかどうか。\verb|USE_NESTING|が真, \verb|OFFLINE|が偽のときのみ有効。 \\ \hline
% \verb|ATMOS_PHY_RD_RESTART_OUT_BASENAME| & 論理値 & .false. & 安全な同期通信を行うかどうか。\verb|USE_NESTING|が真, \verb|OFFLINE|が偽のときのみ有効。 \\ \hline
% \verb|ATMOS_PHY_RD_RESTART_OUT_TITLE| & 論理値 & .false. & 安全な同期通信を行うかどうか。\verb|USE_NESTING|が真, \verb|OFFLINE|が偽のときのみ有効。 \\ \hline
% \verb|ATMOS_PHY_RD_RESTART_OUT_DTYPE| & 論理値 & .false. & 安全な同期通信を行うかどうか。\verb|USE_NESTING|が真, \verb|OFFLINE|が偽のときのみ有効。 \\ \hline
%\end{tabularx}
%
%*
%\subsubsection{PARAM\_ATMOS\_PHY\_SF\_VARS}
%\begin{tabularx}{150mm}{|l|c|c|X|} \hline
% \rowcolor[gray]{0.9} 名称 & 種類 & 初期値 & 説明 \\ \hline
% \verb|ATMOS_PHY_SF_RESTART_IN_BASENAME| & 論理値 & .false. & 安全な同期通信を行うかどうか。\verb|USE_NESTING|が真, \verb|OFFLINE|が偽のときのみ有効。 \\ \hline
% \verb|ATMOS_PHY_SF_RESTART_OUTPUT| & 論理値 & .false. & 安全な同期通信を行うかどうか。\verb|USE_NESTING|が真, \verb|OFFLINE|が偽のときのみ有効。 \\ \hline
% \verb|ATMOS_PHY_SF_RESTART_OUT_BASENAME| & 論理値 & .false. & 安全な同期通信を行うかどうか。\verb|USE_NESTING|が真, \verb|OFFLINE|が偽のときのみ有効。 \\ \hline
% \verb|ATMOS_PHY_SF_RESTART_OUT_TITLE| & 論理値 & .false. & 安全な同期通信を行うかどうか。\verb|USE_NESTING|が真, \verb|OFFLINE|が偽のときのみ有効。 \\ \hline
% \verb|ATMOS_PHY_SF_RESTART_OUT_DTYPE| & 論理値 & .false. & 安全な同期通信を行うかどうか。\verb|USE_NESTING|が真, \verb|OFFLINE|が偽のときのみ有効。 \\ \hline
% \verb|ATMOS_PHY_SF_DEFAULT_SFC_TEMP| & 論理値 & .false. & 安全な同期通信を行うかどうか。\verb|USE_NESTING|が真, \verb|OFFLINE|が偽のときのみ有効。 \\ \hline
% \verb|ATMOS_PHY_SF_DEFAULT_SFC_ALBEDO| & 論理値 & .false. & 安全な同期通信を行うかどうか。\verb|USE_NESTING|が真, \verb|OFFLINE|が偽のときのみ有効。 \\ \hline
%\end{tabularx}
%
%*
%\subsubsection{PARAM\_ATMOS\_PHY\_TB\_VARS}
%\begin{tabularx}{150mm}{|l|c|c|X|} \hline
% \rowcolor[gray]{0.9} 名称 & 種類 & 初期値 & 説明 \\ \hline
% \verb|ATMOS_PHY_TB_RESTART_IN_BASENAME| & 論理値 & .false. & 安全な同期通信を行うかどうか。\verb|USE_NESTING|が真, \verb|OFFLINE|が偽のときのみ有効。 \\ \hline
% \verb|ATMOS_PHY_TB_RESTART_OUTPUT| & 論理値 & .false. & 安全な同期通信を行うかどうか。\verb|USE_NESTING|が真, \verb|OFFLINE|が偽のときのみ有効。 \\ \hline
% \verb|ATMOS_PHY_TB_RESTART_OUT_BASENAME| & 論理値 & .false. & 安全な同期通信を行うかどうか。\verb|USE_NESTING|が真, \verb|OFFLINE|が偽のときのみ有効。 \\ \hline
% \verb|ATMOS_PHY_TB_RESTART_OUT_TITLE| & 論理値 & .false. & 安全な同期通信を行うかどうか。\verb|USE_NESTING|が真, \verb|OFFLINE|が偽のときのみ有効。 \\ \hline
% \verb|ATMOS_PHY_TB_RESTART_OUT_DTYPE| & 論理値 & .false. & 安全な同期通信を行うかどうか。\verb|USE_NESTING|が真, \verb|OFFLINE|が偽のときのみ有効。 \\ \hline
%\end{tabularx}
%
%*
%\subsubsection{PARAM\_ATMOS\_PHY\_CP\_VARS}
%\begin{tabularx}{150mm}{|l|c|c|X|} \hline
% \rowcolor[gray]{0.9} 名称 & 種類 & 初期値 & 説明 \\ \hline
% \verb|ATMOS_PHY_CP_RESTART_IN_BASENAME| & 論理値 & .false. & 安全な同期通信を行うかどうか。\verb|USE_NESTING|が真, \verb|OFFLINE|が偽のときのみ有効。 \\ \hline
% \verb|ATMOS_PHY_CP_RESTART_OUTPUT| & 論理値 & .false. & 安全な同期通信を行うかどうか。\verb|USE_NESTING|が真, \verb|OFFLINE|が偽のときのみ有効。 \\ \hline
% \verb|ATMOS_PHY_CP_RESTART_OUT_BASENAME| & 論理値 & .false. & 安全な同期通信を行うかどうか。\verb|USE_NESTING|が真, \verb|OFFLINE|が偽のときのみ有効。 \\ \hline
% \verb|ATMOS_PHY_CP_RESTART_OUT_TITLE| & 論理値 & .false. & 安全な同期通信を行うかどうか。\verb|USE_NESTING|が真, \verb|OFFLINE|が偽のときのみ有効。 \\ \hline
% \verb|ATMOS_PHY_CP_RESTART_OUT_DTYPE| & 論理値 & .false. & 安全な同期通信を行うかどうか。\verb|USE_NESTING|が真, \verb|OFFLINE|が偽のときのみ有効。 \\ \hline
%\end{tabularx}
%
%*
%\subsubsection{PARAM\_OCEAN\_VARS}
%\begin{tabularx}{150mm}{|l|c|c|X|} \hline
% \rowcolor[gray]{0.9} 名称 & 種類 & 初期値 & 説明 \\ \hline
% \verb|OCEAN_RESTART_IN_BASENAME| & 論理値 & .false. & 安全な同期通信を行うかどうか。\verb|USE_NESTING|が真, \verb|OFFLINE|が偽のときのみ有効。 \\ \hline
% \verb|OCEAN_RESTART_OUTPUT| & 論理値 & .false. & 安全な同期通信を行うかどうか。\verb|USE_NESTING|が真, \verb|OFFLINE|が偽のときのみ有効。 \\ \hline
% \verb|OCEAN_RESTART_OUT_BASENAME| & 論理値 & .false. & 安全な同期通信を行うかどうか。\verb|USE_NESTING|が真, \verb|OFFLINE|が偽のときのみ有効。 \\ \hline
% \verb|OCEAN_RESTART_OUT_TITLE| & 論理値 & .false. & 安全な同期通信を行うかどうか。\verb|USE_NESTING|が真, \verb|OFFLINE|が偽のときのみ有効。 \\ \hline
%\end{tabularx}
%
%*
%\subsubsection{PARAM\_OCEAN\_VARS}
%\begin{tabularx}{150mm}{|l|c|c|X|} \hline
% \rowcolor[gray]{0.9} 名称 & 種類 & 初期値 & 説明 \\ \hline
% \verb|LAND_RESTART_IN_BASENAME| & 論理値 & .false. & 安全な同期通信を行うかどうか。\verb|USE_NESTING|が真, \verb|OFFLINE|が偽のときのみ有効。 \\ \hline
% \verb|LAND_RESTART_OUTPUT| & 論理値 & .false. & 安全な同期通信を行うかどうか。\verb|USE_NESTING|が真, \verb|OFFLINE|が偽のときのみ有効。 \\ \hline
% \verb|LAND_RESTART_OUT_BASENAME| & 論理値 & .false. & 安全な同期通信を行うかどうか。\verb|USE_NESTING|が真, \verb|OFFLINE|が偽のときのみ有効。 \\ \hline
% \verb|LAND_RESTART_OUT_TITLE| & 論理値 & .false. & 安全な同期通信を行うかどうか。\verb|USE_NESTING|が真, \verb|OFFLINE|が偽のときのみ有効。 \\ \hline
% \verb|LAND_RESTART_OUT_DTYPE| & 論理値 & .false. & 安全な同期通信を行うかどうか。\verb|USE_NESTING|が真, \verb|OFFLINE|が偽のときのみ有効。 \\ \hline
% \verb|LAND_VARS_CHECKRANGE| & 論理値 & .false. & 安全な同期通信を行うかどうか。\verb|USE_NESTING|が真, \verb|OFFLINE|が偽のときのみ有効。 \\ \hline
% \verb|LAND_RESTART_OUT_TITLE| & 論理値 & .false. & 安全な同期通信を行うかどうか。\verb|USE_NESTING|が真, \verb|OFFLINE|が偽のときのみ有効。 \\ \hline
%\end{tabularx}
%
%*
%\subsubsection{PARAM\_LAND\_PROPERTY}
%\begin{tabularx}{150mm}{|l|c|c|X|} \hline
% \rowcolor[gray]{0.9} 名称 & 種類 & 初期値 & 説明 \\ \hline
% \verb|LAND_PROPERTY_IN_FILENAME| & 論理値 & .false. & 安全な同期通信を行うかどうか。\verb|USE_NESTING|が真, \verb|OFFLINE|が偽のときのみ有効。 \\ \hline
%\end{tabularx}
%
%*
%\subsubsection{PARAM\_URBAN\_VARS}
%\begin{tabularx}{150mm}{|l|c|c|X|} \hline
% \rowcolor[gray]{0.9} 名称 & 種類 & 初期値 & 説明 \\ \hline
% \verb|URBAN_RESTART_IN_BASENAME| & 論理値 & .false. & 安全な同期通信を行うかどうか。\verb|USE_NESTING|が真, \verb|OFFLINE|が偽のときのみ有効。 \\ \hline
% \verb|URBAN_RESTART_OUTPUT| & 論理値 & .false. & 安全な同期通信を行うかどうか。\verb|USE_NESTING|が真, \verb|OFFLINE|が偽のときのみ有効。 \\ \hline
% \verb|URBAN_RESTART_OUT_TITLE| & 論理値 & .false. & 安全な同期通信を行うかどうか。\verb|USE_NESTING|が真, \verb|OFFLINE|が偽のときのみ有効。 \\ \hline
% \verb|URBAN_RESTART_OUT_DTYPE| & 論理値 & .false. & 安全な同期通信を行うかどうか。\verb|USE_NESTING|が真, \verb|OFFLINE|が偽のときのみ有効。 \\ \hline
% \verb|URBAN_VARS_CHECKRANGE| & 論理値 & .false. & 安全な同期通信を行うかどうか。\verb|USE_NESTING|が真, \verb|OFFLINE|が偽のときのみ有効。 \\ \hline
%\end{tabularx}
%
%*
%\subsubsection{PARAM\_ATMOS\_SOLARINS}
%\begin{tabularx}{150mm}{|l|c|c|X|} \hline
% \rowcolor[gray]{0.9} 名称 & 種類 & 初期値 & 説明 \\ \hline
% \verb|ATMOS_SOLARINS_CONSTANT| & 論理値 & .false. & 安全な同期通信を行うかどうか。\verb|USE_NESTING|が真, \verb|OFFLINE|が偽のときのみ有効。 \\ \hline
% \verb|ATMOS_SOLARINS_FIXEDLATLON| & 論理値 & .false. & 安全な同期通信を行うかどうか。\verb|USE_NESTING|が真, \verb|OFFLINE|が偽のときのみ有効。 \\ \hline
% \verb|ATMOS_SOLARINS_FIXEDDATE| & 論理値 & .false. & 安全な同期通信を行うかどうか。\verb|USE_NESTING|が真, \verb|OFFLINE|が偽のときのみ有効。 \\ \hline
% \verb|ATMOS_SOLARINS_LON| & 論理値 & .false. & 安全な同期通信を行うかどうか。\verb|USE_NESTING|が真, \verb|OFFLINE|が偽のときのみ有効。 \\ \hline
% \verb|ATMOS_SOLARINS_LAT| & 論理値 & .false. & 安全な同期通信を行うかどうか。\verb|USE_NESTING|が真, \verb|OFFLINE|が偽のときのみ有効。 \\ \hline
% \verb|ATMOS_SOLARINS_DATE| & 論理値 & .false. & 安全な同期通信を行うかどうか。\verb|USE_NESTING|が真, \verb|OFFLINE|が偽のときのみ有効。 \\ \hline
%\end{tabularx}

\subsubsection{PARAM\_ATMOS\_REFSTATE}
\begin{tabularx}{150mm}{|l|c|c|X|} \hline
 \rowcolor[gray]{0.9} 名称 & 種類 & 初期値 & 説明 \\ \hline
 \verb|ATMOS_REFSTATE_TYPE| & 文字列 & UNIFORM & Reference stateに何を用いるか。(UNIFORM:一定値で与える。値は\verb|ATMOS_REFSTATE_**|によって決まる、INIT:初期値(リスタートの時はリスタート時の値)、FILE:\verb|ATMOS_REFSTATE_IN_BASENAME|から読み出す、ISA:International Standard Atmosphere) \\ \hline
 \verb|ATMOS_REFSTATE_IN_BASENAME| & 文字列 & & Reference stateの入力ファイル名 \\ \hline
 \verb|ATMOS_REFSTATE_OUT_BASENAME| & 文字列 & & Reference stateの出力ファイル名 \\ \hline
% \verb|ATMOS_REFSTATE_OUT_TITLE| & 論理値 & .false. & 安全な同期通信を行うかどうか。\verb|USE_NESTING|が真, \verb|OFFLINE|が偽のときのみ有効。 \\ \hline
% \verb|ATMOS_REFSTATE_OUT_DTYPE| & 論理値 & .false. & 安全な同期通信を行うかどうか。\verb|USE_NESTING|が真, \verb|OFFLINE|が偽のときのみ有効。 \\ \hline
% \verb|ATMOS_REFSTATE_TEMP_SFC| & 論理値 & .false. & 安全な同期通信を行うかどうか。\verb|USE_NESTING|が真, \verb|OFFLINE|が偽のときのみ有効。 \\ \hline
% \verb|ATMOS_REFSTATE_RH| & 論理値 & .false. & 安全な同期通信を行うかどうか。\verb|USE_NESTING|が真, \verb|OFFLINE|が偽のときのみ有効。 \\ \hline
% \verb|ATMOS_REFSTATE_POTT_UNIFORM| & 論理値 & .false. & 安全な同期通信を行うかどうか。\verb|USE_NESTING|が真, \verb|OFFLINE|が偽のときのみ有効。 \\ \hline
% \verb|ATMOS_REFSTATE_UPDATE_FLAG| & 論理値 & .false. & 安全な同期通信を行うかどうか。\verb|USE_NESTING|が真, \verb|OFFLINE|が偽のときのみ有効。 \\ \hline
 \verb|ATMOS_REFSTATE_UPDATE_DT| & 実数 & 0.D0 & Reference Stateをアップデートする時間間隔(0の時はReference stateが更新されない) \\ \hline
\end{tabularx}

\subsubsection{PARAM\_ATMOS\_BOUNDARY}
\begin{tabularx}{150mm}{|l|c|c|X|} \hline
 \rowcolor[gray]{0.9} 名称 & 種類 & 初期値 & 説明 \\ \hline
 \verb|ATMOS_BOUNDARY_TYPE| & 文字列 & NONE & バッファ層でどの値に緩和させるか。(CONST:一定値(\verb|ATMOS_BOUNDARY_VALUE_**|に緩和)、INIT:初期値に緩和、REAL:Nesting時、FILE:入力ファイル(\verb|ATMOS_BOUNDARY_IN_BASENAME|)から読み込む)) \\ \hline
 \verb|ATMOS_BOUNDARY_IN_BASENAME| & 文字列 &  & \verb|ATMOS_BOUNDARY_TYPE|=FILEの時に読み込む外部ファイル  \\ \hline
 \verb|ATMOS_BOUNDARY_OUT_BASENAME| & 文字列 &  & バッファ層で緩和させる値を出力する際の出力ファイル名(設定しない場合は出力されない) \\ \hline
% \verb|ATMOS_BOUNDARY_OUT_TITLE| & 論理値 & .false. & 安全な同期通信を行うかどうか。\verb|USE_NESTING|が真, \verb|OFFLINE|が偽のときのみ有効。 \\ \hline
 \verb|ATMOS_BOUNDARY_USE_VELZ| & 論理値 & .false. &  鉛直方向の運動量にダンピングを作用させるか否か。 \\ \hline
 \verb|ATMOS_BOUNDARY_USE_VELY| & 論理値 & .false. &  Y方向の運動量にダンピングを作用させるか否か。 \\ \hline
 \verb|ATMOS_BOUNDARY_USE_VELX| & 論理値 & .false. &  X方向の運動量にダンピングを作用させるか否か。 \\ \hline
 \verb|ATMOS_BOUNDARY_USE_POTT| & 論理値 & .false. &  温位にダンピングを作用させるか否か。 \\ \hline
 \verb|ATMOS_BOUNDARY_USE_QV| & 論理値 & .false. &  水蒸気混合比にダンピングを作用させるか否か。 \\ \hline
 \verb|ATMOS_BOUNDARY_USE_QHYD| & 論理値 & .false. &  水物質の混合比ににダンピングを作用させるか否か。 \\ \hline
 \verb|ATMOS_BOUNDARY_VALUE_VELZ| & 実数 & 0.D0 & 鉛直方向の運動量を緩和させる値(\verb|ATMOS_BOUNDARY_TYPE|=.true.の時に有効) \\ \hline
 \verb|ATMOS_BOUNDARY_VALUE_VELX| & 実数 & 0.D0 & X方向の運動量を緩和させる値(\verb|ATMOS_BOUNDARY_TYPE|=.true.の時に有効) \\ \hline
 \verb|ATMOS_BOUNDARY_VALUE_VELY| & 実数 & 0.D0 & Y方向の運動量を緩和させる値(\verb|ATMOS_BOUNDARY_TYPE|=.true.の時に有効) \\ \hline
 \verb|ATMOS_BOUNDARY_VALUE_POTT| & 実数 & 3.D0 & 温位を緩和させる値(\verb|ATMOS_BOUNDARY_TYPE|=.true.の時に有効) \\ \hline
 \verb|ATMOS_BOUNDARY_VALUE_QV| & 実数 & 0.D0 & 水蒸気混合比を緩和させる値(\verb|ATMOS_BOUNDARY_TYPE|=.true.の時に有効) \\ \hline
 \verb|ATMOS_BOUNDARY_VALUE_QHYD| & 実数 & 0.D0 & 水物質の混合比を緩和させる値(\verb|ATMOS_BOUNDARY_TYPE|=.true.の時に有効) \\ \hline
% \verb|ATMOS_BOUNDARY_SMOOTHER_FACT| & 論理値 & .false. & 安全な同期通信を行うかどうか。\verb|USE_NESTING|が真, \verb|OFFLINE|が偽のときのみ有効。 \\ \hline
% \verb|ATMOS_BOUNDARY_FRACZ| & 論理値 & .false. & 安全な同期通信を行うかどうか。\verb|USE_NESTING|が真, \verb|OFFLINE|が偽のときのみ有効。 \\ \hline
% \verb|ATMOS_BOUNDARY_FRACX| & 論理値 & .false. & 安全な同期通信を行うかどうか。\verb|USE_NESTING|が真, \verb|OFFLINE|が偽のときのみ有効。 \\ \hline
% \verb|ATMOS_BOUNDARY_FRACY| & 論理値 & .false. & 安全な同期通信を行うかどうか。\verb|USE_NESTING|が真, \verb|OFFLINE|が偽のときのみ有効。 \\ \hline
% \verb|ATMOS_BOUNDARY_TAUZ| & 論理値 & .false. & 安全な同期通信を行うかどうか。\verb|USE_NESTING|が真, \verb|OFFLINE|が偽のときのみ有効。 \\ \hline
% \verb|ATMOS_BOUNDARY_TAUX| & 論理値 & .false. & 安全な同期通信を行うかどうか。\verb|USE_NESTING|が真, \verb|OFFLINE|が偽のときのみ有効。 \\ \hline
% \verb|ATMOS_BOUNDARY_TAUY| & 論理値 & .false. & 安全な同期通信を行うかどうか。\verb|USE_NESTING|が真, \verb|OFFLINE|が偽のときのみ有効。 \\ \hline
% \verb|ATMOS_BOUNDARY_UPDATE_DT| & 論理値 & .false. & 安全な同期通信を行うかどうか。\verb|USE_NESTING|が真, \verb|OFFLINE|が偽のときのみ有効。 \\ \hline
% \verb|ATMOS_BOUNDARY_START_DATE| & 論理値 & .false. & 安全な同期通信を行うかどうか。\verb|USE_NESTING|が真, \verb|OFFLINE|が偽のときのみ有効。 \\ \hline
 \verb|ATMOS_BOUNDARY_LINEAR_V| & 論理値 & .false. &  わかりません\\ \hline
 \verb|ATMOS_BOUNDARY_LINEAR_H| & 論理値 & .true.  &  わかりません\\ \hline
 \verb|ATMOS_BOUNDARY_EXP_H| & 実数 & 2.D0 & わかりません \\ \hline
% \verb|ATMOS_BOUNDARY_INCREMENT_TYPE| & 論理値 & .false. & 安全な同期通信を行うかどうか。\verb|USE_NESTING|が真, \verb|OFFLINE|が偽のときのみ有効。 \\ \hline
\end{tabularx}


\subsubsection{PARAM\_ATMOS\_DYN}
\begin{tabularx}{150mm}{|l|c|c|X|} \hline
 \rowcolor[gray]{0.9} 名称 & 種類 & 初期値 & 説明 \\ \hline
 \verb|ATMOS_DYN_NUMERICAL_DIFF_ORDER| & 実数 & 1.D0. & 数値粘性のオーダーを4で割ったもの(\verb|ATMOS_DYN_NUMERICAL_DIFF_ORDER|=1なら4次の数値粘性になる) \\ \hline
 \verb|ATMOS_DYN_NUMERICAL_DIFF_COEF| & 実数 & 1.D-4 & 混合比以外の予報変数にかかる数値粘性の強さを決める無次元定数\\ \hline
 \verb|ATMOS_DYN_NUMERICAL_DIFF_COEF_Q| & 実数 & 1.D-4 & 混合比にかかる数値粘性の強さを決める無次元定数\\ \hline
% \verb|ATMOS_DYN_NUMERICAL_DIFF_SFC_FACT| & 論理値 & .false. & 安全な同期通信を行うかどうか。\verb|USE_NESTING|が真, \verb|OFFLINE|が偽のときのみ有効。 \\ \hline
% \verb|ATMOS_DYN_NUMERICAL_DIFF_USE_REFSTATE| & 論理値 & .false. & 安全な同期通信を行うかどうか。\verb|USE_NESTING|が真, \verb|OFFLINE|が偽のときのみ有効。 \\ \hline
 \verb|ATMOS_DYN_ENABLE_CORIOLIS| & 論理値 & .false. & コリオリ力を考慮するか \\ \hline
% \verb|ATMOS_DYN_DIVDMP_COEF| & 論理値 & .false. & 安全な同期通信を行うかどうか。\verb|USE_NESTING|が真, \verb|OFFLINE|が偽のときのみ有効。 \\ \hline
 \verb|ATMOS_DYN_FLAG_FCT_RHO| & 論理値 & .false. & 密度にFCTをかけるか \\ \hline
 \verb|ATMOS_DYN_FLAG_FCT_MOMENTUM| & 論理値 & .false. & 運動量にFCTをかけるか。 \\ \hline
 \verb|ATMOS_DYN_FLAG_FCT_T| & 論理値 & .false. & 温位にFCTをかけるか。\\ \hline
% \verb|ATMOS_DYN_FLAG_FCT_ALONG_STREAM| & 論理値 & .false. & 安全な同期通信を行うかどうか。\verb|USE_NESTING|が真, \verb|OFFLINE|が偽のときのみ有効。 \\ \hline
% \verb|ATMOS_DYN_ADJUST_FLUX_CELL| & 論理値 & .false. & 安全な同期通信を行うかどうか。\verb|USE_NESTING|が真, \verb|OFFLINE|が偽のときのみ有効。 \\ \hline
\end{tabularx}

\subsubsection{PARAM\_ATMOS\_PHY\_MP}
\begin{tabularx}{150mm}{|l|c|c|X|} \hline
 \rowcolor[gray]{0.9} 名称 & 種類 & 初期値 & 説明 \\ \hline
 \verb|MP_DOPRECIPITATION| & 論理値 & .true. & 水物質の重力による落下を考慮するか否か \\ \hline
 \verb|MP_DONEGATIVE_FIXER| & 論理値 & .false. & 雲物理スキームの前後で負値を0にするか否か。 \\ \hline
 \verb|MP_NTMAX_SEDIMENTATION| & 整数 & 1 & 水物質の重力落下の計算をする際に区切るステップ数 \\ \hline
\end{tabularx}


%\subsubsection{PARAM\_ATMOS\_PHY\_MP\_TOMITA08}
%\begin{tabularx}{150mm}{|l|c|c|X|} \hline
% \rowcolor[gray]{0.9} 名称 & 種類 & 初期値 & 説明 \\ \hline
% \verb|AUTOCONV_NC| & 論理値 & .false. & 安全な同期通信を行うかどうか。\verb|USE_NESTING|が真, \verb|OFFLINE|が偽のときのみ有効。 \\ \hline
% \verb|AUTOCONV_USEKK| & 論理値 & .false. & 安全な同期通信を行うかどうか。\verb|USE_NESTING|が真, \verb|OFFLINE|が偽のときのみ有効。 \\ \hline
% \verb|PARAM_ROH14| & 論理値 & .false. & 安全な同期通信を行うかどうか。\verb|USE_NESTING|が真, \verb|OFFLINE|が偽のときのみ有効。 \\ \hline
% \verb|DENS_S| & 論理値 & .false. & 安全な同期通信を行うかどうか。\verb|USE_NESTING|が真, \verb|OFFLINE|が偽のときのみ有効。 \\ \hline
% \verb|DENS_G| & 論理値 & .false. & 安全な同期通信を行うかどうか。\verb|USE_NESTING|が真, \verb|OFFLINE|が偽のときのみ有効。 \\ \hline
% \verb|RE_QC| & 論理値 & .false. & 安全な同期通信を行うかどうか。\verb|USE_NESTING|が真, \verb|OFFLINE|が偽のときのみ有効。 \\ \hline
% \verb|RE_QI| & 論理値 & .false. & 安全な同期通信を行うかどうか。\verb|USE_NESTING|が真, \verb|OFFLINE|が偽のときのみ有効。 \\ \hline
% \verb|DEBUG| & 論理値 & .false. & 安全な同期通信を行うかどうか。\verb|USE_NESTING|が真, \verb|OFFLINE|が偽のときのみ有効。 \\ \hline
%\end{tabularx}


\subsubsection{PARAM\_ATMOS\_PHY\_RD\_MSTRN}
\begin{tabularx}{150mm}{|l|c|c|X|} \hline
 \rowcolor[gray]{0.9} 名称 & 種類 & 初期値 & 説明 \\ \hline
 \verb|ATMOS_PHY_RD_MSTRN_KADD| & 整数 & 0 & 放射スキームで計算する際にモデル上端から上に加える層数。 \\ \hline
 \verb|ATMOS_PHY_RD_MSTRN_GASPARA_IN_FILENAME| & 文字列 & 'PARAG.29' & 放射スキームで用いる希ガスパラメータの入力ファイル名 \\ \hline
 \verb|ATMOS_PHY_RD_MSTRN_AEROPARA_IN_FILENAME| & 文字列 & 'PARAC.29' & 放射スキームで用いるエアロゾルパラメータの入力ファイル名 \\ \hline
 \verb|ATMOS_PHY_RD_MSTRN_HYGROPARA_IN_FILENAME| & 文字列 & 'VARDATA.RM29' & 放射スキームで用いる水蒸気と水物質パラメータの入力ファイル名 \\ \hline
 \verb|ATMOS_PHY_RD_MSTRN_NBAND| & 整数 & 29 & MSTRNXで計算に用いる波長バンド数(\verb|ATMOS_PHY_RD_MSTRN_GAS(AERO,HYGRO)PARA_IN_FILENAME|のバンド数と対応する必要がある) \\ \hline
\end{tabularx}

\subsubsection{PARAM\_ATMOS\_PHY\_RD\_PROFILE}
\begin{tabularx}{150mm}{|l|c|c|X|} \hline
 \rowcolor[gray]{0.9} 名称 & 種類 & 初期値 & 説明 \\ \hline
 \verb|ATMOS_PHY_RD_PROFILE_TOA| & 実数 & 100.D0 & 放射スキームで用いるTop Of Atmospher (TOA)の高度(km) \\ \hline
 \verb|ATMOS_PHY_RD_PROFILE_USE_CLIMATOLOGY| & 論理値 & .true. & 放射スキームに必要な大気のプロファイルに気候値を用いるかどうか。 \\ \hline
 \verb|ATMOS_PHY_RD_PROFILE_CIRA86_IN_FILENAME| & 文字列 & ./ & cira.ncの相対パス \\ \hline
 \verb|ATMOS_PHY_RD_PROFILE_MIPAS2001_IN_BASENAME| & 文字列 &  & MIPAS2001が保存されているディレクトリ名(相対パス) \\ \hline
 \verb|ATMOS_PHY_RD_PROFILE_USER_IN_FILENAME| & 文字列 &  & 放射スキームに用いる大気のプロファイルの入力ファイル名(省略された場合は気候値が用いられる) \\ \hline
 \verb|ATMOS_PHY_RD_PROFILE_USE_CO2| & 論理値 & .true. & 放射計算で二酸化炭素を考慮するか \\ \hline
 \verb|ATMOS_PHY_RD_PROFILE_USE_O3| & 論理値 & .true. & 放射計算でオゾンを考慮するか \\ \hline
 \verb|ATMOS_PHY_RD_PROFILE_USE_CO| & 論理値 & .true. & 放射計算で一酸化炭素を考慮するか \\ \hline
 \verb|ATMOS_PHY_RD_PROFILE_USE_N2O| & 論理値 & .true. & 放射計算で一酸化二窒素を考慮するか \\ \hline
 \verb|ATMOS_PHY_RD_PROFILE_USE_CH4| & 論理値 & .true. & 放射計算でメタンを考慮するか \\ \hline
 \verb|ATMOS_PHY_RD_PROFILE_USE_O2| & 論理値 & .true. & 放射計算で酸素を考慮するか \\ \hline
 \verb|ATMOS_PHY_RD_PROFILE_USE_CFC| & 論理値 & .true. & 放射計算でCFCを考慮するか \\ \hline
 \verb|DEBUG| & 論理値 & .false. & 放射計算においてDEBUG情報を出力するか否か \\ \hline
\end{tabularx}

\subsubsection{PARAM\_OCEAN\_PHY\_SLAB}
\begin{tabularx}{150mm}{|l|c|c|X|} \hline
 \rowcolor[gray]{0.9} 名称 & 種類 & 初期値 & 説明 \\ \hline
 \verb|OCEAN_PHY_SLAB_DEPTH| & 実数 & 10.D0 & スラブモデルにおける海洋の深さ \\ \hline
\end{tabularx}

\subsubsection{PARAM\_LAND\_PHY\_SLAB}
\begin{tabularx}{150mm}{|l|c|c|X|} \hline
 \rowcolor[gray]{0.9} 名称 & 種類 & 初期値 & 説明 \\ \hline
 \verb|LAND_PHY_UPDATE_BOTTOM_TEMP| & 論理値 & .false. & わかりません \\ \hline
 \verb|LAND_PHY_UPDATE_BOTTOM_WATER| & 論理値 & .false. & わかりません \\ \hline
\end{tabularx}

%\subsubsection{PARAM\_LAND\_SFC\_SLAB}
%\begin{tabularx}{150mm}{|l|c|c|X|} \hline
% \rowcolor[gray]{0.9} 名称 & 種類 & 初期値 & 説明 \\ \hline
% \verb|LAND_SFC_SLAB_ITR_MAX| & 論理値 & .false. & 安全な同期通信を行うかどうか。\verb|USE_NESTING|が真, \verb|OFFLINE|が偽のときのみ有効。 \\ \hline
% \verb|LAND_SFC_SLAB_RES_MIN| & 論理値 & .false. & 安全な同期通信を行うかどうか。\verb|USE_NESTING|が真, \verb|OFFLINE|が偽のときのみ有効。 \\ \hline
% \verb|LAND_SFC_SLAB_DTS_MAX| & 論理値 & .false. & 安全な同期通信を行うかどうか。\verb|USE_NESTING|が真, \verb|OFFLINE|が偽のときのみ有効。 \\ \hline
%\end{tabularx}


\subsubsection{PARAM\_URBAN\_PHY\_SLC足立さんよろしくです}
\begin{tabularx}{150mm}{|l|c|c|X|} \hline
 \rowcolor[gray]{0.9} 名称 & 種類 & 初期値 & 説明 \\ \hline
 \verb|DTS_MAX| & 論理値 & .false. & 安全な同期通信を行うかどうか。\verb|USE_NESTING|が真, \verb|OFFLINE|が偽のときのみ有効。 \\ \hline
 \verb|ZR| & 論理値 & .false. & 安全な同期通信を行うかどうか。\verb|USE_NESTING|が真, \verb|OFFLINE|が偽のときのみ有効。 \\ \hline
 \verb|ROOF_WIDTH| & 論理値 & .false. & 安全な同期通信を行うかどうか。\verb|USE_NESTING|が真, \verb|OFFLINE|が偽のときのみ有効。 \\ \hline
 \verb|ROAD_WIDTH| & 論理値 & .false. & 安全な同期通信を行うかどうか。\verb|USE_NESTING|が真, \verb|OFFLINE|が偽のときのみ有効。 \\ \hline
 \verb|SIGMA_ZED| & 論理値 & .false. & 安全な同期通信を行うかどうか。\verb|USE_NESTING|が真, \verb|OFFLINE|が偽のときのみ有効。 \\ \hline
 \verb|AH| & 論理値 & .false. & 安全な同期通信を行うかどうか。\verb|USE_NESTING|が真, \verb|OFFLINE|が偽のときのみ有効。 \\ \hline
 \verb|AHL| & 論理値 & .false. & 安全な同期通信を行うかどうか。\verb|USE_NESTING|が真, \verb|OFFLINE|が偽のときのみ有効。 \\ \hline
 \verb|BETR| & 論理値 & .false. & 安全な同期通信を行うかどうか。\verb|USE_NESTING|が真, \verb|OFFLINE|が偽のときのみ有効。 \\ \hline
 \verb|BETB| & 論理値 & .false. & 安全な同期通信を行うかどうか。\verb|USE_NESTING|が真, \verb|OFFLINE|が偽のときのみ有効。 \\ \hline
 \verb|BETG| & 論理値 & .false. & 安全な同期通信を行うかどうか。\verb|USE_NESTING|が真, \verb|OFFLINE|が偽のときのみ有効。 \\ \hline
 \verb|STRGR| & 論理値 & .false. & 安全な同期通信を行うかどうか。\verb|USE_NESTING|が真, \verb|OFFLINE|が偽のときのみ有効。 \\ \hline
 \verb|STRGB| & 論理値 & .false. & 安全な同期通信を行うかどうか。\verb|USE_NESTING|が真, \verb|OFFLINE|が偽のときのみ有効。 \\ \hline
 \verb|STRGG| & 論理値 & .false. & 安全な同期通信を行うかどうか。\verb|USE_NESTING|が真, \verb|OFFLINE|が偽のときのみ有効。 \\ \hline
 \verb|CAPR| & 論理値 & .false. & 安全な同期通信を行うかどうか。\verb|USE_NESTING|が真, \verb|OFFLINE|が偽のときのみ有効。 \\ \hline
 \verb|CAPB| & 論理値 & .false. & 安全な同期通信を行うかどうか。\verb|USE_NESTING|が真, \verb|OFFLINE|が偽のときのみ有効。 \\ \hline
 \verb|CAPG| & 論理値 & .false. & 安全な同期通信を行うかどうか。\verb|USE_NESTING|が真, \verb|OFFLINE|が偽のときのみ有効。 \\ \hline
 \verb|AKSR| & 論理値 & .false. & 安全な同期通信を行うかどうか。\verb|USE_NESTING|が真, \verb|OFFLINE|が偽のときのみ有効。 \\ \hline
 \verb|AKSB| & 論理値 & .false. & 安全な同期通信を行うかどうか。\verb|USE_NESTING|が真, \verb|OFFLINE|が偽のときのみ有効。 \\ \hline
 \verb|AKSG| & 論理値 & .false. & 安全な同期通信を行うかどうか。\verb|USE_NESTING|が真, \verb|OFFLINE|が偽のときのみ有効。 \\ \hline
 \verb|ALBR| & 論理値 & .false. & 安全な同期通信を行うかどうか。\verb|USE_NESTING|が真, \verb|OFFLINE|が偽のときのみ有効。 \\ \hline
 \verb|ALBB| & 論理値 & .false. & 安全な同期通信を行うかどうか。\verb|USE_NESTING|が真, \verb|OFFLINE|が偽のときのみ有効。 \\ \hline
 \verb|ALBG| & 論理値 & .false. & 安全な同期通信を行うかどうか。\verb|USE_NESTING|が真, \verb|OFFLINE|が偽のときのみ有効。 \\ \hline
 \verb|EPSR| & 論理値 & .false. & 安全な同期通信を行うかどうか。\verb|USE_NESTING|が真, \verb|OFFLINE|が偽のときのみ有効。 \\ \hline
 \verb|EPSB| & 論理値 & .false. & 安全な同期通信を行うかどうか。\verb|USE_NESTING|が真, \verb|OFFLINE|が偽のときのみ有効。 \\ \hline
 \verb|EPSG| & 論理値 & .false. & 安全な同期通信を行うかどうか。\verb|USE_NESTING|が真, \verb|OFFLINE|が偽のときのみ有効。 \\ \hline
 \verb|Z0R| & 論理値 & .false. & 安全な同期通信を行うかどうか。\verb|USE_NESTING|が真, \verb|OFFLINE|が偽のときのみ有効。 \\ \hline
 \verb|Z0B| & 論理値 & .false. & 安全な同期通信を行うかどうか。\verb|USE_NESTING|が真, \verb|OFFLINE|が偽のときのみ有効。 \\ \hline
 \verb|Z0G| & 論理値 & .false. & 安全な同期通信を行うかどうか。\verb|USE_NESTING|が真, \verb|OFFLINE|が偽のときのみ有効。 \\ \hline
 \verb|TRLEND| & 論理値 & .false. & 安全な同期通信を行うかどうか。\verb|USE_NESTING|が真, \verb|OFFLINE|が偽のときのみ有効。 \\ \hline
 \verb|TBLEND| & 論理値 & .false. & 安全な同期通信を行うかどうか。\verb|USE_NESTING|が真, \verb|OFFLINE|が偽のときのみ有効。 \\ \hline
 \verb|TGLEND| & 論理値 & .false. & 安全な同期通信を行うかどうか。\verb|USE_NESTING|が真, \verb|OFFLINE|が偽のときのみ有効。 \\ \hline
 \verb|BOUND| & 論理値 & .false. & 安全な同期通信を行うかどうか。\verb|USE_NESTING|が真, \verb|OFFLINE|が偽のときのみ有効。 \\ \hline
\end{tabularx}


%\subsubsection{PARAM\_ATMOS\_PHY\_TB\_MYNN}
%\begin{tabularx}{150mm}{|l|c|c|X|} \hline
% \rowcolor[gray]{0.9} 名称 & 種類 & 初期値 & 説明 \\ \hline
% \verb|ATMOS_PHY_TB_MYNN_TKE_INIT| & 論理値 & .false. & 安全な同期通信を行うかどうか。\verb|USE_NESTING|が真, \verb|OFFLINE|が偽のときのみ有効。 \\ \hline
% \verb|ATMOS_PHY_TB_MYNN_KMAX_PBL| & 論理値 & .false. & 安全な同期通信を行うかどうか。\verb|USE_NESTING|が真, \verb|OFFLINE|が偽のときのみ有効。 \\ \hline
% \verb|ATMOS_PHY_TB_MYNN_NU_MIN| & 論理値 & .false. & 安全な同期通信を行うかどうか。\verb|USE_NESTING|が真, \verb|OFFLINE|が偽のときのみ有効。 \\ \hline
% \verb|ATMOS_PHY_TB_MYNN_KH_MIN| & 論理値 & .false. & 安全な同期通信を行うかどうか。\verb|USE_NESTING|が真, \verb|OFFLINE|が偽のときのみ有効。 \\ \hline
%\end{tabularx}

%&PARAM_ATMOS_VARS
% ATMOS_VARS_CHECKRANGE = .false.,
%/
%
%&PARAM_ATMOS_REFSTATE
% ATMOS_REFSTATE_TYPE        = "INIT",
% ATMOS_REFSTATE_UPDATE_FLAG = .true.,
% ATMOS_REFSTATE_UPDATE_DT   = 10800.D0,
%/
%
%&PARAM_ATMOS_BOUNDARY
% ATMOS_BOUNDARY_TYPE        = "REAL",
% ATMOS_BOUNDARY_IN_BASENAME = "boundary_d01",
% ATMOS_BOUNDARY_UPDATE_DT   = 21600.D0,
% ATMOS_BOUNDARY_USE_VELZ    = .true.,
% ATMOS_BOUNDARY_USE_QHYD    = .false.,
% ATMOS_BOUNDARY_VALUE_VELZ  = 0.0D0,
% ATMOS_BOUNDARY_LINEAR_H    = .false.,
% ATMOS_BOUNDARY_EXP_H       = 2.d0,
%/
%
%&PARAM_ATMOS_DYN
% ATMOS_DYN_NUMERICAL_DIFF_COEF   = 1.D-2,
% ATMOS_DYN_NUMERICAL_DIFF_COEF_Q = 1.D-2,
% ATMOS_DYN_enable_coriolis       = .true.,
%/
%
%&PARAM_ATMOS_PHY_RD_MSTRN
% ATMOS_PHY_RD_MSTRN_KADD                  = 30,
% ATMOS_PHY_RD_MSTRN_GASPARA_IN_FILENAME   = "PARAG.29",
% ATMOS_PHY_RD_MSTRN_AEROPARA_IN_FILENAME  = "PARAPC.29",
% ATMOS_PHY_RD_MSTRN_HYGROPARA_IN_FILENAME = "VARDATA.RM29",
% ATMOS_PHY_RD_MSTRN_NBAND                 = 29,
%/
%
%&PARAM_ATMOS_PHY_RD_PROFILE
% ATMOS_PHY_RD_PROFILE_TOA                   = 100.D0,
% ATMOS_PHY_RD_PROFILE_CIRA86_IN_FILENAME    = "cira.nc",
% ATMOS_PHY_RD_PROFILE_MIPAS2001_IN_BASENAME = "MIPAS",
%/
%
%#################################################
%#
%# model configuration: ocean
%#
%#################################################
%
%&PARAM_OCEAN_VARS
% OCEAN_VARS_CHECKRANGE = .false.,
%/
%
%&PARAM_OCEAN_PHY_SLAB
% OCEAN_PHY_SLAB_DEPTH = 10.D0,
%/
%
%#################################################
%#
%# model configuration: land
%#
%#################################################
%
%&PARAM_LAND_VARS
% LAND_VARS_CHECKRANGE = .false.,
%/
%
%&PARAM_LAND_PHY_SLAB
% LAND_PHY_UPDATE_BOTTOM_TEMP  = .false.,
% LAND_PHY_UPDATE_BOTTOM_WATER = .true.,
%/
%
%#################################################
%#
%# model configuration: urban
%#
%#################################################
%
%&PARAM_URBAN_VARS
% URBAN_VARS_CHECKRANGE = .false.,
%/
%
%&PARAM_URBAN_PHY_SLC
% ZR         = 15.0D0,
% roof_width = 7.5D0,
% road_width = 22.5D0,
% AH         = 0.0D0,
% ALH        = 0.0D0,
% STRGR      = 0.24D0,
% STRGB      = 0.009D0,
% STRGG      = 0.24D0,
% AKSR       = 2.28D0,
% AKSB       = 2.28D0,
% AKSG       = 2.28D0,
% ALBR       = 0.20D0,
% ALBB       = 0.20D0,
% ALBG       = 0.20D0,
% EPSR       = 0.97D0,
% EPSB       = 0.97D0,
% EPSG       = 0.97D0,
% Z0R        = 0.005D0,
% Z0B        = 0.005D0,
% Z0G        = 0.005D0,
% CAPR       = 2.01D6,
% CAPB       = 2.01D6,
% CAPG       = 2.01D6,
%/
