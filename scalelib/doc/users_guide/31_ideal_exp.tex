
理想実験は
\begin{verbatim}
  scale/uguide/scale/scale-les/test/case/
\end{verbatim}
にいくつか用意されている。
\ref{sec:req_env}の実行環境で動かせる理想実験は、
今日時点ではないため、使い方のみ説明する。


\subsubsection{コンパイル}
実験したい実験のディレクトリに移動して、コンパイルする.
\begin{verbatim}
  $ cd scale/uguide/scale/scale-les/test/case/HOGEHOGE/
  $ make -j 4
\end{verbatim}

\subsubsection{実行方法}
理想実験は,コンパイルが完了したディレクトリ上で,下記のコマンドによって
実行することができる.
\begin{verbatim}
  $ make run
\end{verbatim}
この場合,デフォルトのconfigurationファイルを使用して自動的に実験設定に
合った初期値・境界値を作成し,その後scale-lesモデル本体を実行する.
"make run"のコマンドを使用せず,以下のように手動で実行過程を進めることもできる.

\begin{enumerate}
\item 実験設定を記述したconfigurationファイル,init.confを編集して目的の実験にあった設定を構築する.

\item 下記のコマンドによって事前処理(初期値・境界値作成)を実行する.ここの例ではMPI並列として6プロセスを使用している.
\begin{verbatim}
$ mpirun -n 6 ./scale-les_init init.conf
\end{verbatim}
正常にJOBが終了すれば,
\verb|init_****.pe#####.nc|,および\verb|boundary.pe#####.nc|
といったファイルが,それぞれMPIプロセス数ずつ生成される(\verb|#####|はMPIプロセスの番号).

\item モデルを実行するためにrun.confを適宜編集する.MPIプロセスの数や,格子,境界の取り方の設定については,init.confの時の設定と相違ないように注意すること.
正常にJOBが終了すれば,\verb|init_****.pe#####.nc|,
および\verb|boundary.pe#####.nc|といったファイルが,それぞれMPIプロセス数ずつ生成される(\verb|#####|はMPIプロセスの番号).

\item 下記のコマンドによってモデルを実行する.
\begin{verbatim}
$ mpirun -n 6 ./scale-les run.conf
\end{verbatim}
configurationファイルの設定によるが,\verb|history.pe#####.nc|
という名前のファイルが作成され,この中に出力変数が含まれている.
\end{enumerate}

