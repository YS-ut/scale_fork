\subsection{Setting of output variables to history file}
出力ファイルへの変数の追加は、2段階の手続きが必要である。
\begin{enumerate}
\item ソースファイル内の設定。対象の変数をhistory出力するための設定で、この手続きにより、出力のための準備が行われる。
\item run.conf内の設定。実際にhistoryに出力するかどうかを指定。コンパイルし直すことなく、実験毎に変更可能。
\end{enumerate}
予報変数や主要な変数についてはすでに1の手続きは行われているため、2だけを行えばよい。
対象となる変数リストは、Appendix\ref{sec:vari_hist}を参照。
これ以外の変数をユーザーが書き出したい場合は、1の手続きが必要であるが、ここでは2の手続きのみ説明する。


出力ファイルへの変数の追加は、下記のフォーマットに従ってrun.confに追記すればよい。
%\begin{screen}
\begin{verbatim}
 &HISTITEM ITEM     = "character"
         (,BASENAME = "character",
           TINTERVAL= real,
           TUNIT    = "character", 
           TAVERAGE = logical,
           ZINTERP  = logical, 
           DATATYPE = "character")
 /
\end{verbatim}
%\end{screen}
$\left( \right)$内はオプション。
特に指定がない場合には、\verb|&PARAM_HISTORY|の設定がdefaultとして設定される。
%\begin{screen}
\begin{verbatim}
 &PARAM_HISTORY
  HISTORY_DEFAULT_BASENAME  = "character",
  HISTORY_DEFAULT_TINTERVAL = real,
  HISTORY_DEFAULT_TUNIT     = "character",
  HISTORY_DEFAULT_TAVERAGE  = logical,
  HISTORY_DEFAULT_ZINTERP   = logical,
  HISTORY_DEFAULT_DATATYPE  = "character",
  HISTORY_OUTPUT_STEP0      = logical,
 /
\end{verbatim}
%\end{screen}
各namelist内変数の説明は以下の通りである。\\
%\begin{table}[h]
{\renewcommand\arraystretch{1.2}
\begin{tabular}{ll}
\hline
ITEM                   & 変数名。 Appendix\ref{sec:vari_hist}の\ref{tb:vari_hist}を参照。\\                       
BASENEME               & 出力ファイル名。\\
                       & BASENAME\_xxxxxx.ncとなる。xxxxxxはノード番号。\\
TINTERVAL              & 出力間隔。\\
TUNIT                  & TINTERVALで指定した出力間隔の単位。\\
TAVERAGE               & .false.=瞬間値、.true.=平均値として出力。\\
                       & 平均値の場合、出力タイミングの直前のTINTERVAL間の平均値。\\
DATATYPE               & 出力値の型。''REAL4'',''REAL8''など。\\
ZINTERP                & .false.=モデル面、.true.=Z面(FZ面$?$、CZ面$?$)の値として出力。\\
HISTORY\_OUTPUT\_STEP0 & .false.=初期時刻(t=0)の値を出力、.true.=出力しない。\\
                       & \verb|&PARAM_HISTORY|でのみ有効。\\
\hline
\end{tabular}
}\\
%\end{table}

\vspace{1cm}
\noindent {\Large\em Example}
\begin{verbatim}
 &PARAM_HISTORY
  HISTORY_DEFAULT_BASENAME  = "history_d03",
  HISTORY_DEFAULT_TINTERVAL = 3600.D0,
  HISTORY_DEFAULT_TUNIT     = "SEC",
  HISTORY_DEFAULT_TAVERAGE  = .false.,
  HISTORY_DEFAULT_DATATYPE  = "REAL4",
  HISTORY_DEFAULT_ZINTERP   = .false.,
  HISTORY_OUTPUT_STEP0      = .true.,
 /

 &HISTITEM item="RAIN", taverage=.true., tinterval=600.D0 /
\end{verbatim}




