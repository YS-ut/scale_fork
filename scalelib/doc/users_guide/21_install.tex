本章では,SCALE-LESの使用方法と、
SCALEチュートリアルを実行するための環境の準備について説明する。
なお、本ユーザーズガイド内の説明で、コマンドラインのシンボル(\verb|$|)があれば、コマンドの実行を示す.

\section{必要なシステム環境}
\label{sec:req_env}
%====================================================================================
ここでは、SCALEのインストール、および実行に必要な環境について説明する。

{\bf 必須のシステム環境}
\begin{itemize}
  \item {\bf システム構成}
\begin{itemize}
  \item {\bf CPU} : 理想実験は物理コアが2コア以上、現実大気実験は4コア以上を搭載したシステムが望ましい。
  \item {\bf メモリ} : 理想実験は512MB以上、現実大気実験は2GB以上のメモリ容量が搭載されたシステムが必要となる。
  \item {\bf HDD} : 実験内容によって変化するがTutorialの現実大気実験を実行するには約3GBのディスク空き容量が必要となる。
\end{itemize}

  \item {\bf OS} : Linux OS(Fedora, CentOS, SUSE等),Mac OS X.
        チュートリアルでは、Linux (CentOS6)を使用して説明する.
  \item {\bf コンパイラ} : Fortran 2003をサポートするC,Fortranコンパイラを必要とする.
        GNU 4.6.x以上,Intel compiler 2012以上を推奨する.
        チュートリアルでは,gcc/gfortranを使用して説明する.
  \item {\bf MPIライブラリ} : MPICH2, OpenMPI, Intel MPI等をサポートする.
        チュートリアルではopenMPIを使用して説明する.
  \item {\bf netcdf3 もしくは HDF5/netcdf4} : gzip, szipをサポートするHDF5,
        およびそのHDF5をサポートするnetcdf4を必要とする.
        ただし,netcdf3の環境下ではscaleライブラリが提供する全ての機能をサポートできない可能性がある.
  \item {\bf 描画環境(非必須)} : Dennou Club提供のRuby DCL/GPhysに含まれるgpviewがあると
        計算結果を簡単にチェックできる.チュートリアルではgpviewを使用する.
        それ以外に,netcdfからGrADS用にフォーマットを変換するためのpostprocess(\verb|netcdf2grads_h|)も用意している.
  \item SCALEは演算性能評価のためにPAPIライブラリを使用が可能.
        PAPIライブラリがインストールされている環境下では,
        以下で説明するconfigureファイルの編集によってPAPIを適用することができます.
\end{itemize}



\section{ライブラリ環境のインストール}
%====================================================================================
実行環境に合わせて、SCALEライブラリのインストールに必要な環境、ライブラリのインストールを行う。詳細は、Appendix \ref{sec:env_setting}を参照。
\ref{sec:source_code}以降のチュートリアルは,
それらのライブラリ環境がインストールされていることを想定して進める.



\section{Building the source code} \label{sec:source_code}
%====================================================================================


以下の説明で使用した環境は次のとおりである。
\begin{itemize}
\item CPU: Intel Core i5 2410M (sandybridge)
\item Memory: DDR3-1333 4GB
\item OS: CentOS 7.1 (kernel: 3.10.0-229.7.2.el7.x86\_64)\\
{\small *インストール時、"日本語"、 "Gnome デスクトップ"、"Kdump有り"を選択}
\end{itemize}

\subsubsection{ソースコードの入手}
%-----------------------------------------------------------------------------------
安定版ソースコードは,\\
 \url{http://scale.aics.riken.jp/ja/download/index.html}\\
よりダウンロードできる.
ソースコードのtarballファイルを展開すると
\verb|scale/|というディレクトリができる.
\begin{verbatim}
 $ tar -zxvf SCALE_v020.tar.gz
 $ ls
  scale/
\end{verbatim}



\subsubsection{configure ファイルと環境変数の設定}
%-----------------------------------------------------------------------------------

\verb|scale/sysdep/|内にいくつかのコンフィグファイル(\verb|Makedef.***|)が準備されている.
これらの中から自分の環境にあったものを設定する.
チュートリアルでは,OSはLinux,コンパイラはgcc/gfortran,およびopenMPIを使用するため,
\verb|"Makedef.Linux64-gnu-ompi"|が対応するファイルとなる.
常にこのコンフィグファイルを使用するために、
\verb|Makedef.***|の\verb|"***"|の部分を、\verb|SCALE_SYS|という環境変数として設定し、
\verb|.bashrc|などのファイルに記述しておくと便利である.
\begin{verbatim}
 $ export SCALE_SYS="Linux64-gnu-ompi"
\end{verbatim}
自分の環境に合うものがなければ既存ファイルをベースにして作成する.

さらに、SCALEをコンパイルするのに必要な外部ライブラリについても
下記のようにPATHを設定する.
ここでは,Appendix \ref{sec:env_setting}(Linux編)に従ったとして,
HDF5,netcdf4ともに\verb|/usr|の下にインストールされている場合の例を示す.
\begin{verbatim}
 $ export HDF5="/usr"
 $ export NETCDF4="/usr"
\end{verbatim}


\subsubsection{コンパイル}
%-----------------------------------------------------------------------------------
コンパイルは、各テストケースのディレクトリ内で実行可能である.
ここではコンパイルに必要な環境が整っているかどうかの確認だけ行い、
詳細な説明は、次章以降に譲る.

チュートリアル用のディレクトリに移動して,makeコマンドによってコンパイルを行う.
\begin{verbatim}
 $ cd scale/scale-les/test/tutorial/bin
 $ make -j 4
\end{verbatim}
\verb|make|のあとの \verb|"-j 4"| は,
コンパイル時の並列数を示しており,
例では4並列コンパイルを行うことを指示している.
実行環境によっては並列数を増やすこともできる.
このmakeによってSCALEライブラリ,およびSCALE-LESモデルのコンパイルが行われ,
結果として
\begin{verbatim}
 scale-les  scale-les_init  scale-les_pp
\end{verbatim}
の3つの実行ファイルが生成されていればコンパイルは成功である.\\


{\bf 注意点}
\begin{itemize}
\item SCALEライブラリは,scaleのTOPディレクトリ直下の
 \verb|scale/scalelib/|というディレクトリ内でコンパイルとアーカイブが行われ,
 \verb|".lib"|という名前の隠しディレクトリとして
 \verb|bin/|ディレクトリ内へコピーされる.
\item Debugモードでコンパイルしたい場合や,
 コンパイルオプションを変更したい場合は,
 \verb|Makedef.***|のファイルを編集する.
\item 開発版ソースコードをコンパイルしている場合,
 一部のコンパイラバージョンにおいて、コンパイルが正常に終了しないケースがある.
 そのような場合はSCALE開発チームまでご報告ください.
\end{itemize}


%####################################################################################

