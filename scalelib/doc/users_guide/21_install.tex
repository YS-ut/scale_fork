本章では,SCALE-LESの使用方法と、
SCALEチュートリアルを実行するための環境の準備について説明する。

\section{SCALEのインストール}
%####################################################################################

\subsection{Required Environment}
\label{sec:req_env}
%====================================================================================
SCALEの実行環境は下記を想定している。

\begin{itemize}
  \item {\bf 計算機環境} : Unix互換OS (Mac OS Xを含む)が動作する環境.
        マルチコアCPU環境以上を推奨する.
        実験サイズによるが、最低4GB以上のメモリが
        インストールされているマシン環境が好ましい.
  \item {\bf OS} : Linux OS(Fedora, CentOS, SUSE等),Mac OS X.
        チュートリアルでは、Linux (CentOS6)を使用して説明する.
  \item {\bf コンパイラ} : Fortran 2003をサポートするC,Fortranコンパイラを必要とする.
        GNU 4.6.x以上,Intel compiler 2012以上を推奨する.
        チュートリアルでは,gcc/gfortranを使用して説明する.
  \item {\bf MPIライブラリ} : MPICH2, OpenMPI, Intel MPI等をサポートする.
        チュートリアルではopenMPIを使用して説明する.
  \item {\bf netcdf3 もしくは HDF5/netcdf4} : gzip, szipをサポートするHDF5,
        およびそのHDF5をサポートするnetcdf4を必要とする.
        ただし,netcdf3の環境下ではscaleライブラリが提供する全ての機能をサポートできない可能性がある.
  \item {\bf 描画環境(非必須)} : Dennou Club提供のRuby DCL/GPhysに含まれるgpviewがあると
        計算結果を簡単にチェックできる.チュートリアルではgpviewを使用する.
        それ以外に,netcdfからGrADS用にフォーマットを変換するためのpostprocess(\verb|netcdf2grads_h|)も用意している.
  \item SCALEは演算性能評価のためにPAPIライブラリを使用が可能.
        PAPIライブラリがインストールされている環境下では,
        以下で説明するconfigureファイルの編集によってPAPIを適用することができます.
\end{itemize}



\subsection{ライブラリ環境のインストール}
%====================================================================================
SCALEライブラリのインストールに必要な環境、ライブラリのインストールを行う。
実行環境に合わせて、Appendix \ref{sec:env_setting}を参照してインストールを行う。
\ref{sec:source_code}以降のチュートリアルは,
それらのライブラリ環境がインストールされていることを想定して進める.



\subsection{Building the source code} \label{sec:source_code}
%====================================================================================

\subsubsection{ソースコードの入手}
%-----------------------------------------------------------------------------------

安定版ソースコードは,\url{http://scale.aics.riken.jp/ja/download/index.html}
よりダウンロードすることができる.
ソースコードのtarballファイルを展開すると
\begin{verbatim}
  scale/
\end{verbatim}
というディレクトリができる.


\subsubsection{configure ファイルと環境変数の設定}
%-----------------------------------------------------------------------------------

\verb|scale/sysdep/|内にいくつかのコンフィグファイル(\verb|Makedef.***|)が準備されている.
これらの中から自分の環境にあったものを設定する.
チュートリアルでは,OSはLinux,コンパイラはgcc/gfortran,およびopenMPIを使用するため,
\verb|"Makedef.Linux64-gnu-ompi"|が対応するファイルとなる.
自分の環境に合うものがなければ既存ファイルをベースにして作成する.

常にこのコンフィグファイルを使用するために、
\verb|Makedef.***|の\verb|"***"|の部分を、\verb|SCALE_SYS|という環境変数として設定し、
\verb|.bashrc|などのファイルに記述しておくと便利である.
さらに、SCALEをコンパイルするのに必要な外部ライブラリについても
下記のようにPATHを設定する.
ここでは,Appendix \ref{sec:env_setting}(Linux編)に従ったとして,
HDF5,netcdf4ともに\verb|/usr|の下にインストールされている場合の例を示す.

\begin{verbatim}
 $ export SCALE_SYS="Linux64-gnu-ompi"
 $ export HDF5="/usr"
 $ export NETCDF4="/usr"
\end{verbatim}


\subsubsection{コンパイル}
%-----------------------------------------------------------------------------------
コンパイルは、各テストケースのディレクトリ内で実行可能である.
ここではコンパイルに必要な環境が整っているかどうかの確認だけ行い、
詳細な説明は、次章以降に譲る.

チュートリアル用のディレクトリに移動して,makeコマンドによってコンパイルを行う.
\begin{verbatim}
 $ cd scale/scale-les/test/tutorial/bin
 $ make -j 4
\end{verbatim}
\verb|make|のあとの \verb|"-j 4"| は,
コンパイル時の並列数を示しており,
例では4並列コンパイルを行うことを指示している.
実行環境によっては並列数を増やすこともできる.
このmakeによってSCALEライブラリ,およびSCALE-LESモデルのコンパイルが行われ,
結果として
\begin{verbatim}
 scale-les  scale-les_init  scale-les_pp
\end{verbatim}
の3つの実行ファイルが生成されていればコンパイルは成功である.\\


{\bf 注意点}
\begin{itemize}
\item SCALEライブラリは,scaleのTOPディレクトリ直下の
 \verb|scale/scalelib/|というディレクトリ内でコンパイルとアーカイブが行われ,
 \verb|".lib"|という名前の隠しディレクトリとして
 \verb|bin/|ディレクトリ内へコピーされている.
\item Debugモードでコンパイルしたい場合や,
 コンパイルオプションを変更したい場合は,
 \verb|Makedef.***|のファイルを編集してください.
\item 開発版ソースコードをコンパイルしている場合,
 一部のコンパイラバージョンにおいてコンパイルが正常に終了しないケースがあります.
 そのような場合はぜひSCALE開発チームまでご報告ください.
\end{itemize}


%####################################################################################

