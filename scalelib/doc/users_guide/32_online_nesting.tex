\section{実験設定の変更方法}
%\subsection{Domain setting}

\subsection{Online nesting}

現在,scale-lesでは,単一ドメインの他に、複数ドメインをもつNesting計算をサポートしている.
Nesting計算は1-way(親ドメインから娘ドメインへのデータ受け渡し)のみをサポートしている.

Online Nestingを行う場合は,Nestingの段数分だけpp.conf,init.conf,run.confファイルをそれぞれ用意する必要がある.
それぞれのドメイン毎に地形,土地利用,初期値・境界値を作成し,一番外側のドメインを下記のコマンドによって起動すると,
外側のドメインが順次,内側のドメインを起動する.
configureファイルの記述例は\verb|scale-les/test/case_real/kobe_prod|の下にあるので適宜参照して欲しい.
\begin{verbatim}
 $ mpirun -n 6 ./scale-les run.d01.conf
\end{verbatim}
このとき注意することは,Online Nesting計算の場合,実際に起動されるMPIプロセス数は外側のドメインを起動する時に指定したプロセス数(np)とドメイン段数を掛けた数になる.
たとえば,上記のコマンドで3段ドメインのOnline Nesting計算を行うことを考えると,total\_mpi\_processes = 3 (domain levels)×6 (np) = 36 processesとなる.
実行時のマシン環境に合わせて実験を行う必要がある.

