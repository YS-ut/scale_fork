\subsection{netcdf2grads}

SCALE-LESの出力ファイル(\verb|history.******.nc|)を
GrADSで図化するためのnetcdf2grads の使用方法について説明する。
並列処理に対応したnetcdf2grads\_Hも用意されているが、
これは最後に簡単に説明する。
ここではシングルノード用のnetcdf2gradsの使用方法を示す。
ソースファイルは \verb|scale/scale-les/util/netcdf2grads/|にある。

\subsubsection{コンパイル}

\begin{description}
\item[Intel compiler]\mbox{}\\
 \begin{verbatim}
  ifort -convert big_endian -assume byterecl
    -I${NETCDF4}/include -L${NETCDF4}/lib -lnetcdff -lnetcdf make_grads_file.f90 -o convine
  \end{verbatim}
\item[gfortran]\mbox{}\\
\begin{verbatim}
gfortran
\end{verbatim}
\end{description}

\subsubsection{使用方法}


\subsubsection{並列処理: netcdf2grads\_h}

スパコンなどの大型計算機で並列計算を行った場合、
出力ファイルの数が多く、それぞれのファイルのデータ容量も大きい。
netcdf2gradsを並列処理したい場合にはこちらを使用する。
ここでは、K上での使用方法を簡単に説明する。

\begin{enumerate}
\item ファイルをコピー\\
 \verb|scale/scale-les/util/netcdf2grads_h| を \verb|/data/GROUP/USER/WORK_directory/| など、作業したいディレクトリにコピー。
\item コンパイル\\
 \verb|.bashrc| などに下記を設定
 \begin{verbatim}
  #--SCALE
   export SCALE_SYS=''K''
   export AGGRESIVE=''F''
   export FAST=''T''
   export DEBUG=''F''
 \end{verbatim}
 そして、コンパイル.\\
 \verb|$ make|\\
 うまくいけば、\verb|net2g|が作成される。
\item \verb|microで実行する| \\
 \verb|/scratch/GROUP/USER/|の下に、作業ディレクトリを用意する。そこに、
 \begin{verbatim}
   net2g         : copy executive file
   net2g.conf    : copy configure file
   run.12hours/  : link to directory with scale history files
   bindata/      : create directory for grads file output
   job.sh        : job script for K (option)
 \end{verbatim}
 を用意する。\verb|net2g.conf|の設定と\verb|job.sh|の設定をする。
 使用するノード数は、計算に使用したノード数の約数である必要がある。
\end{enumerate}


