%%%%%%%%%%%%%%%%%%%%%%%%%%%%%%%%%%%%%%%%%%%%%%%%%%%%%%%%%%%%%%%%%%%%%%%%%%%%%%%%%%%%%%%%%%%%
\chapter{ライブラリ環境のインストール}
\label{sec:env_setting}
%%%%%%%%%%%%%%%%%%%%%%%%%%%%%%%%%%%%%%%%%%%%%%%%%%%%%%%%%%%%%%%%%%%%%%%%%%%%%%%%%%%%%%%%%%%%

ここでは、SCALEのインストールに必要なライブラリ環境のインストール方法について説明する。
本書で説明するライブラリ環境のインストールにあたっては、root権限が必要になる。したがって、
想定する環境は、ユーザーがroot権限を所持しているサーバーや、自前のデスクトップマシンといった環境である。
別途サーバー管理者が存在し、root権限を取得できない場合等は、必要な環境条件が整っているか問い合わせること。

また本節では、HDF5, NetCDF, MPIについてGNU compilerでコンパイルされたライブラリの説明を行う。
GNU compiler以外のIntel compilerなどを利用する場合は、各自でインストール方法を調べてインストールすること。\\

\noindent ここでインストールするライブラリ環境は、主に下記の4点である。
\begin{itemize}
\item GNU C/C++, fortran compiler
\item HDF5 Library (https://www.hdfgroup.org/HDF5/)
\item NetCDF Library (http://www.unidata.ucar.edu/software/netcdf/)
\item Message Passing Interface (MPI) Library (openMPI版、http://www.open-mpi.org/)
\end{itemize}
これらのインストール方法について、本書では下記の5種類のOperating System (OS)について説明する。
\begin{itemize}
\item Linux CentOS 6.6 x86-64
\item Linux CentOS 7.1 x86-64
\item Linux openSUSE 13.2 x86-64
\item Apple MacOSX \textcolor{red}{(version XXX)←誰かMacユーザーの人に埋めてもらう}
\item スーパーコンピュータ京 (言語環境: K-1.2.0-18)
\end{itemize}
他のOSディストリビューション(下記参照)でもSCALEを利用可能だが、
本書でサポートするのは上記の範囲とする。\\

\noindent{\bf 動作確認済みの他のOSディストリビューション}
\begin{itemize}
\item Linux SUSE Enterprise Linux 11.1, 11.3 x86-64
\item Linux Vine Linux 6.3 x86-64
\item Linux Fedora 16 x86-64
\end{itemize}


\section{インストールに関する基礎知識}
%==========================================================================================

Linuxをインストール後、各種プログラムのインストールにはコマンドライン端末を使う。
コマンドラインのシンボル(\verb|$, #|)は、コマンドの実行を示す。
下記の表記の違いは、プログラムを実行する権限の違いを示している。

\begin{verbatim}
 #        <- root権限で実行するコマンド
 $        <- ユーザ権限で実行するコマンド
\end{verbatim}
権限の一時的な切り替えにはsuコマンドを用いる。
\verb|{User_Name}|は実際のユーザ名に読み替えること。
\begin{verbatim}
 $ su {User_Name}   <- {User_Name}のユーザー名でログイン
 $ exit             <- {User_Name}のユーザー名でログインを修了
 $ su -             <- root権限に変更
 #
\end{verbatim}

コマンドオプションにハイフンを用いると、そのユーザでのログインを行う。
用いない場合、権限のみの変更となる。またユーザ名を省略するとrootでのログインを試す。
ユーザの一時切り替えを終わるには、exitコマンドを用いる。
各プログラムをインストールするための圧縮ファイルは、/tmpにダウンロードされていると仮定する。
他のディレクトリにダウンロードしてある場合は、mvコマンド等を用いて/tmpに移動しておくことを勧める。
コード表記のうち、ダブルスラッシュ(//)で始まる行は解説のためのもので、実際に記述する必要はない。


\begin{verbatim}
 $ vi
\end{verbatim}

上記のコマンドは、vi(汎用テキストエディタ)プログラムを実行する。
他に、gedit、emacsなどのテキストエディタがある。
マニュアルではviコマンドの実行指示があるが、各自の使いやすいエディタへ適宜読み替えること。


\section{インストール方法 (Linux - CentOS 6.6 編)}
%==========================================================================================

\subsubsection{ライブラリのインストール}

CentOS 6.6では、一部のライブラリを外部リポジトリからインストールする。
このため、まずepelリポジトリをシステムにインストールし登録する。
CentOS 6.6では、ソフトウェアのインストールに"yum"コマンドを利用する。
ルート権限になり、下記のコマンドを実行することでインストールが可能である。
\begin{verbatim}
 # yum install epel-release
\end{verbatim}
実行時のコマンドラインの様子は以下のようになる。
インストール対象がリストされるので、確認して"y"をタイプして先へ進める。
\begin{verbatim}
 ここに実行結果を貼り付け
\end{verbatim}
{\small *この時点で、yumによるインストールに失敗する場合は、
プロキシ設定等を含めた通信環境、yumリポジトリの登録状況等を再確認すること。}

\noindent yumのグループインストール機能を用いて,開発ツール(ここでの対象は主にGNU compilerとmakeシステム)をまとめてインストールする。
\begin{verbatim}
 # yum groupinstall "development tools"
\end{verbatim}

\noindent グループインストールではインストールされないライブラリを個別に追加する。
\begin{verbatim}
 # yum install hdf5-devel hdf5-static
 # yum install netcdf-devel netcdf-static
 # yum install openmpi-devel
 # yum install lapack lapack-devel
\end{verbatim}

\noindent {\small *GNU compilerでSCALEをコンパイルする際には、
デフォルトでLapackを利用する設定であるため、依存関係のあるライブラリとしてインストールが必要になる。}

\noindent {\small *"yum -y install package name"として実行することで、インストール前の再確認をスキップできる。}


\subsubsection{環境変数の設定}

ローカルシステムでMPI並列プログラムを実行するために、OpenMPIライブラリの環境変数設定を行う。
ユーザ権限に移動して.bashrcをエディタで開き,
\begin{verbatim}
 $ vi ~/.bashrc
\end{verbatim}
下記をファイルの最後に追加して,環境変数の設定を記述する。
\begin{verbatim}
 // ---------------- Add to end of the file ----------------
 # OpenMPI
 export MPI="/usr/lib64/openmpi"
 export PATH="$PATH:$MPI/bin"
 export LD_LIBRARY_PATH="$LD_LIBRARY_PATH:$MPI/lib"
\end{verbatim}
編集が終わったら、環境設定を有効にする。
\begin{verbatim}
 $ . ~/.bashrc
\end{verbatim}


\subsubsection{Installation of Gphys}

\textcolor{red}{\large 本節は削除予定;Gphysのインストールは書かない。
書くなら参考URLのみを挙げてGrads、ncview等と同じ取り扱いとする。}

%CentOSの場合、yumリポジトリに地球電脳倶楽部のGFD-Dennouリポジトリを登録することで、
%簡単にGphysをインストールできる。
%root権限で、GFD-Dennouリポジトリを次のような内容で登録する。

%\begin{verbatim}
% # vi /etc/yum.repos.d/GFD-Dennou.repo
%\end{verbatim}

%\begin{verbatim}
% // ---------------- Edit the file ----------------
% [gfd-dennou]
% name=GFD DENNOU Club RPMS for CentOS $releasever - $basearch
% baseurl=http://www.gfd-dennou.org/library/cc-env/rpm-dennou/CentOS/$releasever/$basearch/
% enabled=1
% gpgcheck=0
%\end{verbatim}
%編集が終わったら、yumでGphysをインストールする。
%\begin{verbatim}
% # yum install gphys
%\end{verbatim}


\section{インストール方法 (Linux - CentOS 7.1 編)}
%==========================================================================================

以下の説明で使用した環境は次のとおりである。
\begin{itemize}
\item CPU: Intel Core i5 2410M (sandybridge)
\item Memory: DDR3-1333 4GB
\item OS: CentOS 7.1 (kernel: 3.10.0-229.7.2.el7.x86\_64)\\
{\small *インストール時、"日本語"、 "Gnome デスクトップ"、"Kdump有り"を選択}
\end{itemize}

\subsubsection{ライブラリのインストール}

CentOS 7.1では、一部のライブラリを外部リポジトリからインストールする。
このため、まずepelリポジトリをシステムにインストールし登録する。
CentOS 7.1では、ソフトウェアのインストールに"yum"コマンドを利用する。
ルート権限になり、下記のコマンドを実行することでインストールが可能である。
\begin{verbatim}
 # yum install epel-release
\end{verbatim}
実行時のコマンドラインの様子は以下のようになる。
インストール対象がリストされるので、確認して"y"をタイプして先へ進める。
\begin{verbatim}
 ここに実行結果を貼り付け
\end{verbatim}
{\small *この時点で、yumによるインストールに失敗する場合は、
プロキシ設定等を含めた通信環境、yumリポジトリの登録状況等を再確認すること。}

\noindent yumのグループインストール機能を用いて,開発ツール(ここでの対象は主にGNU compilerとmakeシステム)をまとめてインストールする。
\begin{verbatim}
 # yum groupinstall "development tools"
\end{verbatim}

\noindent グループインストールではインストールされないライブラリを個別に追加する。
\begin{verbatim}
 # yum install hdf5-devel hdf5-static
 # yum install netcdf-devel netcdf-static
 # yum install netcdf-fortran-devel
 # yum install openmpi-devel
 # yum install lapack lapack-devel
\end{verbatim}
{\small *fortran用のモジュールファイルは別パッケージになっている。
"netcdf-fortran-devel"のインストールを忘れないこと。}

\noindent {\small *GNU compilerでSCALEをコンパイルする際には、
デフォルトでLapackを利用する設定であるため、依存関係のあるライブラリとしてインストールが必要になる。}

\noindent {\small *"yum -y install package name"として実行することで、インストール前の再確認をスキップできる。}

\subsubsection{環境変数の設定}

ローカルシステムでMPI並列プログラムを実行するために、OpenMPIライブラリの環境変数設定を行う。
ユーザ権限に移動して.bashrcをエディタで開き,
\begin{verbatim}
 $ vi ~/.bashrc
\end{verbatim}
下記をファイルの最後に追加して,環境変数の設定を記述する。
\begin{verbatim}
 // ---------------- Add to end of the file ----------------
 # OpenMPI
 export MPI="/usr/lib64/openmpi"
 export PATH="$PATH:$MPI/bin"
 export LD_LIBRARY_PATH="$LD_LIBRARY_PATH:$MPI/lib"
\end{verbatim}
編集が終わったら、環境設定を有効にする。
\begin{verbatim}
 $ . ~/.bashrc
\end{verbatim}


\section{インストール方法(MacOSX編)}
%==========================================================================================

\subsubsection{macportsを用いたインストール}

Apple MacOSXでのSCALE実行環境を整備する方法について説明する。
ここではMacOSXのパッケージマネージャの一つであるmacportsを用いる方法を紹介する。
その他の主要なパッケージマネージャとしては、homebrewが挙げられる。homebrewを利用しても環境は手軽に揃えられるので、
興味のある方は利用してもらいたい。

まずはAppleの開発ツールであるXcodeをインストールする。
大元のgccコンパイラを導入するために、必ずインストールする必要がある。
最近のOSのバージョンのものは、App Store経由で入手できる(無料)。
古いOSでは、インストールディスクから追加することが出来る。
最近のOSのXcodeの場合、最初に以下の様な設定をターミナルから行う必要がある。
\begin{verbatim}
 コマンドラインツールのインストール
 # xcode-select --install
\end{verbatim}
\begin{verbatim}
 ライセンス条項の承認
 # xcodebuild -license
\end{verbatim}

次にmacports本体をインストールする。
\url{https://www.macports.org/}

macportsとmacportsが管理するパッケージは/opt/local以下に配置される。
インストール時に\verb|.bash_profile|に、/opt/local/binへのパスが張られているので確認されたし。
macportsはコマンドラインから操作する。主要なコマンドは以下の通り。

\begin{verbatim}
 インストール可能なソフトウェアを検索する
 $ port search <検索文字>
\end{verbatim}
\begin{verbatim}
 ソフトウェアのインストール時に選択可能なオプション(variants)を確認する
 $ port variants <アプリ名>
\end{verbatim}
\begin{verbatim}
 ソフトウェアのインストール(root権限必要)
 $ sudo port install <アプリ名> [variants]
\end{verbatim}
\begin{verbatim}
 ソフトウェアのアンインストール(root権限必要)
 $ sudo port uninstall <アプリ名> [variants]
\end{verbatim}
\begin{verbatim}
 macports本体とパッケージカタログの更新(root権限必要)
 $ sudo port selfupdate
\end{verbatim}
\begin{verbatim}
 パッケージの更新(root権限必要)
 $ sudo port upgrade outdated
\end{verbatim}
\begin{verbatim}
 不要なパッケージ(activateされていない過去のバージョン等)の削除
 $ sudo port -u uninstall
\end{verbatim}

\subsubsection{gccからNetCDFまでのインストール}

macportsはパッケージの依存関係を解決してくれるが、必要なvariantsを備えたセットを作るには、
順番にインストールしていく方が問題が少ない。以下にsudo port installしていく順番とvariantsの設定を示す。
今回はgcc4.9の利用を想定している。
\begin{verbatim}
 $ gcc49
 $ openmpi-gcc49 +threads
 $ hdf4 +gcc49 +szip
 $ hdf5 +gcc49 +szip +fortran +cxx +openmpi +threadsafe
 $ netcdf +gcc49 +openmpi +netcdf4 +hdf4
 $ netcdf-fortran +gcc49 +openmpi
\end{verbatim}

最近のmacportsでは、gccとmpiライブラリはselectで選択する。
この操作を行うと、gfortran等の一般的な名前でエイリアスが作られてパスが通るようになる。
\begin{verbatim}
 $ sudo port select --set gcc mp-gcc49
 $ sudo port select --set mpi openmpi-gcc49-fortran
\end{verbatim}

SCALEは陰解法計算の部分で、数値計算ライブラリを利用するオプションがある。
もし必要ならば、macportsからATLASをインストールすることが出来る。
\begin{verbatim}
 $ atlas +gcc49
\end{verbatim}

\subsubsection{rubyからGphysまでのインストール}

\textcolor{red}{\large 本節は削除予定;Gphysのインストールは書かない。
書くなら参考URLのみを挙げてGrads、ncview等と同じ取り扱いとする。
ただし、ここに挙げているインストール方法がGphys本家のインストール方法と異なる場合は残す。}

macports-JPが作成しているリポジトリから、MacOSX用のGPhysをインストールできる。
以下のように、MacPorts-JPのリポジトリをチェックアウトして配置する。場所は好きな場所でよい。
\begin{verbatim}
 $ sudo svn co http://svn.sourceforge.jp/svnroot/macports-jp/trunk/ \
 $ /opt/local/var/sources/svn.sourceforge.jp/svnroot/macports-jp/trunk
\end{verbatim}
/opt/local/etc/macports/sources.confに、チェックアウトしたリポジトリを追加する。
\begin{verbatim}
 file:///opt/local/var/sources/svn.sourceforge.jp/svnroot/macports-jp/trunk/dports
\end{verbatim}
これで、Gphysと関連ライブラリをmacportsで一括インストールできるようになる。
今回はruby1.9向けのrb19-gphysをインストールすることにする。

\begin{verbatim}
 rubyのインストールと1.9の選択
 $ sudo port install ruby19
 $ sudo port select --set ruby ruby19
\end{verbatim}
\begin{verbatim}
 依存関係の解決を任せるかたちで、gphysのインストール
 $ sudo port install rb19-gphys
\end{verbatim}

インストールされたruby1.9版gphysコマンドは、末尾に1.9が付いている。
これを無くしたい場合は/opt/local/bin内でシンボリックリンクを張る。
以下では特に頻繁に使う3つを例に挙げる。
\begin{verbatim}
 $ sudo ln -s gplist1.9 gplist
 $ sudo ln -s gpvect1.9 gpvect
 $ sudo ln -s gpview1.9 gpview
\end{verbatim}

\subsubsection{その他のインストール}

\textcolor{red}{\large 本節は部分的に削除予定;UsersGuideの中で使用していないものや、
単純なユーザーにとって必要のないものには触れないようにする。}

macportsでインストールできるパッケージのうち、重要なものを以下に紹介する。
%
%\begin{itemize}
%\item git, git-flow : ソースコード開発時に利用する。
%\item coreutils : GNU core utils。様々な場面で必要になる。
%\item gawk, gsed : MacOSXに付属するawk,sedはGNU版と機能が異なるため、こちらが必要になる時がある。
%\item gnuplot : グラフ描画に利用。
%\item ImageMagick : Gphysから出力したPostScriptファイルをpngやアニメーションgifに変換する時に利用。
%\end{itemize}


\section{描画ツールのインストール}
%==========================================================================================

SCALEの計算結果や、初期値/境界値データなどを描画するのに利用可能である描画ツールの例を挙げる。
個人の好みでどのツールを使ってもよいし、出力形式を理解していれば、
ここに挙げた以外のツールで解析・描画することももちろん可能である。

\begin{itemize}
\item Gphys / Ruby-DCL by 地球電脳倶楽部\\
 \begin{itemize}
  \item URL: \verb|http://ruby.gfd-dennou.org/products/gphys/|
  \item 概略:SCALEの出力ファイルは、MPI並列の計算領域分割に従ってMPIプロセスごとに
              NetCDF形式の分割ファイルとして出力される。Gphysの"gpview"や"gpvect"といった
              描画ツールを使えば、分割ファイルを後処理なしに直接開いて描画することができる。
  \item インストール方法:本書で使用したCentOS6、CentOS7については、下記のWebページにインストール方法が記載されている。\\
                         \verb|http://www.gfd-dennou.org/arch/davis/gfdnavi/doc/install/install-ruby-gphys.htm|
 \end{itemize}
\item Grid Analysis and Display System (GrADS) by COLA\\
 \begin{itemize}
  \item URL: \verb|http://iges.org/grads/|
  \item 概略:言わずと知れた描画ツール。SCALEのNetCDF形式の分割ファイルをそのまま読むことはできない
             ため、SCALEで提供している出力データの後処理ツール"\verb|netcdf2grads_h|"を使用して分割ファイルを結合し、
             GrADSで読み込めるファイル形式に変換する必要ある。"\verb|netcdf2grads_h|"の使用方法は本書の第3章、
             および第4章を参照のこと。
  \item インストール方法:\verb|http://iges.org/grads/downloads.html|を参照のこと。
 \end{itemize}
\item Ncview: a netCDF visual browser by David W. Pierce\\
 \begin{itemize}
  \item URL: \verb|http://meteora.ucsd.edu/~pierce/ncview_home_page.html|
  \item 概略:NetCDF形式ファイルのクイックビューアーである。SCALEの分割ファイルを結合して描画することは
             できないが、分割ファイルを1つずつ描画してチェックすることはできる。
  \item インストール方法:\verb|http://meteora.ucsd.edu/~pierce/ncview_home_page.html|を参照のこと。
                        また、CentOS6、CentOS7などでepelリポジトリを登録していれば、yumコマンドによって
                        インストールできる。
 \end{itemize}
\end{itemize}





