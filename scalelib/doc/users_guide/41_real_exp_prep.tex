
このチュートリアルでは,Fig. \ref{fig:domain}に示した日本域を対象とした
現実大気実験を行う.
計算領域(ドメイン)の設定はTable \ref{tab:grids}のようになっている.

\begin{figure}[h]
\begin{center}
  \includegraphics[width=0.5\hsize]{./figure/domain.eps}\\
  \caption{計算領域.コンターは海岸線,カラーシェードは地形の高度を示す.}
  \label{fig:domain}
\end{center}
\end{figure}

\begin{table}[h]
\begin{center}
  \caption{実験設定の概略}
  \label{tab:grids}
  \begin{tabularx}{150mm}{|l|X|} \hline
    \rowcolor[gray]{0.9} 項目 & 設定 \\ \hline
    MPIプロセス分割 (東西 x 南北) & 3 x 3 (合計9プロセス) \\ \hline
    水平格子数 (東西 x 南北) & 180格子点 x 180格子点 \\ \hline
    鉛直層数                 & 36層                  \\ \hline
    水平格子間隔             & dx = dy = 7500m       \\ \hline
    積分期間 & 1999年5月5日 00UTC~12UTC (12時間積分) \\ \hline
    時間ステップ間隔 & 30 sec (1440 steps) \\ \hline
  \end{tabularx}
\end{center}
\end{table}

%-------------------------------------------------------%
\section{境界データの入手: AICS内部用、最終版では削ります}
%-------------------------------------------------------%

現実大気実験のシミュレーションを行う場合,SCALE本体に加えて境界値データが必要になる.
本チュートリアル用の気象場のデータ,日本領域の地形・土地利用のデータを\\
 \url{http://scale.aics.riken.jp/download/tutorial_data.tar.gz}\\
より入手し,チュートリアルの入力ファイル用ディレクトリ
\begin{verbatim}
  scale/scale-les/test/tutorial/data/
\end{verbatim}
の下に展開しておく.

以降の説明で\verb|${TOPDIR}|は,\verb|scale/scale-les/test/tutorial/|がある絶対PATHを指す.

\begin{verbatim}
  ${TOPDIR}/data/tutorial_data/input_atom/    <- 気象場データ
  ${TOPDIR}/data/tutorial_data/input_topo/    <- 地形データ
  ${TOPDIR}/data/tutorial_data/input_landuse/ <- 土地利用データ
\end{verbatim}
\verb|tutorial_data/|には,本チュートリアルに必要な最低限のデータのみが納めされているため,
その他の設定で実験を行う場合には別途,気象場,地形,および土地利用データが必要となる.


%-------------------------------------------------------%
\section{境界データの準備: 一般ユーザー用、公開時タイトル注意}
%-------------------------------------------------------%

現実大気実験のシミュレーションを行う場合,SCALE本体に加えて境界値データが必要になる。
境界値データとしては下記が必要である。
\begin{itemize}
\item 標高データ
\item 土地利用データ
\item 大気・地表面データ
\end{itemize}

ここでは、ユーザーが全球の任意の地域を対象とした計算できるよう、
標高データはUSGS(U.S. Geological Survey) のGTOPO30、
土地利用データはGLCCv2、
大気・地表面データはFinal Analysis の使い方を示す.


\subsubsection{地形データ: GTOPO30}

USGSのサイト\\
 \url{https://lta.cr.usgs.gov/GTOPO30}\\
からGTOPO30のデータをダウンロードする。
ダウンロードにはregistrationが必要である。

%以降の説明で\verb|${TOPDIR}|は,\verb|scale/scale-les/test/tutorial/|がある絶対PATHを指す.

つづく。


