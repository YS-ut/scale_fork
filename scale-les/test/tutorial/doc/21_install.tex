本章では、SCALE-LESの使用方法と、SCALEチュートリアルを実行するための環境の準備について説明する。

\section{必要なシステム環境}
\label{sec:req_env}
%====================================================================================
ここでは、SCALEのインストール、および実行に必要な環境について説明する。
以下に説明するもの以外に、SCALEは、理化学研究所 計算科学研究機構のスーパーコンピュータ「京」、
及びFujitsu PRIME-HPC FX10の環境に対応している。

\noindent {\bf 必須のシステム環境}
\begin{itemize}
  \item {\bf システムハードウェア構成}
  \begin{itemize}
    \item {\bf CPU} : 理想実験は物理コアが2コア以上、現実大気実験は4コア以上を搭載したシステムが望ましい。
    \item {\bf Memory} : 理想実験は512MB以上、現実大気実験は2GB以上のメモリ容量が搭載されたシステムが必要となる。
    \item {\bf HDD} : 実験内容によって変化するが、Tutorialの現実大気実験を実行するには約3GBのディスク空き容量が必要となる。
  \end{itemize}

  \item {\bf システムソフトウェア構成}
  \begin{itemize}
  \item {\bf OS} : Linux OS、Mac OS X\\
        対応確認済みOSについては、表\ref{tab:compatible_os}を参照のこと。
  \item {\bf コンパイラ} : Cコンパイラ、Fortran\\
        FortranコンパイラはFortran2003をサポートするコンパイラを必要とする。
        対応確認済みコンパイラについては、表\ref{tab:compatible_compiler}を参照のこと。
  \item {\bf MPIライブラリ} : MPI1.0/2.0 に対応するMPIライブラリを必要とする。
        対応確認済みMPIライブラリについては、表\ref{tab:compatible_mpi}を参照のこと。
  \item {\bf ファイルI/Oライブラリ} : gzip、HDF5、NetCDF4を必要とする。\\
        HDF5/NetCDF4の代わりに、NetCDF3でも動作するが、この場合はSCALEが提供する全ての機能が有効にならない可能性がある。
  \end{itemize}
\end{itemize}


\noindent {\bf あると便利なシステム環境}
\begin{itemize}
  \item {\bf データ変換ツール}:wgrib、wgrib2やNCL等があれば、SCALEで直接サポートしないデータを外部入力データとして利用できる。現実大気実験のTutorialでは、実際にwgrib2を利用する。
  \item {\bf 描画環境}:GrADS、GPhys/Ruby-DCL、ncviewなどがあれば計算結果を描画することができる。
  \item {\bf 演算性能評価}:演算性能評価としてPAPIライブラリを使用が可能である。
\end{itemize}


\begin{table}[htb]
\begin{center}
\caption{対応確認済みOS(全てx86-64の64bit版)}
\begin{tabularx}{150mm}{|l|l|X|} \hline
 \rowcolor[gray]{0.9} OS名 & 確認済みバージョン & 備考 \\ \hline
 CentOS                & 6.5、6.6、7.0、7.1 & 左記以外も正常動作する可能性が高い \\ \hline
 openSUSE              & 13.2               &  \\ \hline
 SUSE Enterprise Linux & 11.1、11.3         & 11.1ではGNUコンパイラは使用不可 \\ \hline
 fedora                & 20                 & 左記以外も正常動作する可能性が高い \\ \hline
 Vine Linux            & 6.2、6.3           &  \\ \hline
 Mac OS X              & 10.10 Yosemite     &  \\ \hline
\end{tabularx}
\label{tab:compatible_os}
\end{center}
\end{table}

\begin{table}[htb]
\begin{center}
\caption{対応確認済みコンパイラ}
\begin{tabularx}{150mm}{|l|l|X|} \hline
 \rowcolor[gray]{0.9} コンパイラ名 & 確認済みバージョン & 備考 \\ \hline
 GNU (gcc/gfortran)    & 4.4.x, 4.8.x           & 4.4.xではコンパイル時にWarningが出ることがある \\ \hline
 Intel (icc/ifort)     & 13.0.1、14.0.2、15.0.0 & 2013年以降のバージョンであれば左記以外も正常動作する可能性が高い\\ \hline
 PGI (gcc/pgfortran)   & 14.3                   &  \\ \hline
\end{tabularx}
\label{tab:compatible_compiler}
\end{center}
\end{table}

\begin{table}[htb]
\begin{center}
\caption{対応確認済みMPIライブラリ}
\begin{tabularx}{150mm}{|l|l|X|} \hline
 \rowcolor[gray]{0.9} MPIライブラリ名 & 確認済みバージョン & 備考 \\ \hline
 openMPI   & 1.8.5                    & 左記以外も正常動作する可能性が高い \\ \hline
 Intel MPI & 4.0 Update 3、4.1.0、5.0 & 2013年以降のバージョンであれば左記以外も正常動作する可能性が高い \\ \hline
 SGI MPT   & 2.05、2.09               & Intel Compilerとの組み合わせで確認済み \\ \hline
\end{tabularx}
\label{tab:compatible_mpi}
\end{center}
\end{table}


\subsection{本書で想定する環境}
\label{sec:assumed_env}
%====================================================================================
本書の以降すべての説明において、下記の環境であることを前提として説明をすすめる。

\begin{itemize}
 \item {\bf CPU} : 第\ref{sec:tuto_ideal}章は物理コアが2コア、第\ref{sec:tuto_real}章は4コアを搭載
 \item {\bf Memory} : 第\ref{sec:tuto_ideal}章は512MB、第\ref{sec:tuto_real}章は2GB
 \item {\bf OS} : Linux OS x86-64 (CentOS 6.6、CentOS 7.1、openSUSE 13.2のいずれか)
 \item {\bf コンパイラ} : GNU コンパイラ(gcc/gfortran)
 \item {\bf MPIライブラリ} : openMPI(リポジトリ経由でのインストール)
\end{itemize}


\section{ライブラリ環境のインストール}
\label{sec:inst_env}
%====================================================================================
実行環境に合わせて、SCALEライブラリのインストールに必要な環境、ライブラリのインストールを行う。詳細は、Appendix \ref{sec:env_setting}章を参照のこと。
第\ref{sec:tuto_ideal}章、第\ref{sec:tuto_real}章のチュートリアルは、
それらのライブラリ環境がインストールされていることを想定して進める。

描画ツールは何を使っても良いが、対応を確認している描画ツールについての詳細とインストール方法についての情報は
Appendix \ref{sec:env_vis_tools}を参照のこと。
第3章および第4章のチュートリアルではGrADSを使って描画する方法を紹介する。
その他にGphysを使って、結果のクイックビューを行う方法については、\ref{sec:quicklook}節にて説明する。



\section{SCALEのコンパイル} \label{sec:source_code}
%====================================================================================


以下の説明で使用した環境は次のとおりである。
\begin{itemize}
\item CPU: Intel Core i5 2410M (sandybridge)
\item Memory: DDR3-1333 4GB
\item OS: CentOS 6.6 x86-64, CentOS 7.1 x86-64, openSUSE 13.2 x86-64
\end{itemize}

\subsubsection{ソースコードの入手}
%-----------------------------------------------------------------------------------
安定版ソースコードは,\\
 \url{http://scale.aics.riken.jp/ja/download/index.html}\\
よりダウンロードできる.
ソースコードのtarballファイルを展開すると
\verb|scale/|というディレクトリができる.
\begin{verbatim}
 $ tar -zxvf SCALE_v020.tar.gz
 $ ls
  scale/
\end{verbatim}



\subsubsection{Makedefファイルと環境変数の設定}
%-----------------------------------------------------------------------------------

SCALEはコンパイルするとき、独自の環境変数"\verb|SCALE_SYS|"で与えられた名称に対応した、
Makedefファイルを使用してコンパイルが行われる。
Makdefファイルは、\verb|scale/sysdep/|内にいくつかの計算機環境に
対応するファイル(\verb|Makedef.***|)が準備されており、これらの中から自分の環境にあったものを設定する。
ここでは、先述のとおりLinux OS、GNUコンパイラ、およびopenMPIを使用することを想定しているため、
\verb|"Makedef.Linux64-gnu-ompi"|が対応するファイルとなる。下記のコマンドによって環境変数の値を設定する。
\begin{verbatim}
 $ export SCALE_SYS="Linux64-gnu-ompi"
\end{verbatim}
自分の環境に合致するものがなければ、既存ファイルをベースにして作成する。
実行環境がいつも同じに決まっている場合は、常に同じMakedefファイルを使用するために、
環境変数設定を\verb|.bashrc|などの環境設定ファイルに記述しておくと便利である。\\


チュートリアル通り進めている場合は、Appendix \ref{sec:env_setting}章で説明する以外にPATHの設定をする
必要はないが、環境が異なる場合には上述に加えて次の設定が必要になる。
HDF5とNetCDF4について、下記のようにPATHを設定する。
例えば、Intel コンパイラを利用して自分で"\verb|/opt|"下にHDF5とNetCDF4を
インストールした場合を想定して、HDF5は\verb"/opt/hdf5"、netcdf4は\verb|/opt/netcdf|に
それぞれインストールされている場合の例を示す。
\begin{verbatim}
 $ export HDF5="/opt/hdf5"
 $ export NETCDF4="/opt/netcdf"
\end{verbatim}


\subsubsection{コンパイル}
%-----------------------------------------------------------------------------------

チュートリアル用のディレクトリに移動して,makeコマンドによってコンパイルを行う.
\begin{verbatim}
 $ cd scale/scale-les/test/tutorial/bin
 $ make -j 4
\end{verbatim}
\verb|make|のあとの \verb|"-j 4"| は,
コンパイル時の並列数を示しており,
例では4並列コンパイルを行うことを指示している.
実行環境によっては並列数を増やすこともできる.
このmakeによってSCALEライブラリ,およびSCALE-LESモデルのコンパイルが行われ,
結果として
\begin{verbatim}
 scale-les  scale-les_init  scale-les_pp
\end{verbatim}
の3つの実行ファイルが生成されていればコンパイルは成功である.\\

下記のコマンドで作成された実行バイナリを消去できる。
\begin{verbatim}
 $ make clean
\end{verbatim}
ただし、この場合はコンパイルされたライブラリは消去されないため、全てのコンパイル済みファイルを消去したい場合は、
\begin{verbatim}
 $ make allclean
\end{verbatim}
とする。コンパイラを変更したり、コンパイルオプションを変更して再コンパイルする場合は、"allclean"を実行すること。


{\bf 注意点}
\begin{itemize}
\item SCALEライブラリは,scaleのTOPディレクトリ直下の
 \verb|scale/scalelib/|というディレクトリ内でコンパイルとアーカイブが行われ,
 \verb|".lib"|という名前の隠しディレクトリとして
 \verb|bin/|ディレクトリ内へコピーされる.
\item Debugモードでコンパイルしたい場合は、\verb|"make -j 4 DEBUG=T"|としてコンパイルする。
\item 細かくコンパイルオプションを変更したい場合は、\verb|Makedef.***|のファイルを編集する。
%\item 開発版ソースコードをコンパイルしている場合,
% 一部のコンパイラバージョンにおいて、コンパイルが正常に終了しないケースがある.
% そのような場合はSCALE開発チームまでご報告ください.
\end{itemize}


\section{後処理ツールのコンパイル} \label{sec:source_net2g}
%====================================================================================

第\ref{sec:tuto_ideal}章、第\ref{sec:tuto_real}章のチュートリアルでは、
GrADSを使用して図化する方法を紹介している。このとき、SCALE-LESの分割
出力ファイル(\verb|history.******.nc|)を結合し、GrADSで直接読み込めるデータ形式へ
変換する後処理ツール「net2g」を使用する。
ここでは、このnet2gのコンパイル方法について説明する。

net2gはSCALE本体から独立したツールになっている(ただしMakedefファイルを除く)ため、
任意の場所へコピーしてコンパイルすることができるが、
コンパイルにはNetCDFライブラリが必要であり、また並列実行するためにはMPIライブラリが必要である。
従って、以降はこれらのライブラリがインストールされている環境であることを想定して進める。\\


まず、Makedefファイル設定のために環境変数を設定する。
SCALE本体のコンパイル時と同じように、使用する環境にあったMakedefファイルを
選択し、下記のコマンドにて設定する。

\begin{verbatim}
 $ export SCALE_SYS="Linux64-gnu-ompi"
\end{verbatim}

その後、net2gのディレクトリに移動して、makeコマンドによってコンパイルを行う。
MPIライブラリを用いた並列実行を行うためのバイナリは、下記のコマンドによって生成される。
\begin{verbatim}
 $ cd scale/scale-les/util/netcdf2grads_h
 $ make -j 2
\end{verbatim}
MPIライブラリが無い場合など、逐次実行バイナリを生成するためには、\\
\begin{verbatim}
 $ make -j 2 NOMPI=T
\end{verbatim}
としてコンパイルを行う。
結果として、"net2g"という名前の実行ファイルが生成されていればコンパイルは成功である.\\

下記のコマンドで作成された実行バイナリを消去できる。
\begin{verbatim}
 $ make clean
\end{verbatim}

%####################################################################################

