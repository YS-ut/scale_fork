
%-------------------------------------------------------%
\section{時間積分を行う:run}
%-------------------------------------------------------%
\subsubsection{run.confの準備}
ここではいよいよSCALE-LESモデルを実行する。
まず、runディレクトリへ移動する。
\begin{verbatim}
 $ cd ${Tutrial_DIR}/real/run
\end{verbatim}

runディレクトリの中には、\verb|run.conf|という名前の
コンフィグファイルが準備されており、
ドメインの位置や格子点数など、チュートリアル用の設定(Table\ref{tab:grids})に合わせてある。
モデル本体の実行には事前に作成した地形・土地利用データや初期値・境界値データを利用する。
これらのファイルの場所は、\verb|run.conf|内の
\verb|TOPO_IN_BASENAME|、\verb|LANDUSE_IN_BASENAME|、
\verb|RESTART_IN_BASENAME|、および\verb|ATMOS_BOUNDARY_IN_BASENAME|で
指定している。\\

\noindent {\gt
\ovalbox{
\begin{tabularx}{140mm}{l}
\verb|&PARAM_TOPO| \\
\verb|   TOPO_IN_BASENAME = "../pp/topo_d01",| \\
\verb|/| \\
 \\
\verb|&PARAM_LANDUSE| \\
\verb|   LANDUSE_IN_BASENAME  = "../pp/landuse_d01",| \\
\verb|/| \\
 \\
\verb|&PARAM_RESTART| \\
\verb|   RESTART_OUTPUT      = .false.,| \\
\verb|   RESTART_IN_BASENAME = "../init/init_d01_00019094400.000",| \\
\verb|/| \\
 \\
\verb|&PARAM_ATMOS_BOUNDARY| \\
\verb|   ATMOS_BOUNDARY_TYPE        = "REAL",| \\
\verb|   ATMOS_BOUNDARY_IN_BASENAME = "../init/boundary_d01",| \\
\verb|   ATMOS_BOUNDARY_USE_VELZ    = .true.,| \\
\verb|   ATMOS_BOUNDARY_USE_QHYD    = .false.,| \\
\verb|   ATMOS_BOUNDARY_VALUE_VELZ  = 0.0D0,| \\
\verb|   ATMOS_BOUNDARY_UPDATE_DT   = 21600.0D0,| \\
\verb|/| \\
\end{tabularx}
}}\\


\verb|run.conf|の設定の中で時間積分に関する設定は、\verb|PARAM_TIME|の項目にある。\\

\noindent {\gt
\ovalbox{
\begin{tabularx}{140mm}{l}
\verb|&PARAM_TIME| \\
\verb| TIME_STARTDATE          = 2014, 8, 10, 0, 0, 0,|  ← 時間積分を開始する時刻 \\
\verb| TIME_STARTMS            = 0.D0,| \\
\verb| TIME_DURATION           = 12.0D0,|      ← 積分期間 \\
\verb| TIME_DURATION_UNIT      = "HOUR",|  ← \verb|TIME_DURATION|の単位\\
\verb| TIME_DT                 = 60.0D0,|            ← 移流計算の時間ステップ\\
\verb| TIME_DT_UNIT            = "SEC",|          ← \verb|TIME_DT|の単位\\
\verb| TIME_DT_ATMOS_DYN       = 15.0D0,|      ← 移流計算以外の力学過程の時間ステップ\\
\verb| TIME_DT_ATMOS_DYN_UNIT  = "SEC",|  ← \verb|TIME_DT_ATMOS_DYN|の単位\\
 \\
\verb| ~~中略~~| \\
 \\
\verb|/| \\
\end{tabularx}
}}\\

\noindent 初期時刻\verb|TIME_STARTDATE|はUTCで指定する。
チュートリアルでは2014年8月10日0時UTCに設定している。
積分のための時間ステップは、上記の他、
それぞれの物理スキーム毎に設定できるようになっている。


計算結果の出力に関する設定は\verb|PARAM_HISTORY|で行う。\\

\noindent {\gt
\ovalbox{
\begin{tabularx}{140mm}{l}
\verb|&PARAM_HISTORY| \\
\verb|   HISTORY_DEFAULT_BASENAME  = "history_d01",|  ← 出力するファイル名\\
\verb|   HISTORY_DEFAULT_TINTERVAL = 1800.D0,|      ← 出力時間間隔\\
\verb|   HISTORY_DEFAULT_TUNIT     = "SEC",|          ← 出力時間間隔の単位\\
\verb|   HISTORY_DEFAULT_TAVERAGE  = .false.,| \\
\verb|   HISTORY_DEFAULT_DATATYPE  = "REAL4",| \\
\verb|   HISTORY_DEFAULT_ZINTERP   = .false.,|  ← 出力時に高さ面へ内挿するかどうか\\
\verb|   HISTORY_OUTPUT_STEP0      = .true.,|  ← 初期時刻(t=0)の値を出力するかどうか\\
\verb|/| \\
\end{tabularx}
}}\\

\noindent 上記の設定に従って、下記の\verb|HISTITEM|に羅列された変数が出力される。
\verb|HISTITEM|ではオプション変数を加えることで、変数毎に、出力間隔を変更したり、
平均値を出力したりすることも可能である。
これらの説明は\ref{sec:output}を参照されたい。\\

\noindent {\small {\gt
\ovalbox{
\begin{tabularx}{150mm}{l}
\verb|&HISTITEM item="DENS" /              ! density (3D)| \\
\verb|&HISTITEM item="MOMZ" /              ! vertical momentum (3D)| \\
\verb|&HISTITEM item="MOMX" /              ! horizontal momentum-x (3D)| \\
\verb|&HISTITEM item="MOMY" /              ! horizontal momentum-y (3D)| \\
\verb|&HISTITEM item="RHOT" /              ! density * potential-temperature (3D)| \\
\verb|&HISTITEM item="QV"   /               ! mixing ratio for vapor (3D)| \\
\verb|&HISTITEM item="QHYD" /              ! mixing ratio for hydrometeor (3D)| \\
\verb|&HISTITEM item="T"    /               ! temperature (3D)| \\
\verb|&HISTITEM item="PRES" /              ! pressure (3D)| \\
\verb|&HISTITEM item="U"    /               ! horizontal wind component-x (3D)| \\
\verb|&HISTITEM item="V"    /               ! horizontal wind component-y (3D)| \\
\verb|&HISTITEM item="W"    /               ! vertical wind component (3D)| \\
\verb|&HISTITEM item="PT"   /               ! potential temperature (3D)| \\
\verb|&HISTITEM item="RH"   /               ! relative humidity (3D)| \\
\verb|&HISTITEM item="PREC" /              ! precipitation (2D)| \\
\verb|&HISTITEM item="OLR"  /                ! out-going longwave radiation(2D)| \\
\verb|&HISTITEM item="U10" /                 ! horizontal wind component-x at 10m height(2D)| \\
\verb|&HISTITEM item="V10" /                 ! horizontal wind component-y at 10m height(2D)| \\
\verb|&HISTITEM item="T2"  /                ! temperature at 2m height (2D)| \\
\verb|&HISTITEM item="Q2"  /                ! mixing ratio for vapor at 2m height (2D)| \\
\verb|&HISTITEM item="SFC_PRES"   /       ! pressure at the bottom surface (2D)| \\
\verb|&HISTITEM item="SFC_TEMP"   /       ! temperature a the bottom surface (2D)| \\
\verb|&HISTITEM item="LAND_SFC_TEMP" /  ! temperature a the bottom surface for land model (2D)| \\
\verb|&HISTITEM item="URBAN_SFC_TEMP"/ ! temperature a the bottom surface for urban model (2D)| \\
\end{tabularx}
}}}\\

\noindent その他に実験で使用される物理過程の設定は、
\verb|PARAM_TRACER,PARAM_ATMOS,PARAM_OCEAN,PARAM_LAND,PARAM_URBAN|の項目に
記述されているので、実行前にチェックすること。
詳細なコンフィグファイルの内容については、Appendix \ref{app:namelist}を参照されたい。

%
\subsubsection{実行}
コンパイル済みのバイナリをrunディレクトリへリンクする。

\begin{verbatim}
 $ ln -s ../../bin/scale-les ./
\end{verbatim}
陸面過程や放射過程のモデルを起動するためのパラメータファイルに
リンクを張る。
\begin{verbatim}
 $ ln -s ../../../data/land/* ./   <- 陸面スキーム用のパラメータファイル
 $ ln -s ../../../data/rad/*  ./   <- 放射スキーム用のパラメータファイル
\end{verbatim}
準備が整ったら、4つのMPIプロセスを使用してscale-lesを実行する。
\begin{verbatim}
  $ mpirun -n 4 ./scale-les run.conf < /dev/null >&log&
\end{verbatim}

実行にはおおよそ1時間を要するため、上記のように標準出力をファイルへ
書き出すようにしてバックグラウンドで実行すると便利である。
計算が開始されれば,処理内容のログとして、
\verb|"LOG_d01.pe000000"|ファイルが生成されるので、
例えば下記のようなコマンドで\verb|"LOG_d01.pe000000"|ファイルを参照すれば、
どこまで計算が進んでいるかチェックすることができる。
\begin{verbatim}
 $ tail -f LOG_d01.pe000000
\end{verbatim}
正常にジョブが終了すれば、\verb|history_d01.pe######.nc|と
\verb|restart_d01.pe######.nc|という名前のファイルがMPIプロセス数だけ、
つまり4つずつ生成される(\verb|######|にはMPIプロセスの番号が入る)。
historyファイルは計算結果の出力ファイルであり、
\verb|HISTITEM|に指定した変数のみ書き出される。
restartファイルは対応する時刻を開始時刻として
再計算を開始するための初期値ファイルである。
次節でhistoryデータをGrADSで描画可能なバイナリーデータに変換して
結果を確認する方法について説明する。

%####################################################################################

